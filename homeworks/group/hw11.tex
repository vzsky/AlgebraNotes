% chktex-file 44
% chktex-file 8

\documentclass{report}
\usepackage{amsthm}
\usepackage{amsmath}
\usepackage{amssymb}
\usepackage{amssymb}
\usepackage{amsfonts}
\usepackage{xcolor}
\usepackage{tikz}
\usepackage{fancyhdr}
\usepackage{enumerate}
\usepackage{graphicx}
\usepackage[normalem]{ulem}
\usepackage[most,many,breakable]{tcolorbox}
\usepackage[a4paper, top=80pt, foot=25pt, bottom=50pt, left=0.5in, right=0.5in]{geometry}
\usepackage{hyperref, theoremref}
\hypersetup{
	pdftitle={Assignment},
	colorlinks=true, linkcolor=b!90,
	bookmarksnumbered=true,
	bookmarksopen=true
}
\usepackage{nameref}
\usepackage{parskip}
\pagestyle{fancy}

\usepackage[explicit,compact]{titlesec}
\titleformat{\chapter}[block]{\bfseries\huge}{\thechapter. }{\compact}{#1}
        

%%%%%%%%%%%%%%%%%%%%%
%% Defining colors %%
%%%%%%%%%%%%%%%%%%%%%

\definecolor{lr}{RGB}{188, 75, 81}
\definecolor{r}{RGB}{249, 65, 68}
\definecolor{dr}{RGB}{174, 32, 18}
\definecolor{lo}{RGB}{255, 172, 129}
\definecolor{do}{RGB}{202, 103, 2}
\definecolor{o}{RGB}{238, 155, 0}
\definecolor{ly}{RGB}{255, 241, 133}
\definecolor{y}{RGB}{255, 229, 31}
\definecolor{dy}{RGB}{143, 126, 0}
\definecolor{lb}{RGB}{148, 210, 189}
\definecolor{bg}{RGB}{10, 147, 150}
\definecolor{b}{RGB}{39, 125, 161}
\definecolor{db}{RGB}{0, 95, 115}
\definecolor{p}{RGB}{229, 152, 155}
\definecolor{dp}{RGB}{181, 101, 118}
\definecolor{pp}{RGB}{142, 143, 184}
\definecolor{v}{RGB}{109, 89, 122}
\definecolor{lg}{RGB}{144, 190, 109}
\definecolor{g}{RGB}{64, 145, 108}
\definecolor{dg}{RGB}{45, 106, 79}

\colorlet{mysol}{g}
\colorlet{mythm}{lr}
\colorlet{myqst}{db}
\colorlet{myclm}{lb}
\colorlet{mywrong}{r}
\colorlet{mylem}{o}
\colorlet{mydef}{lg}
\colorlet{mycor}{lb}
\colorlet{myrem}{dr}

%%%%%%%%%%%%%%%%%%%%%

\newcommand{\col}[2]{
  \color{#1}#2\color{black}\,
}

\newcommand{\TODO}[1][5cm]{
  \color{red}TODO\color{black}
  \vspace{#1}
}

\newcommand{\wans}[1]{
	\noindent\color{mywrong}\textbf{Wrong answer: }\color{black}
	#1 


}

\newcommand{\wreason}[1]{
	\noindent\color{mywrong}\textbf{Reason: }\color{black}
	#1 

  
}

\newcommand{\sol}[1]{
	\noindent\color{mysol}\textbf{Solution: }\color{black}
	#1


}

\newcommand{\nt}[1]{
  \begin{note}Note: #1\end{note}
}

\newcommand{\ky}[1]{
  \begin{key}#1\end{key}
}

\newcommand{\pf}[1]{
  \begin{myproof}#1\end{myproof}
}

\newcommand{\qs}[3][]{
  \begin{question}{#2}{#1}#3\end{question}
}

\newcommand{\df}[3][]{
  \begin{definition}{#2}{#1}#3\end{definition}
}

\newcommand{\thm}[3][]{
  \begin{theorem}{#2}{#1}#3\end{theorem}
}

\newcommand{\clm}[3][]{
  \begin{claim}{#2}{#1}#3\end{claim} 
}

\newcommand{\lem}[3][]{
  \begin{lemma}{#2}{#1}#3\end{lemma}
}

\newcommand{\cor}[3][]{
  \begin{corollary}{#2}{#1}#3\end{corollary}
}

\newcommand{\rem}[3][]{
  \begin{remark}{#2}{#1}#3\end{remark}
}

\newcommand{\twoways}[2]{
  \leavevmode\\
  ($\Longrightarrow$): 
  \begin{shift}#1\end{shift}
  ($\Longleftarrow$):
  \begin{shift}#2\end{shift} 
}

\newcommand{\nways}[2]{
  \leavevmode\\
  ($#1$): 
  \begin{shift}#2\end{shift}
}

%%%%%%%%%%%%%%%%%%%%%%%%%%%%%% ENVRN

\newenvironment{myproof}[1][\proofname]{%
	\proof[\bfseries #1: ]
}{\endproof}

\tcbuselibrary{theorems,skins,hooks}
\newtcolorbox{shift}
{%
  before upper={\setlength{\parskip}{5pt}},
  blanker,
	breakable,
	width=0.95\textwidth,
  enlarge left by=0.03\textwidth,
}

\tcbuselibrary{theorems,skins,hooks}
\newtcolorbox{key}
{%
	breakable,
	width=0.95\textwidth,
  enlarge left by=0.03\textwidth,
}

\tcbuselibrary{theorems,skins,hooks}
\newtcolorbox{note}
{%
	enhanced,
	breakable,
	colback = white,
	width=\textwidth,
	frame hidden,
	borderline west = {2pt}{0pt}{black},
	sharp corners,
}

\tcbuselibrary{theorems,skins,hooks}
\newtcbtheorem[]{remark}{Remark}
{%
	enhanced,
	breakable,
	colback = white,
	frame hidden,
	boxrule = 0sp,
	borderline west = {2pt}{0pt}{myrem},
	sharp corners,
	detach title,
  before upper={\setlength{\parskip}{5pt}\tcbtitle\par\smallskip},
	coltitle = myrem,
	fonttitle = \bfseries\sffamily,
	description font = \mdseries,
	separator sign none,
	segmentation style={solid, myrem},
}{rem}

\tcbuselibrary{theorems,skins,hooks}
\newtcbtheorem[number within=section]{lemma}{Lemma}
{%
	enhanced,
	breakable,
	colback = white,
	frame hidden,
	boxrule = 0sp,
	borderline west = {2pt}{0pt}{mylem},
	sharp corners,
	detach title,
  before upper={\setlength{\parskip}{5pt}\tcbtitle\par\smallskip},
	coltitle = mylem,
	fonttitle = \bfseries\sffamily,
	description font = \mdseries,
	separator sign none,
	segmentation style={solid, mylem},
}{lem}

\tcbuselibrary{theorems,skins,hooks}
\newtcbtheorem{claim}{Claim}
{%
  parbox=false,
	enhanced,
	breakable,
	colback = white,
	frame hidden,
	boxrule = 0sp,
	borderline west = {2pt}{0pt}{myclm},
	sharp corners,
	detach title,
  before upper={\setlength{\parskip}{5pt}\tcbtitle\par\smallskip},
	coltitle = myclm,
	fonttitle = \bfseries\sffamily,
	description font = \mdseries,
	separator sign none,
	segmentation style={solid, myclm},
}{clm}

\makeatletter
\newtcbtheorem[number within=section, use counter from=lemma]{theorem}{Theorem}{enhanced,
	breakable,
	colback=white,
	colframe=mythm,
	attach boxed title to top left={yshift*=-\tcboxedtitleheight},
	fonttitle=\bfseries,
	title={#2},
	boxed title size=title,
	boxed title style={%
			sharp corners,
			rounded corners=northwest,
			colback=mythm,
			boxrule=0pt,
		},
	underlay boxed title={%
			\path[fill=mythm] (title.south west)--(title.south east)
			to[out=0, in=180] ([xshift=5mm]title.east)--
			(title.center-|frame.east)
			[rounded corners=\kvtcb@arc] |-
			(frame.north) -| cycle;
		},
	#1
}{thm}
\makeatother

\makeatletter
\newtcbtheorem{question}{Question}{enhanced,
	breakable,
	colback=white,
	colframe=myqst,
	attach boxed title to top left={yshift*=-\tcboxedtitleheight},
	fonttitle=\bfseries,
	title={#2},
	boxed title size=title,
	boxed title style={%
			sharp corners,
			rounded corners=northwest,
			colback=myqst,
			boxrule=0pt,
		},
	underlay boxed title={%
			\path[fill=myqst] (title.south west)--(title.south east)
			to[out=0, in=180] ([xshift=5mm]title.east)--
			(title.center-|frame.east)
			[rounded corners=\kvtcb@arc] |-
			(frame.north) -| cycle;
		},
	#1
}{qs}
\makeatother

\makeatletter
\newtcbtheorem[number within=section]{definition}{Definition}{enhanced,
	breakable,
	colback=white,
	colframe=mydef,
	attach boxed title to top left={yshift*=-\tcboxedtitleheight},
	fonttitle=\bfseries,
	title={#2},
	boxed title size=title,
	boxed title style={%
			sharp corners,
			rounded corners=northwest,
			colback=mydef,
			boxrule=0pt,
		},
	underlay boxed title={%
			\path[fill=mydef] (title.south west)--(title.south east)
			to[out=0, in=180] ([xshift=5mm]title.east)--
			(title.center-|frame.east)
			[rounded corners=\kvtcb@arc] |-
			(frame.north) -| cycle;
		},
	#1
}{def}
\makeatother

\makeatletter
\newtcbtheorem[number within=section, use counter from=lemma]{corollary}{Corollary}{enhanced,
	breakable,
	colback=white,
	colframe=mycor,
	attach boxed title to top left={yshift*=-\tcboxedtitleheight},
	fonttitle=\bfseries,
	title={#2},
	boxed title size=title,
	boxed title style={%
			sharp corners,
			rounded corners=northwest,
			colback=mycor,
			boxrule=0pt,
		},
	underlay boxed title={%
			\path[fill=mycor] (title.south west)--(title.south east)
			to[out=0, in=180] ([xshift=5mm]title.east)--
			(title.center-|frame.east)
			[rounded corners=\kvtcb@arc] |-
			(frame.north) -| cycle;
		},
	#1
}{cor}
\makeatother

% Basic
  \DeclareMathOperator{\lcm}{lcm}
  \newcommand{\Real}{\mathbb{R}}
  \newcommand{\Comp}{\mathbb{C}}
  \newcommand{\Nat}{\mathbb{N}}
  \newcommand{\Rat}{\mathbb{Q}}
  \newcommand{\Int}{\mathbb{Z}}
  \newcommand{\set}[1]{\left\{\, #1 \,\right\}}
  \newcommand{\paren}[1]{\left( \; #1 \; \right)}
  \newcommand{\abs}[1]{\left\lvert #1 \right\rvert}
  \newcommand{\ang}[1]{\left\langle #1 \right\rangle}
  \renewcommand{\to}[1][]{\xrightarrow{\text{#1}}}
  \newcommand{\tol}[1][]{\to{$#1$}}
  \newcommand{\curle}{\preccurlyeq}
  \newcommand{\curge}{\succcurlyeq}
  \newcommand{\mapsfrom}{\leftarrow\!\shortmid}

  \newcommand{\mat}[1]{\begin{bmatrix} #1 \end{bmatrix}}
  \newcommand{\pmat}[1]{\begin{pmatrix} #1 \end{pmatrix}}
  \newcommand{\eqs}[1]{\begin{align*} #1 \end{align*}}
  \newcommand{\case}[1]{\begin{cases} #1 \end{cases}}
  

  % Algebra
  \newcommand{\normSg}[0]{\vartriangleleft}
  \newcommand{\ZMod}[1][n]{\mathbb{Z}/#1\mathbb{Z}}
  \newcommand{\isom}{\simeq}
  \newcommand{\mapHom}{\xrightarrow{\text{hom}}}
  \DeclareMathOperator{\Inn}{Inn}
  \DeclareMathOperator{\Aut}{Aut}
  \DeclareMathOperator{\im}{im}
  \DeclareMathOperator{\ord}{ord}
  \DeclareMathOperator{\Gal}{Gal}
  \DeclareMathOperator{\chr}{char}
  \newcommand{\surjto}{\twoheadrightarrow}
  \newcommand{\injto}{\hookrightarrow}

  % Analysis 
  \newcommand{\limty}[1][k]{\lim_{#1\to\infty}}
  \newcommand{\norm}[1]{\left\lVert#1\right\rVert}
  \newcommand{\darrow}{\rightrightarrows}


\fancyhead[L]{Homework 11 - Modern Algebra MAS311}
\fancyhead[R]{\textbf{Touch Sungkawichai} 20210821}

\begin{document}
\qs{}{
  Show that two rings $\Int[x]$ and $\Rat[x]$ are not isomorphic.
}
\sol{
  There is no element in $\Int[x]$ such that $a + a = 1$.
  This is because for $a \in \Int[x]$, it follows that 
  \[a + a = \sum_{i=0}^n a_ix^i + \sum_{i=0}^n a_ix^i = \sum_{i=0}^n (a_i + a_i)x^i\] 
  If $ a + a = 1 $, then $a_0 + a_0 = 1$ for $a_0 \in \Int$. However, there is no such element in $\Int$. 

  But there is such an element in $\Rat[x]$, which is $\frac{1}{2}x^0$.
} 

\qs{}{
  Let $R$ and $S$ be nonzero rings and let $\phi: R \to S$ be a nonzero ring homomorphism (so $\phi(1_R) = 1_S$, where $1_R$ and $1_S$
  denote the identities of $R$ and $S$, respectively).

  Show that $\phi(u)$ is a unit in $S$ and $\phi(u^{-1}) = \phi(u)^{-1}$ for each unit $u$ of $R$.
}
\sol{
  Let $u$ be a unit of $R$, then it follows that there is an element $u^{-1}$ such that $uu^{-1} = u^{-1}u = 1_R$.
  Since $\phi$ is an homomorphism, then, $\phi(ab) = \phi(a)\phi(b)$ for any element $a, b \in R$ by definition.
  This means that $1_S = \phi(1_R) = \phi(uu^{-1}) = \phi(u)\phi(u^{-1})$, and similarly, $1_S = \phi(1_R) = \phi(u^{-1}u) = 
  \phi(u^{-1})\phi(u)$.

  As $\phi(u)\phi(u^{-1}) = 1_S = \phi(u^{-1})\phi(u)$, then $\phi(u^{-1}) = \phi(u)^{-1}$ by definition.
} 

\qs{}{
  In problem 2, prove that if $\phi(1_R) \ne 1_S$ then $\phi(1_R)$ is a zero divisor in $S$.
}
\sol{
  Since $1_S \in S$ and $\phi(1_R) \in S$, then 
  \[ 1_S \phi(1_R) = 1_S \phi(1_R \cdot 1_R) = 1_S \phi(1_R) \phi_(1_R) = \phi(1_R) \phi(1_R) \]
  Which is that $1_S \phi(1_R) - \phi(1_R) \phi(1_R) = 0$, or $(1_S - \phi(1_R)) \phi(1_R) = 0$. 
  If $\phi(1_R) \ne 1_S$, then $1_S - \phi(1_R) \ne 0$. And $\phi(1_R) \ne 0$ since if $1_R = 0$, 
  then $\phi(x) = \phi(1_R x) = 0 \phi(x) = 0$ which means that $\phi$ is a zero homomorphism.
  
  But as $\phi(1_R) (1_S - \phi(1_R)) = 0$ while both terms are non-zero, it must be the case that $\phi(1_R)$ is a zero-divisor.
}

\qs{}{
  Let $R$ be a ring. Prove that the center of the ring $M_n(R)$ is the set of all scalar matrices $aI$, where $I$ is the identity
  matrix and $a$ is an element of the center of $R$.
}
\newcommand{\lmat}[4]{\mat{#1 & \cdots & #2 \\ \vdots & \ddots & \vdots \\ #3 & \cdots & #4}}
\sol{
  Consider if $a \in Z(R)$, then $a$ commutes with every element in $R$, so
  \[ \lmat a 0 0 a \lmat {b_{11}} {b_{n1}} {b_{n1}} {b_{nn}} = \lmat{ab_{11}}{ab_{1n}}{ab_{n1}}{ab_{nn}} 
  = \lmat{b_{11}a} {b_{1n}a} {b_{n1}a} {b_{nn}a} = \lmat{b_{11}}{b_{1n}}{b_{n1}}{b_{nn}}\lmat{a}{0}{0}{a} \]
  
  Therefore, $aI$ is an element of $Z(M_n(R))$.

  Now, consider an element $A$ of $Z(M_n(R))$, then $AB = BA$ for all $B \in M_n(R)$.

  \begin{center}
  Let consider arbitrary $A = \lmat{a_{11}}{a_{1n}}{a_{n1}}{a_{nn}}$ and $B = \lmat{b_{11}}{b_{1n}}{b_{n1}}{b_{nn}}$
  \end{center}

  Then, $(AB)_{ik} = \sum_{j = 0}^n a_{ij}b_{jk}$ and $(BA)_{ik} = \sum_{j=0}^n b_{ij}a_{jk}$.
  Therefore, if $A \in Z(M_n(R))$, then it must follow that 
  \[ \sum_{j=0}^n a_{ij}b_{jk} = \sum_{j=0}^n b_{ij}a_{jk} \text{ for any } B \in Z(R) \] 
  
  Since $b$ is arbitrary, consider if $B = bI$, for any $b \in R$ and $b \not\in Z(R)$, then this yields 
  \[ \lmat{a_{11}b}{a_{1n}b}{a_{n1}b}{a_{nn}b} = \lmat{ba_{11}}{ba_{1n}}{ba_{n1}}{ba_{nn}} \]
  So, $a_{ij}b = ba_{ij}$, but since $b$ is arbitrary, then $a_{ij} \in Z(R)$ for any $i, j$.

  Therefore it must follow that $\sum_{j=0}^n a_{ij}b_{jk} = \sum_{j=0}^n b_{ij}a_{jk} = \sum_{j=0}^n a_{jk}b_{ij}$. 
  
  Now, if $B$ were chosen so that $b_{jk} = b_{ij} = 0$ except for $b_{ij'} \ne b_{j'k}$ only at certain $j'$, 
  then \[ 
    a_{ij'}b_{j'k} = \sum_{j\ne j'} a_{ij}b_{ij} + a_{ij'}b_{j'k} = \sum_{j=0}^n a_{ij}b_{ij} 
    = \sum_{j=0}^n a_{jk}b_{ij} = \sum_{j \ne j'} a_{jk}b_{ij} + a_{j'k}b_{ij'} = a_{j'k}b_{ij'}
  \]
  If $i = j' = k$, then $a_{j'j'}b_{j'j'} = a_{j'j'}b_{j'j'}$ trivially, 
  But otherwise, since $b_{ij'} \ne b_{j'k}$, it must be the case that $a_{ij'} = a_{j'k} = 0$
  Thus, $A$ must be diagonal.

  Then consider $b_{ij} = 1$ for all entries.
  So,  
  \[ \lmat{a_1}{0}{0}{a_n}\lmat 1111 = \lmat{a_1}{a_1}{a_n}{a_n} = \lmat{a_1}{a_n}{a_1}{a_n} = \lmat 1111 \lmat{a_1}00{a_n} \]
  Which, by considering the element in the $i^{th}$ row and $j^{th}$ column, yeilds $a_i = a_j$.
  Which means that $a_1 = a_2 = \cdots = a_n$.
  So, $A = aI$ for some $a \in Z(R)$
} 

\qs{}{
  Let $R$ be a commutative ring. Let $f(x) = a_nx^n + a_{n-1}x^{n-1} + \cdots + a_1x + a_0$ be an element of the polynomial ring $R[x]$.
  Prove that $f(x)$ is nilpotent in $R[x]$ if and only if $a_0, a_1, \ldots, a_n$ are nilpotent elements of $R$.
}
\sol{
  \twoways{
    \clm{
      If $a$ and $b$ are nilpotent, then $a-b$ is nilpotent.
      \pf{
        If $a$ and $b$ are nilpotent, then let $m$ be an integer such that $a^m = b^m = 0$, 
        Then $(a-b)^{2m} = a^{2m} -\begin{pmatrix} 2m-1 \\ 1 \end{pmatrix} a^{2m-1}b + \cdots + b^{2m}$. 
        And since each term contain either $a^m$ or $b^m$. Therefore, all the terms are zero. Thus, $(a-b)^{2m} = 0$. 
        So, $a-b$ is nilpotent.
      }
    }
    Assume that $f_k(x)$ be a degree $k$ nilpotent polynomial such that $f_k(x)^m = 0$. 
    For induction, assume that $f_{k-1}(x)$ is nilpotent implies $a_{k-1}, \ldots, a_0$ are all nilpotent, then
    \[f_k(x)^m = (a_kx^k + f_{k-1}(x))^m = (a_k^m x^km + f_{k-1}(x)^m) \]
    Therefore, $a_k^m x^km = 0$ follows, so, $a_k$ must be nilpotent. 
    This means that $f_k(x)$ is nilpotent implies $a_{k}$ is nilpotent and $f_k(x) - a_{k}x^k$ is nilpotent by claim.
    So, by the induction hypothesis, $a_k, a_{k-1}, \ldots a_0$ are all nilpotent.
    Note that if $f_0(k) = a_0$ is nilpotent, then $a_0$ is nilpotent trivially.

    Therefore, if $f(x)$ is nilpotent, then $a_0, a_1, \ldots a_n$ are all nilpotent.
  }{
    If $a_0, \ldots, a_n$ are all nilpotent, then there is an integer $m$ such that $a_0^m = a_1^m = \cdots = a_n^m$.
    Now, consider $(a_nx^n + \cdots + a_0)^{nm}$.
    The coefficients of $x^k$ is a summation of all product that contains 
    $a_n^{i_n} \cdots a_0^{i_0}$ for which $\sum_{j=0}^n ji_j = k$ and 
    $\sum_{j=0}^n i_j = nm$

    But since $\sum_{j=0}^n i_j = nm$ then by pigeon hole principle, there must be at least one $j$ such that $i_j \ge m$
    Therefore, the product 
    \[ a_n^{i_n} \cdots a_0^{i_0} = a_n^{i_n} \cdots a_j^{i_j} \cdots a_0^{i_0} = a_n^{i_n} \cdots 0 \cdots a_n^{i_n} = 0\]
    So the coefficient of $x^k$ is a summation of all products, which are all zeros, so the coefficient of $x^k$ is zero, for any $k$.

    Therefore, $(a_nx^n + \cdots + a_0)^{nm} = 0$, so $a_nx^n + \cdots + a_0$ is nilpotent. 
  }
} 

\qs{}{
  Let $R$ be a commutative ring. Let $g(x)$ be a nonzero polynomial in $R[x]$. Prove that $g(x)$ is a zero divisor in $R[x]$ if and 
  only if there is a nonzero $b \in R$ such that $b \cdot g(x) = 0$.
}
\sol{
  \twoways{
    Let $g(x)$ be a nonzero polynomial, written $g(x) = g_0 + \cdots + g_nx^n$ and $g(x)f(x) = 0$ for $f$ with smallest degree. 
    Assume that degree of $f$ is greater than $0$. Let that degree of $f$ be $m$.
    Then consider $f(x) = f_0 + f_1x + \cdots + f_mx^m$. 
    
    If, for every $i$, $g_ix^i \cdot f(x) = 0$, then $g_ix^i \cdot f_mx^m = 0$, so $g(x)\cdot f_m = 0$. So, the statement was proved. 
    Otherwise, let $i$ be the highest degree such that $g_ix^i \cdot f(x) \ne 0$.
    So, 
    \[ g(x) f(x) = (g_0 + \cdots + g_ix^i + \cdots + g_nx^n)(f_0 + \cdots + f_mx^m) = (g_0 + \cdots + g_ix^i)(f_0 + \cdots + f_mx^m)\]
    But then,  $g(x)f(x) = 0$ implies that $g_i f_m = 0$. 
    So, $g_i f(x) = g_if_0 + \cdots + g_if_m x^m = g_if_0 + \cdots + g_if_{m-1}x^{m-1}$ is a degree $m-1$ polynomial.
    
    This means that $g(x) g_i f(x) = 0$, but $g_if(x)$ is a degree $m-1$ polynomial, contradicting that $f$ is the smallest degree.
    Therefore, $f$ must be a degree 0 polynomial, or simply $g(x)f_0 = 0$ for some $f_0 \in R$.
  }{
    if $b \in R$, then $b = bx^0 \in R[x]$. So, if $b \cdot g(x) = 0$, then $g(x)$ is a zero-divisor.
  }
} 

\qs{}{
  Let $R$ be a commutative ring. Let $R[[x]$ be formal power series in the indeterminate $x$ with coefficients in $R$.
  Show that $1-x$ is a unit in $R[[x]]$. 
}
\sol{
  Consider an element $p = \sum_{i=0}^\infty x^i = 1 + x + x^2 + \cdots$. Then 
  $$p(1-x) = (1-x)p = 1 - x + x - x^2 + x^2 - \cdots = 1 - (0) - (0) - \cdots = 1$$
  So, $p$ is the inverse of $1-x$. Thus, $1-x$ is a unit.
} 

\qs{}{
  In problem 7, prove that $\sum_{n=0}^\infty a_nx^n$ is a unit in $R[[x]]$ if and only if $a_0$ is a unit in $R$.
}
\sol{
  \twoways{
    If $\sum_{n=0}^\infty a_nx^n$ have an inverse, then $(\sum_{n=0}^\infty a_nx^n)(\sum_{n=0}^\infty b_nx^n) = 1$. 
    Then the constant term of the equality is $a_0 \times b_0 = 1$
    Therefore, $a_0$ must be a unit in $R$
  }{
    If $a_0$ is a unit, then there exists $b$ such that $a_0b = 1$. 
    Then assume that a degree $n$ polynomial with the constant term being $a_0$ is a unit for induction. This holds for $n=0$.
    Let that degree $n$ polynomial be $f(x)$, and $b \in R[[x]]$ such that $b \cdot f(x) = 1$ 
    then 
    \eqs{ (f(x) + a_{n+1}x^{n+1}) 
      &\cdot (b - b^2a_{n+1}x^{n+1} + b^3a_{n+1}^2x^{(n+1)2} + \cdots) \\ 
      &= bf(x) \; +\; ba_{n+1}x^{n+1}\; -\; bba_{n+1}f(x)x^{n+1}\; -\; bba_{n+1}^2 x^{(n+1)2}\; +\; \cdots \\ 
      &= 1 \; + \; ba_{n+1}x^{n+1}\; - \; ba_{n+1}x^{n+1} \; - \; b^2 a_{n+1}^2 x^{(n+1)2} \; + \; \cdots \\
      &= 1 \; + \; 0 + \cdots \; 0 \; \cdots \\ 
      &= 1
    }
    Now, as $b - b^2a_{n+1}x^{n+1} + b^3a_{n+1}^2x^{(n+1)2} + \cdots$ is an element of $R[[x]]$, then the argument holds for 
    a polynomial of degree $n+1$.
    Thus, the argument holds for any element of $R[[x]]$ by induction.
  }
} 

\qs{}{
  Prove that if $R$ is an integral domain, then the ring of formal power series $R[[x]]$ is also an integral domain. 
}
\sol{
  If $R[[x]]$ is not an integral domain, then there is a zero-divisor. Let there be nonzero elements $A, B$ such that 
  $A = \sum_{i=0}^\infty a_nx^n$ and $B = \sum_{i=0}^\infty b_nx^n$ and $AB = 0$

  As $A$ and $B$ are nonzero, let $A = A'x^{k_a}$ and $B = B'x^{k_b}$ for which $k_a$ and $k_b$ is the smallest non-zero term of 
  $A$ and $B$. There exist this number by the well-ordering principle.

  So, \[ 
    0 = (\sum_{i=0}^\infty a_nx^n)(\sum_{i=0}^\infty b_nx^n) 
    = (a_{k_a}b_{k_b} + a_{k_a+1}b_{k_b}x + b_{k_b+1}a_{k_a}x + \cdots)x^{k_a+k_b} 
  \]
  Hence, $a_{k_a} \cdot b_{k_b} = 0$, which means that $a_{k_a} \in R$ is a zero-divisor since $a_{k_a} \ne 0$,
  and $b_{k_b} \ne 0$ by construction, thus $R$ is not an integral domain.
  So, the argument is proved by contraposition.
}

\qs{}{
  Let $R$ be a commutative ring. Let $G = \set{g_1, \ldots g_n}$ be a finite group. Prove that the element $N = g_1 + g_2 + \cdots + 
  g_n$ is in the center of the group ring $RG$ 
}
\sol{
  As $RG = \set{\sum_{g} r_g \cdot g \mid \forall r}$, then let $N = g_1 + g_2 + \cdots + g_n$.
  Consider an arbitrary element $a = \sum_{g} r_g \cdot g = r_1g_1 + \cdots + r_ng_n$.
  Then, \eqs{Na &= N(r_1g_1 + \cdots + r_ng_n) \\ 
                &= (g_1 + \cdots g_n)(r_1g_1 + \cdots + r_ng_n) \\ 
                &= r_1g_1g_1 + \cdots + r_1g_1g_n + r_2g_2g_1 + \cdots + r_2g_2g_n + \cdots + r_ng_ng_n \\ 
                &= r_1g_1g_1 + r_2g_2g_1 + \cdots + r_ng_ng_1 + \cdots + r_ng_ng_n \\ 
                &= (r_1g_1 + \cdots + r_ng_n)(g_1 + \cdots + g_n) \\ 
                &= aN}
  So, $N \in Z(RG)$
}

\end{document}
