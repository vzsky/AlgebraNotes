% chktex-file 44
% chktex-file 8

\documentclass{report}
\usepackage{amsthm}
\usepackage{amsmath}
\usepackage{amssymb}
\usepackage{amssymb}
\usepackage{amsfonts}
\usepackage{xcolor}
\usepackage{tikz}
\usepackage{fancyhdr}
\usepackage{enumerate}
\usepackage{graphicx}
\usepackage[normalem]{ulem}
\usepackage[most,many,breakable]{tcolorbox}
\usepackage[a4paper, top=80pt, foot=25pt, bottom=50pt, left=0.5in, right=0.5in]{geometry}
\usepackage{hyperref, theoremref}
\hypersetup{
	pdftitle={Assignment},
	colorlinks=true, linkcolor=b!90,
	bookmarksnumbered=true,
	bookmarksopen=true
}
\usepackage{nameref}
\usepackage{parskip}
\pagestyle{fancy}

\usepackage[explicit,compact]{titlesec}
\titleformat{\chapter}[block]{\bfseries\huge}{\thechapter. }{\compact}{#1}
        

%%%%%%%%%%%%%%%%%%%%%
%% Defining colors %%
%%%%%%%%%%%%%%%%%%%%%

\definecolor{lr}{RGB}{188, 75, 81}
\definecolor{r}{RGB}{249, 65, 68}
\definecolor{dr}{RGB}{174, 32, 18}
\definecolor{lo}{RGB}{255, 172, 129}
\definecolor{do}{RGB}{202, 103, 2}
\definecolor{o}{RGB}{238, 155, 0}
\definecolor{ly}{RGB}{255, 241, 133}
\definecolor{y}{RGB}{255, 229, 31}
\definecolor{dy}{RGB}{143, 126, 0}
\definecolor{lb}{RGB}{148, 210, 189}
\definecolor{bg}{RGB}{10, 147, 150}
\definecolor{b}{RGB}{39, 125, 161}
\definecolor{db}{RGB}{0, 95, 115}
\definecolor{p}{RGB}{229, 152, 155}
\definecolor{dp}{RGB}{181, 101, 118}
\definecolor{pp}{RGB}{142, 143, 184}
\definecolor{v}{RGB}{109, 89, 122}
\definecolor{lg}{RGB}{144, 190, 109}
\definecolor{g}{RGB}{64, 145, 108}
\definecolor{dg}{RGB}{45, 106, 79}

\colorlet{mysol}{g}
\colorlet{mythm}{lr}
\colorlet{myqst}{db}
\colorlet{myclm}{lb}
\colorlet{mywrong}{r}
\colorlet{mylem}{o}
\colorlet{mydef}{lg}
\colorlet{mycor}{lb}
\colorlet{myrem}{dr}

%%%%%%%%%%%%%%%%%%%%%

\newcommand{\col}[2]{
  \color{#1}#2\color{black}\,
}

\newcommand{\TODO}[1][5cm]{
  \color{red}TODO\color{black}
  \vspace{#1}
}

\newcommand{\wans}[1]{
	\noindent\color{mywrong}\textbf{Wrong answer: }\color{black}
	#1 


}

\newcommand{\wreason}[1]{
	\noindent\color{mywrong}\textbf{Reason: }\color{black}
	#1 

  
}

\newcommand{\sol}[1]{
	\noindent\color{mysol}\textbf{Solution: }\color{black}
	#1


}

\newcommand{\nt}[1]{
  \begin{note}Note: #1\end{note}
}

\newcommand{\ky}[1]{
  \begin{key}#1\end{key}
}

\newcommand{\pf}[1]{
  \begin{myproof}#1\end{myproof}
}

\newcommand{\qs}[3][]{
  \begin{question}{#2}{#1}#3\end{question}
}

\newcommand{\df}[3][]{
  \begin{definition}{#2}{#1}#3\end{definition}
}

\newcommand{\thm}[3][]{
  \begin{theorem}{#2}{#1}#3\end{theorem}
}

\newcommand{\clm}[3][]{
  \begin{claim}{#2}{#1}#3\end{claim} 
}

\newcommand{\lem}[3][]{
  \begin{lemma}{#2}{#1}#3\end{lemma}
}

\newcommand{\cor}[3][]{
  \begin{corollary}{#2}{#1}#3\end{corollary}
}

\newcommand{\rem}[3][]{
  \begin{remark}{#2}{#1}#3\end{remark}
}

\newcommand{\twoways}[2]{
  \leavevmode\\
  ($\Longrightarrow$): 
  \begin{shift}#1\end{shift}
  ($\Longleftarrow$):
  \begin{shift}#2\end{shift} 
}

\newcommand{\nways}[2]{
  \leavevmode\\
  ($#1$): 
  \begin{shift}#2\end{shift}
}

%%%%%%%%%%%%%%%%%%%%%%%%%%%%%% ENVRN

\newenvironment{myproof}[1][\proofname]{%
	\proof[\bfseries #1: ]
}{\endproof}

\tcbuselibrary{theorems,skins,hooks}
\newtcolorbox{shift}
{%
  before upper={\setlength{\parskip}{5pt}},
  blanker,
	breakable,
	width=0.95\textwidth,
  enlarge left by=0.03\textwidth,
}

\tcbuselibrary{theorems,skins,hooks}
\newtcolorbox{key}
{%
	breakable,
	width=0.95\textwidth,
  enlarge left by=0.03\textwidth,
}

\tcbuselibrary{theorems,skins,hooks}
\newtcolorbox{note}
{%
	enhanced,
	breakable,
	colback = white,
	width=\textwidth,
	frame hidden,
	borderline west = {2pt}{0pt}{black},
	sharp corners,
}

\tcbuselibrary{theorems,skins,hooks}
\newtcbtheorem[]{remark}{Remark}
{%
	enhanced,
	breakable,
	colback = white,
	frame hidden,
	boxrule = 0sp,
	borderline west = {2pt}{0pt}{myrem},
	sharp corners,
	detach title,
  before upper={\setlength{\parskip}{5pt}\tcbtitle\par\smallskip},
	coltitle = myrem,
	fonttitle = \bfseries\sffamily,
	description font = \mdseries,
	separator sign none,
	segmentation style={solid, myrem},
}{rem}

\tcbuselibrary{theorems,skins,hooks}
\newtcbtheorem[number within=section]{lemma}{Lemma}
{%
	enhanced,
	breakable,
	colback = white,
	frame hidden,
	boxrule = 0sp,
	borderline west = {2pt}{0pt}{mylem},
	sharp corners,
	detach title,
  before upper={\setlength{\parskip}{5pt}\tcbtitle\par\smallskip},
	coltitle = mylem,
	fonttitle = \bfseries\sffamily,
	description font = \mdseries,
	separator sign none,
	segmentation style={solid, mylem},
}{lem}

\tcbuselibrary{theorems,skins,hooks}
\newtcbtheorem{claim}{Claim}
{%
  parbox=false,
	enhanced,
	breakable,
	colback = white,
	frame hidden,
	boxrule = 0sp,
	borderline west = {2pt}{0pt}{myclm},
	sharp corners,
	detach title,
  before upper={\setlength{\parskip}{5pt}\tcbtitle\par\smallskip},
	coltitle = myclm,
	fonttitle = \bfseries\sffamily,
	description font = \mdseries,
	separator sign none,
	segmentation style={solid, myclm},
}{clm}

\makeatletter
\newtcbtheorem[number within=section, use counter from=lemma]{theorem}{Theorem}{enhanced,
	breakable,
	colback=white,
	colframe=mythm,
	attach boxed title to top left={yshift*=-\tcboxedtitleheight},
	fonttitle=\bfseries,
	title={#2},
	boxed title size=title,
	boxed title style={%
			sharp corners,
			rounded corners=northwest,
			colback=mythm,
			boxrule=0pt,
		},
	underlay boxed title={%
			\path[fill=mythm] (title.south west)--(title.south east)
			to[out=0, in=180] ([xshift=5mm]title.east)--
			(title.center-|frame.east)
			[rounded corners=\kvtcb@arc] |-
			(frame.north) -| cycle;
		},
	#1
}{thm}
\makeatother

\makeatletter
\newtcbtheorem{question}{Question}{enhanced,
	breakable,
	colback=white,
	colframe=myqst,
	attach boxed title to top left={yshift*=-\tcboxedtitleheight},
	fonttitle=\bfseries,
	title={#2},
	boxed title size=title,
	boxed title style={%
			sharp corners,
			rounded corners=northwest,
			colback=myqst,
			boxrule=0pt,
		},
	underlay boxed title={%
			\path[fill=myqst] (title.south west)--(title.south east)
			to[out=0, in=180] ([xshift=5mm]title.east)--
			(title.center-|frame.east)
			[rounded corners=\kvtcb@arc] |-
			(frame.north) -| cycle;
		},
	#1
}{qs}
\makeatother

\makeatletter
\newtcbtheorem[number within=section]{definition}{Definition}{enhanced,
	breakable,
	colback=white,
	colframe=mydef,
	attach boxed title to top left={yshift*=-\tcboxedtitleheight},
	fonttitle=\bfseries,
	title={#2},
	boxed title size=title,
	boxed title style={%
			sharp corners,
			rounded corners=northwest,
			colback=mydef,
			boxrule=0pt,
		},
	underlay boxed title={%
			\path[fill=mydef] (title.south west)--(title.south east)
			to[out=0, in=180] ([xshift=5mm]title.east)--
			(title.center-|frame.east)
			[rounded corners=\kvtcb@arc] |-
			(frame.north) -| cycle;
		},
	#1
}{def}
\makeatother

\makeatletter
\newtcbtheorem[number within=section, use counter from=lemma]{corollary}{Corollary}{enhanced,
	breakable,
	colback=white,
	colframe=mycor,
	attach boxed title to top left={yshift*=-\tcboxedtitleheight},
	fonttitle=\bfseries,
	title={#2},
	boxed title size=title,
	boxed title style={%
			sharp corners,
			rounded corners=northwest,
			colback=mycor,
			boxrule=0pt,
		},
	underlay boxed title={%
			\path[fill=mycor] (title.south west)--(title.south east)
			to[out=0, in=180] ([xshift=5mm]title.east)--
			(title.center-|frame.east)
			[rounded corners=\kvtcb@arc] |-
			(frame.north) -| cycle;
		},
	#1
}{cor}
\makeatother

% Basic
  \DeclareMathOperator{\lcm}{lcm}
  \newcommand{\Real}{\mathbb{R}}
  \newcommand{\Comp}{\mathbb{C}}
  \newcommand{\Nat}{\mathbb{N}}
  \newcommand{\Rat}{\mathbb{Q}}
  \newcommand{\Int}{\mathbb{Z}}
  \newcommand{\set}[1]{\left\{\, #1 \,\right\}}
  \newcommand{\paren}[1]{\left( \; #1 \; \right)}
  \newcommand{\abs}[1]{\left\lvert #1 \right\rvert}
  \newcommand{\ang}[1]{\left\langle #1 \right\rangle}
  \renewcommand{\to}[1][]{\xrightarrow{\text{#1}}}
  \newcommand{\tol}[1][]{\to{$#1$}}
  \newcommand{\curle}{\preccurlyeq}
  \newcommand{\curge}{\succcurlyeq}
  \newcommand{\mapsfrom}{\leftarrow\!\shortmid}

  \newcommand{\mat}[1]{\begin{bmatrix} #1 \end{bmatrix}}
  \newcommand{\pmat}[1]{\begin{pmatrix} #1 \end{pmatrix}}
  \newcommand{\eqs}[1]{\begin{align*} #1 \end{align*}}
  \newcommand{\case}[1]{\begin{cases} #1 \end{cases}}
  

  % Algebra
  \newcommand{\normSg}[0]{\vartriangleleft}
  \newcommand{\ZMod}[1][n]{\mathbb{Z}/#1\mathbb{Z}}
  \newcommand{\isom}{\simeq}
  \newcommand{\mapHom}{\xrightarrow{\text{hom}}}
  \DeclareMathOperator{\Inn}{Inn}
  \DeclareMathOperator{\Aut}{Aut}
  \DeclareMathOperator{\im}{im}
  \DeclareMathOperator{\ord}{ord}
  \DeclareMathOperator{\Gal}{Gal}
  \DeclareMathOperator{\chr}{char}
  \newcommand{\surjto}{\twoheadrightarrow}
  \newcommand{\injto}{\hookrightarrow}

  % Analysis 
  \newcommand{\limty}[1][k]{\lim_{#1\to\infty}}
  \newcommand{\norm}[1]{\left\lVert#1\right\rVert}
  \newcommand{\darrow}{\rightrightarrows}


\DeclareMathOperator{\rad}{rad}

\fancyhead[L]{Homework 12 - Modern Algebra MAS311}
\fancyhead[R]{\textbf{Touch Sungkawichai} 20210821}

\begin{document}
  \qs{}{
    Show that every left ideal of the product $R \times S$ of two rings is a product $I \times J$ of 
    left ideals $I$ and $J$ of $R$ and $S$, respectively.
  }
  \sol{
    Let $H$ be a left ideal of the product $R \times S$. Then, $H = A \times B$ for some set $A$ and $B$. 
    Consider $(1_R, 0) \in R \times S$, so, $(1_R, 0)(a, b) \in H$ for $(a, b \in H)$. 
    But $(1_R, 0)(a, b) = (a, 0) \in A \times B$, implying that $a \in A$. And similarly, it is possible, by considering the product 
    with $(0, 1_S) \in R \times S$ to concludes that $(a, b) \in H$ implies $b \in B$.

    Now, considering that for any $r \in R$, $(r, s)(a, b) = (ra, sb) \in A \times B$. Therefore, $ra \in A$ and $sb \in B$. 
    Thus, $A$ and $B$ are left ideals. This proves that the ideal is a product $I \times J$ of left ideals $I$ of $R$ and $J$ of $S$
  }
  
  \qs{}{
    Find all prime and maximal ideals in $\ZMod[n]$
  }
  \sol{
    Notice that $\ZMod[n]$ is a commutative ring.
    Let $p_1^{a_1}, \ldots, p_k^{a_k}$ be the prime decomposistion of $n$.

    Firstly, notice that $p_i\ZMod[n] = \set{p_i k \mid \forall k \in \ZMod[n]}$ is an ideal.
    This is because for any $a \in \ZMod[n]$, it follows that $akp_i = (ak)p_i \in p_i\ZMod[n]$. 
    Moreover, they are prime ideal, as if $ab \in p_i\ZMod[n]$, then $p_i \vert ab$, which means either $p_i \vert a$ or $p_i \vert b$
    because $p_i$ is a prime number. This means that either $a \in p_i\ZMod[n]$ or $b \in p_i\ZMod[n]$

    Then for any other non-zero $I$ such that $I \ne \ZMod[n]$ and $(I, +) < (G, +)$.
    If there is $p_i \ZMod[n] \subset I$ and $p_j \ZMod[n] \subset I$ for $i \ne j$, then $p_i \in I$ and $p_j \in I$, 
    so $1 = ap_i+bp_j \in I$ since $\gcd(p_i, p_j) = 1$. Thus, $I = R$.
    In this case, for $ap_i \in p_i\ZMod[n]$ and $bp_j \in p_j$ gives $(ap_i)\cdot(bp_j) = abp_ip_j \in (p_ip_j)\ZMod[n]$. 
    Therefore, $I$ cannot be prime.

    And also, if for the ideal of the form, $I = p_i^r\ZMod[n]$, with $r > 1$, it would follow that $p_i^{r-1}$ or $p_i \in I$ 
    but neither $p_i$ nor $p_i^{r-1} \in I$. Therefore, $I$ is not prime. 

    So, the only possible prime ideal of $\ZMod[n]$ are $p_i\ZMod[n]$ for each prime divisor of $n$.

    Note that no other $I$ is possible as $I$ must be an additive subgroup of $R$.
    
    Now, consider any bigger ideal $I$ that contain $p_i\ZMod[n]$, if it contains any other number not divisible by $p_i$,
    then that number must be divisible by $p_j$ with some $j \ne i$,
    then $I = \ZMod[n]$ was shown. Therefore, $p_i\ZMod[n]$ are all maximal.

    Lastly, since a maximal ideal must be prime, then $p_i\ZMod[n]$ are all prime ideal implies that they are all of the maximal ideals
    of $\ZMod[n]$.
  }

  \qs{}{
    Let $I$ be a proper ideal of a commutative ring $R$. Show that there is a maximal ideal of $R$ containing $I$.
  }
  \sol{
    Let $S$ be a set of all proper ideal of $R$ that contains $I$, and define $\curle$ operator as an ordering of $S$ by set inclusion.
    Then, $(S, \curle)$ is a poset.

    If $I$ is maximal, then the statment holds trivially, so assume that $I$ is non-maximal.
    Given an ideal $I$, $I \in S$, there exists a chain $I \curle I_1 \curle \cdots \curle I_k \curle \cdots$.
    This is true because if there is no ideal (not neccessary maximal) that contains $I$, then $I$ must be maximal by definition.

    Now, consider that the union $\bigcup I_i$ is an ideal such that it contains $I, I_1, \cdots, I_k \cdots$. This is due to the 
    fact that if $t \in \bigcup I_i$, it follows that $t \in I_i$ for some $i$, then for all $r \in R$, $rt, tr \in I_i$, thus 
    $t \in \bigcup I_i$. And for $s, t \in \bigcup I_i$, it holds that $s \in I_i$ and $t \in I_j$ for some $i$ and $j$.
    Let, without loss of generality, $i \ge j$, then $s, t \in I_i$ as $I_j \curle I_i$, so $s\pm t \in I_i \subset \bigcup I_i$

    Therefore, by zorn's lemma, there exists a maximal element $M$ of $S$.
    That maximal element $M$ is a proper ideal that contains $I$, and is not contained in any other proper ideal of $R$ 
    that contains $I$. However, any ideal that might contains $M$ will always contains $I$, thus, there is no such proper ideal 
    containing $M$. Therefore, $M$ is the maximal ideal containing $I$ by definition.
  }

  \qs{}{
    Let $I$ and $J$ be ideals of a commutative ring $R$ such that $I + J = R$. Show that $I^n + J^m = R$ for any positive integers $n,m$.
  }
  \sol{
    Consider that $I + J$ is the ideal $\set{i + j \mid i \in I, j \in J}$. Since $I + J = R$, it follows that $1 = i + j$ for some 
    $i \in I, j \in J$. 

    \[ \begin{aligned} 
      1 = 1^{n+m} = (i+j)^{n+m} &= (i^{n+m} + (n+m)i^{n+(m-1)}j + \cdots + j^{n+m}) \\ 
                                &= i^n(i^m + (n+m)i^{m-1}j + \cdots + \frac{(n+m)!}{n!m!}j^{m})  + j^m(j^n + \cdots + \frac{(n+m)!}{n!m!}i^n)
    \end{aligned}\]
    Moreover, $I^n$ and $J^m$ are ideals as it was proven that for any ideal $S,T$, $ST$ is an ideal. 
    Then an induction can be made to argue that $I^n$ is ideal as $I^n = I^{n-1}I$.

    But $i^n \in I^n$ and $j^m \in J^m$. 
    Therefore, $1 \in I^n + J^m$, and since $I^n + J^m$ is also an ideal, it follows that $I^n + J^m = R$
  }

  \qs{}{
    Let $S$ and $T$ be multiplicative subsets of a commutative ring $R$. Let $\phi: R \to S^{-1}R$ be the ring homomorphism given 
    by $r \mapsto r/1$. Show that two localizations $(ST)^{-1}R$ and $\phi(T)^{-1}(S^{-1}R)$ are isomorphic.
  }
  \sol{
    If $0 \in T$ or $0 \in S$, then $(ST)^{-1}R$ is a zero ring, and $\phi(T)^{-1}(S^{-1}R)$ is also a zero ring, 
    thus they are isomorphic. Now, assume $0 \not\in T$ and $0 \not \in S$

    Firstly, notice that as $S, T$ are multiplicative subset, then $ST = \set{st \mid s \in S, t \in T}$ is a multiplicative subset, as 
    $(st)(s't') = ss'tt' \in ST$ for $st, s't' \in ST$, and $1 \in S$, $1 \in T$, so $1 \in ST$.

    Notice that $S^{-1}R$ is a domain.

    Consider a homomorphism $\psi: (ST)^{-1}R \to \phi(T)^{-1}(S^{-1}R)$ defined by $r/st \mapsto \frac{r/s}{t/1})$.
    Then, $\psi$ is well-define since for $r'/s't' = r/st$, it follows that $u(r'st - rs't') = 0$ for some $u \in S$.
    But then, $\frac{r/s}{t/1} = \frac{r'/s'}{t'/1}$ because $(r/s)(t'/1) = (r/s')(t/1)$ as $rt'/s = rt/s'$ due to the fact that 
    $u(rt's' - r'ts) = 0$ from above statment.

    Next, $\psi$ is an homomorphism because 
    \[\begin{aligned} 
      \psi(r/st + r'/s't') = \psi(\frac{rs't' + r'st}{sts't'}) &= \frac{(rs't' + r'st)/ss'}{tt'/1} \\ 
                                                               &= \frac{rt'/s + r't/s'}{tt'/1} \\
                                                               &= \frac{r/s}{t/1} + \frac{r'/s'}{t'/1} \\
                                                               &= \psi(r/st) + \psi(r'/s't')
    \end{aligned}\]
    shows the that addition is preserved and
    \[\begin{aligned}
      \psi((r/st)(r'/s't')) = \psi(\frac{rr'}{ss'tt'}) &= \frac{rr'/ss'}{tt'/1} \\ 
                                                       &= \frac{(r/s)(r'/s')}{(t/1)(t'/1)} \\ 
                                                       &= \frac{r/s}{t/1} \frac{r'/s'}{t'/1} \\
                                                       &= \psi(r/st)\psi(r'/s't')
    \end{aligned}\]
    shows that the multiplication is preserved.

    Then, consider $\ker(\psi) = \set{r/st \mid \frac{r/s}{t/1} = 0} = \set{0}$.
    This is due to the equivalence: 
    \[\begin{aligned}
      \frac{r/s}{t/1} = 0 &\iff \frac{r/s}{t/1} = \frac{0}{1} \\
                          &\iff \exists u \mid u(\frac{r}{s}) = 0 \\ 
                          &\iff 0 \in \phi(T) \lor \frac{r}{s} = 0  \quad \text{ as $S^{-1}T$ is a domain}\\
                          &\iff \exists u' \in S \mid u'r = 0 \text{ since $0 \not\in \phi(T)$} \\
                          &\iff \exists u' \mid u'r = 0 \iff \frac{r}{st} = 0
    \end{aligned}\]
    And consider that $\psi$ is surjective as for any element $\frac{r/s}{t/1} = \psi(r/st)$ and $r/st \in (ST)^{-1}R$, there is a 
    corresponding element in the domain that maps to that desired element.
    Thus, by the first ring isomorphism theorem, \[ (ST)^{-1}R  \isom \phi(T)^{-1}(S^{-1}R)  \]
  }

  \qs{}{
    Let $I$ be an ideal of a commutative ring $R$. Prove that 
    \[ \rad I := \set{r \in R \mid r^n \in I \text{ for some } n \in \Int^+ } \]
    is an ideal containing $I$. Prove also that $(\rad I)/I$ is an ideal of nilpotent elements of the factor ring $R/I$.
  }
  \sol{
    If $i \in \rad I$, and let $i^n \in I$, it follows that $(ri)^n = r^ni^n \in r^nI = I$, therefore, $ri \in \rad I$.
    Moreover, for $i, j \in \rad I$, with $i^n \in I$ and $j^m \in J$, it follows that 
    \[ (i-j)^{n+m} = i^{n+m} + (n+m)i^{n+m-1}j + \cdots + j^{n+m} 
    = i^{n}(i^m + \cdots + \frac{(m+n)!}{n!m!}j^m) + (\pm\frac{(m+n)!}{n!m!}i^n \cdots \pm j^n)j^m \in I\]
    So, $i-j \in \rad I$.
    Hence, $\rad I$ is an ideal.

    Now, consider if $i \in I$, then $i^n = i(i^{n-1}) \in I$, so $i \in \rad I$.
    Thus $\rad I$ is and ideal that contains $I$.

    Since $I$ is an ideal of $R$ then it must be an ideal of $\rad I$.
    Then $\rad I/I \subset R/I$ as subgroup because $\rad I \subset R$. 
    Consider $d + I \in \rad I/I$, for some $d \in \rad I$ and $r + I \in R/I$ for some $r\in R$.
    Now, notice that $(d+I)(r+I) = (rd + I)$ as $I$ is an ideal. And since $\rad I$ is an ideal, it follows that 
    $rd \in \rad I$, which is that $(rd + I) \in \rad/I$. Therefore, $\rad I/I$ is an ideal of $R/I$. 
  }

  \qs{}{
    Let $M$ be a maximal ideal of a commutative ring $R$. Prove that the quotient ring $R/M^n$ is local for any $n \ge 1$.
  }
  \sol{
    Assume that there is a maximal ideal $N$ of $R$ such that $N \ne M$ and $N$ contains $M^n$. Then, as $N$ is a prime 
    ideal and $m^n \in N$ for all $m \in M$, it must follow that $m^{n-1}m \in N$, which is $m^{n-1} \in N$. Since $n$ is 
    finite and $n-1 < n$, it is possible to conclude that $m \in N$. Therefore, $M \subset N$. But this yield a 
    contradiction, therefore, there must be no such $N \ne M$ that contains $M^n$.

    Consider a map $\phi: R \to R/M^n$ given by $\phi: x \mapsto x + M^n$ which is a homomorphism.
    The kernel of the map is $\ker \phi = M^n$.
    Then if $I$ is an ideal of $R$, then for $i\in I$ and $r \in R$, it follows that $\phi(ir) = \phi(i)\phi(r)$. 
    So, $\phi(I)$ is an ideal.
    And conversely, if $I$ is an ideal of $R/M^n$, then the preimage $\phi^{-1}(I) = \set{i + m \mid i \in I, m \in M^n}$
    is an ideal as $r(i+m) = ri + rm = \phi^{-1}(ri) \in \phi^{-1}(I)$.

    This shows that there is a bijection between the set of ideal of $R/M^n$ and ideal of $R$ containing $M^n$.

    Moreover, consider if $\phi(I)$ is a maximal ideal of $R/M^n$ and there is a proper ideal $J > I$ of $R$. 
    Then let $j \in J-I$. 
    So, $\phi(j)$ must be in ideal $\phi(J)$ of $R/M^n$ but not in $\phi(I)$. Since $\phi(I)$ is maximal, then $\phi(J)$
    must be the whole ring $R/M^n$. However, $J$ is a proper ideal of $R$, so there is $r \in R-J$. 
    And $\phi(r) \not\in \phi(J)$, which contradicts that $\phi(J)$ is the whole ring. Therefore, $I$ must be maximal.
    
    The contraposition yields that if $I$ is not maximal, then $\phi(I)$ cannot be maximal.

    However, there is a unique maximal ideal of $R$ containing $M^n$, therefore, there cannot be two maximal ideals 
    in the ring $R/M^n$. As there is a unique maximal element of $R/M^n$, it is local.
  }

  \qs{}{
    Let $F$ be a field. Define the ring $F((x))$ of formal Laurant series by 
    \[ F((x)) = \set{\sum_{n \ge N}^\infty a_nx^n \mid a_n \in F \text{ and } N-1 \in \Int} \]
    Prove that the field of fractions of $F[[x]]$ is $F((x))$. 
    Prove also that the field of fractions of the power series ring $\Int[[x]]$ is properly contained in the field of 
    formal Laurant series $\Rat((x))$.
  }
  \sol{
    Firstly, notice that a unit in $F[[x]]$ is any element such that the constant term is non-zero.
    As if that is the case, then \[ (a_0 + a_1x + \cdots) (a_0^{-1} + a_0^{-2}a_1x + \cdots) = 1 \]
    But if the constant term is zero, then the element can be written as $x^k(a_k + a_{k+1}x + \cdots)$
    Which means that the product \[x^k(a_k + a_{k+1}x + \cdots) b(x) = x^k c(x) \ne 1 \] for some $b, c \in F[[x]]$

    As there is a natural embedding (injective) of $F[[x]] \to F((x))$ defined by $f \mapsto f$.
    Consider an element $p'(x) = \sum_{i=0}^\infty p_ix^i$ and $q'(x) = \sum_{i=0}^\infty q_ix^i \ne 0$ are elements of $F[[x]]$, 
    and their corresponding elements $p(x), q(x) \in F((x))$.
    Now, rewrite $q(x)$ in the form of $x^k(a_k + a_{k+1}x + \cdots)$ where $a_k$ is non-zero, as $q(x)$ is non-zero.
    Then, as $(a_k + a_{k+1}x + \cdots)$ is invertible, and $(x^k)^{-1} = x^{-k}$, it follows that 
    \[ \frac{p(x)}{q(x)} = p(x) \cdot x^{-k}(a_k + a_{k+1}x + \cdots)^{-1} = x^{-k}f(x) \]
    for some $f(x)$ being an embedding of $f'(x) \in F[[x]]$.
    By rewriting, \[ \frac{p(x)}{q(x)} = \sum_{i=-k}^{\infty} f_{i+k}x^i \]
    Thus, a fraction of any element in $F[[x]]$ can be written as an element in the ring $F((x))$. 

    As $F((x))$ is a ring of Laurant series, it has closure of addition and multiplication, thus, the fraction of $F[[x]]$ is 
    the ring $F((x))$.

    Next, since there is a natural homomorphism $\Int[[x]] \to \Rat[[x]]$ given by $x \mapsto \frac{x}{1}$ that is injective.
    Then the fraction of $\Int[[x]]$ must be contained in the fraction of $\Rat[[x]]$, which is $\Rat((x))$.
  }

  \qs{}{
    Find all idempotents in $\ZMod[p^n]$, where $p$ is a prime integer and $n \ge 1$. Find also the number of idempotents of $\ZMod[n]$.
  }
  \sol{
    Consider an element $a \in \ZMod[p^n]$ such that $a^2 = a$. Clearly, $0^2 = 0$ and $1^2 = 1$.
    Apart from that, $a^2 \equiv a \pmod{p^n}$, which implies $a(a-1) \equiv 0 \pmod{p^n}$. But as $a$ and $(a-1)$ are coprime, then 
    $p^n \mid a$ or $p^n \mid a-1$, which yields two solutions. If $p^n \mid a$, then $a = 0$, otherwise, a = 1 (in $\ZMod[p^n]$).
    Thus, only $0$ and $1$ are the idempotents of $\ZMod[p^n]$

    For $\ZMod[n]$. Write $n = p_1^{k_1}p_2^{k_2}\cdots p_m^{k_m}$ for distinct prime $p_i$.
    Then, consider the system 
    \[\begin{gathered}
      a(a-1) \equiv 0 \pmod{p_1^{k_1}} \\
      a(a-1) \equiv 0 \pmod{p_2^{k_2}} \\
      \cdots \\
      a(a-1) \equiv 0 \pmod{p_m^{k_m}}
    \end{gathered}\]
    Where each congruence equation yields two solutions.
    Thus, as $p_1^{k_1}, p_2^{k_2}, \ldots, p_m^{k_m}$ are pairwise relatively prime, then by the chineses remainder theorem, 
    one can construct $a(a-1) \equiv 0 \pmod{n}$ in $2^m$ ways.

    Therefore, there are $2^m$ idempotents in $\ZMod[n]$
  }

  \qs{}{
    Let $f_1(x), f_2(x), \ldots f_k(x)$ be polynomials with integer coefficients of the same degree $d$.
    Let $n_1, n_2, \ldots, n_k$ be integers which are relatively prime in pairs (ie. $\gcd(n_i, n_j) = 1$ for all $i \ne j$).
    Use the Chinese Remainder Theorem to prove that there exists a polynomial $f(x)$ with integer coefficients of degree $d$ with 
    \[ f(x) \equiv f_1(x) \pmod{n_1}, \quad f(x) \equiv f_2(x) \pmod{n_2}, \quad \ldots, \quad f(x) \equiv f_k(x) \pmod{n_k} \]
    ie. the coefficients of $f(x)$ agree with the coefficients of $f_i(x) \pmod{n_i}$. Show that if all the $f_i(x)$ are monic, then 
    $f(x)$ may also be chosen monic.
  }
  \sol{
    For polynomials with degree $d=0$, the statement is holds trivially as it resembles the chinese remainder theorem. 
    Now, assume for induction that the statement holds for any polynomials of degree $d$.

    Let $g_1(x), g_2(x), \ldots, g_k(x)$ be polynomials of degree $d+1$, then $g_i(x) = a_ix^{d+1} + f_i(x)$ for some $f_i(x)$ being a 
    degree $d$ polynomial.
    Since $f_i(x)$ are degree $d$, then there is a degree $d$ polynomial $f(x)$ satisfying that 
    \[ f(x) \equiv f_1(x) \pmod{n_1}, \quad f(x) \equiv f_2(x) \pmod{n_2}, \quad \ldots, \quad f(x) \equiv f_k(x) \pmod{n_k} \]
    by the induction hypothesis.

    Now, as $n_1, \ldots, n_k$ are coprime, then there exists $a$ such that 
    \[ a \equiv a_1 \pmod{n_1}, \quad a \equiv a_2 \pmod{n_2}, \quad \ldots, \quad a \equiv a_k \pmod{n_k} \]
    by the chinese remainder theorem.

    By multiplying $x^{d+1}$ gives 
    \[ ax^{d+1} \equiv a_1x^{d+1} \pmod{n_1}, \quad ax^{d+1} \equiv a_2x^{d+1} \pmod{n_2}, \quad \ldots, \quad ax^{d+1} \equiv a_kx^{d+1} \pmod{n_k} \]
    Which, upon setting $g(x) = ax^{d+1} + f(x)$ gives 
    \[ g(x) \equiv g_1(x) \pmod{n_1}, \quad g(x) \equiv g_2(x) \pmod{n_2}, \quad \ldots, \quad g(x) \equiv g_k(x) \pmod{n_k} \]
    Therefore, the statement holds generally by induction.

    Now, if $g_1(x), \ldots g_k(x)$ are all monic of degree $d$, then consider that there is $f(x)$ of degree $d-1$ that satisfies 
    the condition for all $f_i(x) = g_i(x)-x^d$. Then $g(x) = x^d + f(x)$ satisfies the condition, and is monic.
    Therefore, if all the functions $g_i(x)$ are monic, the $g(x)$ can be chosen to be monic.
  }


  \end{document}
