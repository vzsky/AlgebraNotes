% chktex-file 44
% chktex-file 8

\documentclass{report}
\usepackage{amsthm}
\usepackage{amsmath}
\usepackage{amssymb}
\usepackage{amssymb}
\usepackage{amsfonts}
\usepackage{xcolor}
\usepackage{tikz}
\usepackage{fancyhdr}
\usepackage{enumerate}
\usepackage{graphicx}
\usepackage[normalem]{ulem}
\usepackage[most,many,breakable]{tcolorbox}
\usepackage[a4paper, top=80pt, foot=25pt, bottom=50pt, left=0.5in, right=0.5in]{geometry}
\usepackage{hyperref, theoremref}
\hypersetup{
	pdftitle={Assignment},
	colorlinks=true, linkcolor=b!90,
	bookmarksnumbered=true,
	bookmarksopen=true
}
\usepackage{nameref}
\usepackage{parskip}
\pagestyle{fancy}

\usepackage[explicit,compact]{titlesec}
\titleformat{\chapter}[block]{\bfseries\huge}{\thechapter. }{\compact}{#1}
        

%%%%%%%%%%%%%%%%%%%%%
%% Defining colors %%
%%%%%%%%%%%%%%%%%%%%%

\definecolor{lr}{RGB}{188, 75, 81}
\definecolor{r}{RGB}{249, 65, 68}
\definecolor{dr}{RGB}{174, 32, 18}
\definecolor{lo}{RGB}{255, 172, 129}
\definecolor{do}{RGB}{202, 103, 2}
\definecolor{o}{RGB}{238, 155, 0}
\definecolor{ly}{RGB}{255, 241, 133}
\definecolor{y}{RGB}{255, 229, 31}
\definecolor{dy}{RGB}{143, 126, 0}
\definecolor{lb}{RGB}{148, 210, 189}
\definecolor{bg}{RGB}{10, 147, 150}
\definecolor{b}{RGB}{39, 125, 161}
\definecolor{db}{RGB}{0, 95, 115}
\definecolor{p}{RGB}{229, 152, 155}
\definecolor{dp}{RGB}{181, 101, 118}
\definecolor{pp}{RGB}{142, 143, 184}
\definecolor{v}{RGB}{109, 89, 122}
\definecolor{lg}{RGB}{144, 190, 109}
\definecolor{g}{RGB}{64, 145, 108}
\definecolor{dg}{RGB}{45, 106, 79}

\colorlet{mysol}{g}
\colorlet{mythm}{lr}
\colorlet{myqst}{db}
\colorlet{myclm}{lb}
\colorlet{mywrong}{r}
\colorlet{mylem}{o}
\colorlet{mydef}{lg}
\colorlet{mycor}{lb}
\colorlet{myrem}{dr}

%%%%%%%%%%%%%%%%%%%%%

\newcommand{\col}[2]{
  \color{#1}#2\color{black}\,
}

\newcommand{\TODO}[1][5cm]{
  \color{red}TODO\color{black}
  \vspace{#1}
}

\newcommand{\wans}[1]{
	\noindent\color{mywrong}\textbf{Wrong answer: }\color{black}
	#1 


}

\newcommand{\wreason}[1]{
	\noindent\color{mywrong}\textbf{Reason: }\color{black}
	#1 

  
}

\newcommand{\sol}[1]{
	\noindent\color{mysol}\textbf{Solution: }\color{black}
	#1


}

\newcommand{\nt}[1]{
  \begin{note}Note: #1\end{note}
}

\newcommand{\ky}[1]{
  \begin{key}#1\end{key}
}

\newcommand{\pf}[1]{
  \begin{myproof}#1\end{myproof}
}

\newcommand{\qs}[3][]{
  \begin{question}{#2}{#1}#3\end{question}
}

\newcommand{\df}[3][]{
  \begin{definition}{#2}{#1}#3\end{definition}
}

\newcommand{\thm}[3][]{
  \begin{theorem}{#2}{#1}#3\end{theorem}
}

\newcommand{\clm}[3][]{
  \begin{claim}{#2}{#1}#3\end{claim} 
}

\newcommand{\lem}[3][]{
  \begin{lemma}{#2}{#1}#3\end{lemma}
}

\newcommand{\cor}[3][]{
  \begin{corollary}{#2}{#1}#3\end{corollary}
}

\newcommand{\rem}[3][]{
  \begin{remark}{#2}{#1}#3\end{remark}
}

\newcommand{\twoways}[2]{
  \leavevmode\\
  ($\Longrightarrow$): 
  \begin{shift}#1\end{shift}
  ($\Longleftarrow$):
  \begin{shift}#2\end{shift} 
}

\newcommand{\nways}[2]{
  \leavevmode\\
  ($#1$): 
  \begin{shift}#2\end{shift}
}

%%%%%%%%%%%%%%%%%%%%%%%%%%%%%% ENVRN

\newenvironment{myproof}[1][\proofname]{%
	\proof[\bfseries #1: ]
}{\endproof}

\tcbuselibrary{theorems,skins,hooks}
\newtcolorbox{shift}
{%
  before upper={\setlength{\parskip}{5pt}},
  blanker,
	breakable,
	width=0.95\textwidth,
  enlarge left by=0.03\textwidth,
}

\tcbuselibrary{theorems,skins,hooks}
\newtcolorbox{key}
{%
	breakable,
	width=0.95\textwidth,
  enlarge left by=0.03\textwidth,
}

\tcbuselibrary{theorems,skins,hooks}
\newtcolorbox{note}
{%
	enhanced,
	breakable,
	colback = white,
	width=\textwidth,
	frame hidden,
	borderline west = {2pt}{0pt}{black},
	sharp corners,
}

\tcbuselibrary{theorems,skins,hooks}
\newtcbtheorem[]{remark}{Remark}
{%
	enhanced,
	breakable,
	colback = white,
	frame hidden,
	boxrule = 0sp,
	borderline west = {2pt}{0pt}{myrem},
	sharp corners,
	detach title,
  before upper={\setlength{\parskip}{5pt}\tcbtitle\par\smallskip},
	coltitle = myrem,
	fonttitle = \bfseries\sffamily,
	description font = \mdseries,
	separator sign none,
	segmentation style={solid, myrem},
}{rem}

\tcbuselibrary{theorems,skins,hooks}
\newtcbtheorem[number within=section]{lemma}{Lemma}
{%
	enhanced,
	breakable,
	colback = white,
	frame hidden,
	boxrule = 0sp,
	borderline west = {2pt}{0pt}{mylem},
	sharp corners,
	detach title,
  before upper={\setlength{\parskip}{5pt}\tcbtitle\par\smallskip},
	coltitle = mylem,
	fonttitle = \bfseries\sffamily,
	description font = \mdseries,
	separator sign none,
	segmentation style={solid, mylem},
}{lem}

\tcbuselibrary{theorems,skins,hooks}
\newtcbtheorem{claim}{Claim}
{%
  parbox=false,
	enhanced,
	breakable,
	colback = white,
	frame hidden,
	boxrule = 0sp,
	borderline west = {2pt}{0pt}{myclm},
	sharp corners,
	detach title,
  before upper={\setlength{\parskip}{5pt}\tcbtitle\par\smallskip},
	coltitle = myclm,
	fonttitle = \bfseries\sffamily,
	description font = \mdseries,
	separator sign none,
	segmentation style={solid, myclm},
}{clm}

\makeatletter
\newtcbtheorem[number within=section, use counter from=lemma]{theorem}{Theorem}{enhanced,
	breakable,
	colback=white,
	colframe=mythm,
	attach boxed title to top left={yshift*=-\tcboxedtitleheight},
	fonttitle=\bfseries,
	title={#2},
	boxed title size=title,
	boxed title style={%
			sharp corners,
			rounded corners=northwest,
			colback=mythm,
			boxrule=0pt,
		},
	underlay boxed title={%
			\path[fill=mythm] (title.south west)--(title.south east)
			to[out=0, in=180] ([xshift=5mm]title.east)--
			(title.center-|frame.east)
			[rounded corners=\kvtcb@arc] |-
			(frame.north) -| cycle;
		},
	#1
}{thm}
\makeatother

\makeatletter
\newtcbtheorem{question}{Question}{enhanced,
	breakable,
	colback=white,
	colframe=myqst,
	attach boxed title to top left={yshift*=-\tcboxedtitleheight},
	fonttitle=\bfseries,
	title={#2},
	boxed title size=title,
	boxed title style={%
			sharp corners,
			rounded corners=northwest,
			colback=myqst,
			boxrule=0pt,
		},
	underlay boxed title={%
			\path[fill=myqst] (title.south west)--(title.south east)
			to[out=0, in=180] ([xshift=5mm]title.east)--
			(title.center-|frame.east)
			[rounded corners=\kvtcb@arc] |-
			(frame.north) -| cycle;
		},
	#1
}{qs}
\makeatother

\makeatletter
\newtcbtheorem[number within=section]{definition}{Definition}{enhanced,
	breakable,
	colback=white,
	colframe=mydef,
	attach boxed title to top left={yshift*=-\tcboxedtitleheight},
	fonttitle=\bfseries,
	title={#2},
	boxed title size=title,
	boxed title style={%
			sharp corners,
			rounded corners=northwest,
			colback=mydef,
			boxrule=0pt,
		},
	underlay boxed title={%
			\path[fill=mydef] (title.south west)--(title.south east)
			to[out=0, in=180] ([xshift=5mm]title.east)--
			(title.center-|frame.east)
			[rounded corners=\kvtcb@arc] |-
			(frame.north) -| cycle;
		},
	#1
}{def}
\makeatother

\makeatletter
\newtcbtheorem[number within=section, use counter from=lemma]{corollary}{Corollary}{enhanced,
	breakable,
	colback=white,
	colframe=mycor,
	attach boxed title to top left={yshift*=-\tcboxedtitleheight},
	fonttitle=\bfseries,
	title={#2},
	boxed title size=title,
	boxed title style={%
			sharp corners,
			rounded corners=northwest,
			colback=mycor,
			boxrule=0pt,
		},
	underlay boxed title={%
			\path[fill=mycor] (title.south west)--(title.south east)
			to[out=0, in=180] ([xshift=5mm]title.east)--
			(title.center-|frame.east)
			[rounded corners=\kvtcb@arc] |-
			(frame.north) -| cycle;
		},
	#1
}{cor}
\makeatother

% Basic
  \DeclareMathOperator{\lcm}{lcm}
  \newcommand{\Real}{\mathbb{R}}
  \newcommand{\Comp}{\mathbb{C}}
  \newcommand{\Nat}{\mathbb{N}}
  \newcommand{\Rat}{\mathbb{Q}}
  \newcommand{\Int}{\mathbb{Z}}
  \newcommand{\set}[1]{\left\{\, #1 \,\right\}}
  \newcommand{\paren}[1]{\left( \; #1 \; \right)}
  \newcommand{\abs}[1]{\left\lvert #1 \right\rvert}
  \newcommand{\ang}[1]{\left\langle #1 \right\rangle}
  \renewcommand{\to}[1][]{\xrightarrow{\text{#1}}}
  \newcommand{\tol}[1][]{\to{$#1$}}
  \newcommand{\curle}{\preccurlyeq}
  \newcommand{\curge}{\succcurlyeq}
  \newcommand{\mapsfrom}{\leftarrow\!\shortmid}

  \newcommand{\mat}[1]{\begin{bmatrix} #1 \end{bmatrix}}
  \newcommand{\pmat}[1]{\begin{pmatrix} #1 \end{pmatrix}}
  \newcommand{\eqs}[1]{\begin{align*} #1 \end{align*}}
  \newcommand{\case}[1]{\begin{cases} #1 \end{cases}}
  

  % Algebra
  \newcommand{\normSg}[0]{\vartriangleleft}
  \newcommand{\ZMod}[1][n]{\mathbb{Z}/#1\mathbb{Z}}
  \newcommand{\isom}{\simeq}
  \newcommand{\mapHom}{\xrightarrow{\text{hom}}}
  \DeclareMathOperator{\Inn}{Inn}
  \DeclareMathOperator{\Aut}{Aut}
  \DeclareMathOperator{\im}{im}
  \DeclareMathOperator{\ord}{ord}
  \DeclareMathOperator{\Gal}{Gal}
  \DeclareMathOperator{\chr}{char}
  \newcommand{\surjto}{\twoheadrightarrow}
  \newcommand{\injto}{\hookrightarrow}

  % Analysis 
  \newcommand{\limty}[1][k]{\lim_{#1\to\infty}}
  \newcommand{\norm}[1]{\left\lVert#1\right\rVert}
  \newcommand{\darrow}{\rightrightarrows}


\fancyhead[L]{Modern Algebra 1 - MAS311}
\fancyhead[R]{\textbf{Touch Sungkawichai} 20210821}

\begin{document}
  \qs{}{
    Show that $S_3$ is isomorphic to $\Aut(\ZMod[2] \times \ZMod[2])$. 
  }
  \sol{
    Firstly, an automorphism of $K_4 \isom \ZMod[2] \times \ZMod[2] = \set{0, 1, 2, 3}$ 
    must map $e \to e$. And as the rest of the elements are all of order 2, 
    then it must map a non-trivial element $1$ with either $1 \to 1, 2, 3$. Next, to maintain that the map is an isomorphism,
    it must map $2$ to either $1, 2, 3$ with the condition that it must map $1$ and $2$ to different element. Therefore it must map $3$ 
    to the remaining element. This shows that $\abs{\Aut(K_4)} \le 6$

    Next, consider a group $S_3$ that permutes element $\set{1, 2, 3}$ and the action of $S_3$ on $K_4$ such that 
    $\sigma \cdot 0 = 0$ and $\sigma \cdot g = \sigma(g)$. Then the action is well defined as 
    $\sigma \cdot \sigma' \cdot g = \sigma(\sigma'(g)) = \sigma \circ \sigma'(g)$ and $id \cdot g = g$.
    Now, the group action corresponds bijectively to a homomorphism from $S_3$ to $\Aut(K_4)$. 
    The kernel of the homomorphism is the set $\set{\sigma \mid \sigma g = g \forall g} = \set{id}$.
    Lastly, as $\abs{\Aut(K_4)} \le 6$, it follows that the image of the homomorphism must be $\Aut(K_4)$ so lagrange's theorem holds. 
    Therefore, by the first isomorphism theorem, $S_3 \isom \Aut(K_4)$. 
  }

  \qs{}{
    A subgroup $C$ of $G$ is called characteristic if $f(C) = C$ for any automorphism $f$ of $G$. 
    Show that a characteristic subgroup $C$ is normal in $G$. 
  }
  \sol{
    As $\Inn(G) < \Aut(G)$, it holds that $f(C) = C$ for all $f \in \Inn(G)$. As $\Inn(G)$ is the group of all automorphism given by 
    conjugation, and $f(C) = C$ for any $f \in \Inn(G)$, then $C$ is normal by definition.
  } 

  \qs{}{
    Let $f, g: K \to \Aut(H)$ be two homomorphism. Assume that there exists an automorphism $\phi: K \to K$ such that 
    $f = g \circ \phi$. Prove that the map $H \rtimes_g K \to H \rtimes_f K$ given by $(h,k) \mapsto (h, \phi^{-1}(k))$ is an 
    isomorphism.
  }
  \sol{
    Consider a map $\psi: H \rtimes_g K \to H \rtimes_f K$ given by $\psi: (h, k) \mapsto (h, \phi^{-1}(k))$.
    Notice that $f(k) = g(\phi(k))$, and since $f$ and $g$ are automorphisms, $f(\phi^{-1}(k)) = g(k)$ for any $k$.
    Then, \eqs{
      \psi((h, k)(h', k')) &= \psi((hg(k)(h'), kk')) \\ 
                           &= (hg(k)(h'), \phi^{-1}(kk')) \\ 
                           &= (h f(\phi^{-1}(k))(h'), \phi^{-1}(k) \phi^{-1}(k')) \\ 
                           &= (h, \phi^{-1}(k)) \cdot (h', \phi^{-1}(k')) \\ 
                           &= \psi(h, k) \cdot \psi(h', k')
    }
    shows that $\psi$ is a homomorphism. 

    Furthermore, If $\psi((h, k)) = \psi((h', k'))$, then $(h, \phi^{-1}(k)) = (h', \phi^{-1}(k'))$. Thus, $h = h'$ and $k = k'$ as 
    $\phi$ is an automorphism, which means that it is an isomorphism.

    Lastly, $\im(\psi) = \set{(h, k) \mid h \in H, k \in \im(\phi)}$. However, as $\phi$ is an automorphism, then $\im(\phi) = K$. 
    Thus, $\im(\psi) = \set{(h, k) \mid h \in H, k \in K} = H \rtimes_g K$. 

    As the map $\psi$ is a homomorphism that is surjective and injective, then it is an isomorphism.
  }

  \qs{}{
    Classify all groups of order 325 upto isomorphism.
  }
  \sol{
    Let $G$ be a group of order $325 = 5^2 \cdot 13$. Assume that $G$ is non-abelian. 
    Then, consider the sylow 5-subgroup of $G$ of order 25.
    The number of such subgroup satisfies $n_5 \equiv 1 \pmod5$ and $n_5 \vert 13$, so $n_5 = 1$. 
    Since there is a unique sylow 5-subgroup of $G$, then the group is normal.
    In the same way, the number of sylow 13-subgroup satisfies $n_{13} \equiv 1 \pmod{13}$ and $n_{13} \vert 25$, so $n_13 = 1$ or $26$.
    However, if there is also a unique sylow 13-subgroup of $G$, then $G$ must be abelian as all of sylow subgroups are unique.
    Thus, $n_13 = 26$. Now, since all sylow 13-subgroup are cyclic, then they intersect trivially, 
    so there must be $12 \times 26 = 312$ elements of order 13 in $G$.

    Now, the unique sylow 5-subgroup contains 24 elements, none of which has order 13 by lagrange's theorem. So, $G$ must contains 
    at least $312 + 24 = 336$ non-trivial elements, which is not possible. This concludes that $G$ must be abelian.

    Consider that if $G$ is abelian, then by the fundamental theorem of finite abelian group, 
    $$G \isom \ZMod[325] \text{ or } G \isom \ZMod[5] \times \ZMod[65]$$
    This is because $\ZMod[p] \times \ZMod[q] \isom \ZMod[pq]$ if $\gcd(p, q) = 1$.

    Moreover, $\ZMod[325]$ and $\ZMod[5] \times \ZMod[65]$ is not isomorphism as the first one is cyclic but the latter is not.

    To conclude, they are only two groups of order 325 upto isomorphism.
  } 

  \qs{}{
    A ring $R$ is called a Boolean ring if $a^2 = a$ for all $a \in R$. Prove that every Boolean ring is commutative. And prove that 
    only Boolean ring that is an integral domain is $\ZMod[2]$. 
  }
  \sol{
    Let $a$ and $b$ be two elements in $R$. Then $a + b \in R$ by closure, 
    hence $a + b = (a + b)^2 = a^2 + ab + ba + b^2 = a + ab + ba + b$. This shows that $ab + ba = 0$, so, $ab = -ba$.
    But, $ab \in R$ by closure, so $ab + ab = (ab+ab)^2 = ab + ab + ab + ab$, which is that $ab + ab = 0$, or $ab = -ab$.
    Therefore, $-ab = -ba$, or equivalently, $ab = ba$ for every $a$ and $b$ in $R$.
    
    Now, as $a^2 = a$, then $a(a-1) = a^2 - a = 0$. If there exist two non-zero elements $a$ and $(a-1)$, then $R$ is not an 
    integral domain. That is equivalent to saying that if there is a non-zero element $x$ such that $x + 1$ is also non-zero, then 
    $R$ would not be an integral domain.
    But since $1$ is not the additive identity, if there is at least 3 elements in $R$, then there must be an element that is non-zero 
    and that is not the inverse of $1$. Consider that the element, say, $x$ satisfies that $x+1 \ne 0$ and $x \ne 0$. 
    So, $(x+1)(x+1-1) = (x+1)^2 - (x+1) = 0$ shows that $(x+1)$ is a zero-divisor.

    For the case of $R = \ZMod[2]$, it is trivial to check that $1\cdot1 = 1^2 = 1 \ne 0$, since $1$ is the only non-zero element. 
  } 

  \qs{}{
    Let $R$ be a ring with an identity. Prove that the center $\set{z \in R \mid zr = rz \text{ for all } r \in R}$ of $R$ is a 
    subring that contains the identity. And prove that the center of a division ring is a field. 
  }
  \sol{
    Let $Z = \set{z \in R \mid zr = rz \text{ for all } r \in R}$ be the center of $R$.
    Then, for $z, z' \in Z$, the sum $z + z'$ satisfies $(z+z')r = zr + z'r = rz + rz' = r(z+z')$. So, $(z + z') \in Z$. 
    Also, inverse $-z$ satisfies $(-z)r = -zr = -rz = r(-z)$. So, $(-z) \in Z$. 
    In addition, the product $zz'$ also satisfies $zz'r = zrz' = rzz'$, so $(zz') \in Z$. 
    This shows that $Z$ is a subring of $R$.
    In addition, $1 \in R$ since $1r = r = r1$ for every $r \in R$ by definition.

    Next, if $R$ is a division ring, then any element in $R$ has an inverse.
    The center $Z$ contains the element that is commutative over the multiplicative operation and that $Z$ is subring implies that 
    $Z$ is a commutative ring. Moreover, since any element in $R$ has an inverse, any element in $Z$ must have an inverse.
    Therefore, $Z$ is a commutative division ring, which proves that $Z$ is a field.
  } 

  \qs{}{
    Let $x$ be a nilpotent element (i.e. $x^m = 0$ for some $m \in \Int^+$) of the commutative ring $R$ with an identity.
    Prove that $x$ is either zero or a zero divisor. 
  }
  \sol{
    Assume that $x \ne 0$, then let $m$ be the least integer such that $x^m = 0$, so that $x^{m-1} \ne 0$. 
    Then $x \cdot x^{m-1} = x^m = 0$ where neither $x$ nor $x^{m-1}$ is zero. So, $x$ is a zero divisor.

    Therfore, $x=0$ or $x$ is a zero divisor must holds.
  }

  \qs{}{
    Let $x$ be a nilpotent element (i.e. $x^m = 0$ for some $m \in \Int^+$) of the commutative ring $R$ with an identity.
    Prove that $1 + rx$ is a unit in $R$ for all $r \in R$. 
  }
  \sol{
    For any element $x$, consider an integer $m$ such that $x^m = 0$. If $m$ is even, then consider $x^{m + 1} = x^m \cdot x = 0$. 

    Now, $$1 = 1 + r^mx^m = 1 + (rx)^m = (1 + rx)(1 - rx + (rx)^2 - \cdots + (rx)^{m-1})$$.
    holds for any value of $r$ as $x^m = 0$. Then $(1+rx)^{-1} = (1 - rx + (rx)^2 - \cdots + (rx)^{m-1})$ by definition.
    Hence, $(1+rx)$ is invertible.
  } 

  \qs{}{
    Let $K = \Rat$ and let $p$ be a prime integer. For any $x \in \Rat$, we can write uniquely as $x = p^n\frac{c}{d}$ 
    where $p \nmid c$ and $p\nmid d$. Define $v_p(x) = n$. Prove that $v_p$ is a discrete valuation on $K$. 
  }
  \sol{
    Define $v_p$ as in the problem statement. Note that since $x = p^n \frac{c}{d}$ is a unique representation (according to the 
    fundamental theorem of arithmetric), then $v_p$ is well-defined.

    Let $x = p^n\frac{a}{b}$, and $y = p^m\frac{c}{d}$, where $p$ does not divide any of $a, b, c, d$.
    Without loss of generality, let $n < m$.

    Consider $v_p(xy) = v_p(p^n\frac{a}{b}p^m \frac{c}{d}) = v_p(p^{n+m} \frac{ac}{bd})$ where $p \nmid ac$ and $p\nmid bd$.
    So, $v_p(xy) = n+m = v_p(x) + v_p(y)$

    Next, consider $v_p(x + y) = v_p(\frac{p^n a }{b} + \frac{p^m c}{d}) = v_p(\frac{p^n ad + p^m cb}{bd} = p^n
    \frac{ad + p^{m-n}cb}{bd})$.
    So $v_p(x + y) = v_p(p^{n}) + v_p(\frac{ad + p^{m-n} cb}{bd}) \ge n = \min(n, m) = \min(v_p(x), v_p(y))$.

    Thus, $v_p$ is a discrete valuation on $K$.
  }
  
  \qs{}{
    In problem 9, prove that the ring of all rational numbers whose denominators are relatively prime to $p$ is a 
    discrete valuation ring.
  }
  \sol{
    As $v_p: p^n\frac{a}{c} \mapsto n$ is a discrete valuation on $\Rat$, then $\Rat_v = \set{a \in \Rat^\times \mid v_p(a) \le 0} 
    \cup \set{0}$ is a discrete valuation ring by definition. 

    Consider that for a rational number $r$ in its simplest form that the denominator is divisible by $p$, then $r = \frac{a}{p^n b}$ 
    for some positive integer $n$, and $a, b$ relatively prime to $p$. Thus, $v_p(r) = -n < 0$. 

    Otherwise, if the denominator is not divisible by $p$, then $f = \frac{p^n a}{b}$ for some non-negative integer $n$, and $a, b$
    relatively prime to $p$. Thus, $v_p(r) = n \ge 0$. 

    So, $\Rat_v = \set{\frac{c}{d} \in \Rat^\times \mid d \text{ is not divisible by } p} \cup \set{0}$. 
    Hence, the set of all rational numbers whose denominators are relatively prime to $p$ is a discrete valuation ring.
  } 

\end{document}
