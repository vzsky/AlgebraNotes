% chktex-file 44
% chktex-file 8

\documentclass{report}
\usepackage{amsthm}
\usepackage{amsmath}
\usepackage{amssymb}
\usepackage{amssymb}
\usepackage{amsfonts}
\usepackage{xcolor}
\usepackage{tikz}
\usepackage{fancyhdr}
\usepackage{enumerate}
\usepackage{graphicx}
\usepackage[normalem]{ulem}
\usepackage[most,many,breakable]{tcolorbox}
\usepackage[a4paper, top=80pt, foot=25pt, bottom=50pt, left=0.5in, right=0.5in]{geometry}
\usepackage{hyperref, theoremref}
\hypersetup{
	pdftitle={Assignment},
	colorlinks=true, linkcolor=b!90,
	bookmarksnumbered=true,
	bookmarksopen=true
}
\usepackage{nameref}
\usepackage{parskip}
\pagestyle{fancy}

\usepackage[explicit,compact]{titlesec}
\titleformat{\chapter}[block]{\bfseries\huge}{\thechapter. }{\compact}{#1}
        

%%%%%%%%%%%%%%%%%%%%%
%% Defining colors %%
%%%%%%%%%%%%%%%%%%%%%

\definecolor{lr}{RGB}{188, 75, 81}
\definecolor{r}{RGB}{249, 65, 68}
\definecolor{dr}{RGB}{174, 32, 18}
\definecolor{lo}{RGB}{255, 172, 129}
\definecolor{do}{RGB}{202, 103, 2}
\definecolor{o}{RGB}{238, 155, 0}
\definecolor{ly}{RGB}{255, 241, 133}
\definecolor{y}{RGB}{255, 229, 31}
\definecolor{dy}{RGB}{143, 126, 0}
\definecolor{lb}{RGB}{148, 210, 189}
\definecolor{bg}{RGB}{10, 147, 150}
\definecolor{b}{RGB}{39, 125, 161}
\definecolor{db}{RGB}{0, 95, 115}
\definecolor{p}{RGB}{229, 152, 155}
\definecolor{dp}{RGB}{181, 101, 118}
\definecolor{pp}{RGB}{142, 143, 184}
\definecolor{v}{RGB}{109, 89, 122}
\definecolor{lg}{RGB}{144, 190, 109}
\definecolor{g}{RGB}{64, 145, 108}
\definecolor{dg}{RGB}{45, 106, 79}

\colorlet{mysol}{g}
\colorlet{mythm}{lr}
\colorlet{myqst}{db}
\colorlet{myclm}{lb}
\colorlet{mywrong}{r}
\colorlet{mylem}{o}
\colorlet{mydef}{lg}
\colorlet{mycor}{lb}
\colorlet{myrem}{dr}

%%%%%%%%%%%%%%%%%%%%%

\newcommand{\col}[2]{
  \color{#1}#2\color{black}\,
}

\newcommand{\TODO}[1][5cm]{
  \color{red}TODO\color{black}
  \vspace{#1}
}

\newcommand{\wans}[1]{
	\noindent\color{mywrong}\textbf{Wrong answer: }\color{black}
	#1 


}

\newcommand{\wreason}[1]{
	\noindent\color{mywrong}\textbf{Reason: }\color{black}
	#1 

  
}

\newcommand{\sol}[1]{
	\noindent\color{mysol}\textbf{Solution: }\color{black}
	#1


}

\newcommand{\nt}[1]{
  \begin{note}Note: #1\end{note}
}

\newcommand{\ky}[1]{
  \begin{key}#1\end{key}
}

\newcommand{\pf}[1]{
  \begin{myproof}#1\end{myproof}
}

\newcommand{\qs}[3][]{
  \begin{question}{#2}{#1}#3\end{question}
}

\newcommand{\df}[3][]{
  \begin{definition}{#2}{#1}#3\end{definition}
}

\newcommand{\thm}[3][]{
  \begin{theorem}{#2}{#1}#3\end{theorem}
}

\newcommand{\clm}[3][]{
  \begin{claim}{#2}{#1}#3\end{claim} 
}

\newcommand{\lem}[3][]{
  \begin{lemma}{#2}{#1}#3\end{lemma}
}

\newcommand{\cor}[3][]{
  \begin{corollary}{#2}{#1}#3\end{corollary}
}

\newcommand{\rem}[3][]{
  \begin{remark}{#2}{#1}#3\end{remark}
}

\newcommand{\twoways}[2]{
  \leavevmode\\
  ($\Longrightarrow$): 
  \begin{shift}#1\end{shift}
  ($\Longleftarrow$):
  \begin{shift}#2\end{shift} 
}

\newcommand{\nways}[2]{
  \leavevmode\\
  ($#1$): 
  \begin{shift}#2\end{shift}
}

%%%%%%%%%%%%%%%%%%%%%%%%%%%%%% ENVRN

\newenvironment{myproof}[1][\proofname]{%
	\proof[\bfseries #1: ]
}{\endproof}

\tcbuselibrary{theorems,skins,hooks}
\newtcolorbox{shift}
{%
  before upper={\setlength{\parskip}{5pt}},
  blanker,
	breakable,
	width=0.95\textwidth,
  enlarge left by=0.03\textwidth,
}

\tcbuselibrary{theorems,skins,hooks}
\newtcolorbox{key}
{%
	breakable,
	width=0.95\textwidth,
  enlarge left by=0.03\textwidth,
}

\tcbuselibrary{theorems,skins,hooks}
\newtcolorbox{note}
{%
	enhanced,
	breakable,
	colback = white,
	width=\textwidth,
	frame hidden,
	borderline west = {2pt}{0pt}{black},
	sharp corners,
}

\tcbuselibrary{theorems,skins,hooks}
\newtcbtheorem[]{remark}{Remark}
{%
	enhanced,
	breakable,
	colback = white,
	frame hidden,
	boxrule = 0sp,
	borderline west = {2pt}{0pt}{myrem},
	sharp corners,
	detach title,
  before upper={\setlength{\parskip}{5pt}\tcbtitle\par\smallskip},
	coltitle = myrem,
	fonttitle = \bfseries\sffamily,
	description font = \mdseries,
	separator sign none,
	segmentation style={solid, myrem},
}{rem}

\tcbuselibrary{theorems,skins,hooks}
\newtcbtheorem[number within=section]{lemma}{Lemma}
{%
	enhanced,
	breakable,
	colback = white,
	frame hidden,
	boxrule = 0sp,
	borderline west = {2pt}{0pt}{mylem},
	sharp corners,
	detach title,
  before upper={\setlength{\parskip}{5pt}\tcbtitle\par\smallskip},
	coltitle = mylem,
	fonttitle = \bfseries\sffamily,
	description font = \mdseries,
	separator sign none,
	segmentation style={solid, mylem},
}{lem}

\tcbuselibrary{theorems,skins,hooks}
\newtcbtheorem{claim}{Claim}
{%
  parbox=false,
	enhanced,
	breakable,
	colback = white,
	frame hidden,
	boxrule = 0sp,
	borderline west = {2pt}{0pt}{myclm},
	sharp corners,
	detach title,
  before upper={\setlength{\parskip}{5pt}\tcbtitle\par\smallskip},
	coltitle = myclm,
	fonttitle = \bfseries\sffamily,
	description font = \mdseries,
	separator sign none,
	segmentation style={solid, myclm},
}{clm}

\makeatletter
\newtcbtheorem[number within=section, use counter from=lemma]{theorem}{Theorem}{enhanced,
	breakable,
	colback=white,
	colframe=mythm,
	attach boxed title to top left={yshift*=-\tcboxedtitleheight},
	fonttitle=\bfseries,
	title={#2},
	boxed title size=title,
	boxed title style={%
			sharp corners,
			rounded corners=northwest,
			colback=mythm,
			boxrule=0pt,
		},
	underlay boxed title={%
			\path[fill=mythm] (title.south west)--(title.south east)
			to[out=0, in=180] ([xshift=5mm]title.east)--
			(title.center-|frame.east)
			[rounded corners=\kvtcb@arc] |-
			(frame.north) -| cycle;
		},
	#1
}{thm}
\makeatother

\makeatletter
\newtcbtheorem{question}{Question}{enhanced,
	breakable,
	colback=white,
	colframe=myqst,
	attach boxed title to top left={yshift*=-\tcboxedtitleheight},
	fonttitle=\bfseries,
	title={#2},
	boxed title size=title,
	boxed title style={%
			sharp corners,
			rounded corners=northwest,
			colback=myqst,
			boxrule=0pt,
		},
	underlay boxed title={%
			\path[fill=myqst] (title.south west)--(title.south east)
			to[out=0, in=180] ([xshift=5mm]title.east)--
			(title.center-|frame.east)
			[rounded corners=\kvtcb@arc] |-
			(frame.north) -| cycle;
		},
	#1
}{qs}
\makeatother

\makeatletter
\newtcbtheorem[number within=section]{definition}{Definition}{enhanced,
	breakable,
	colback=white,
	colframe=mydef,
	attach boxed title to top left={yshift*=-\tcboxedtitleheight},
	fonttitle=\bfseries,
	title={#2},
	boxed title size=title,
	boxed title style={%
			sharp corners,
			rounded corners=northwest,
			colback=mydef,
			boxrule=0pt,
		},
	underlay boxed title={%
			\path[fill=mydef] (title.south west)--(title.south east)
			to[out=0, in=180] ([xshift=5mm]title.east)--
			(title.center-|frame.east)
			[rounded corners=\kvtcb@arc] |-
			(frame.north) -| cycle;
		},
	#1
}{def}
\makeatother

\makeatletter
\newtcbtheorem[number within=section, use counter from=lemma]{corollary}{Corollary}{enhanced,
	breakable,
	colback=white,
	colframe=mycor,
	attach boxed title to top left={yshift*=-\tcboxedtitleheight},
	fonttitle=\bfseries,
	title={#2},
	boxed title size=title,
	boxed title style={%
			sharp corners,
			rounded corners=northwest,
			colback=mycor,
			boxrule=0pt,
		},
	underlay boxed title={%
			\path[fill=mycor] (title.south west)--(title.south east)
			to[out=0, in=180] ([xshift=5mm]title.east)--
			(title.center-|frame.east)
			[rounded corners=\kvtcb@arc] |-
			(frame.north) -| cycle;
		},
	#1
}{cor}
\makeatother

% Basic
  \DeclareMathOperator{\lcm}{lcm}
  \newcommand{\Real}{\mathbb{R}}
  \newcommand{\Comp}{\mathbb{C}}
  \newcommand{\Nat}{\mathbb{N}}
  \newcommand{\Rat}{\mathbb{Q}}
  \newcommand{\Int}{\mathbb{Z}}
  \newcommand{\set}[1]{\left\{\, #1 \,\right\}}
  \newcommand{\paren}[1]{\left( \; #1 \; \right)}
  \newcommand{\abs}[1]{\left\lvert #1 \right\rvert}
  \newcommand{\ang}[1]{\left\langle #1 \right\rangle}
  \renewcommand{\to}[1][]{\xrightarrow{\text{#1}}}
  \newcommand{\tol}[1][]{\to{$#1$}}
  \newcommand{\curle}{\preccurlyeq}
  \newcommand{\curge}{\succcurlyeq}
  \newcommand{\mapsfrom}{\leftarrow\!\shortmid}

  \newcommand{\mat}[1]{\begin{bmatrix} #1 \end{bmatrix}}
  \newcommand{\pmat}[1]{\begin{pmatrix} #1 \end{pmatrix}}
  \newcommand{\eqs}[1]{\begin{align*} #1 \end{align*}}
  \newcommand{\case}[1]{\begin{cases} #1 \end{cases}}
  

  % Algebra
  \newcommand{\normSg}[0]{\vartriangleleft}
  \newcommand{\ZMod}[1][n]{\mathbb{Z}/#1\mathbb{Z}}
  \newcommand{\isom}{\simeq}
  \newcommand{\mapHom}{\xrightarrow{\text{hom}}}
  \DeclareMathOperator{\Inn}{Inn}
  \DeclareMathOperator{\Aut}{Aut}
  \DeclareMathOperator{\im}{im}
  \DeclareMathOperator{\ord}{ord}
  \DeclareMathOperator{\Gal}{Gal}
  \DeclareMathOperator{\chr}{char}
  \newcommand{\surjto}{\twoheadrightarrow}
  \newcommand{\injto}{\hookrightarrow}

  % Analysis 
  \newcommand{\limty}[1][k]{\lim_{#1\to\infty}}
  \newcommand{\norm}[1]{\left\lVert#1\right\rVert}
  \newcommand{\darrow}{\rightrightarrows}


\fancyhead[L]{HW 9 - Modern Algebra MAS311}
\fancyhead[R]{\textbf{Touch Sungkawichai} 20210821}

\begin{document}
  \qs{}{
    Find the number of all Sylow 7-subgroups of $S_7$.
  }  
  \sol{
    Notice that there are $6!$ elements of order $7$ given by counting an element in the form of $(1 a b c d e f)$. As an 
    element in this form has order $7$, and no other element have order 7 if it is not written in this form. This is 
    because when considering a disjoint cycle notation, the order of the element is determined by the lcm of the size of 
    each disjoint cycles.

    Next, a sylow 7-subgroups of $S_7$ has order 7 because $\abs{S_7} = 7!$. And every two distinct sylow subgroup are cyclic, 
    thus they must intersects trivially. Since there is exactly $6!$ elements, and $6$ non-trivial elements in each cyclic
    group, then there should be exactly $6!/6 = 5! = 120$ sylow 7-subgroups. 
  } 

  \qs{}{
    Given $\sigma \in S_n$, we define a permutation $\tau: \tau = \sigma(n+1 \; n+2)$ if $\sigma$ is odd and 
    $\tau = \sigma$ if $\sigma$ is even. Show that the morphism $S_n \to S_{n+2}$ by $\sigma \mapsto \tau$ is injective
    and the image is contained in $A_{n+2}$. Conclude that any finite group is isomorphic to a subgroup of an
    alternating group. 
  }
  \sol{
    Firsly, define a morphism $\phi: S_n \to S_{n+2}$ by $\phi: \sigma \mapsto \tau$ as defined in the statement.
    $\phi$ is a homomorphism as if $\sigma$ and $\sigma'$ are both even, then $\phi(\sigma\sigma') = \phi(\sigma)\phi(\sigma')$ 
    trivially, if exactly one of them is even, without loss of generality, let $\sigma$ be odd and $\sigma'$ is even then 
    $\phi(\sigma\sigma') = \sigma(n+1\; n+2)\sigma' = \phi(\sigma)\phi(\sigma')$. And if both of them are odd, then 
    $\phi(\sigma\sigma') = \sigma\sigma' = \sigma(n+1 \; n+2)^2\sigma' = \sigma(n+1 \; n+2) \sigma'(n+1 \; n+2) = 
    \phi(\sigma)\phi(\sigma')$. This verifies that $\phi$ is a homomorphism.

    Consider that if $\sigma$ moves $x$ (to $y$), then $\phi(\sigma)$ must also moves $x$ (to $y$). So, if $\phi(\sigma)$ 
    does not move any element (ie. identity), then the only possible $\sigma$ is the identity. 
    Now, $\phi(id) = id$ since $id$ is an even permutation. 
    Therefore, $\ker\phi = \set{id}$. Which proves that $\phi$ is injective. 

    Next, the image $\im\phi$ is the set $\set{ \tau =  \begin{cases} \sigma & \text{ if $\sigma$ is even} \\ 
        \sigma(n+1 \; n+2) & \text{ if $\sigma$ is odd} \end{cases}}$
    Which is always even. Thus, $\im\phi$ must be a subset, and therefore a subgroup of $A_{n+2}$ as it is the image of $\phi$.

    Lastly, as from the Cayley's theorem, there is an isomorphism from any finite group $G$ to a subgroup of $S_{\abs{G}}$. 
    Let that isomorphism be $\psi$. Then, $\psi: G \to S_{\abs{G}}$ is an injective homomorphism. 
    Consider a composition $\phi \circ \psi$. Then this homomorphism is an injective homomorphism of $G \to S_{\abs{G} + 2}$. 
    Furthermore, the image of $\phi \circ \psi$ is a subgroup of $A_{\abs{G} + 2}$. Thus, $\psi \circ \phi$ is an injective 
    homomorphism from $G$ to $A_{\abs{G} + 2}$. 

    Therefore, since there is an injective homomorphism from $G$ to $A_n$, it follows that $G$ is isomorphic to a subgroup 
    $\im(\phi \circ \psi)$ of an alternating group. 
  }
  
  \qs{}{
    Let $\sigma \in S_3 \backslash A_3$. Show that the automorphsim of $A_3$ given by conjugation by $\sigma$ is 
    not an inner automorphism of $A_3$. 
  }
  \sol{
    Firstly, consider that $A_3 = \set{id, (1 2 3), (1 3 2)}$ and $S_3 \backslash A_3 = \set{(1 2), (2 3), (1 3)}$.
    Let $\psi_\sigma: \tau \mapsto \sigma \tau \sigma^{-1}$ for $\sigma \in S_3 \backslash A_3$
    
    With the following information.
    \eqs{
      (12)(1 2 3)(12) = (1 3 2) \quad &\text{ and }\quad (12)(1 3 2)(12) = (1 2 3) \\
      (13)(1 2 3)(13) = (1 3 2) \quad &\text{ and }\quad (13)(1 3 2)(13) = (1 2 3) \\
      (23)(1 2 3)(23) = (1 3 2) \quad &\text{ and }\quad (23)(1 3 2)(23) = (1 2 3)
    }

    Therefore, $\psi_{(12)}$, $\psi_{(23)}$, $\psi_{(13)}$ are an element of $\Aut(S_3)$. 

    Next, an inner automorphism of $A_3$ is $\Inn(A_3) = \set{ \phi_\sigma  \mid  \sigma \in A_3 \text{ and }
    \phi_\sigma : \tau \mapsto \sigma \tau \sigma^{-1}}$
    As $\abs{A_3} = 3!/2 = 3$, then $A_3 \isom \ZMod[3]$. So $A_3$ is abelian, thus 
    \[ \phi_\sigma : \tau \mapsto \sigma\tau\sigma^{-1} = \tau\sigma\sigma^{-1} = \tau \]
    So, $\Inn(A_3)$ is trivial
    Hence, $\sigma_{(12)}$, $\sigma_{(13)}$, $\sigma{(23)}$ are not an element of the inner automorphism as they are not 
    the identity automorphism.
    \vspace{2cm}
  }

  \qs{}{
    Let $p$ and $q$ be a prime numbers. Show that any group of order $pq$ is solvable.
  }
  \sol{
    Let a group $G$ be a group of order $pq$ where $p$ and $q$ are prime numbers.
    If $p = q$, then $\abs{G} = p^2$. As $G$ is a $p$ group, then it is nilpotent, and the factor between the subgroups 
    must be abelian (order $p$ or $p^2$). So $G$ must be solvable.

    Assume without loss of generality that $p > q$. 
    Let $H$ be a sylow subgroup of order $p$. Then there is $n_p \equiv 1 \pmod{p}$ subgroups where $n_p \vert q$. 
    The only possible conclusion is that $n_p = 1$, since otherwise $n_p \not\vert q$, or $q = p+1$. 

    Therefore, $H$ is a normal subgroup of $G$. So, $G/H$ is a group of order $q$. Since a group of order $p$ is cylic, and 
    a group of order $q$ is also cyclic, then they must be abelian, then they must be solvable. Since a subgroup $H$ of $G$, 
    and the quotient $G/H$ are both solvable, it must be the case that the group $G$ must also be solvable. 
  }
  
  \qs{}{
    Find the subgroup of all torsion elements in $\Real/\Int$. 
  }
  \sol{
    Consider that an element of $G = \Real/\Int$ is $r + \Int$ for some $r \in \Real$. 
    If $r$ is irregular, then for any integer $n$, $n(r + \Int) = nr + \Int$. Assuming that $nr$ is an integer yields that 
    $nr = m$ for some integer $m$. Which means that $r = \frac{m}{n}$ is not an irregular number. Thus, is a contradiction. 
    Therefore, $nr$ is not an integer, which means that $n(r + \Int) \ne \Int$ for any finite integer $n$. 
    This means that $(r+\Int) \not\in Tors(\Real/\Int)$

    Next, if $r$ is regular, then let $r = \frac{m}{n}$ without loss of generality. 
    Consider that $n(r + \Int) = nr + \Int = m + \Int = \Int$, so the order of $(r + \Int)$ is less than or equal to $n$.
    As the order is finite, then $r + \Int \in Tors(\Real/\Int)$

    As a real number is either regular or irregular, it follows that $T = \set{r + \Int \mid r \in \Rat}$ is the subset of all 
    torsion elements in $\Real/\Int$.

    Lastly, it can be shown that the subset is a subgroup since if $\frac{n}{m} + \Int \in T$, and $\frac{u}{v} + \Int \in T$ 
    then \[ (\frac{n}{m} + \Int) - (\frac{u}{v} + \Int) = \frac{n}{m}-\frac{u}{v} + \Int = \frac{nv - uv}{mv} + \Int \in T\]
    In conclusion, $T$ is the subgroup of all torsion elements in $\Real/\Int$.
  }

  \qs{}{
    Prove that if $\phi: K \to \Aut(H)$ is a nontrivial group homomorphism, then $H \rtimes_\phi K$ is nonabelian.
  }
  \sol{
    If $\phi$ is nontrivial, then there exists an element $k \in K$ such that $\phi(k)$ is not trivial. Thus, there 
    exists an element $h \in H$ such that $\phi(k)(h) \ne h$. 

    For that element $k$ and $h$, consider $(h, e) \cdot (h, k) = (h\phi(e)(h), k) = (hh, k)$ since $\phi(e)$ must 
    be the identity element in $\Aut(H)$. So, $\phi(e)(h) = h$. 

    However, $(h, k)\cdot (h, e) = (h \phi(k)(h), ke) = (h\phi(k)(h), k) \ne (hh, k)$ as $\phi(k)(h) \ne h$ by 
    construction. 

    As $(h,e)\cdot(h,k) \ne (h,k)\cdot(h,e)$ and $(h,e)$ and $(h,k)$ are both an element of $H \rtimes_\phi K$. 
    Therefore, the group $H \rtimes_\phi K$ is nonabelian.
  }

  \newcommand{\GL}{GL_n(\Real)}
  \newcommand{\SL}{SL_n(\Real)}
  \qs{}{
    Show that $\GL \isom \SL \rtimes \Real^\times$.
  }
  \sol{
    Firstly, $\SL$ is a subgroup of $\GL$. Furthermore, consider that for an element $g \in \GL$ 
    and $s \in \SL$, $\det(gsg^{-1}) = \det(g)\det(s)\det(g^{-1}) = \det(s)$. Thus, $gsg^{-1} \in \SL$ be definition. 
    So, $\SL$ is a normal subgroup of $\GL$.

    Now, consider an isomorphism $\phi: \Real^\times \to \im\phi \subset \GL$ given by
    $\phi: r \mapsto \mat{r & 0 & \cdots& 0 \\ 0 & 1 & \cdots& 0 \\ \vdots & & \ddots & & \\ 0 & 0 & \cdots & 1}$, 
    of $\GL$. Then, $\phi$ is well-defined obviously, since $\phi(r) = \phi(r')$ implies that $r = r'$, as 
    $r = (\phi(r))_{11} = (\phi(r'))_{11} = r'$ was required. Then, $\phi$ is a homomorphism as 
    \[ \phi(rs) = \mat{rs & \cdots & 0 \\ 0 & \ddots & 0 \\ 0 & 0 & 1 } = 
      \mat{r & \cdots & 0 \\ 0 & \ddots & 0 \\ 0 & 0 & 1 }\mat{s & \cdots & 0 \\ 0 & \ddots & 0 \\ 0 & 0 & 1 } 
      = \phi(r)\phi(s) \]
    Next, if $\phi(r) = \phi(r')$, then $r = (\phi(r))_{11} = (\phi(r'))_{11} = r'$, thus $\phi$ is injective. 
    Therefore, $\phi$ is an isomorphism. 

    Now, consider an element $g \in \GL$, let $\abs{\det(g)} = r$ for some real number $r$, 
    then there is a matrix $g'$ such that $g' \cdot \phi(r) = g$ given by diving each element in the first column of $g$ by $r$, 
    Since $\det(g') \det(\phi(r)) = \det(g)$, then $\det(g') = 1$, so $g' \in \SL$, and $\phi(r) \in \im\phi$. 
    Thus, $\GL \subset \SL\im\phi$. But as $\SL < \GL$ and $\im\phi < \GL$, it must be the case that $\GL = \SL\im\phi$

    Next, consider $g \in \SL$ and $g \in \im\phi$. Then $\det(g) = 1$, and $g = \phi(r)$, therefore, $\det(\phi(r)) = r = 1$. 
    Hence, it follows that $r = 1$. Thus, $\SL \cap \im\phi = \set{\phi(1) = I}$. 

    Now, as $\SL \normSg \GL$, $\im\phi < \GL$, where the intersection is trivial and $\GL = \SL\im\phi$. It follows that 
    $\GL \isom \SL \rtimes \im\phi$. Next, as $\im\phi \isom \Real^\times$, then $\GL \isom \SL \rtimes \Real^\times$. 
  } 

  \qs{}{
    Explain why two groups $D_{24}$ and $S_4$ are not isomorphic. 
  }
  \sol{
    An element in $D_{24}$ is of the form $r^if^j$, for $j = 0$ or $1$ and $i \in \set{0, \ldots, 11}$
    If $j = 1$, then $r^if \cdot r^if = r^i f^2 r^{-i} = e$, so $r^if$ has order 2.
    If $j = 0$, then the order of $r^i$ is $12/\gcd(i, 12)$. So, $\gcd(i, 12) = 3$ only when $i = 3, 9$.

    Consider that there is only two elements $g \in D_{24}$, that has order $4$, which are $r^3$, $r^9$. 
    But there is more than two elements of order 4 in $S_4$, for examples, $(1 2 3 4)$, $(1 3 2 4)$ and $(1 4 3 2)$. 
  }

  \qs{}{
    Explain why two groups $\ZMod[12] \times \ZMod[4] \times \ZMod[3]$ and $\ZMod[6] \times \ZMod[6] \times \ZMod[4]$ 
    are not isomorphic. 
  }
  \sol{
    Consider if $(A \times C) \isom (B \times C)$, then since $C \isom C$, it follows that $A \isom B$. This is due to 
    the fact that $\times$ and $\oplus$ behave similarly for finite groups, (and it is proven that 
    $(G \oplus H) \isom (G' \oplus H')$ with $G \isom G'$ implies $H \isom H'$). 

    Now, as $\ZMod[12] \times \ZMod[4] \times \ZMod[3]$ is isomorphic to $\ZMod[12] \times \ZMod[3] \times \ZMod[4]$ by a 
    homomorphism $\phi: (a, b, c) \mapsto (a, c, b)$.

    Assuming that $\ZMod[12] \times \ZMod[4] \times \ZMod[3]$ is isomorphic to $\ZMod[6] \times \ZMod[6] \times \ZMod[4]$ 
    yields that $\ZMod[12] \times \ZMod[3]$ is isomorphic to $\ZMod[6] \times \ZMod[6]$.
    However, the second one possess no element of order 12. As for $(a, b) \in \ZMod[6] \times \ZMod[6]$, it follows that 
    $6(a, b) = (6a, 6b) = (0, 0)$. But the former possess at least one element of order 12, which is $(1, 0)$ as $6(1, 0) = 
    (6, 0) \ne (0, 0)$, and $12(1, 0) = (0, 0)$

    By contraposition, the two groups $\ZMod[12] \times \ZMod[4] \times \ZMod[3]$ and $\ZMod[6] \times \ZMod[6] \times \ZMod[4]$
    must be non-isomorphic.
  }

  \qs{}{
    Show that nonabelian groups $A_4$, $D_{12}$, and $\ZMod[3] \rtimes \ZMod[4]$ are not isomorphic.
  }
  \sol{
    Firstly, consider between the group $A_4$ and $\ZMod[3] \rtimes \ZMod[4]$. 
    It is evidence that $\ZMod[3]$ is a normal subgroup of $\ZMod[3] \rtimes \ZMod[4]$, and $\ZMod[3]$ is non-trivial. 
    Therefore, the later group is not simple, but the first group is simple. Thus, they cannot be isomorphic.

    Secondly, consider between the group $A_4$ and $D_{12}$. Notice that $\set{1, r, \ldots, r^5} \normSg D_{12}$ as 
    for $fr^i \in D_{12} - \set{1, r, \ldots, r^5}$, it follows that 
    \begin{align*} 
      fr^i \set{1, r, \ldots, r^5} (fr^i)^{-1} &= fr^i \set{1, r, \ldots, r^5} r^{-i}f \\ 
                                               &= \set{fr^ir^{-i}f, fr^i r r^{-i}f, \ldots, fr^i r^5 r^{-i} f} \\ 
                                               &= \set{1, frf, fr^2f, \ldots, fr^5f} \\ 
                                               &= \set{1, r^{-1}, r^{-2}, \ldots, r^{-5}} \\ 
                                               &= \set{1, r, \ldots, r^5}
    \end{align*}
    So, $D_{12}$ has a non-trivial normal subgroup, thus is non-simple, but $A_4$ is simple, so there is no isomorphism between
    the two. 

    Lastly, between the group $D_{12}$ and $\ZMod[3] \rtimes \ZMod[4]$. Consider that for $fr^i \in D_{12}$, the order are all 
    2. However, when considering element $(e,e) = (h,k)^2 = (h\phi(k)(h), k^2)$ of the later group, it must follow that 
    $k$ is of order 2, which is only 2. Thus, there are AT MOST 3 elements (which are $(0, 2), (1, 2), (2, 2)$) of order 2 
    in $\ZMod[3] \rtimes \ZMod[4]$. Therefore $D_{12}$ and $\ZMod[3] \rtimes \ZMod[4]$ cannot be isomorphic. 
  }
\end{document}
