% chktex-file 44
% chktex-file 8

\documentclass{report}
\usepackage{amsthm}
\usepackage{amsmath}
\usepackage{amssymb}
\usepackage{amssymb}
\usepackage{amsfonts}
\usepackage{xcolor}
\usepackage{tikz}
\usepackage{fancyhdr}
\usepackage{enumerate}
\usepackage{graphicx}
\usepackage[normalem]{ulem}
\usepackage[most,many,breakable]{tcolorbox}
\usepackage[a4paper, top=80pt, foot=25pt, bottom=50pt, left=0.5in, right=0.5in]{geometry}
\usepackage{hyperref, theoremref}
\hypersetup{
	pdftitle={Assignment},
	colorlinks=true, linkcolor=b!90,
	bookmarksnumbered=true,
	bookmarksopen=true
}
\usepackage{nameref}
\usepackage{parskip}
\pagestyle{fancy}

\usepackage[explicit,compact]{titlesec}
\titleformat{\chapter}[block]{\bfseries\huge}{\thechapter. }{\compact}{#1}
        

%%%%%%%%%%%%%%%%%%%%%
%% Defining colors %%
%%%%%%%%%%%%%%%%%%%%%

\definecolor{lr}{RGB}{188, 75, 81}
\definecolor{r}{RGB}{249, 65, 68}
\definecolor{dr}{RGB}{174, 32, 18}
\definecolor{lo}{RGB}{255, 172, 129}
\definecolor{do}{RGB}{202, 103, 2}
\definecolor{o}{RGB}{238, 155, 0}
\definecolor{ly}{RGB}{255, 241, 133}
\definecolor{y}{RGB}{255, 229, 31}
\definecolor{dy}{RGB}{143, 126, 0}
\definecolor{lb}{RGB}{148, 210, 189}
\definecolor{bg}{RGB}{10, 147, 150}
\definecolor{b}{RGB}{39, 125, 161}
\definecolor{db}{RGB}{0, 95, 115}
\definecolor{p}{RGB}{229, 152, 155}
\definecolor{dp}{RGB}{181, 101, 118}
\definecolor{pp}{RGB}{142, 143, 184}
\definecolor{v}{RGB}{109, 89, 122}
\definecolor{lg}{RGB}{144, 190, 109}
\definecolor{g}{RGB}{64, 145, 108}
\definecolor{dg}{RGB}{45, 106, 79}

\colorlet{mysol}{g}
\colorlet{mythm}{lr}
\colorlet{myqst}{db}
\colorlet{myclm}{lb}
\colorlet{mywrong}{r}
\colorlet{mylem}{o}
\colorlet{mydef}{lg}
\colorlet{mycor}{lb}
\colorlet{myrem}{dr}

%%%%%%%%%%%%%%%%%%%%%

\newcommand{\col}[2]{
  \color{#1}#2\color{black}\,
}

\newcommand{\TODO}[1][5cm]{
  \color{red}TODO\color{black}
  \vspace{#1}
}

\newcommand{\wans}[1]{
	\noindent\color{mywrong}\textbf{Wrong answer: }\color{black}
	#1 


}

\newcommand{\wreason}[1]{
	\noindent\color{mywrong}\textbf{Reason: }\color{black}
	#1 

  
}

\newcommand{\sol}[1]{
	\noindent\color{mysol}\textbf{Solution: }\color{black}
	#1


}

\newcommand{\nt}[1]{
  \begin{note}Note: #1\end{note}
}

\newcommand{\ky}[1]{
  \begin{key}#1\end{key}
}

\newcommand{\pf}[1]{
  \begin{myproof}#1\end{myproof}
}

\newcommand{\qs}[3][]{
  \begin{question}{#2}{#1}#3\end{question}
}

\newcommand{\df}[3][]{
  \begin{definition}{#2}{#1}#3\end{definition}
}

\newcommand{\thm}[3][]{
  \begin{theorem}{#2}{#1}#3\end{theorem}
}

\newcommand{\clm}[3][]{
  \begin{claim}{#2}{#1}#3\end{claim} 
}

\newcommand{\lem}[3][]{
  \begin{lemma}{#2}{#1}#3\end{lemma}
}

\newcommand{\cor}[3][]{
  \begin{corollary}{#2}{#1}#3\end{corollary}
}

\newcommand{\rem}[3][]{
  \begin{remark}{#2}{#1}#3\end{remark}
}

\newcommand{\twoways}[2]{
  \leavevmode\\
  ($\Longrightarrow$): 
  \begin{shift}#1\end{shift}
  ($\Longleftarrow$):
  \begin{shift}#2\end{shift} 
}

\newcommand{\nways}[2]{
  \leavevmode\\
  ($#1$): 
  \begin{shift}#2\end{shift}
}

%%%%%%%%%%%%%%%%%%%%%%%%%%%%%% ENVRN

\newenvironment{myproof}[1][\proofname]{%
	\proof[\bfseries #1: ]
}{\endproof}

\tcbuselibrary{theorems,skins,hooks}
\newtcolorbox{shift}
{%
  before upper={\setlength{\parskip}{5pt}},
  blanker,
	breakable,
	width=0.95\textwidth,
  enlarge left by=0.03\textwidth,
}

\tcbuselibrary{theorems,skins,hooks}
\newtcolorbox{key}
{%
	breakable,
	width=0.95\textwidth,
  enlarge left by=0.03\textwidth,
}

\tcbuselibrary{theorems,skins,hooks}
\newtcolorbox{note}
{%
	enhanced,
	breakable,
	colback = white,
	width=\textwidth,
	frame hidden,
	borderline west = {2pt}{0pt}{black},
	sharp corners,
}

\tcbuselibrary{theorems,skins,hooks}
\newtcbtheorem[]{remark}{Remark}
{%
	enhanced,
	breakable,
	colback = white,
	frame hidden,
	boxrule = 0sp,
	borderline west = {2pt}{0pt}{myrem},
	sharp corners,
	detach title,
  before upper={\setlength{\parskip}{5pt}\tcbtitle\par\smallskip},
	coltitle = myrem,
	fonttitle = \bfseries\sffamily,
	description font = \mdseries,
	separator sign none,
	segmentation style={solid, myrem},
}{rem}

\tcbuselibrary{theorems,skins,hooks}
\newtcbtheorem[number within=section]{lemma}{Lemma}
{%
	enhanced,
	breakable,
	colback = white,
	frame hidden,
	boxrule = 0sp,
	borderline west = {2pt}{0pt}{mylem},
	sharp corners,
	detach title,
  before upper={\setlength{\parskip}{5pt}\tcbtitle\par\smallskip},
	coltitle = mylem,
	fonttitle = \bfseries\sffamily,
	description font = \mdseries,
	separator sign none,
	segmentation style={solid, mylem},
}{lem}

\tcbuselibrary{theorems,skins,hooks}
\newtcbtheorem{claim}{Claim}
{%
  parbox=false,
	enhanced,
	breakable,
	colback = white,
	frame hidden,
	boxrule = 0sp,
	borderline west = {2pt}{0pt}{myclm},
	sharp corners,
	detach title,
  before upper={\setlength{\parskip}{5pt}\tcbtitle\par\smallskip},
	coltitle = myclm,
	fonttitle = \bfseries\sffamily,
	description font = \mdseries,
	separator sign none,
	segmentation style={solid, myclm},
}{clm}

\makeatletter
\newtcbtheorem[number within=section, use counter from=lemma]{theorem}{Theorem}{enhanced,
	breakable,
	colback=white,
	colframe=mythm,
	attach boxed title to top left={yshift*=-\tcboxedtitleheight},
	fonttitle=\bfseries,
	title={#2},
	boxed title size=title,
	boxed title style={%
			sharp corners,
			rounded corners=northwest,
			colback=mythm,
			boxrule=0pt,
		},
	underlay boxed title={%
			\path[fill=mythm] (title.south west)--(title.south east)
			to[out=0, in=180] ([xshift=5mm]title.east)--
			(title.center-|frame.east)
			[rounded corners=\kvtcb@arc] |-
			(frame.north) -| cycle;
		},
	#1
}{thm}
\makeatother

\makeatletter
\newtcbtheorem{question}{Question}{enhanced,
	breakable,
	colback=white,
	colframe=myqst,
	attach boxed title to top left={yshift*=-\tcboxedtitleheight},
	fonttitle=\bfseries,
	title={#2},
	boxed title size=title,
	boxed title style={%
			sharp corners,
			rounded corners=northwest,
			colback=myqst,
			boxrule=0pt,
		},
	underlay boxed title={%
			\path[fill=myqst] (title.south west)--(title.south east)
			to[out=0, in=180] ([xshift=5mm]title.east)--
			(title.center-|frame.east)
			[rounded corners=\kvtcb@arc] |-
			(frame.north) -| cycle;
		},
	#1
}{qs}
\makeatother

\makeatletter
\newtcbtheorem[number within=section]{definition}{Definition}{enhanced,
	breakable,
	colback=white,
	colframe=mydef,
	attach boxed title to top left={yshift*=-\tcboxedtitleheight},
	fonttitle=\bfseries,
	title={#2},
	boxed title size=title,
	boxed title style={%
			sharp corners,
			rounded corners=northwest,
			colback=mydef,
			boxrule=0pt,
		},
	underlay boxed title={%
			\path[fill=mydef] (title.south west)--(title.south east)
			to[out=0, in=180] ([xshift=5mm]title.east)--
			(title.center-|frame.east)
			[rounded corners=\kvtcb@arc] |-
			(frame.north) -| cycle;
		},
	#1
}{def}
\makeatother

\makeatletter
\newtcbtheorem[number within=section, use counter from=lemma]{corollary}{Corollary}{enhanced,
	breakable,
	colback=white,
	colframe=mycor,
	attach boxed title to top left={yshift*=-\tcboxedtitleheight},
	fonttitle=\bfseries,
	title={#2},
	boxed title size=title,
	boxed title style={%
			sharp corners,
			rounded corners=northwest,
			colback=mycor,
			boxrule=0pt,
		},
	underlay boxed title={%
			\path[fill=mycor] (title.south west)--(title.south east)
			to[out=0, in=180] ([xshift=5mm]title.east)--
			(title.center-|frame.east)
			[rounded corners=\kvtcb@arc] |-
			(frame.north) -| cycle;
		},
	#1
}{cor}
\makeatother

% Basic
  \DeclareMathOperator{\lcm}{lcm}
  \newcommand{\Real}{\mathbb{R}}
  \newcommand{\Comp}{\mathbb{C}}
  \newcommand{\Nat}{\mathbb{N}}
  \newcommand{\Rat}{\mathbb{Q}}
  \newcommand{\Int}{\mathbb{Z}}
  \newcommand{\set}[1]{\left\{\, #1 \,\right\}}
  \newcommand{\paren}[1]{\left( \; #1 \; \right)}
  \newcommand{\abs}[1]{\left\lvert #1 \right\rvert}
  \newcommand{\ang}[1]{\left\langle #1 \right\rangle}
  \renewcommand{\to}[1][]{\xrightarrow{\text{#1}}}
  \newcommand{\tol}[1][]{\to{$#1$}}
  \newcommand{\curle}{\preccurlyeq}
  \newcommand{\curge}{\succcurlyeq}
  \newcommand{\mapsfrom}{\leftarrow\!\shortmid}

  \newcommand{\mat}[1]{\begin{bmatrix} #1 \end{bmatrix}}
  \newcommand{\pmat}[1]{\begin{pmatrix} #1 \end{pmatrix}}
  \newcommand{\eqs}[1]{\begin{align*} #1 \end{align*}}
  \newcommand{\case}[1]{\begin{cases} #1 \end{cases}}
  

  % Algebra
  \newcommand{\normSg}[0]{\vartriangleleft}
  \newcommand{\ZMod}[1][n]{\mathbb{Z}/#1\mathbb{Z}}
  \newcommand{\isom}{\simeq}
  \newcommand{\mapHom}{\xrightarrow{\text{hom}}}
  \DeclareMathOperator{\Inn}{Inn}
  \DeclareMathOperator{\Aut}{Aut}
  \DeclareMathOperator{\im}{im}
  \DeclareMathOperator{\ord}{ord}
  \DeclareMathOperator{\Gal}{Gal}
  \DeclareMathOperator{\chr}{char}
  \newcommand{\surjto}{\twoheadrightarrow}
  \newcommand{\injto}{\hookrightarrow}

  % Analysis 
  \newcommand{\limty}[1][k]{\lim_{#1\to\infty}}
  \newcommand{\norm}[1]{\left\lVert#1\right\rVert}
  \newcommand{\darrow}{\rightrightarrows}


\fancyhead[L]{HW 5 - Modern Algebra MAS311}
\fancyhead[R]{\textbf{Touch Sungkawichai} 20210821}

\begin{document}
  \qs{}{
    Show that $(\ZMod[13])^\times \isom \ZMod[12]$.
  }  
  \sol{
    Consider $x \in (\ZMod[13])^\times$ such that $x \ne 0$, then $x^12 \equiv 1 \pmod{13}$ by the fermat's little theorem.
    Therefore, there must exists an element $x^{-1} = x^11 \in (\ZMod[13])^\times$ by the closure of group. 
    Hence, the order of $(\ZMod[13])^\times$ is 12. Then, since we know that for any element $x$, 
    $\abs{x} \vert 12$ by lagrange's theorem, and in $(\ZMod[13])^\times$, the followings hold 
    \begin{align*}
      [2^1] &\ne [1] \\ 
      [2^2] = [4] &\ne [1] \\
      [2^3] = [8] &\ne [1] \\ 
      [2^4] = [3] &\ne [1] \\ 
      [2^6] = [12] &\ne [1] \\ 
      [2^12] &= [1] 
    \end{align*}. Therefore, the order of 2 is 12. Thus, 2 generates $(\ZMod[13])^\times$. 
    This means that $(\ZMod[13])^\times$ is cyclic, and have the same order as $\ZMod[12]$. Thus, they are isomorphic.

    Alternatively, one can construct an isomorphism $\varphi: (\ZMod[13])^\times \to \ZMod[12]$ by 
    \[ \varphi([2]) = [1] \] Which will follows that 
    \begin{align*}
      \varphi([4]) = [2] \\ 
      \varphi([8]) = [3] \\ 
      \varphi([3]) = [4] \\ 
      \varphi([6]) = [5] \\ 
      \varphi([12]) = [6] \\ 
      \varphi([11]) = [7] \\ 
      \varphi([9]) = [8] \\ 
      \varphi([5]) = [9] \\ 
      \varphi([10]) = [10] \\ 
      \varphi([7]) = [11] \\ 
      \varphi([1]) = [0] 
    \end{align*}
    Which is a homomorphism, that is apparently surjective and injective. Therefore, $\varphi$ is an isomorphism.
  } 

  \qs{}{
    Let $H < G$. Prove that the map $gH \mapsto Hg^{-1}$ is a bijection between the sets of left and right cosets.
  }
  \sol{
    Let $f$ be a function that maps from $gH$ to $Hg^{-1}$. Then firstly, for $gH = g'H$, we have that 
    \[ Hg^{-1} = \set{hg^{-1} \mid h \in H} = \set{(gh)^{-1} \mid h \in H} = \set{(g'h)^{-1} \mid h \in H} = Hg'^{-1}\] So 
    $f$ is well-defined. 

    Next, for the surjectivity of $f$, let there be some right coset $Hg$. Then there must always be $g^{-1}$ such that 
    $f(g^{-1}H) = H(g^{-1})^{-1} = Hg$ by the property of inverse. 

    Lastly, for the injectivity, let consider that if $Hg = Hg'$, then it must follows that $g' \in \set{hg \mid h \in H}$. Now, 
    we can further deduce that \[ g'^{-1} \in \set{(hg)^{-1} \mid h \in H} = \set{g^{-1}h \mid h \in H} \] 
    So $g'gH = gH$, which means that $g'H = gH$. 

    Since $f$ is a function from the set of left cosets to the set of right cosets such that $f$ is surjective and injective, it 
    must follows that $f$ is a bijection between those two sets. 
  }

  \qs{}{
    Let $H < G$. Prove that $H \normSg G$ if and only if for any $g \in G$ there exists $g' \in G$ such that $gH = Hg'$.
  }
  \sol{
    \twoways{
      Assume that $H \normSg G$. Then for any $g \in G$, we know that $gHg^{-1} = H$, so $gH = Hg$. Thus there exists $g' = g$
      such that $gH = Hg'$
    }{
      Assume there for any $g \in G$ there is an element $g' \in G$ such that $gH = Hg'$. Then, since $g \in gH, g = hg'$ for 
      some element $h \in H$. This means that $h^{-1}g = g'$ for that element $h \in H$. 

      Now, since $gH = Hg'$, and $g' = h^{-1}g$, we get that $gH = Hh^{-1}g = Hg$. Thus, $gHg^{-1} = H$, which means that 
      $H \normSg G$ by definition. 
    }
  }

  \qs{}{
    Let $N$ be a subgroup of a cyclic group $G$. Prove that the quotient group $G/N$ is cyclic.
  }
  \clm[subcyiscy]{
    All subgroups of a cyclic group is cylic.
    \pf{
      Since a cyclic group $G$ is generated by a single element, $g$, then each element 
      in the subgroup $S \le G$ is in $G$, which means that it must be some power of $g$. 
      Then, let $S = \set{g^{a_1}, g^{a_2}, \ldots}$, so that it is possible to choose 
      the smallest possitive $a_i$ by the well ordering principle since $\abs{S} < \abs{G}$ is always countable, here 
      note that the only infinite cyclic group $G$ must be isomorphic to $\Int$, 
      so let $b$ be that element.

      With the division algorithm, we know that for any $a$, $g^{a} = g^{mb + c}$ for some integer $m$ and $0 \le c < b$. 
      The closure of the group asserts that $g^{c}$ must be an element of $S$, but $0 \le c < b$, so $c = 0$. 

      Hence, $g^b$ generates $S$.
    }
  }
  \sol{
    Firstly, let notice that a cyclic group is an abelian group, since $r^{n}r^{m} = r^{n+m} = r^{m}r^{n}$ for any $r, m, n$. 
    Now, let $N$ be a subgroup of $G$, then $N$ must be cyclic as per claim~\ref{clm:subcyiscy}, so $N$ must be abelian, 
    and thus, normal.

    Then $G/N$ is a group of the left cosets. Let $G = \ang{g}$. Then we know that $gN$ 
    generates $G/N$. Firstly, notice if $gN = N$, then $g^kN = N$ for every $k$, making $GN = N$,
    which means that $G/N = {1}$ is trivially cyclic. 

    So we are left with the case where $gN \ne N$.
    Now, let $N$ ge generated by $g^n$ with some integer $n$. Then we know that $g^kN \ne N$ for any $k < n$ since
    $g^k \not\in N$, and moreover, $g^k \not\in g^jN$ for any $j < k$ since $g$ generates $G$ so there is no element $g^k$ in 
    $g^jN$ for all $j < k < n$. This asserts that all of \[ N, gN, g^2N, \ldots, g^{n-1}N \] are all pairwise different. 

    Moreover, if $N$ is generated by $g^n$, then there must be exactly $n$ element of $G/N$ which are as listed above, since 
    any element of $g^\alpha \in G$ is in one of the partition of the $n$ cosets as $g^\alpha = g^{kn + a} \in g^aN$ for some 
    integer $k$ and $0 \le a < n$ by the division algorithm.

    Since those $n$ left cosets of $G$ is all of the left cosets, we get that $G/N = \set{N, gN, g^2N, \ldots, g^{n-1}N}$, and
    that $G/N$ is generated by $gN$, thus, $G/N$ is cyclic.
  } 

  \qs{}{
    Let $H < G$ be a subgroup of finite index. Show that the set $\set{gHg^{-1} \mid g \in G}$ is a finite set.
  }
  \sol{
    Consider that since $[G:H]$ is finite, then we can let $n = [G:H]$, which means that the quotient group
    $G/H = \set{g_1H, g_2H, \ldots, g_nH}$.
    Now, since $G/H$ partitions $G$ into $n$ partitions, we know that for $g \in G$, $gH = g_iH$ for some $g_i$. This means that 
    $g = g_ih$ for some $h \in H$. Thus, \[ gHg^{-1} = g_ihHh^{-1}g_i^{-1} = g_iHg_i^{-1} \] From the equation, 
    we know that $\set{gHg^{-1} \mid g \in G} = \set{g_iHg_i^{-1}}$ for previously defined $g_i$. Thus, 
   \[ \abs{\set{gHg^{-1}}} \le \abs{G/H} = [G:H] \] So, the set $\set{gHg^{-1} \mid g \in G}$ is a finite set.
  } 

  \qs{}{
    Let $N$ be a normal subgroup of a finite group $G$. Prove that $N$ is the unique subgroup of order $\abs{N}$ 
    if $gcd(\abs{N}, [G:N]) = 1$.
  }
  \sol{
    Let $K$ be a normal subgroup of $G$ with $n = \abs{K} = \abs{N}$. Then $KN$ is must be subgroup of $G$ since for 
    $KN = \set{ kn \mid k \in K, n \in N }$, we have that for $kn$ and $k'n'$ in $KN$, $kn(k'n')^{-1} = knn'^{-1}k'^{-1}$. 
    But since $N$ normal subgroup, $k'nn'^{-1}k'^{-1} \in N$, thus, $knn'^{-1}k'^{-1}$ is in $KN$. 

    Moreover, by the second isomorphism theorem, $\abs{KN} = \frac{\abs{K}\abs{N}}{\abs{K\cap N}} = \frac{n^2}{\abs{H\cap K}}$
    and $ \abs{KN} $ divides $\abs{G}$ by lagrange's theorem. Moreover, since $[G:N] = \frac{\abs{G}}{n}$, we can deduce that 
    $\abs{KN}$ must be $n$ since otherwise $\gcd(n, \abs{G}{n}) \ne 1$.

    Since $\abs{KN} = n$, we get that $\abs{K \cap N} = n$, thus, $K = N$. So, $N$ is the unique subgroup of order $\abs{N}$. 
  }

  \qs{}{
    Let $p$ be an odd prime integer. Show that $p \equiv  1 \pmod4$ 
    if and only if $x^2 \equiv -1 \pmod{p}$ has an integer solution.
  }
  \sol{
    \twoways{
      if $p \equiv 1 \pmod4$ then consider that $(\ZMod[p])^\times$ is a cyclic group of order $p-1$. 
      Thus, let $(\ZMod[p])^\times$ be generated by $\ang{r}$, then there exists an element, $x = r^{\frac{p-1}{4}}$ such that 
      the order of the element is 4. Hence, $x^4 \equiv 1 \pmod{p}$ but $x^2 \not\equiv 1 \pmod{p}$, 
      so $x^2 \equiv -1 \pmod{p}$
    }{
      if $x^2 \equiv -1 \pmod{p}$ for some integer $x$. Then $x \not\equiv 1 \pmod{p}$ and $x^3 \equiv -x \not\equiv 1 \pmod{p}$.
      Therefore, there exists an element $x \in (\ZMod[p])^\times$ that is of degree 4. Now, by lagrange's theorem, $4$ must 
      divides $\abs{(\ZMod[p])} = p-1$. So, it follows that $p \equiv 1 \pmod4$. 
    }
  } 

  \qs{}{
    Find all homomorphisms from $S_3$ to $\ZMod[3]$.
  }
  \sol{
    Note that we write the operator of both group using $\cdot$, abbrieviate using juxtaposition, and using $x^n$ to denotes 
    repeated operation even for $\ZMod[3]$.

    Let us consider $f$ as the homomorphism from $S_3$ ot $\ZMod[3]$. Trivially, we know that $f(id) = 0$.
    Next, let us consider $f((1 2)) = x$, then it must follows that $x^2 = 0$ since $(1 2)^2 = 0$. And as there is no element
    $x \ne 0$ in $ZMod[3]$ such that $x^2 = 0$. (As $1^2 = 2$ and $2^2 = 1$). So $f((1 2)) = 0$. Similar argument can be 
    applied for $(1 3)$ and $(2 3)$ yielding that $f((1 2)) = f((2 3)) = f((1 3)) = 0$. 

    Next, let us consider the remaining two element of $S_3$, namely $(1 2 3)$ and $(1 3 2)$. We know that $(1 2 3) = (1 3)(2 3)$
    and $(1 3 2) = (1 2)(1 3)$. So, $f((1 2 3)) = f((1 3)(2 3)) = 0 \cdot 0 = 0$, and similarly, $f((1 3 2)) = f((1 2)(1 3)) = 
    0 \cdot 0 = 0$. Thus, $f(x) = 0 \quad \forall x \in S_3$. is the only homomorphism from $S_3$ to $\ZMod[3]$.  
  } 

  \newcommand{\U}{\set{z \in \Comp \mid \abs{z} = 1}}
  \qs{}{
    Show that the quotient group $\Real/\Int$ is isomorphic to the unit circle $\U$.
  }
  \sol{
    Consider $\varphi$ to be a epimorphism from $\Real$ to the unit circle $\U$, with the following definition. 
    \[ \varphi: x \mapsto f(x) = \cos(2\pi x) + i\sin(2\pi x) \] 
    Now, we will show that $\varphi$ is well defined. Consider some $x = x'$ be a real number. Then it follows that
    $\cos(2\pi x) = \cos(2 \pi x')$ and $\sin(2\pi x) = \sin(2\pi x')$ by the definition of the $\cos$ and $\sin$ functions.
    Hence, $f(x) = f(x')$, so $\varphi$ is well-defined. 

    To show the surjectivity, let $z = (a + bi) \in \U$. Now, we know that $\abs{z} = \sqrt{a^2 + b^2} \ge a$, so $a \le 1$ 
    and $b \le 1$. Moreover, $1 - a^2 = b^2$. So, it is possible to find such $\theta$ for which $a = \cos\theta$. This will 
    ensure that $1 - a^2 = b^2 = 1 - \cos^2\theta = \sin^2\theta$. So, $z = (a + bi) = \cos\theta + i\sin\theta = f(\theta)$ 
    for some $\theta \in \Real$. Hence, $\varphi$ is surjective. 

    Lastly, consider the kernel $\ker\varphi = \set{x \mid f(x) = 1}$. Then we know that $f(x) = 1$ if and only if 
    $\cos(2\pi x) = 1$ and $\sin(2\pi x) = 0$. This occur if and only if $x \in \Int$ holds, by the periodicity of the 
    functions. Hence, $\ker\varphi = \Int$. 

    Therefore, by the first isomorphism theorem, we have that $\Real/\Int \isom \U$. 
  } 

  \qs{}{
    Find the class equation for $D_{2n}$ when $n \in \Int$ with $n \ge 3$.
  }
  \sol{
    Firstly, consider that an element $g$ of $D_{2n}$ can always be written in the form of $f^ir^j$ for $i \in \set{0, 1}$
    and $j \in \set{0, 1, \ldots, n-1}$. Now, consider the conjugacy classes of $D_{2n}$. 

    \begin{align*} 
      r^\alpha \cdot r^j    &= r^{\alpha + j} \\
                            &= r^j \cdot r^\alpha \\  
      r^\alpha \cdot fr^j   &= fr^{\alpha-j} \\ 
                            &= fr^j \cdot r^{\alpha - 2j} \\
      fr^\alpha \cdot r^j   &= fr^{\alpha + j} \\
                            &= r^{-\alpha-j}f \\ 
                            &= r^j \cdot r^{-\alpha-2j}f \\ 
                            &= r^j \cdot fr^{\alpha + 2j} \\
      fr^\alpha \cdot fr^j  &= ffr^{-\alpha}r^j \\ 
                            &= ffr^{-j}r^{-\alpha +2j} \\
                            &= fr^j \cdot fr^{-\alpha + 2j}
    \end{align*}

    From this, we split the problem into two cases of whether $n$ is odd or even. We first discuss the scenario where $n$ is odd.

    If $n$ is odd, then we can make a conjugacy class of $r^k$ as $\set{r^k, r^{-k}}$ since $\forall k, r^k \ne r^{-k}$ as $n$
    is odd. and the class containing $fr$ must contain $fr^{-1}$ and thus, contains $fr^j$ for all $0 \le j < n$.
    We could write the conjugacy classes of $D_{2n}$ as
    \[ \set{1},\; \set{r, r^{-1}},\; \set{r^2, r^{-2}},\; \ldots,\; \set{r^{\frac{n-1}{2}}, r^{\frac{n+1}{2}}},\; 
    \set{f, fr, \ldots, fr^{n-1}} \]

    Now, since all of the conjugacy classes, except for $\set{1}$ has 2 or more elements, the center of the group is $\set{1}$.

    Since the class equation is
    \[ \abs{D_{2n}} = \abs{Z(D_{2n}} + [D_{2n}:C_{D_{2n}}(x)] \] we need to find the centralizer of $x$ for $x$ in each class. 
    Firstly, consider $x = r^\alpha$, then the centralizer $C_{D_{2n}}(x)$ is any $r^j$ since they commute. But not $fr^j$ since 
    as shown above, $r^\alpha$and $fr^j$ only if $r^\alpha = r^{\alpha - 2j}$, which is when $j = 0$. 
    And the centralizer of $x = f$ contains just $f$ and $1$ since $f \cdot r^j = r^j \cdot fr^{2j}$ 
    and $f \cdot fr^j = fr^j \cdot fr^{2j}$. 
    Since the index $[D_{2n}:C_{D_{2n}}(x)] = \frac{2n}{\abs{C_{D_{2n}}(x)}}$, we get the following class equation. 
    \[ \abs{D_{2n}} = 1 + \frac{2n}{n} + \cdots + \frac{2n}{n} + \frac{2n}{2} \]
    Where there is exactly $\frac{n-1}{2}$ numbers of $\frac{2n}{n}$ as there is $\frac{n-1}{2}$ classes of $\set{r^i, r^{-i}}$, 
    the number 1 represents the size of the center, and the number $\frac{2n}{2}$ is the index of the class that contain $f$.

    Now, for the case that $n$ is even. we can deduce the conjugacy classes by similar arguments, $fr$ and $fr^{-1}$ is not 
    in the same conjugacy class. This is because $i$ and $n-i$ will always be the same parity since $n$ is even. This makes it 
    so that $fr^i$ and $fr^{n-i}$ is of different parity, thus is in different class. 
    As $fr$ and $fr^{-1}$ is not in the same class, the structure changed to
    \[ \set{1},\; \set{r, r^{-1}},\; \set{r^2, r^{-2}},\; \ldots,\; \set{r^{\frac{n}{2}}, r^{-\frac{n}{2}}},\; 
    \set{f, fr^2, \ldots, f^{n-2}},\; \set{fr, fr^3, \ldots, fr^{n-1}} \]

    Now, since $\frac{n}{2} = n - \frac{n}{2}$, we get another element in the center $Z(D_{2n})$. Then, let us consider the 
    centralizer of each representative of the classes. Firstly, notice that the centralizer of $r^j$ remains the same except
    for $r^{\frac{n}{2}}$ which is in the center. Now, for $C_{D_{2n}}(f)$, we know that $f$ commutes with $1$, $f$ and 
    $r^{\frac{n}{2}}$ and $fr^{\frac{n}{2}}$ from the above equations. 
    Since $f \cdot fr^{\frac{n}{2}} = fr^{\frac{n}{2}} \cdot fr^{-0 + 2\frac{n}{2}}$. Moreover, the equation above omit no 
    extra solution apart from this 4 solutions as $f = fr^{2j}$ only at $j = \frac{n}{2}$ or $j = 0$.
    Similarly, for the centralizer $C_{D_{2n}}(fr)$, we know that $fr$ commutes with $1$, $fr$, $r^{\frac{n}{2}}$ and 
    $fr^{\frac{n}{2}+1}$. And this are the only four solutions to the equation by similar arguments, which is that 
    $fr = fr^{-1 + 2j}$ solves only at $j = 1, \frac{n}{2}+1$, and $fr^{1 + 2j}$ solves only at $j = 0, \frac{n}{2}$. 

    This makes the class equation of this case to be 
    \[ \abs{D_{2n}} = 2 + \frac{2n}{n} + \cdots + \frac{2n}{n} + \frac{2n}{4} + \frac{2n}{4} \]
    Where the number 2 represents the size of the center, $\frac{n-2}{2}$ numbers of $\frac{2n}{n} = 2$ are the indices of each 
    of the classes in the form $\set{r^j, r^{-j}}$ from $j = 1, \ldots, n-2$, and two numbers of $\frac{2n}{4}$ are the indices 
    of the classes with $f$ and $fr$, ie. the class of $fr^j$ for even and odd $j$ respectively.  
  }

  \end{document}
