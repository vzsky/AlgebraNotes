% chktex-file 44
% chktex-file 8

\documentclass{report}
\usepackage{amsthm}
\usepackage{amsmath}
\usepackage{amssymb}
\usepackage{amssymb}
\usepackage{amsfonts}
\usepackage{xcolor}
\usepackage{tikz}
\usepackage{fancyhdr}
\usepackage{enumerate}
\usepackage{graphicx}
\usepackage[normalem]{ulem}
\usepackage[most,many,breakable]{tcolorbox}
\usepackage[a4paper, top=80pt, foot=25pt, bottom=50pt, left=0.5in, right=0.5in]{geometry}
\usepackage{hyperref, theoremref}
\hypersetup{
	pdftitle={Assignment},
	colorlinks=true, linkcolor=b!90,
	bookmarksnumbered=true,
	bookmarksopen=true
}
\usepackage{nameref}
\usepackage{parskip}
\pagestyle{fancy}

\usepackage[explicit,compact]{titlesec}
\titleformat{\chapter}[block]{\bfseries\huge}{\thechapter. }{\compact}{#1}
        

%%%%%%%%%%%%%%%%%%%%%
%% Defining colors %%
%%%%%%%%%%%%%%%%%%%%%

\definecolor{lr}{RGB}{188, 75, 81}
\definecolor{r}{RGB}{249, 65, 68}
\definecolor{dr}{RGB}{174, 32, 18}
\definecolor{lo}{RGB}{255, 172, 129}
\definecolor{do}{RGB}{202, 103, 2}
\definecolor{o}{RGB}{238, 155, 0}
\definecolor{ly}{RGB}{255, 241, 133}
\definecolor{y}{RGB}{255, 229, 31}
\definecolor{dy}{RGB}{143, 126, 0}
\definecolor{lb}{RGB}{148, 210, 189}
\definecolor{bg}{RGB}{10, 147, 150}
\definecolor{b}{RGB}{39, 125, 161}
\definecolor{db}{RGB}{0, 95, 115}
\definecolor{p}{RGB}{229, 152, 155}
\definecolor{dp}{RGB}{181, 101, 118}
\definecolor{pp}{RGB}{142, 143, 184}
\definecolor{v}{RGB}{109, 89, 122}
\definecolor{lg}{RGB}{144, 190, 109}
\definecolor{g}{RGB}{64, 145, 108}
\definecolor{dg}{RGB}{45, 106, 79}

\colorlet{mysol}{g}
\colorlet{mythm}{lr}
\colorlet{myqst}{db}
\colorlet{myclm}{lb}
\colorlet{mywrong}{r}
\colorlet{mylem}{o}
\colorlet{mydef}{lg}
\colorlet{mycor}{lb}
\colorlet{myrem}{dr}

%%%%%%%%%%%%%%%%%%%%%

\newcommand{\col}[2]{
  \color{#1}#2\color{black}\,
}

\newcommand{\TODO}[1][5cm]{
  \color{red}TODO\color{black}
  \vspace{#1}
}

\newcommand{\wans}[1]{
	\noindent\color{mywrong}\textbf{Wrong answer: }\color{black}
	#1 


}

\newcommand{\wreason}[1]{
	\noindent\color{mywrong}\textbf{Reason: }\color{black}
	#1 

  
}

\newcommand{\sol}[1]{
	\noindent\color{mysol}\textbf{Solution: }\color{black}
	#1


}

\newcommand{\nt}[1]{
  \begin{note}Note: #1\end{note}
}

\newcommand{\ky}[1]{
  \begin{key}#1\end{key}
}

\newcommand{\pf}[1]{
  \begin{myproof}#1\end{myproof}
}

\newcommand{\qs}[3][]{
  \begin{question}{#2}{#1}#3\end{question}
}

\newcommand{\df}[3][]{
  \begin{definition}{#2}{#1}#3\end{definition}
}

\newcommand{\thm}[3][]{
  \begin{theorem}{#2}{#1}#3\end{theorem}
}

\newcommand{\clm}[3][]{
  \begin{claim}{#2}{#1}#3\end{claim} 
}

\newcommand{\lem}[3][]{
  \begin{lemma}{#2}{#1}#3\end{lemma}
}

\newcommand{\cor}[3][]{
  \begin{corollary}{#2}{#1}#3\end{corollary}
}

\newcommand{\rem}[3][]{
  \begin{remark}{#2}{#1}#3\end{remark}
}

\newcommand{\twoways}[2]{
  \leavevmode\\
  ($\Longrightarrow$): 
  \begin{shift}#1\end{shift}
  ($\Longleftarrow$):
  \begin{shift}#2\end{shift} 
}

\newcommand{\nways}[2]{
  \leavevmode\\
  ($#1$): 
  \begin{shift}#2\end{shift}
}

%%%%%%%%%%%%%%%%%%%%%%%%%%%%%% ENVRN

\newenvironment{myproof}[1][\proofname]{%
	\proof[\bfseries #1: ]
}{\endproof}

\tcbuselibrary{theorems,skins,hooks}
\newtcolorbox{shift}
{%
  before upper={\setlength{\parskip}{5pt}},
  blanker,
	breakable,
	width=0.95\textwidth,
  enlarge left by=0.03\textwidth,
}

\tcbuselibrary{theorems,skins,hooks}
\newtcolorbox{key}
{%
	breakable,
	width=0.95\textwidth,
  enlarge left by=0.03\textwidth,
}

\tcbuselibrary{theorems,skins,hooks}
\newtcolorbox{note}
{%
	enhanced,
	breakable,
	colback = white,
	width=\textwidth,
	frame hidden,
	borderline west = {2pt}{0pt}{black},
	sharp corners,
}

\tcbuselibrary{theorems,skins,hooks}
\newtcbtheorem[]{remark}{Remark}
{%
	enhanced,
	breakable,
	colback = white,
	frame hidden,
	boxrule = 0sp,
	borderline west = {2pt}{0pt}{myrem},
	sharp corners,
	detach title,
  before upper={\setlength{\parskip}{5pt}\tcbtitle\par\smallskip},
	coltitle = myrem,
	fonttitle = \bfseries\sffamily,
	description font = \mdseries,
	separator sign none,
	segmentation style={solid, myrem},
}{rem}

\tcbuselibrary{theorems,skins,hooks}
\newtcbtheorem[number within=section]{lemma}{Lemma}
{%
	enhanced,
	breakable,
	colback = white,
	frame hidden,
	boxrule = 0sp,
	borderline west = {2pt}{0pt}{mylem},
	sharp corners,
	detach title,
  before upper={\setlength{\parskip}{5pt}\tcbtitle\par\smallskip},
	coltitle = mylem,
	fonttitle = \bfseries\sffamily,
	description font = \mdseries,
	separator sign none,
	segmentation style={solid, mylem},
}{lem}

\tcbuselibrary{theorems,skins,hooks}
\newtcbtheorem{claim}{Claim}
{%
  parbox=false,
	enhanced,
	breakable,
	colback = white,
	frame hidden,
	boxrule = 0sp,
	borderline west = {2pt}{0pt}{myclm},
	sharp corners,
	detach title,
  before upper={\setlength{\parskip}{5pt}\tcbtitle\par\smallskip},
	coltitle = myclm,
	fonttitle = \bfseries\sffamily,
	description font = \mdseries,
	separator sign none,
	segmentation style={solid, myclm},
}{clm}

\makeatletter
\newtcbtheorem[number within=section, use counter from=lemma]{theorem}{Theorem}{enhanced,
	breakable,
	colback=white,
	colframe=mythm,
	attach boxed title to top left={yshift*=-\tcboxedtitleheight},
	fonttitle=\bfseries,
	title={#2},
	boxed title size=title,
	boxed title style={%
			sharp corners,
			rounded corners=northwest,
			colback=mythm,
			boxrule=0pt,
		},
	underlay boxed title={%
			\path[fill=mythm] (title.south west)--(title.south east)
			to[out=0, in=180] ([xshift=5mm]title.east)--
			(title.center-|frame.east)
			[rounded corners=\kvtcb@arc] |-
			(frame.north) -| cycle;
		},
	#1
}{thm}
\makeatother

\makeatletter
\newtcbtheorem{question}{Question}{enhanced,
	breakable,
	colback=white,
	colframe=myqst,
	attach boxed title to top left={yshift*=-\tcboxedtitleheight},
	fonttitle=\bfseries,
	title={#2},
	boxed title size=title,
	boxed title style={%
			sharp corners,
			rounded corners=northwest,
			colback=myqst,
			boxrule=0pt,
		},
	underlay boxed title={%
			\path[fill=myqst] (title.south west)--(title.south east)
			to[out=0, in=180] ([xshift=5mm]title.east)--
			(title.center-|frame.east)
			[rounded corners=\kvtcb@arc] |-
			(frame.north) -| cycle;
		},
	#1
}{qs}
\makeatother

\makeatletter
\newtcbtheorem[number within=section]{definition}{Definition}{enhanced,
	breakable,
	colback=white,
	colframe=mydef,
	attach boxed title to top left={yshift*=-\tcboxedtitleheight},
	fonttitle=\bfseries,
	title={#2},
	boxed title size=title,
	boxed title style={%
			sharp corners,
			rounded corners=northwest,
			colback=mydef,
			boxrule=0pt,
		},
	underlay boxed title={%
			\path[fill=mydef] (title.south west)--(title.south east)
			to[out=0, in=180] ([xshift=5mm]title.east)--
			(title.center-|frame.east)
			[rounded corners=\kvtcb@arc] |-
			(frame.north) -| cycle;
		},
	#1
}{def}
\makeatother

\makeatletter
\newtcbtheorem[number within=section, use counter from=lemma]{corollary}{Corollary}{enhanced,
	breakable,
	colback=white,
	colframe=mycor,
	attach boxed title to top left={yshift*=-\tcboxedtitleheight},
	fonttitle=\bfseries,
	title={#2},
	boxed title size=title,
	boxed title style={%
			sharp corners,
			rounded corners=northwest,
			colback=mycor,
			boxrule=0pt,
		},
	underlay boxed title={%
			\path[fill=mycor] (title.south west)--(title.south east)
			to[out=0, in=180] ([xshift=5mm]title.east)--
			(title.center-|frame.east)
			[rounded corners=\kvtcb@arc] |-
			(frame.north) -| cycle;
		},
	#1
}{cor}
\makeatother

% Basic
  \DeclareMathOperator{\lcm}{lcm}
  \newcommand{\Real}{\mathbb{R}}
  \newcommand{\Comp}{\mathbb{C}}
  \newcommand{\Nat}{\mathbb{N}}
  \newcommand{\Rat}{\mathbb{Q}}
  \newcommand{\Int}{\mathbb{Z}}
  \newcommand{\set}[1]{\left\{\, #1 \,\right\}}
  \newcommand{\paren}[1]{\left( \; #1 \; \right)}
  \newcommand{\abs}[1]{\left\lvert #1 \right\rvert}
  \newcommand{\ang}[1]{\left\langle #1 \right\rangle}
  \renewcommand{\to}[1][]{\xrightarrow{\text{#1}}}
  \newcommand{\tol}[1][]{\to{$#1$}}
  \newcommand{\curle}{\preccurlyeq}
  \newcommand{\curge}{\succcurlyeq}
  \newcommand{\mapsfrom}{\leftarrow\!\shortmid}

  \newcommand{\mat}[1]{\begin{bmatrix} #1 \end{bmatrix}}
  \newcommand{\pmat}[1]{\begin{pmatrix} #1 \end{pmatrix}}
  \newcommand{\eqs}[1]{\begin{align*} #1 \end{align*}}
  \newcommand{\case}[1]{\begin{cases} #1 \end{cases}}
  

  % Algebra
  \newcommand{\normSg}[0]{\vartriangleleft}
  \newcommand{\ZMod}[1][n]{\mathbb{Z}/#1\mathbb{Z}}
  \newcommand{\isom}{\simeq}
  \newcommand{\mapHom}{\xrightarrow{\text{hom}}}
  \DeclareMathOperator{\Inn}{Inn}
  \DeclareMathOperator{\Aut}{Aut}
  \DeclareMathOperator{\im}{im}
  \DeclareMathOperator{\ord}{ord}
  \DeclareMathOperator{\Gal}{Gal}
  \DeclareMathOperator{\chr}{char}
  \newcommand{\surjto}{\twoheadrightarrow}
  \newcommand{\injto}{\hookrightarrow}

  % Analysis 
  \newcommand{\limty}[1][k]{\lim_{#1\to\infty}}
  \newcommand{\norm}[1]{\left\lVert#1\right\rVert}
  \newcommand{\darrow}{\rightrightarrows}


\fancyhead[L]{HW 4 - Modern Algebra MAS311}
\fancyhead[R]{\textbf{Touch Sungkawichai} 20210821}

\begin{document}
  \qs{}{
    Let $K < H < G$ be subgroups. Show that $[G:K] = [G:H][H:K]$
  } 
  \sol{
    Assuming that $[G:H]$ and $[H:K]$ is finite.
    By definition, $[H:K]$ is the number of left coset of $K$ in $H$, so let $n = [G:H]$, 
    we know that $\set{g_1H, g_2H, \ldots, g_nH}$ forms a partition of $G$. and let $m = [H:K]$
    so that $\set{h_1K, h_2K, \ldots, h_mK}$ forms a partition of $H$.

    \newcommand{\dju}{\; \bar\cup \;}
    Now, let $\dju$ denote the operation union, which an assertion that the set is disjoint.
    Then, $g_i(h_1K \dju h_2K \dju \cdots \dju h_mK)$ a left coset of $H$ in $G$. Moreover, for 
    each value of $i$, we must get a disjoint set, since it is a member of the partition of $G$ mentioned above.
    Therefore, \[g_1(h_1K \dju h_2K \dju \cdots \dju h_mK) \dju \cdots \dju 
    g_n(h_1K \dju h_2K \dju \cdots \dju h_mK) = G\] and by distributing the $g_i$ inside of the parentheses, 
    a partition of $G$ into cosets of $K$ is created. 

    Note: Cconsider any $g_ih_jK$ and $g_{i'}h_{j'}$. If $i \ne i'$ that $g_ih_jK$ is in a partition $g_i$ of $G$ 
    but $g_{i'}h_{j'}K$ cannot be in the same coset in $G$, as it would contradict that the set of $\set{g_iH}$ 
    forms a partition.
    Similarly, if $j \ne j'$ then $g_ih_jK$ would be in the partition $h_jK$ of $g_iH$ but $g_{i'}h_{j'}K$ cannot be in the 
    same coset even if $g_{i'}H = g_iH$ as it would contradict that $\set{h_j}$ partitions $H$.
    So $g_ih_jK \cap g_{i'}h_{j'}K = \emptyset$ whenever $(i, j) \ne (i', j')$

    Since the parition has $nm$ elements, $[G:K] = nm = [G:H][H:K]$ by definition.

    Otherwise, if $[G:H]$ or $[H:K]$ is infinite, then there $G$ should be partition into an infinite number of partitions by $K$
    Hence, $[G:K]$ should be infinite, and $[G:K] = [G:H][H:K]$ still holds, taking that multiplication with infinite number 
    returns infinite number.
  } 

  \qs{}{
    Assume that both $H$ and $K$ have finite index in $G$. Prove that $H \cap K$ has finite index in $G$.
  }
  \sol{
    Firstly, if $H$ and $K$ have finite index in $G$, then denote the index of $H$ as $n$ and the index of $K$ as $m$.
    Then, there is a partition of $G$ into disjoint $\set{g_1H, g_2H, \ldots, g_nH}$
    and a partition of $G$ into dishoint $\set{g'_1K, g'_2K, \ldots, g'_mK}$. 

    Consider if $gK = K$ and $gH = H$, then $g \in K$ and $g \in H$, which means $g \in H \cap K$. 
    Now, \[\set{g_1, \ldots, g_n}(H \cap K) = \set{g_1 \ldots, g_n}H \cap \set{g_1, \ldots, g_n}K = \set{g_1, \ldots, g_n}K \]
    Since it is possible to choose $g_1$ so that $g_1 \in H \cap K$ without loss of generality. 
    So $\set{g_1, \ldots, g_n}(H \cap K) = \set{g_1, \ldots, g_n}K$ but $g_1 \in K$. 
    So $\set{g_1(H \cap K), \ldots, g_n(H \cap K)}$ must cover a partition of $K$ into cosets of $H \cap K$ in the sense that
    it might cover some other cosets of $K$.
    However, the size of the set of coset of $(H \cap K)$ that partitions $K$ must be less than or equal to $n$, hence finite.

    Now, since $[K: H\cap K]$ is finite, $[G:K]$ is finite, and $K \cap H < K < G$ as subgroup. Then the problem 1 asserts 
    that $[G:H \cap K]$ is finite.
  } 

  \qs{}{
    Show that $S_n = \ang{(1 \; 2), (1 \; 2 \; 3 \cdots n)}$ for all $n \ge 2$
  }
  \sol{
    For $n = 2$, $S_2 = \ang{(1 \; 2)}$ trivially, as $(1 \; 2)^2 = id$.

    Now, for $n > 2$, notice that $(1 \; 2 \; 3 \cdots n)^{k-1} = 
    \begin{pmatrix}
      1 & 2   & \cdots & k-1 & k & \cdots & n
      \\ k & k+1 & \cdots & n & 1 & \cdots & k-1
    \end{pmatrix}$
    and that 
    \begin{align*} 
      (1 \; 2 \cdots n)^{-k}(1 2)(1 \; 2 \cdots n)^k 
      & = (1 \; 2 \cdots n)^{-k} 
      \begin{pmatrix}
        1 & 2   & \cdots & k-1 & k & k+1 &  \cdots & n
        \\ k & k+1 & \cdots & n & 2 & 1 &\cdots & k-1
      \end{pmatrix} \\ &= 
      \begin{pmatrix}
        1 & 2 & \cdots & k & k+1 & \cdots & n
        \\ 1 & 2 & \cdots & k+1 & k & \cdots & n
      \end{pmatrix} \\ &= (k \; k+1) 
    \end{align*}

    Now, A transposition $(a \; b)$ is the product $(b-1 \; b)\cdots(a+1 \; a+2)(a \; a+1)(a+1 \; a+2)\cdots(b-1 \; b)$
    Therefore, $(a \; b)$ for any $a, b$ is in the group $\ang{(1 \; 2), (1 \;l 2 \cdots n)}$

    Now, since $S_n$ is generated by transposition, $S_n = \ang{(a \; b) \mid \forall a, b}$, 
    then $S_n$ is generated by $\ang{(1 \; 2), (1 \; 2 \cdots n)}$
  } 

  \qs{}{
    Let $G$ be a group of order $pq$, where $p$ and $q$ are primes. Prove that every proper subgroup of $G$ is cyclic.
  }
  \sol{
    Firstly, we know that $G$ is not trivial, as if it is the case, the order of $G$ will be $1 \ne pq$ for any prime $p, q$.
    Let $S$ be a subgroup of $G$, then by lagrange's theorem, $\abs{G} = [G:S]\abs{S}$, 
    which means that $\abs{S} = 1$, $p$, or $q$ must hold. If $\abs{S} = 1$, then $S$ is trivially cyclic. Otherwise, 
    An assumption that $\abs{S} = p$ can be made without loss of generality. 

    Consider an element $s \in S$ such that $s$ is not the identity. 
    Now, $\ang{s} < S$ and $\abs{\ang{s}} > 1$, so $\abs{\ang{s}}$ must divides $\abs{S}$ by the lagrange's theorem. 
    However, since $\abs{S} = p$, then $\abs{\ang{s}}$ must be $p$. 

    Since $\abs{S} = \abs{\ang{s}}$ with $\ang{s} \le S$, then, $S = \ang{s}$ is generated by one element, hence cyclic.
  } 

  \qs{}{
    Find a homomorphism $\phi: G \to H$ such that the image of $\phi$ is not normal in $H$.
  }
  \sol{
    Let $H = D_6 = \set{1, r, r^2, f, fr, fr^2}$. Then, consider $\phi: G = \ZMod[2] = \set{\bar0, \bar1} \to \set{1, f}$ 
    be a homomorphism where
    \begin{align*}
      \phi(\bar0) &= 1  \\
      \phi(\bar1) &= f
    \end{align*}
    Then $\phi$ is a homomorphism since $\bar0$ is the identity of $G$ and $1$ is the identity of $H$. 
    and $\phi(\bar1 + \bar1) = \phi(\bar0) = 1 = f \cdot f = \phi(\bar1) \cdot \phi(\bar1)$
    Then, $\im\phi = \set{1, f}$ and $rfr^{-1} = r^2f \not\in \set{1, f}$. This asserts that $\im\phi$ is not normal in $H$.
  }

  \qs{}{
    Show that every subgroup of index 2 is normal.
  }
  \sol{
    Let $S$ be a subgroup of $G$ with index 2. Then the coset $\set{S, gS}$ partitions $G$ for some element $g \in G$.
    Note that $g \not\in S$, since otherwise, $gS = S$ contradicts that $S$ has index 2.

    Now, consider the right coset $Sg$. We know that $g \not\in S$. By assuming that $\exists s \in S, sg \in S$, we get
    $s^{-1}sg = g \in S$ by closure of $S$ since $sg$ and $s^{-1}$ are both in $S$. Therefore, $\forall s \in S, sg \not\in S$
    Since $\set{S, gS}$ partitions $G$, and $Sg \ne S$ with $Sg \subset G$. The only possibility is that $Sg = gS$. Thus, 
    $S$ must be normal.
  } 

  \qs{}{
    Let $\phi: G \to G'$ be an epimorphism and $H \normSg G'$ a normal subgroup. Prove that $\phi^{-1}(H) \normSg G$ and
    $(G/\phi^{-1}(H)) \isom G'/H$.
  }
  \sol{
    Consider $\varphi: G' \to G'/H$ such that $\varphi: g' \mapsto g'H$, then the homomorphism is an epimorphism since 
    for every $g'H \in G'/H$, there is such $g'$ that $\varphi(g') = g'H$. Consider 
    $\ker\varphi = \set { h \mid \varphi(h) = H } = \set{ h \mid h \in H } = H$. 
    This means that $\varphi \circ \phi: G \to G'/H$ with $\varphi \circ \phi: g \mapsto \phi(g)H$ is an epimorphism with 
    \[ \ker(\varphi \circ \phi) = \set{h \mid \phi(h) \in H} = \phi^{-1}(H) \]  
    Then by the first isomorphism theorem, $G'/\ker(\varphi\circ\phi) \isom \im(\varphi\circ\phi)$,
    Hence $G/\phi^{-1}(H) \isom G/H$ 
  }

  \qs{}{
    Let $m$ be a positive integer. Show that the map $\phi: \Rat/\Int \to \Rat/\Int$ given by $q + \Int \mapsto mq + \Int$
    is a homomorphism and find $\ker(\phi)$
  }
  \sol{
    Consider the given definition of $\phi$. Let $[q]$ denotes the equivalent class of $q + \Int$ in $\Rat/\Int$. 
    In other words, $[q] = q+n$ for some integer $n$. 
    Then for $[q] = [q']$, \[\exists n \quad  \phi([q]) = [mq] = [mq + mn] = [mq'] = \phi([q']) \]
    asserts that $\phi$ is well defined.

    Now, consider \[ \phi([a] + [b]) = \phi([a + b]) = m[a+b] = [ma + mb] = [ma] + [mb] = \phi([a]) + \phi([b]) \]
    With the above equation, $\phi([a] + [b]) = \phi([a]) + \phi([b])$, so $\phi$ is a homomorphism. 

    Lastly, consider $\ker\phi = \set{[q] \mid \phi([q]) = [0]}$. It follows that $\ker\phi = \set{[q] \mid [mq] = 0}$.
    However, $[mq] = 0$ is the same as $mq \in \Int$, and $q = \frac{a}{b}$ for some $a, b \in \Int$. Now, $mq \in \Int$ means
    that $m\frac{a}{b} \in \Int$, which is true whenever $b \vert m$.

    So $\ker\phi = \set{[\frac{a}{b}] \mid b \vert m, a \in \Int}$. Note that for $a \ge b$, $[\frac{a}{b}] = [\frac{a-b}{b}]$. 
    And similarly for $a \le -b$.
    Therefore, $\ker\phi = \set{[\frac{a}{b}] \mid a, b \in \Int, b \vert m, -b < a < b}$
  } 

  \qs{}{
    Let $K \normSg G$ and $H \normSg G'$ be a normal subgroups. Show that $K \times H \normSg G \times G'$ and 
    $(G \times G')/(K \times H) \isom (G/K)\times(G'/H)$.
  }
  \sol{
    Firstly, for any any $(g, g') \in G \times G'$, 
    \[ (g, g') K \times H = (g, g') \set{(k, h) \mid k \in K, h \in H} = gK \times g'H = Kg \times Hg' = K \times H (g, g')\]
    Therefore, $K \times H$ is a normal subgroup of $G \times G'$

    Now, consider $\varphi$, a homomorphism from $G \times G'$ to $(G/K) \times (G'/H)$ where 
    $\varphi: (g, g') \mapsto (gK, g'H)$.
    Then, since $K$ and $H$ are normal subgroups, it is clear that $\varphi$ is surjective,
    as $\set{gK}$ partitions $G$, so $\varphi_G: g \mapsto gK$ is surjective, and similar for $H$. 

    For $\ker\varphi$, consider that $\varphi((g, g')) = e$ means that $gK = e$ and $g'H = e$. This situation happens when 
    $g \in K$ and $g' \in H$, hence, when $(g, g') \in K \times H$. 

    Therefore, by the first isomorphism theorem, $(G \times G')/\ker\varphi \isom \im\varphi$, which in this case is 
    $(G \times G')/(K \times H) \isom (G/K) \times (G'/H)$
  } 

  \qs{}{
    Let $G$ be a group of order $p^2$, where $p$ is a prime. Show that either $G$ is cyclic or 
    every nontrivial element of $G$ has order $p$.
  }
  \sol{
    For a group $G$ of order $p^2$, if not every nontrivial element of $G$ has order $p$, then there must 
    be some element $g$ that the order of $g$ is not $p$ and not $1$. 
    \clm[orddivgord]{
      For a finite group $G$, every element $g \in G$, $\abs{g}$ divides $\abs{G}$. 
      \pf{
        For an element $g$ with order $n$, $\ang{g}$ is a subgroup of $G$. 
        Moreover, $\abs{\ang{g}} = n$ since for $1 \le i, j \le n$ if $i \ne j$ then $g^i \ne g^j$. 
        This follows from the fact that if $g^i = g^j$ then $g^i-j = g^0$, thus $i-j = 0$ or $i-j \vert n$. 
        However, since $\ang{g}$ is a subgroup of $G$, then $n = \abs{\ang{g}}$ divides $\abs{G}$ by lagrange's theorem.
        Hence, the order of $g$ divides $\abs{G}$.
      }
    }
    However, by claim~\ref{clm:orddivgord}, $\abs{g} \vert \abs{G}$, so $\abs{g} = 1$, $p$, or $p^2$ must hold.
    From the assumption, there must be an element $g$ such that $g \ne p$, and $g \ne 1$. So $\exists g, \; \abs{g} = p^2$. 
    Since there is such element $g$, then $\abs{\ang{g}} = p^2$, so $G$ must be generated by $g$. Hence, $G$ is cyclic.

    Therefore, a group $G$ is either cyclic, or the assumption that not all nontrivial elements are order $p$ does not hold.
    Which is translated naturally into the question statement.
    which means that $\ang{g} = G$.
  }

\end{document}
