% chktex-file 44
% chktex-file 8
\documentclass{report}
\usepackage{amsthm}
\usepackage{amsmath}
\usepackage{amssymb}
\usepackage{amssymb}
\usepackage{amsfonts}
\usepackage{xcolor}
\usepackage{tikz}
\usepackage{fancyhdr}
\usepackage{enumerate}
\usepackage{graphicx}
\usepackage[normalem]{ulem}
\usepackage[most,many,breakable]{tcolorbox}
\usepackage[a4paper, top=80pt, foot=25pt, bottom=50pt, left=0.5in, right=0.5in]{geometry}
\usepackage{hyperref, theoremref}
\hypersetup{
	pdftitle={Assignment},
	colorlinks=true, linkcolor=b!90,
	bookmarksnumbered=true,
	bookmarksopen=true
}
\usepackage{nameref}
\usepackage{parskip}
\pagestyle{fancy}

\usepackage[explicit,compact]{titlesec}
\titleformat{\chapter}[block]{\bfseries\huge}{\thechapter. }{\compact}{#1}
        

%%%%%%%%%%%%%%%%%%%%%
%% Defining colors %%
%%%%%%%%%%%%%%%%%%%%%

\definecolor{lr}{RGB}{188, 75, 81}
\definecolor{r}{RGB}{249, 65, 68}
\definecolor{dr}{RGB}{174, 32, 18}
\definecolor{lo}{RGB}{255, 172, 129}
\definecolor{do}{RGB}{202, 103, 2}
\definecolor{o}{RGB}{238, 155, 0}
\definecolor{ly}{RGB}{255, 241, 133}
\definecolor{y}{RGB}{255, 229, 31}
\definecolor{dy}{RGB}{143, 126, 0}
\definecolor{lb}{RGB}{148, 210, 189}
\definecolor{bg}{RGB}{10, 147, 150}
\definecolor{b}{RGB}{39, 125, 161}
\definecolor{db}{RGB}{0, 95, 115}
\definecolor{p}{RGB}{229, 152, 155}
\definecolor{dp}{RGB}{181, 101, 118}
\definecolor{pp}{RGB}{142, 143, 184}
\definecolor{v}{RGB}{109, 89, 122}
\definecolor{lg}{RGB}{144, 190, 109}
\definecolor{g}{RGB}{64, 145, 108}
\definecolor{dg}{RGB}{45, 106, 79}

\colorlet{mysol}{g}
\colorlet{mythm}{lr}
\colorlet{myqst}{db}
\colorlet{myclm}{lb}
\colorlet{mywrong}{r}
\colorlet{mylem}{o}
\colorlet{mydef}{lg}
\colorlet{mycor}{lb}
\colorlet{myrem}{dr}

%%%%%%%%%%%%%%%%%%%%%

\newcommand{\col}[2]{
  \color{#1}#2\color{black}\,
}

\newcommand{\TODO}[1][5cm]{
  \color{red}TODO\color{black}
  \vspace{#1}
}

\newcommand{\wans}[1]{
	\noindent\color{mywrong}\textbf{Wrong answer: }\color{black}
	#1 


}

\newcommand{\wreason}[1]{
	\noindent\color{mywrong}\textbf{Reason: }\color{black}
	#1 

  
}

\newcommand{\sol}[1]{
	\noindent\color{mysol}\textbf{Solution: }\color{black}
	#1


}

\newcommand{\nt}[1]{
  \begin{note}Note: #1\end{note}
}

\newcommand{\ky}[1]{
  \begin{key}#1\end{key}
}

\newcommand{\pf}[1]{
  \begin{myproof}#1\end{myproof}
}

\newcommand{\qs}[3][]{
  \begin{question}{#2}{#1}#3\end{question}
}

\newcommand{\df}[3][]{
  \begin{definition}{#2}{#1}#3\end{definition}
}

\newcommand{\thm}[3][]{
  \begin{theorem}{#2}{#1}#3\end{theorem}
}

\newcommand{\clm}[3][]{
  \begin{claim}{#2}{#1}#3\end{claim} 
}

\newcommand{\lem}[3][]{
  \begin{lemma}{#2}{#1}#3\end{lemma}
}

\newcommand{\cor}[3][]{
  \begin{corollary}{#2}{#1}#3\end{corollary}
}

\newcommand{\rem}[3][]{
  \begin{remark}{#2}{#1}#3\end{remark}
}

\newcommand{\twoways}[2]{
  \leavevmode\\
  ($\Longrightarrow$): 
  \begin{shift}#1\end{shift}
  ($\Longleftarrow$):
  \begin{shift}#2\end{shift} 
}

\newcommand{\nways}[2]{
  \leavevmode\\
  ($#1$): 
  \begin{shift}#2\end{shift}
}

%%%%%%%%%%%%%%%%%%%%%%%%%%%%%% ENVRN

\newenvironment{myproof}[1][\proofname]{%
	\proof[\bfseries #1: ]
}{\endproof}

\tcbuselibrary{theorems,skins,hooks}
\newtcolorbox{shift}
{%
  before upper={\setlength{\parskip}{5pt}},
  blanker,
	breakable,
	width=0.95\textwidth,
  enlarge left by=0.03\textwidth,
}

\tcbuselibrary{theorems,skins,hooks}
\newtcolorbox{key}
{%
	breakable,
	width=0.95\textwidth,
  enlarge left by=0.03\textwidth,
}

\tcbuselibrary{theorems,skins,hooks}
\newtcolorbox{note}
{%
	enhanced,
	breakable,
	colback = white,
	width=\textwidth,
	frame hidden,
	borderline west = {2pt}{0pt}{black},
	sharp corners,
}

\tcbuselibrary{theorems,skins,hooks}
\newtcbtheorem[]{remark}{Remark}
{%
	enhanced,
	breakable,
	colback = white,
	frame hidden,
	boxrule = 0sp,
	borderline west = {2pt}{0pt}{myrem},
	sharp corners,
	detach title,
  before upper={\setlength{\parskip}{5pt}\tcbtitle\par\smallskip},
	coltitle = myrem,
	fonttitle = \bfseries\sffamily,
	description font = \mdseries,
	separator sign none,
	segmentation style={solid, myrem},
}{rem}

\tcbuselibrary{theorems,skins,hooks}
\newtcbtheorem[number within=section]{lemma}{Lemma}
{%
	enhanced,
	breakable,
	colback = white,
	frame hidden,
	boxrule = 0sp,
	borderline west = {2pt}{0pt}{mylem},
	sharp corners,
	detach title,
  before upper={\setlength{\parskip}{5pt}\tcbtitle\par\smallskip},
	coltitle = mylem,
	fonttitle = \bfseries\sffamily,
	description font = \mdseries,
	separator sign none,
	segmentation style={solid, mylem},
}{lem}

\tcbuselibrary{theorems,skins,hooks}
\newtcbtheorem{claim}{Claim}
{%
  parbox=false,
	enhanced,
	breakable,
	colback = white,
	frame hidden,
	boxrule = 0sp,
	borderline west = {2pt}{0pt}{myclm},
	sharp corners,
	detach title,
  before upper={\setlength{\parskip}{5pt}\tcbtitle\par\smallskip},
	coltitle = myclm,
	fonttitle = \bfseries\sffamily,
	description font = \mdseries,
	separator sign none,
	segmentation style={solid, myclm},
}{clm}

\makeatletter
\newtcbtheorem[number within=section, use counter from=lemma]{theorem}{Theorem}{enhanced,
	breakable,
	colback=white,
	colframe=mythm,
	attach boxed title to top left={yshift*=-\tcboxedtitleheight},
	fonttitle=\bfseries,
	title={#2},
	boxed title size=title,
	boxed title style={%
			sharp corners,
			rounded corners=northwest,
			colback=mythm,
			boxrule=0pt,
		},
	underlay boxed title={%
			\path[fill=mythm] (title.south west)--(title.south east)
			to[out=0, in=180] ([xshift=5mm]title.east)--
			(title.center-|frame.east)
			[rounded corners=\kvtcb@arc] |-
			(frame.north) -| cycle;
		},
	#1
}{thm}
\makeatother

\makeatletter
\newtcbtheorem{question}{Question}{enhanced,
	breakable,
	colback=white,
	colframe=myqst,
	attach boxed title to top left={yshift*=-\tcboxedtitleheight},
	fonttitle=\bfseries,
	title={#2},
	boxed title size=title,
	boxed title style={%
			sharp corners,
			rounded corners=northwest,
			colback=myqst,
			boxrule=0pt,
		},
	underlay boxed title={%
			\path[fill=myqst] (title.south west)--(title.south east)
			to[out=0, in=180] ([xshift=5mm]title.east)--
			(title.center-|frame.east)
			[rounded corners=\kvtcb@arc] |-
			(frame.north) -| cycle;
		},
	#1
}{qs}
\makeatother

\makeatletter
\newtcbtheorem[number within=section]{definition}{Definition}{enhanced,
	breakable,
	colback=white,
	colframe=mydef,
	attach boxed title to top left={yshift*=-\tcboxedtitleheight},
	fonttitle=\bfseries,
	title={#2},
	boxed title size=title,
	boxed title style={%
			sharp corners,
			rounded corners=northwest,
			colback=mydef,
			boxrule=0pt,
		},
	underlay boxed title={%
			\path[fill=mydef] (title.south west)--(title.south east)
			to[out=0, in=180] ([xshift=5mm]title.east)--
			(title.center-|frame.east)
			[rounded corners=\kvtcb@arc] |-
			(frame.north) -| cycle;
		},
	#1
}{def}
\makeatother

\makeatletter
\newtcbtheorem[number within=section, use counter from=lemma]{corollary}{Corollary}{enhanced,
	breakable,
	colback=white,
	colframe=mycor,
	attach boxed title to top left={yshift*=-\tcboxedtitleheight},
	fonttitle=\bfseries,
	title={#2},
	boxed title size=title,
	boxed title style={%
			sharp corners,
			rounded corners=northwest,
			colback=mycor,
			boxrule=0pt,
		},
	underlay boxed title={%
			\path[fill=mycor] (title.south west)--(title.south east)
			to[out=0, in=180] ([xshift=5mm]title.east)--
			(title.center-|frame.east)
			[rounded corners=\kvtcb@arc] |-
			(frame.north) -| cycle;
		},
	#1
}{cor}
\makeatother

% Basic
  \DeclareMathOperator{\lcm}{lcm}
  \newcommand{\Real}{\mathbb{R}}
  \newcommand{\Comp}{\mathbb{C}}
  \newcommand{\Nat}{\mathbb{N}}
  \newcommand{\Rat}{\mathbb{Q}}
  \newcommand{\Int}{\mathbb{Z}}
  \newcommand{\set}[1]{\left\{\, #1 \,\right\}}
  \newcommand{\paren}[1]{\left( \; #1 \; \right)}
  \newcommand{\abs}[1]{\left\lvert #1 \right\rvert}
  \newcommand{\ang}[1]{\left\langle #1 \right\rangle}
  \renewcommand{\to}[1][]{\xrightarrow{\text{#1}}}
  \newcommand{\tol}[1][]{\to{$#1$}}
  \newcommand{\curle}{\preccurlyeq}
  \newcommand{\curge}{\succcurlyeq}
  \newcommand{\mapsfrom}{\leftarrow\!\shortmid}

  \newcommand{\mat}[1]{\begin{bmatrix} #1 \end{bmatrix}}
  \newcommand{\pmat}[1]{\begin{pmatrix} #1 \end{pmatrix}}
  \newcommand{\eqs}[1]{\begin{align*} #1 \end{align*}}
  \newcommand{\case}[1]{\begin{cases} #1 \end{cases}}
  

  % Algebra
  \newcommand{\normSg}[0]{\vartriangleleft}
  \newcommand{\ZMod}[1][n]{\mathbb{Z}/#1\mathbb{Z}}
  \newcommand{\isom}{\simeq}
  \newcommand{\mapHom}{\xrightarrow{\text{hom}}}
  \DeclareMathOperator{\Inn}{Inn}
  \DeclareMathOperator{\Aut}{Aut}
  \DeclareMathOperator{\im}{im}
  \DeclareMathOperator{\ord}{ord}
  \DeclareMathOperator{\Gal}{Gal}
  \DeclareMathOperator{\chr}{char}
  \newcommand{\surjto}{\twoheadrightarrow}
  \newcommand{\injto}{\hookrightarrow}

  % Analysis 
  \newcommand{\limty}[1][k]{\lim_{#1\to\infty}}
  \newcommand{\norm}[1]{\left\lVert#1\right\rVert}
  \newcommand{\darrow}{\rightrightarrows}


\fancyhead[L]{HW 2 - Modern Algebra MAS311}
\fancyhead[R]{\textbf{Touch Sungkawichai} 20210821}

\begin{document}
  \qs{}{
    Let $H$ and $K$ be subgroups of $G$. 
    Prove that $H \cup K$ is a subgroup if and only if $H \subset K$ or $K \subset H$.
  }
  \sol{
    Notice that if $H \subset K$ then $H \cup K = K$ or if $K \subset H$ then $H \cup K = H$
    \\
    $(\Longrightarrow)$~: 
      \begin{shift}
        Assume that $H \not\subset K$ and $K \not\subset H$, where $H$ and $K$ are subgroup of $G$, 
        Then, find an element $h \in H$ such that $h \notin K$ and $k \in K$ such that $k \notin H$.
        It is clear that the elements exists as both $H$ and $K$ are not a subset of the other one.

        Consider $h \cdot k$, if $h \cdot k = \alpha$ for some $\alpha \in H$, 
        then we get that $h\cdot k = \alpha = hh^{-1}\alpha = h(h^{-1}\alpha)$ which means that $k = h^{-1}\alpha \in H$.
        However, it contradicts with the assumption, so $h \cdot k \notin H$

        Consider in the same manner that $h \cdot k = \beta$ for some $\beta \in K$,
        then we get that $h \cdot k = \beta = \beta k^{-1} k = (\beta k^{-1}) k$ which means that $h ^ \beta k^{-1} \in K$. 
        Which, again, contradicts with the assumption, so $h \cdot k \notin K$

        Therefore, $h \cdot k \notin H \cup K$, which asserts that $H \cup K$ cannot be a subgroup of $G$ by contraposition.
      \end{shift}
    $(\Longleftarrow)$~:
      \begin{shift}
        Since either of $K \cup H = K$ or $H \cup K = H$ holds and that $H$ and $K$ are both subgroup,
        then it follows that $K \cup H$ must be a subgroup of $G$
      \end{shift}
  }
  \qs{}{
    Prove that the subset $T_n(\Real) = \set{A \in GL_n(\Real) \mid A_{ij} = 0 \text{ if } i < j}$
    of the group $GL_n(\Real)$ is a subgroup.
  }
  \sol{
    Firstly, $T_n(\Real)$ is not empty, as 
    $I_n = 
      \begin{bmatrix}
        1 & 0 & \cdots & 0 \\
        0 & \ddots &   & 0 \\
        \vdots &   & 1 & 0 \\
        0 & \cdots & 0 & 1
      \end{bmatrix} \in T_n(\Real)$
    \clm{
      $\forall A, B \in T_n(\Real), \; AB \in T_n(\Real)$
      \pf{
        Firstly, From the definition, for all $A, B \in T_n(\Real), \, A_{ij} = B_{ij} = 0$ for $i < j$. 
        And \[{(AB)}_{ij} = \sum_{k=1}^n A_{ik}B_{kj}\] by the definition of matrix multiplication.
        So if $i < k$, then $A_{ik} = 0$ and if $k < j$, then $B_{kj} = 0$. 
        But either $i < k$ or $k < j$ holds if $i < j$. Therefore, 
        \[\text{for } i < j, \; {(AB)}_{ij} = \sum_{k=1}^n 0\]
        Which proves that $AB \in T_n(\Real)$
      }
    }
    \clm{
      $\forall A \in T_n(\Real), \;A^{-1} \in T_n(\Real)$
      \pf{
        Let $M$ be an $n\times n$ matrix, and $M^{-1}$ be the inverse. 
        Then, \[ {(M^{-1}M)}_{ij} = \sum_{k=1}^n M^{-1}_{ik}M_{kj} = I_{ij} \]
        So for $j > i$, \[ 0 = \sum_{k=1}^n M^{-1}_{ik}M_{kj} = \sum_{k=1}^i M^{-1}_{ik}M_{jk} + \sum_{k=i+1}^{n} M^{-1}_{ik}M_{jk} = \sum_{k=i+1}^{n} M^{-1}_{ik}M_{jk} \]

        If an inverse of the matrix would exists, then it would have to be in the form that $\forall j>i \; M^{-1}_{ij} = 0$, 
        which means that it must need to be an element of $T_n(\Real)$
      }
    }
    Since $\forall A, B \in T_n(\Real)$, $A^{-1} \in T_n(\Real)$ and $AB \in T_n(\Real)$, 
    then $T_n(\Real)$ is a subgroup of $GL_n(\Real)$
  }
  \qs{}{
    Let $D_{2n}$ denote the dihedral group of order $2n$. 
    Show that $\set{g \in D_{2n} \mid g^2 = 1}$ is not a subgroup of $D_{2n}$.
  }
  \sol{
    Let $S(G)$ denotes the set $\set{g \in G \mid g^2 = 1}$ for simplicity.
    \\ Consider only for $n\ge 3$, then \[S(D_{2n}) = \set{1, f, fr, \ldots, fr^{n-1}, (r^{n/2} \text{ if n is even})} \subset D_{2n}\]
    By the definition of the operator $(\cdot)$ of $D_{2n}$, $f \cdot fr = r \notin S(D_{2n})$.
    Therefore, $S(D_{2n})$ does not have closure over $\cdot$. Which means that $S(D_{2n})$ is not a subgroup of $D_{2n}$
  }
  \qs{}{
    Show that $GL_n(F)$ is non-abelian for any $n \ge 2$ and any field $F$. 
  }
  \sol{
    Notice that $GL_n(F)$ is a group, since it has closure, associative, has identity $I_n$ and has inverse.
    \clm[zeromul]{
      $a \times 0 = 0$ 
      \pf{ since $a \times 0 = a \times (0 + 0) = (a \times 0) + (a \times 0)$ }
    }
    For $n = 2$, let $A = \begin{bmatrix} 1 & 1 \\ 0 & 1 \end{bmatrix}$ and $B = \begin{bmatrix} 1 & 0 \\ 1 & 1 \end{bmatrix}$
    Then, $AB = \begin{bmatrix} 1 + 1 & 1 \\ 1 & 1 \end{bmatrix}$ but $BA = \begin{bmatrix} 1 & 1 \\ 1 & 1+1 \end{bmatrix}$.
    Therefore, $\cdot$ is commutative if and only if $1+1 = 1$, but $1 + 1 \ne 1$ since $1$ cannot be the additive identity.
    So $GL_2(F)$ is non-abelian, since it is non-commutative.
    \\ Assume for induction, that $GL_n(F)$ is non-abelian, and that $AB \ne BA$.
    Then we can construct \[A' = \begin{bmatrix} A & 0 \\ 0 & 1 \end{bmatrix}
    \text{ and } B' = \begin{bmatrix} B & 0 \\ 0 & 1 \end{bmatrix}\]
    Then \[A'B' = \begin{bmatrix} AB & 0 \\ 0 & 1 \end{bmatrix}
    \text{ but } B'A' = \begin{bmatrix} BA & 0 \\ 0 & 1\end{bmatrix}\]
    However, $AB \ne BA$, so $A'B' \ne B'A'$ certifies that $GL_n(F)$ is not commutative.
    \\ By induction, $GL_n(F)$ is non-commutative, hence, they are non-abelian groups.
  }
  \qs{}{
    Suppose that $G$ is a group such that ${(gh)}^2 = g^2h^2$ for all $g, h \in G$. 
    Show that $G$ is abelian 
  }
  \sol{
    For a group $G$ we get that $g^{-1} \in G$ for $g \in G$.
    Therefore, $\forall g, h$ \[hg = g^{-1}ghghh^{-1} = g^{-1}{(gh)}^2h^{-1} = g^{-1}(g^2h^2)h^{-1} = gh\]
    So, $\forall g, h \in G, \; gh = hg$, which means that $G$ is an abelian group.
  }
  \qs{}{
    Prove that the multiplicative groups $\Real-\set{0}$ and $\mathbb{C}-\set{0}$ are not isomorphic.
  }
  \sol{
    Notice that $i = \sqrt{-1} \in \mathbb{C}-\set{0}$ and the order of $i$ is $4$ since $i \ne 1$, $i^2 = -1 \ne 1$, $i^3 = -i \ne 1$ and $i^4 = 1$

    However, suppose that there is an element $g \in \Real - {0}$ such that $g^4 = 1$, then $g^2 = 1$, which will result in $g$
    having the order 2, or $g^2 = -1$. However, there is no such $g \in \Real - \set{0}$ such that $g^2 = -1$.
    Therefore, there cannot exists an isomorphism $\phi: \mathbb{C} - {0} \to \Real-\set{0}$, since if there is such $\phi$, then ${\phi(i)}^2 = \phi(-1)$, 
    which means that $\phi(i) \notin \Real-\set{0}$.

    Hence, there is no such isomorphism between $\Real-\set{0}$ and $\mathbb{C}-\set{0}$, so they are not isomorphic.
  }
  \qs{}{Let $G$ be a finite group and let $x$ and $y$ be distinct elements of order 2 in $G$ that generate $G$. 
    Prove that $G \isom D_{2n}$, where $n = |xy|$
  }
  \sol{
    For distinct $x$, $y$ of order 2 that generate $G$, notice that $x^{-1} = x$ and $y^{-1} = y$.

    \clm[D2n-normalform]{
      All element $g \in G$ can be written as $x^p{(xy)}^k$ for some $k \in \Nat$ such that $0 \le k \le n-1$ and $p \in \set{0,1}$. 
      \pf{
        Consider if there is some element $g$ that contain consecutive $x$, say $g = axxb$ for certain $a, b \in G$, then $g = axxb = a1b = ab$. 
        Similar argument is valid to show that there will be no consecutive $y$.

        So an element $g \in G$ must be a string of alternating $x$ and $y$. If the string starts with $x$, then it is of the form ${(xy)}^k$, 
        and if it starts with a $y$, then it is of the form $x{(xy)}^k$ since $x(xy) = y$. Note that it will be shown later on that the string will never terminate on $x$

        Additionally, for an element that terminates with $x$, or is in the form of $g = x^{p}{(xy)}^k y$ for some $k \in \Nat$ and $p \in \set{0, 1}$, consider 
        $y {(xy)}^{n-k} {(xy)}^k y = 1$, so $y {(xy)}^{n-k} = {(xy)}^k y$, which implies $x^p {(xy)}^k y = x^{1-p} {(xy)}^{n-k+1}$
        shows the equivalent to the mentioned form
      }
    }

    So, 
    \[\forall g \in G \, \exists k \in \Nat, 0 \le k \le n-1, \quad g = x^p{(xy)}^k  \]
    where $p$ is either 1 or 0

    \clm{
      Furthermore, the representation mentioned above is unique for all element $g \in G$. 
      \pf{
        Assume that for $g \in G$, $g = x^p{(xy)}^k = x^{p'}{(xy)}^{k'}$, then 
        \[x^{p-p'} {(xy)}^{k-k'} = 1\]
        So, if $p \ne p'$, then ${(xy)}^{k-k'} = x$, which is impossible. 
        Since $xy \ne x$ and $xy \ne y$, and ${(xy)}^2 = x(yxy) \ne x(1)$ since $y(xy) \ne y(y)$, 
        and inductively for all integer $m$
        
        Otherwise, if $p = p'$, then ${(xy)}^{k-k'} = 1$, but $0 \le k-k' \le n-1$, so $k-k' = 0$, 
        otherwise, $n$ is not the order of $xy$. 
        Therefore, $p = p'$ and $k = k'$.
      }
    }

    Then consider a homomorphism $\phi: G \to D_{2n}$ such that 
    $\phi: x{(xy)}^n \mapsto fr^n$ and $\phi: {(xy)}^n \mapsto r^n$, for an $r$ and $f$ in $D_{2n}$

    Since the representation of $g \in G$ as $x^p {(xy)}^k$ for $p \in \set{0, 1}, k \in \Nat$ is unique, then $\phi$ is well-defined and injective. 
    And since a representation of $d \in D_{2n}$ as $f^p r^k$ for $p \in \set{0, 1}, k \in \Nat$ is also unique in similar manner, then $\phi$ is surjective. 

    Therefore, $\phi$ is an isomorphism that asserts $G \isom D_{2n}$
  }
  \qs{}{
    Let $\sigma \in S_8$ be a permutation such that 
    \[\sigma(1) = 3, \sigma(2) = 6, \sigma(3) = 4, \sigma(4) = 1, \sigma(5) = 8, \sigma(6) = 2, \sigma(7) = 5, \sigma(8) = 7\]
    Express this permutation as a product of disjoint cycles and a product of transpositions.
  }
  \sol{
    Consider $\sigma^* = (1 3 4)\circ(2 6)\circ(5 8 7)$, where $\circ$ denotes composition and $\phi = (a_1 a_2 \ldots a_n)$ denotes a cyclic permutation of size $n$ such that
    \[ \phi(a_1) = a_2,\; \phi(a_2) = a_3,\; \ldots,\; \phi(a_n) = a_1 \quad \text{and } \phi(x) = x \text{ for other $x \ne a_i$ }\]
    Therefore, it is trivial to check that $\sigma_*$ satisfies that condition of $\sigma$ in the statement. 
    As 
    \begin{align*}
      \sigma^*(1) = [(1 3 4)\circ(2 6)\circ(5 8 7)] (1) = [(1 3 4)\circ(2 6)](1) = [(1 3 4)](1) = 3
    \\\sigma^*(2) = [(1 3 4)\circ(2 6)\circ(5 8 7)] (2) = [(1 3 4)\circ(2 6)](2) = [(1 3 4)](6) = 6
    \\\sigma^*(3) = [(1 3 4)\circ(2 6)\circ(5 8 7)] (3) = [(1 3 4)\circ(2 6)](3) = [(1 3 4)](3) = 4
    \\\sigma^*(4) = [(1 3 4)\circ(2 6)\circ(5 8 7)] (4) = [(1 3 4)\circ(2 6)](4) = [(1 3 4)](4) = 1
    \\\sigma^*(5) = [(1 3 4)\circ(2 6)\circ(5 8 7)] (5) = [(1 3 4)\circ(2 6)](8) = [(1 3 4)](8) = 8
    \\\sigma^*(6) = [(1 3 4)\circ(2 6)\circ(5 8 7)] (6) = [(1 3 4)\circ(2 6)](6) = [(1 3 4)](2) = 2
    \\\sigma^*(7) = [(1 3 4)\circ(2 6)\circ(5 8 7)] (7) = [(1 3 4)\circ(2 6)](5) = [(1 3 4)](5) = 5
    \\\sigma^*(8) = [(1 3 4)\circ(2 6)\circ(5 8 7)] (8) = [(1 3 4)\circ(2 6)](7) = [(1 3 4)](7) = 7
    \end{align*}
    Notice that all of the cycles are disjoint.
    Furthermore, a cycle $(1 3 4)$ is equivalent to a composition of transpositions $(1 4)\circ(1 3)$ 
    and also $(5 8 7)$ is equivalent to a composition of transpositions $(5 7)\circ(5 8)$. I shall prove the claim
    \clm{
      a cycle $(a b c)$ is the same permutation as the composition $(a c)\circ(a b)$
      \pf{
        let $p$ be a permutation of $(a c)\circ(b c)$. Then the followings holds
        \begin{align*}
           p(a) &= [(a c)\circ(a b)] (a) = [(a c)](b) = b
        \\ p(b) &= [(a c)\circ(a b)] (b) = [(a c)](a) = c
        \\ p(c) &= [(a c)\circ(a b)] (c) = [(a c)](c) = a
        \\ \forall x \notin \set{a, b, c} \; p(x) & = [(a c)\circ(a b)] (x) = [(a c)](x) = x
        \end{align*}
        which agrees with the definition of $(a b c)$, therefore $p = (a b c)$.
      }
    }
  Thus, $(1 3 4)\circ(2 6)\circ(5 8 7) = (1 4)\circ(1 3)\circ(2 6)\circ(5 7)\circ(5 8)$.
  And it is possible to verify that $(1 4)\circ(1 3)\circ(2 6)\circ(5 7)\circ(5 8)$ satisfies the condition of $\sigma$ in the statement.
  }
  \qs{}{
    Write out multiplication tables for $D_6$ and $S_3$. Do they have the same structure?
  }
  \sol{
    Note that the multiplication table would be written so that the first column goes to the right hand side of the binary operator.
    \begin{table}[h]
      \centering
      \begin{tabular}{ |c|c c c c c c| }
        \hline
        $\times$ & $e$ & $r$ & $r^2$ & $f$ & $fr$ & $fr^2$ 
      \\\hline
         $e$     & $e$ & $r$ & $r^2$ & $f$ & $fr$ & $fr^2$
      \\ $r$     & $r$ & $r^2$ & $e$ & $fr^2$ & $f$ & $fr$
      \\ $r^2$   & $r^2$ & $e$ & $r$ & $fr$ & $fr^2$ & $f$ 
      \\ $f$     & $f$ & $fr$ & $fr^2$ & $e$ & $r$ & $r^2$
      \\ $fr$    & $fr$ & $fr^2$ & $f$ & $r^2$ & $e$ & $r$
      \\ $fr^2$  & $fr^2$ & $f$ & $fr$ & $r$ & $r^2$ & $e$
      \\\hline
      \end{tabular}
      \caption{the multiplication table of $D_6$}\label{tab:mulD}
    \end{table}
    \begin{table}[h]
      \centering
      \begin{tabular}{ |c|c c c c c c| }
        \hline
        $\times$    & $I$ & $(1 2 3)$ & $(1 3 2)$ & $(1 2)$ & $(2 3)$ & $(1 3)$ 
      \\\hline
         $I$        & $I$ & $(1 2 3)$ & $(1 3 2)$ & $(1 2)$ & $(2 3)$ & $(1 3)$
      \\ $(1 2 3)$  & $(1 2 3)$ & $(1 3 2)$ & $I$ & $(1 3)$ & $(1 2)$ & $(2 3)$
      \\ $(1 3 2)$  & $(1 3 2)$ & $I$ & $(1 2 3)$ & $(2 3)$ & $(1 3)$ & $(1 2)$ 
      \\ $(1 2)$    & $(1 2)$ & $(2 3)$ & $(1 3)$ & $I$ & $(1 2 3)$ & $(1 3 2)$
      \\ $(2 3)$    & $(2 3)$ & $(1 3)$ & $(1 2)$ & $(1 3 2)$ & $I$ & $(1 2 3)$
      \\ $(1 3)$    & $(1 3)$ & $(1 2)$ & $(2 3)$ & $(1 2 3)$ & $(1 3 2)$ & $I$
      \\\hline
      \end{tabular}
      \caption{the multiplication table of $S_3$}\label{tab:mulS}
    \end{table}

    Notice that both table shows similar structure. Also, when applying the map $\phi$ as follows, 
    \begin{align*}
       \phi: \; & r \mapsto (1 2 3)
    \\ \phi: \; & f \mapsto (1 2) 
    \end{align*} 
    and extends according to the multiplication rules.
    We can see that the multiplication table~\ref{tab:mulD} over $\phi$ is exactly the same as table~\ref{tab:mulS}
  }
  \qs{}{
    Let $G$ be an abelian group. Prove that $\set{g \in G \mid |g| < \infty}$ is a subgroup
    of $G$ (called the torsion subgroup of $G$). 
    Give an explicit example where this set is not a subgroup when $G$ is non-abelian.
  }
  \sol{
    Let denote the set $\set{g \in G \mid |g| < \infty}$ as $S$
    Firstly, notice that for $g, h \in S$, $gh \in S$.
    \begin{shift}
      Consider $g, h \in S$, there exists $n, m \in \Nat$ such that $g^n = h^m = 1$ since the order of $g$ and $h$ is finite.
      Therefore, \[{(g^n)}^m = {(h^m)}^n = 1 = g^{nm} = h^{nm}\] Therefore $g^{nm}h^{nm} = 1$. Lastly, since $G$ is abelian, 
      $gh = hg$, which asserts that \[g^{nm}h^{nm} = {(gh)}^{nm} = 1\] 
      Hence, the fact that the order of $gh$ is bounded by $nm$ ensures that \[gh \in \set{g \in G \mid |g| < \infty}\]
    \end{shift}

    Morover, for $g \in S$, the inverse $g^{-1} \in S$ exists
    \begin{shift}
      Consider an element $g \in S$, then there exists some integer $n \in \Nat$ such that $g^n = 1$. 
      From there, $g^{n-1}g = gg^{n-1} = 1$, which means that $g^{n-1} = g^{-1}$ by definition. 
      Note that \[ {g^{n-1}}^n = {g^n}^{n-1} = 1^{n-1} = 1 \] 
      Since the order of $g^{-1}$ is bounded by $n$, then $g^{-1} \in S$.
    \end{shift}

    Therefore, since $S$ has a closure over the operator, and the inverse of each element is contained within $S$, 
    Then $S$ is a subgroup of $G$. 

    For a counterexample when $G$ is not abelian, consider $GL_2(\Real)$.
    There exist $A = \begin{bmatrix}1 & 0 \\ 1 & -1 \end{bmatrix}$ and $B = \begin{bmatrix}1 & 0 \\ 0 & -1\end{bmatrix}$
    such that $AB = \begin{bmatrix}1 & 0 \\ 1 & 1 \end{bmatrix}$. Then it is easy to verify that $A^2 = I = B^2$, but ${(AB)}^n = \begin{bmatrix}1 & 0 \\ n & 1 \end{bmatrix}$
  }
\end{document}