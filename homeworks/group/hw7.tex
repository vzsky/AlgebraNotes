% chktex-file 44
% chktex-file 8

\documentclass{report}
\usepackage{amsthm}
\usepackage{amsmath}
\usepackage{amssymb}
\usepackage{amssymb}
\usepackage{amsfonts}
\usepackage{xcolor}
\usepackage{tikz}
\usepackage{fancyhdr}
\usepackage{enumerate}
\usepackage{graphicx}
\usepackage[normalem]{ulem}
\usepackage[most,many,breakable]{tcolorbox}
\usepackage[a4paper, top=80pt, foot=25pt, bottom=50pt, left=0.5in, right=0.5in]{geometry}
\usepackage{hyperref, theoremref}
\hypersetup{
	pdftitle={Assignment},
	colorlinks=true, linkcolor=b!90,
	bookmarksnumbered=true,
	bookmarksopen=true
}
\usepackage{nameref}
\usepackage{parskip}
\pagestyle{fancy}

\usepackage[explicit,compact]{titlesec}
\titleformat{\chapter}[block]{\bfseries\huge}{\thechapter. }{\compact}{#1}
        

%%%%%%%%%%%%%%%%%%%%%
%% Defining colors %%
%%%%%%%%%%%%%%%%%%%%%

\definecolor{lr}{RGB}{188, 75, 81}
\definecolor{r}{RGB}{249, 65, 68}
\definecolor{dr}{RGB}{174, 32, 18}
\definecolor{lo}{RGB}{255, 172, 129}
\definecolor{do}{RGB}{202, 103, 2}
\definecolor{o}{RGB}{238, 155, 0}
\definecolor{ly}{RGB}{255, 241, 133}
\definecolor{y}{RGB}{255, 229, 31}
\definecolor{dy}{RGB}{143, 126, 0}
\definecolor{lb}{RGB}{148, 210, 189}
\definecolor{bg}{RGB}{10, 147, 150}
\definecolor{b}{RGB}{39, 125, 161}
\definecolor{db}{RGB}{0, 95, 115}
\definecolor{p}{RGB}{229, 152, 155}
\definecolor{dp}{RGB}{181, 101, 118}
\definecolor{pp}{RGB}{142, 143, 184}
\definecolor{v}{RGB}{109, 89, 122}
\definecolor{lg}{RGB}{144, 190, 109}
\definecolor{g}{RGB}{64, 145, 108}
\definecolor{dg}{RGB}{45, 106, 79}

\colorlet{mysol}{g}
\colorlet{mythm}{lr}
\colorlet{myqst}{db}
\colorlet{myclm}{lb}
\colorlet{mywrong}{r}
\colorlet{mylem}{o}
\colorlet{mydef}{lg}
\colorlet{mycor}{lb}
\colorlet{myrem}{dr}

%%%%%%%%%%%%%%%%%%%%%

\newcommand{\col}[2]{
  \color{#1}#2\color{black}\,
}

\newcommand{\TODO}[1][5cm]{
  \color{red}TODO\color{black}
  \vspace{#1}
}

\newcommand{\wans}[1]{
	\noindent\color{mywrong}\textbf{Wrong answer: }\color{black}
	#1 


}

\newcommand{\wreason}[1]{
	\noindent\color{mywrong}\textbf{Reason: }\color{black}
	#1 

  
}

\newcommand{\sol}[1]{
	\noindent\color{mysol}\textbf{Solution: }\color{black}
	#1


}

\newcommand{\nt}[1]{
  \begin{note}Note: #1\end{note}
}

\newcommand{\ky}[1]{
  \begin{key}#1\end{key}
}

\newcommand{\pf}[1]{
  \begin{myproof}#1\end{myproof}
}

\newcommand{\qs}[3][]{
  \begin{question}{#2}{#1}#3\end{question}
}

\newcommand{\df}[3][]{
  \begin{definition}{#2}{#1}#3\end{definition}
}

\newcommand{\thm}[3][]{
  \begin{theorem}{#2}{#1}#3\end{theorem}
}

\newcommand{\clm}[3][]{
  \begin{claim}{#2}{#1}#3\end{claim} 
}

\newcommand{\lem}[3][]{
  \begin{lemma}{#2}{#1}#3\end{lemma}
}

\newcommand{\cor}[3][]{
  \begin{corollary}{#2}{#1}#3\end{corollary}
}

\newcommand{\rem}[3][]{
  \begin{remark}{#2}{#1}#3\end{remark}
}

\newcommand{\twoways}[2]{
  \leavevmode\\
  ($\Longrightarrow$): 
  \begin{shift}#1\end{shift}
  ($\Longleftarrow$):
  \begin{shift}#2\end{shift} 
}

\newcommand{\nways}[2]{
  \leavevmode\\
  ($#1$): 
  \begin{shift}#2\end{shift}
}

%%%%%%%%%%%%%%%%%%%%%%%%%%%%%% ENVRN

\newenvironment{myproof}[1][\proofname]{%
	\proof[\bfseries #1: ]
}{\endproof}

\tcbuselibrary{theorems,skins,hooks}
\newtcolorbox{shift}
{%
  before upper={\setlength{\parskip}{5pt}},
  blanker,
	breakable,
	width=0.95\textwidth,
  enlarge left by=0.03\textwidth,
}

\tcbuselibrary{theorems,skins,hooks}
\newtcolorbox{key}
{%
	breakable,
	width=0.95\textwidth,
  enlarge left by=0.03\textwidth,
}

\tcbuselibrary{theorems,skins,hooks}
\newtcolorbox{note}
{%
	enhanced,
	breakable,
	colback = white,
	width=\textwidth,
	frame hidden,
	borderline west = {2pt}{0pt}{black},
	sharp corners,
}

\tcbuselibrary{theorems,skins,hooks}
\newtcbtheorem[]{remark}{Remark}
{%
	enhanced,
	breakable,
	colback = white,
	frame hidden,
	boxrule = 0sp,
	borderline west = {2pt}{0pt}{myrem},
	sharp corners,
	detach title,
  before upper={\setlength{\parskip}{5pt}\tcbtitle\par\smallskip},
	coltitle = myrem,
	fonttitle = \bfseries\sffamily,
	description font = \mdseries,
	separator sign none,
	segmentation style={solid, myrem},
}{rem}

\tcbuselibrary{theorems,skins,hooks}
\newtcbtheorem[number within=section]{lemma}{Lemma}
{%
	enhanced,
	breakable,
	colback = white,
	frame hidden,
	boxrule = 0sp,
	borderline west = {2pt}{0pt}{mylem},
	sharp corners,
	detach title,
  before upper={\setlength{\parskip}{5pt}\tcbtitle\par\smallskip},
	coltitle = mylem,
	fonttitle = \bfseries\sffamily,
	description font = \mdseries,
	separator sign none,
	segmentation style={solid, mylem},
}{lem}

\tcbuselibrary{theorems,skins,hooks}
\newtcbtheorem{claim}{Claim}
{%
  parbox=false,
	enhanced,
	breakable,
	colback = white,
	frame hidden,
	boxrule = 0sp,
	borderline west = {2pt}{0pt}{myclm},
	sharp corners,
	detach title,
  before upper={\setlength{\parskip}{5pt}\tcbtitle\par\smallskip},
	coltitle = myclm,
	fonttitle = \bfseries\sffamily,
	description font = \mdseries,
	separator sign none,
	segmentation style={solid, myclm},
}{clm}

\makeatletter
\newtcbtheorem[number within=section, use counter from=lemma]{theorem}{Theorem}{enhanced,
	breakable,
	colback=white,
	colframe=mythm,
	attach boxed title to top left={yshift*=-\tcboxedtitleheight},
	fonttitle=\bfseries,
	title={#2},
	boxed title size=title,
	boxed title style={%
			sharp corners,
			rounded corners=northwest,
			colback=mythm,
			boxrule=0pt,
		},
	underlay boxed title={%
			\path[fill=mythm] (title.south west)--(title.south east)
			to[out=0, in=180] ([xshift=5mm]title.east)--
			(title.center-|frame.east)
			[rounded corners=\kvtcb@arc] |-
			(frame.north) -| cycle;
		},
	#1
}{thm}
\makeatother

\makeatletter
\newtcbtheorem{question}{Question}{enhanced,
	breakable,
	colback=white,
	colframe=myqst,
	attach boxed title to top left={yshift*=-\tcboxedtitleheight},
	fonttitle=\bfseries,
	title={#2},
	boxed title size=title,
	boxed title style={%
			sharp corners,
			rounded corners=northwest,
			colback=myqst,
			boxrule=0pt,
		},
	underlay boxed title={%
			\path[fill=myqst] (title.south west)--(title.south east)
			to[out=0, in=180] ([xshift=5mm]title.east)--
			(title.center-|frame.east)
			[rounded corners=\kvtcb@arc] |-
			(frame.north) -| cycle;
		},
	#1
}{qs}
\makeatother

\makeatletter
\newtcbtheorem[number within=section]{definition}{Definition}{enhanced,
	breakable,
	colback=white,
	colframe=mydef,
	attach boxed title to top left={yshift*=-\tcboxedtitleheight},
	fonttitle=\bfseries,
	title={#2},
	boxed title size=title,
	boxed title style={%
			sharp corners,
			rounded corners=northwest,
			colback=mydef,
			boxrule=0pt,
		},
	underlay boxed title={%
			\path[fill=mydef] (title.south west)--(title.south east)
			to[out=0, in=180] ([xshift=5mm]title.east)--
			(title.center-|frame.east)
			[rounded corners=\kvtcb@arc] |-
			(frame.north) -| cycle;
		},
	#1
}{def}
\makeatother

\makeatletter
\newtcbtheorem[number within=section, use counter from=lemma]{corollary}{Corollary}{enhanced,
	breakable,
	colback=white,
	colframe=mycor,
	attach boxed title to top left={yshift*=-\tcboxedtitleheight},
	fonttitle=\bfseries,
	title={#2},
	boxed title size=title,
	boxed title style={%
			sharp corners,
			rounded corners=northwest,
			colback=mycor,
			boxrule=0pt,
		},
	underlay boxed title={%
			\path[fill=mycor] (title.south west)--(title.south east)
			to[out=0, in=180] ([xshift=5mm]title.east)--
			(title.center-|frame.east)
			[rounded corners=\kvtcb@arc] |-
			(frame.north) -| cycle;
		},
	#1
}{cor}
\makeatother

% Basic
  \DeclareMathOperator{\lcm}{lcm}
  \newcommand{\Real}{\mathbb{R}}
  \newcommand{\Comp}{\mathbb{C}}
  \newcommand{\Nat}{\mathbb{N}}
  \newcommand{\Rat}{\mathbb{Q}}
  \newcommand{\Int}{\mathbb{Z}}
  \newcommand{\set}[1]{\left\{\, #1 \,\right\}}
  \newcommand{\paren}[1]{\left( \; #1 \; \right)}
  \newcommand{\abs}[1]{\left\lvert #1 \right\rvert}
  \newcommand{\ang}[1]{\left\langle #1 \right\rangle}
  \renewcommand{\to}[1][]{\xrightarrow{\text{#1}}}
  \newcommand{\tol}[1][]{\to{$#1$}}
  \newcommand{\curle}{\preccurlyeq}
  \newcommand{\curge}{\succcurlyeq}
  \newcommand{\mapsfrom}{\leftarrow\!\shortmid}

  \newcommand{\mat}[1]{\begin{bmatrix} #1 \end{bmatrix}}
  \newcommand{\pmat}[1]{\begin{pmatrix} #1 \end{pmatrix}}
  \newcommand{\eqs}[1]{\begin{align*} #1 \end{align*}}
  \newcommand{\case}[1]{\begin{cases} #1 \end{cases}}
  

  % Algebra
  \newcommand{\normSg}[0]{\vartriangleleft}
  \newcommand{\ZMod}[1][n]{\mathbb{Z}/#1\mathbb{Z}}
  \newcommand{\isom}{\simeq}
  \newcommand{\mapHom}{\xrightarrow{\text{hom}}}
  \DeclareMathOperator{\Inn}{Inn}
  \DeclareMathOperator{\Aut}{Aut}
  \DeclareMathOperator{\im}{im}
  \DeclareMathOperator{\ord}{ord}
  \DeclareMathOperator{\Gal}{Gal}
  \DeclareMathOperator{\chr}{char}
  \newcommand{\surjto}{\twoheadrightarrow}
  \newcommand{\injto}{\hookrightarrow}

  % Analysis 
  \newcommand{\limty}[1][k]{\lim_{#1\to\infty}}
  \newcommand{\norm}[1]{\left\lVert#1\right\rVert}
  \newcommand{\darrow}{\rightrightarrows}


\fancyhead[L]{HW 7 - Modern Algebra MAS311}
\fancyhead[R]{\textbf{Touch Sungkawichai} 20210821}

\begin{document}

  \qs{}{
    Show  subgroups and quotient groups of nilpotent groups are nilpotent
  }
  \sol{
    Let $G$ be a nilpotent group and $H$ be a subgroup of $G$. Then there is a chain
    \[ G = G_0 > G_1 > \cdots > G_n = \set{e} \] such  the commutator $[x,y] \in G_{i+1}$ for all $x \in G$, $y \in G_i$ 
    by the definition of a nilpotent group. 
    Then define $H_i = H \cap G_i$, making the chain \[ H = H_0 > H_1 > \cdots > H_n = \set{e} \]. 
    Now for $x \in H, y \in H_i$, we have  $x, y \in H$ since $H_i < H$ which means that $[x,y] \in H$. 
    And since $x \in G$ and $y \in G_i$ also holds, then $[x,y] \in G_{i+1}$ by the property of nilpotent group $G$.
    Thus, $[x,y] \in H \cap G_{i+1} = H_{i+1}$. Which proves  $H$ is nilpotent.

    Now, let $N$ be a normal subgroup of $G$, and consider  $N \normSg G_{i+1}N \normSg G_iN$ by the property 
    of normal subgroup. Which makes $G_{i+1}N/N$ the set $\set{gN \mid g \in G_{i+1}}$ which is a subgroup of $G_iN/N$
    Therefore, a chain $G/N = G_0N/N > G_1N/N > \cdots > G_nN/N = \set{N}$ is constructed. 
    Moreover, for $x \in G/N, y \in G_iN/N$, we have  $x = gN$, $y = g_iN$ for some $g \in G$, $g_i \in G_i$. 
    Therefore, \[[x,y] = xyx^{-1}y^{-1} = (gN)(g_iN)(gN)^{-1}(g_iN)^{-1} = (gg_ig^{-1}g_i^{-1})N = [g, g_i]N \] by definition
    And since for some element $g_{i+1} \in G_{i+1}$, $[g, g_i] = g_{i+1}$ as $G$ is nilpotent. And $g_{i+1}N \in G_{i+1}N/N$,
    we have  $[x, y] \in G_{i+1}N/N$, which certifies that the quotient group $G/N$ is a nilpotent group.
  }

  \qs{}{
    Prove  a direct product of nilpotent groups is nilpotent
  }
  \sol{
    Let $G$ and $H$ be nilpotent groups, then there exists two chains
    \[ G = G_0 > G_1 > \cdots > G_n = \set{e} \] and \[ H = H_0 > H_1 > \cdots > H_m = \set{e} \]
    where $n \le m$ without loss of generality. 
    and for all $x \in G, y \in G_i$, it follows  $[x,y] \in G_{i+1}$ and for $x \in H, y \in H_i$, it follows that 
    $[x,y] \in H_{i+1}$ by the definition of nilpotent groups

    Now, consider 
    \[ K_i = \begin{cases} G_i \times H_i &\mid 0 \le i \le n \\ G_n \times H_i &\mid \text{otherwise} \end{cases} \]
    making the chain \[G \times H = K_0 > K_1 > \cdots > K_n > \cdots > K_m = \set{e} \]. 
    Moreover, consider  if $(g, h) \in G \times H$ and $(g', h') \in K_i$, then 
    \[ [(g, h), (g', h')] = (g,h)(g',h')(g,h)^{-1}(g',h')^{-1} = (gg'g^{-1}g', hh'h^{-1}h'^{-1}) = ([g,g'], [h,h']) \]
    Now for $i < n$, since $g \in G, g' \in G_i, h \in H, h' \in H_i$ by the definition of $K_i$, it follows  
    $([g,g'], [h,h']) \in G_{i+1} \times H_{i+1} = K_{i+1}$ 
    And for $i \ge n$, $g' = e$ is the only choice, hence $[g,g'] = gg'g^{-1}g'^{-1} = e \in G_n$. And $[h,h'] \in H_{i+1}$ 
    as per above argument, therefore, $([g,g'], [h,h']) \in G_{n} \times H_{i+1} = K_{i+1}$.

    Therefore, the statement was proved. 
  }

  \qs{}{
    Let $G$ be a finite group of order $pqr$, where $p$, $q$, and $r$ are prime numbers.
    Show  $G$ is not simple.
  }
  \sol{
    Let $P, Q, R$ be a sylow $p, q, r$ subgroup of $G$ respectively, then we know  $\abs{P} = p, \abs{Q} = q, \abs{R} = r$.
    Which means  $P, Q, R$ are all cyclic. Thus, the pairwise intersection of each two sylow subgroups must be trivial.  
    This is because the intersection of two disjoint cyclic groups must be a strict subgroup of both groups, but the order of
    the group must divides both, by lagrange's theorem, thus, the order of the intersection must be 1.

    Now, assuming  $G$ is simple, we have that the number of sylow subgroup satisfies $n_p > 1, n_q > 1, n_r > 1$ since 
    if $n_p = 1$, then for any $g \in G$, the conjugate $gPg^{-1} = P$ is another $p$-subgroup, hence itself, so $P \normSg G$. 
    Let $p > q > r$ without loss of generality, then we have  $n_p$ is at least $qr$ as any number $p < k < qr$ does not 
    divide $qr$. As similarly, $n_r$ is at least $q$, and $n_q$ is at least $p$. 

    Since $P$ contains $p-1$ non-trivial elements, and each subgroup intersect trivially,
    there is at least $(p-1)(qr) + (q-1)p + (r-1)q + 1 = pqr - qr + pq - p + qr - q + 1 = pqr + pq - p - q + 1$ elements in $G$.
    Since $pq > p + p > p + q$, it follows  $\abs{G} > pqr$ which is a contradiction, hence $G$ must be non-simple. 
  }

  \qs{}{
    Show any group of order $525 = 3 \cdot 5^2 \cdot 7$ is not simple.
  }
  \sol{
    Let $G$ be a group of order 525. By the Sylow theorems, a Sylow 3-subgroup of $G$ is of order 3, a Sylow 5-subgroup
    of $G$ is of order $25$ and a 7-subgroup of $G$ is of order 7. 
    Moreover, the number of Sylow subgroups follow $n_5 = 1, 21$, and $n_7 = 1, 15$.
    
    Now, assume that $G$ is simple, thus $n_7 \ne 1$ and $n_5 \ne 1$.
    Then, there must be $6 \times 14 = 84$ elements of order 7 since each group intersects trivially. 
    Firstly, if each of the 21 Sylow 5-subgroups intersect trivially pairwisely, then there would be $21 \times 24 = 504$
    non-trivial elements of order dividing 5. And there must be more than $504 + 84 > 524$ nontrivial elements, which is 
    impossible. 

    Moreover, since $G$ is simple, $n_3 \ne 1$, thus $n_3 \ge 7$. Which means that there is at least $14$ elements of 
    order 3.

    Therefore, the intersection of any two Sylow 5-subgroup must be non-trivial. Let $P$ and $Q$ be two distinct 
    (by the above assumption that $n_5 \ne 1$) Sylow 5-subgroup. Then $P \cap Q$ is a subgroup of $P$, thus, $\abs{P \cap Q} = 
    5$ as $\abs{P \cap Q} \ne 1, \abs{P \cap Q} \ne 25, and \abs{P \cap Q} \vert 5^2$. 
    

    Let $X$ be a set of $P \cap Q$ for any two distinct Sylow $p$-subgroup. If $\abs{X} = 1$ then let $N \in X$ be that element.
    It follows that $gNg^{-1} = g(P\cap Q)g^{-1} = gPg^{-1} \cap gQg^{-1} \in X$. So, $gNg^{-1} = N$. Thus, $N$ is a normal 
    subgroup of $G$, which contradicts that $G$ is simple.

    Therefore, $\abs{X} \ge 2$ must holds. In this case, there would be at least 9 nontrivial elements from each element of $X$, 
    $20 \cdot 21 = 420$ nontrivial elements from the rest of 5-subgroups. 84 elements of order 7, and 14 elements of order 3. 
    Thus, combining to $420 + 9 + 84 + 14 = 527$ nontrivial elements. Which is impossible.

    Hence, a group $G$ of order 525 cannot be a simple group.
  }

  \qs{}{
    Let $G$ be a finite group of order $pn$, where $n$ is a natural number such  $2 \le n < p$ and $p$ is prime.
    Show  $G$ is not simple.
  }
  \sol{
    Consider a $p$-subgroup $P$ of $G$, it follows  $G$ must be a cyclic group of order $p$. Now, by the third 
    sylow theorem, $n_p \equiv 1 \pmod{p}$ and $n_p \vert np$, so $n_p \vert n$, thus, $n_p = 1$.
    
    As $n_p = 1$, there is a unique sylow $p$-subgroup of $G$. Since a sylow $p$-subgroup of $G$ is of order $p \ne np$ 
    since $n \ge 2$, it follows that $P \ne G$. Then, from the second sylow theorem, the 
    sylow group $P$ is a normal subgroup of $G$. Hence, $G$ is not simple. 
  }

  \qs{}{
    Let $P$ be a Sylow $p$-subgroup of a finite group $G$. Show  if $N$ is a non-trivial normal subgroup of $G$, 
    then $N \cap P$ is a Sylow $p$-subgroup of $N$.
  }
  \sol{
    Notice that if $p \not\vert \abs{N}$ then $\abs{N \cap P}$ is trivial by lagrange's theorem.

    Now, for the remaining case, it holds that $p \vert \abs{N}$. 
    As $N \cap P$ cannot be trivial, then it is a $p$-subgroup of $N$ by lagrange's theorem. 

    As, $N$ is a normal subgroup of $G$, it follows that $PH$ is a subgroup of $G$. 
    By the second isomorphism theorem, $P/P\cap N \isom PN/N$. Therefore, $\abs{P \cap N} = \frac{\abs{P}\abs{N}}{\abs{PN}}$.
    Let $\abs{G} = p^\alpha q$ where $\gcd(p, q) = 1$, then $\abs{P} = p^\alpha$. And let $\abs{N} = p^\beta r$ 
    where $\gcd(r, p) = 1$.

    Then, as $P < PN$, it follows that $p^\alpha \vert \abs{PN}$, thus $\abs{PN} = p^\alpha s$ for $\gcd(s, p) = 1$. 
    Notice that $p^{\alpha+1} \not\vert \abs{PN}$ by the maximality of a Sylow group.
    Therefore, $\abs{P \cap N} = \frac{p^\alpha \cdot p^\beta r}{p^\alpha s} = p^\beta \frac{r}{s}$, for $\gcd(r/s, p) = 1$. 

    But since $P \cap N$ is a $p$-subgroup of $N$, then $\abs{P \cap N} = p^m$ for some $m$, which restricts the only 
    possibility to that $r/s = 1$, which makes $\abs{P \cap N} = p^{\beta}$.
    Thus, $P \cap N$ is a Sylow $p$-subgroup by definition.
  }

  \qs{}{
    Let $P$ be a Sylow $p$-subgroup of a finite group $G$ and let $Q$ be a $p$-subgrop of $G$.
    Show  $Q \cap N_G(P) = Q \cap P$.
  }
  \sol{
    The statement $Q \cap N_G(P) = Q \cap P$ is equivalent to that $x \in Q \land x \in N_G(P) \iff x \in Q \land x \in P$. 
    The proof of later statement goes as 
    \twoways{
      Notice that for any $g \in N_G(P)$, $gPg^{-1} = P$ by definition, thus $P$ is a unique sylow $p$-subgroup of $N_G(P)$ 
      by the second sylow theorem. Therefore, if $x \in Q$ and $x \in N_G(P)$, then $x$ is in a $p$-subgroup of $G$ and at the 
      same time, in a group which the maximum $p$-subgroup is unique. Thus, $x$ is also in the maximal $p$-subgroup. 
      Therefore, $x \in P$, so it follows that $x \in Q \land x \in P$.
    }{
      Since $P \normSg N_G(P)$, it follows that $P \subseteq N_G(P)$, thus, if $x \in Q \land x \in P$ then 
      $x \in Q \land x \in N_G(P)$.  
    }
  }

  \qs{}{
    Let $G$ be a non-cyclic group of order 21. Find the number of Sylow 3-subgroups of $G$.
  }
  \sol{
    If $G$ is a non-cyclic group of order 21. Then, the number of 3-subgroup of $G$ is $n_3 \equiv 1 \pmod{3}$ with 
    $n_3 \vert 21$ restricting the choice to $n_3 = 1$ or $n_3 = 7$ by the third sylow theorem.
    Note also  the order of a 3 subgroup is 3, since $21 = 3 \times 7$
    Moreover, 7-subgroup of $G$ is unique since $n_7 = 1$ is the only number satisfies the third theorem.
    Now, if $G$ is non-cyclic, then the order of each element in $G$ is either 1, 3, or 7. And there is only one element 
    with only 1, the identity, and 6 elements with order 7, which are the elements in the unique 7-subgroup of $G$.
    Thus, the remaining 14 elements must have order 3, which means  $n_3 = 7$ as each subgroup of order 3 consists of two 
    unique non-trivial elements of order 3 and an identity. 

    Therefore, the number of Sylow 3-subgroup of $G$ is 7. 
  }

  \qs{}{Prove  the number of Sylow $p$-subgroups of $GL_2(\ZMod[p])$ is equal to $p+1$.}
  \sol{
    From the third Sylow theorem, the number of Sylow $p$-subgroup, $n_p$ must satisfies $n_p \equiv 1 \pmod{p}$.
    Let $G$ denotes $GL_2(\ZMod[p])$, then it follows  for $g \in G$, $g = \mat{a & b \\ c & d}$, 
    such  $\det(g) = ad - bc \ne 0$. For $ad = bc$, it follows that $a = bcd^{-1}$, thus $a$ depends uniquely on $b, c, d$
    if $d \ne 0$ as $d^{-1}$ is unique. And for $d = 0$, $ad = bc$ if and only if $b = 0$ or $c = 0$. 

    Thus, if $d = 0$ we have $p \cdot (p-1) \cdot (p-1)$ choices for $a, b, c$ and $d \ne 0$, we have $p \cdot (p-1) \cdot p$ 
    choices for the remaining $a, b, c$. This means  the order $\abs{G} = p\cdot(p+1)\cdot(p-1)^2$

    Therefore, a $p$ subgroup of $G$ is a cyclic group of order $p$. 
    Now, notice  $g = \mat{1 & 1 \\ 0 & 1}$ generates a subgroup of order $p$.
    The corresponding group is $P = \set{\mat{1 & a \\ 0 & 1} \mid 0 \le a < p}$. Now, all the 
    other sylow subgroup must be in the from $gPg^{-1}$ for some $g \in G$.

    \clm{
      for $g = \mat{a & b \\ c & d}$, the inverse $g^{-1}$ is 
      $\frac{1}{\det(g)} \mat{d & -b \\ -c & a}$
      \pf{
        Consider  $\mat{ a & b \\ c & d } \mat{d & -b \\ -c & a} = 
        \mat{ \det(g) & 0 \\ 0 & \det(g) } = \mat{ d & -b \\ -c & a } 
        \mat{ a & b \\ c & d }$
      }
    }

    Now consider a conjugation action of $G$ on $P$.  
    Then, the size of all $p$-subgroup is $n_p = \abs{\set{gPg^{-1} \mid g \in G}} = \abs{GP} = [G:G_P]$ by the orbit 
    stabilizer theorem. Now, since the stabilizer $\abs{G_P} = \abs{\set{g \mid gPg^{-1} = g}} = \abs{N_G(P)}$ 

    For $g\in G$, and $\ang{\bar{p}} = P$ 
    \[g\bar{p}g^{-1} = \mat{a & b \\ c & d}\mat{1 & 1 \\ 0 & 1}\mat{a & b \\ c & d}^{-1} = 
      \frac{1}{ad-bc} \mat{a & b \\ c & d}\mat{1 & 1 \\ 0 & 1}\mat{d & -b \\ -c & a} =
    \frac{1}{ad - bc}\mat{ad-bc-ac & a^2 \\ -c^2 & ad-bc+ac} \]
    is in $P$ only if $c^2 = 0$, which is when $c=0$.  
    And when $c = 0$, $g\bar{p}g^{-1} = \mat{1 & \frac{a^2}{ad-bc} \\ 0 & 1} \in P$
    Thus, it follows that $N_G(P) = \set{\mat{a & b \\ 0 & d} \in G \mid \forall a, b, c}$

    So, $\abs{N_G(P)} = p(p-1)(p-1)$ as $b$ can be any value, but $a$ and $d$ must not be 0 in order for the matrix to be an 
    element of $G = GL_2(\ZMod[p])$. Therefore, $n_p = \abs{G}/\abs{N_G(P)} = p+1$
  }

  \qs{}{
    Let a finite group $G$ act transitively on a finite set $X$. Show  
    \[ \abs{\set{(g, x) \in G \times X \mid g \cdot x = x}} = \abs{G} \]
  }
  \sol{
   Since $G$ acts transitively on $X$, then there is one orbit of the action, which means  $Gx = X$.
   Now, by the orbit stabilizer theorem, $\abs{Gx} = \frac{\abs{G}}{\abs{G_x}}$, thus $\abs{G} = \abs{G_x}\abs{X}$. 
   By the definition, $G_x = \set{g \in G \mid g \cdot x = x}$. Now, consider 
   \[ G_x \times X = \set{(g, x) \mid g \in G_x, x \in X} = \set{(g, x) \mid g \cdot x = x, x \in X}
   = \set{(g, x) \in G \times X \mid g \cdot x = x} \]
   Thus, $\abs{\set{(g, x) \in G\times X \mid g \cdot x = x}} = \abs{G_x \times X} = \abs{G_x} \cdot \abs{X} = \abs{G}$
  }
\end{document}
