% chktex-file 44
% chktex-file 8

\documentclass{report}
\usepackage{amsthm}
\usepackage{amsmath}
\usepackage{amssymb}
\usepackage{amssymb}
\usepackage{amsfonts}
\usepackage{xcolor}
\usepackage{tikz}
\usepackage{fancyhdr}
\usepackage{enumerate}
\usepackage{graphicx}
\usepackage[normalem]{ulem}
\usepackage[most,many,breakable]{tcolorbox}
\usepackage[a4paper, top=80pt, foot=25pt, bottom=50pt, left=0.5in, right=0.5in]{geometry}
\usepackage{hyperref, theoremref}
\hypersetup{
	pdftitle={Assignment},
	colorlinks=true, linkcolor=b!90,
	bookmarksnumbered=true,
	bookmarksopen=true
}
\usepackage{nameref}
\usepackage{parskip}
\pagestyle{fancy}

\usepackage[explicit,compact]{titlesec}
\titleformat{\chapter}[block]{\bfseries\huge}{\thechapter. }{\compact}{#1}
        

%%%%%%%%%%%%%%%%%%%%%
%% Defining colors %%
%%%%%%%%%%%%%%%%%%%%%

\definecolor{lr}{RGB}{188, 75, 81}
\definecolor{r}{RGB}{249, 65, 68}
\definecolor{dr}{RGB}{174, 32, 18}
\definecolor{lo}{RGB}{255, 172, 129}
\definecolor{do}{RGB}{202, 103, 2}
\definecolor{o}{RGB}{238, 155, 0}
\definecolor{ly}{RGB}{255, 241, 133}
\definecolor{y}{RGB}{255, 229, 31}
\definecolor{dy}{RGB}{143, 126, 0}
\definecolor{lb}{RGB}{148, 210, 189}
\definecolor{bg}{RGB}{10, 147, 150}
\definecolor{b}{RGB}{39, 125, 161}
\definecolor{db}{RGB}{0, 95, 115}
\definecolor{p}{RGB}{229, 152, 155}
\definecolor{dp}{RGB}{181, 101, 118}
\definecolor{pp}{RGB}{142, 143, 184}
\definecolor{v}{RGB}{109, 89, 122}
\definecolor{lg}{RGB}{144, 190, 109}
\definecolor{g}{RGB}{64, 145, 108}
\definecolor{dg}{RGB}{45, 106, 79}

\colorlet{mysol}{g}
\colorlet{mythm}{lr}
\colorlet{myqst}{db}
\colorlet{myclm}{lb}
\colorlet{mywrong}{r}
\colorlet{mylem}{o}
\colorlet{mydef}{lg}
\colorlet{mycor}{lb}
\colorlet{myrem}{dr}

%%%%%%%%%%%%%%%%%%%%%

\newcommand{\col}[2]{
  \color{#1}#2\color{black}\,
}

\newcommand{\TODO}[1][5cm]{
  \color{red}TODO\color{black}
  \vspace{#1}
}

\newcommand{\wans}[1]{
	\noindent\color{mywrong}\textbf{Wrong answer: }\color{black}
	#1 


}

\newcommand{\wreason}[1]{
	\noindent\color{mywrong}\textbf{Reason: }\color{black}
	#1 

  
}

\newcommand{\sol}[1]{
	\noindent\color{mysol}\textbf{Solution: }\color{black}
	#1


}

\newcommand{\nt}[1]{
  \begin{note}Note: #1\end{note}
}

\newcommand{\ky}[1]{
  \begin{key}#1\end{key}
}

\newcommand{\pf}[1]{
  \begin{myproof}#1\end{myproof}
}

\newcommand{\qs}[3][]{
  \begin{question}{#2}{#1}#3\end{question}
}

\newcommand{\df}[3][]{
  \begin{definition}{#2}{#1}#3\end{definition}
}

\newcommand{\thm}[3][]{
  \begin{theorem}{#2}{#1}#3\end{theorem}
}

\newcommand{\clm}[3][]{
  \begin{claim}{#2}{#1}#3\end{claim} 
}

\newcommand{\lem}[3][]{
  \begin{lemma}{#2}{#1}#3\end{lemma}
}

\newcommand{\cor}[3][]{
  \begin{corollary}{#2}{#1}#3\end{corollary}
}

\newcommand{\rem}[3][]{
  \begin{remark}{#2}{#1}#3\end{remark}
}

\newcommand{\twoways}[2]{
  \leavevmode\\
  ($\Longrightarrow$): 
  \begin{shift}#1\end{shift}
  ($\Longleftarrow$):
  \begin{shift}#2\end{shift} 
}

\newcommand{\nways}[2]{
  \leavevmode\\
  ($#1$): 
  \begin{shift}#2\end{shift}
}

%%%%%%%%%%%%%%%%%%%%%%%%%%%%%% ENVRN

\newenvironment{myproof}[1][\proofname]{%
	\proof[\bfseries #1: ]
}{\endproof}

\tcbuselibrary{theorems,skins,hooks}
\newtcolorbox{shift}
{%
  before upper={\setlength{\parskip}{5pt}},
  blanker,
	breakable,
	width=0.95\textwidth,
  enlarge left by=0.03\textwidth,
}

\tcbuselibrary{theorems,skins,hooks}
\newtcolorbox{key}
{%
	breakable,
	width=0.95\textwidth,
  enlarge left by=0.03\textwidth,
}

\tcbuselibrary{theorems,skins,hooks}
\newtcolorbox{note}
{%
	enhanced,
	breakable,
	colback = white,
	width=\textwidth,
	frame hidden,
	borderline west = {2pt}{0pt}{black},
	sharp corners,
}

\tcbuselibrary{theorems,skins,hooks}
\newtcbtheorem[]{remark}{Remark}
{%
	enhanced,
	breakable,
	colback = white,
	frame hidden,
	boxrule = 0sp,
	borderline west = {2pt}{0pt}{myrem},
	sharp corners,
	detach title,
  before upper={\setlength{\parskip}{5pt}\tcbtitle\par\smallskip},
	coltitle = myrem,
	fonttitle = \bfseries\sffamily,
	description font = \mdseries,
	separator sign none,
	segmentation style={solid, myrem},
}{rem}

\tcbuselibrary{theorems,skins,hooks}
\newtcbtheorem[number within=section]{lemma}{Lemma}
{%
	enhanced,
	breakable,
	colback = white,
	frame hidden,
	boxrule = 0sp,
	borderline west = {2pt}{0pt}{mylem},
	sharp corners,
	detach title,
  before upper={\setlength{\parskip}{5pt}\tcbtitle\par\smallskip},
	coltitle = mylem,
	fonttitle = \bfseries\sffamily,
	description font = \mdseries,
	separator sign none,
	segmentation style={solid, mylem},
}{lem}

\tcbuselibrary{theorems,skins,hooks}
\newtcbtheorem{claim}{Claim}
{%
  parbox=false,
	enhanced,
	breakable,
	colback = white,
	frame hidden,
	boxrule = 0sp,
	borderline west = {2pt}{0pt}{myclm},
	sharp corners,
	detach title,
  before upper={\setlength{\parskip}{5pt}\tcbtitle\par\smallskip},
	coltitle = myclm,
	fonttitle = \bfseries\sffamily,
	description font = \mdseries,
	separator sign none,
	segmentation style={solid, myclm},
}{clm}

\makeatletter
\newtcbtheorem[number within=section, use counter from=lemma]{theorem}{Theorem}{enhanced,
	breakable,
	colback=white,
	colframe=mythm,
	attach boxed title to top left={yshift*=-\tcboxedtitleheight},
	fonttitle=\bfseries,
	title={#2},
	boxed title size=title,
	boxed title style={%
			sharp corners,
			rounded corners=northwest,
			colback=mythm,
			boxrule=0pt,
		},
	underlay boxed title={%
			\path[fill=mythm] (title.south west)--(title.south east)
			to[out=0, in=180] ([xshift=5mm]title.east)--
			(title.center-|frame.east)
			[rounded corners=\kvtcb@arc] |-
			(frame.north) -| cycle;
		},
	#1
}{thm}
\makeatother

\makeatletter
\newtcbtheorem{question}{Question}{enhanced,
	breakable,
	colback=white,
	colframe=myqst,
	attach boxed title to top left={yshift*=-\tcboxedtitleheight},
	fonttitle=\bfseries,
	title={#2},
	boxed title size=title,
	boxed title style={%
			sharp corners,
			rounded corners=northwest,
			colback=myqst,
			boxrule=0pt,
		},
	underlay boxed title={%
			\path[fill=myqst] (title.south west)--(title.south east)
			to[out=0, in=180] ([xshift=5mm]title.east)--
			(title.center-|frame.east)
			[rounded corners=\kvtcb@arc] |-
			(frame.north) -| cycle;
		},
	#1
}{qs}
\makeatother

\makeatletter
\newtcbtheorem[number within=section]{definition}{Definition}{enhanced,
	breakable,
	colback=white,
	colframe=mydef,
	attach boxed title to top left={yshift*=-\tcboxedtitleheight},
	fonttitle=\bfseries,
	title={#2},
	boxed title size=title,
	boxed title style={%
			sharp corners,
			rounded corners=northwest,
			colback=mydef,
			boxrule=0pt,
		},
	underlay boxed title={%
			\path[fill=mydef] (title.south west)--(title.south east)
			to[out=0, in=180] ([xshift=5mm]title.east)--
			(title.center-|frame.east)
			[rounded corners=\kvtcb@arc] |-
			(frame.north) -| cycle;
		},
	#1
}{def}
\makeatother

\makeatletter
\newtcbtheorem[number within=section, use counter from=lemma]{corollary}{Corollary}{enhanced,
	breakable,
	colback=white,
	colframe=mycor,
	attach boxed title to top left={yshift*=-\tcboxedtitleheight},
	fonttitle=\bfseries,
	title={#2},
	boxed title size=title,
	boxed title style={%
			sharp corners,
			rounded corners=northwest,
			colback=mycor,
			boxrule=0pt,
		},
	underlay boxed title={%
			\path[fill=mycor] (title.south west)--(title.south east)
			to[out=0, in=180] ([xshift=5mm]title.east)--
			(title.center-|frame.east)
			[rounded corners=\kvtcb@arc] |-
			(frame.north) -| cycle;
		},
	#1
}{cor}
\makeatother

% Basic
  \DeclareMathOperator{\lcm}{lcm}
  \newcommand{\Real}{\mathbb{R}}
  \newcommand{\Comp}{\mathbb{C}}
  \newcommand{\Nat}{\mathbb{N}}
  \newcommand{\Rat}{\mathbb{Q}}
  \newcommand{\Int}{\mathbb{Z}}
  \newcommand{\set}[1]{\left\{\, #1 \,\right\}}
  \newcommand{\paren}[1]{\left( \; #1 \; \right)}
  \newcommand{\abs}[1]{\left\lvert #1 \right\rvert}
  \newcommand{\ang}[1]{\left\langle #1 \right\rangle}
  \renewcommand{\to}[1][]{\xrightarrow{\text{#1}}}
  \newcommand{\tol}[1][]{\to{$#1$}}
  \newcommand{\curle}{\preccurlyeq}
  \newcommand{\curge}{\succcurlyeq}
  \newcommand{\mapsfrom}{\leftarrow\!\shortmid}

  \newcommand{\mat}[1]{\begin{bmatrix} #1 \end{bmatrix}}
  \newcommand{\pmat}[1]{\begin{pmatrix} #1 \end{pmatrix}}
  \newcommand{\eqs}[1]{\begin{align*} #1 \end{align*}}
  \newcommand{\case}[1]{\begin{cases} #1 \end{cases}}
  

  % Algebra
  \newcommand{\normSg}[0]{\vartriangleleft}
  \newcommand{\ZMod}[1][n]{\mathbb{Z}/#1\mathbb{Z}}
  \newcommand{\isom}{\simeq}
  \newcommand{\mapHom}{\xrightarrow{\text{hom}}}
  \DeclareMathOperator{\Inn}{Inn}
  \DeclareMathOperator{\Aut}{Aut}
  \DeclareMathOperator{\im}{im}
  \DeclareMathOperator{\ord}{ord}
  \DeclareMathOperator{\Gal}{Gal}
  \DeclareMathOperator{\chr}{char}
  \newcommand{\surjto}{\twoheadrightarrow}
  \newcommand{\injto}{\hookrightarrow}

  % Analysis 
  \newcommand{\limty}[1][k]{\lim_{#1\to\infty}}
  \newcommand{\norm}[1]{\left\lVert#1\right\rVert}
  \newcommand{\darrow}{\rightrightarrows}


\fancyhead[L]{Modern Algebra 2 - MAS312: Homework 9}
\fancyhead[R]{\textbf{Touch Sungkawichai} 20210821}

\begin{document}
  \qs{}{
    Show that an algebraically closed field is infinite.
  }
  \sol{
    Assume that there is a finite field that is algebraically closed. Let the field be $\mathbb F_q$ where $q = p^n$ for some 
    prime $n$.
    Then, there exists a field extension $\mathbb F_{q^2}$ of $\mathbb F_q$ such that it is the splitting field of $x^{q^2} - x$.
    As $\mathbb F_{q^2}$ is a finite extension of $\mathbb F_q$ such that $\mathbb F_q \ne \mathbb F_{q^2}$, then $\mathbb F_q$ is 
    not algebraically closed by definition.
  }
  
  \qs{}{
    Let $\bar F$ be an algebraic closure of the finite field $\mathbb F_q$. Show that $\bar F$ is the union of all finite subfields.
  }
  \sol{
    Notice that the splitting field of $x^{q^n} - x$ over $\mathbb F_q$ is $\mathbb F_q^n$. Therefore, for any $n$, 
    $\mathbb F_{q^n}$ must be contained by the algebraic closure $\bar F$.
    This means that $\bigcup_{n \ge 1} \mathbb F_{q^n} \subset \bar F$.

    Then, since any irreducible polynomial over $\mathbb F_q$ must have finite degree, say $m$, it follows that the splitting 
    field of that polynomial is $\mathbb F_{q^m}$. As the splitting of any irreducible polynomial is a finite field, then it must 
    split in $\bigcup_{n \ge 1} \mathbb F_{q^n}$. Thus, $\bar F \subset \bigcup_{n \ge 1} \mathbb F_{q^n}$.

    This gives that $\bar F = \bigcup_{n \ge 1} \mathbb F_{q^n}$.

    For any $n$, as $\mathbb F_{q^n}/\mathbb F_q$ is an algebraic extension, then $\mathbb F_{q^n}$ is a finite subfield of $\bar F$.
    Thus, $\bar F$ is the union of all of its finite subfields as needed.
  }

  \qs{}{
    Let $E/F$ be a Galois extension. Show that if the quotient group $E^\times/F^\times$ has an element of order $n$, then 
    $E^\times$ has an element of order $n$.
  }
  \sol{
    Since there is an element of order $n$, then, let that element be $\alpha$. So, $\alpha^n \in F^\times$ and 
    $\alpha^k$ for any $k < n$ is not a member of $F^\times$. Let $f = x^n - \alpha^n$. 
    As $f$ has a root in $E$ and $E$ is normal and separable, then $f$ must split in $E$
    So, $f = (x - \beta_1) \cdots (x - \beta_n)$, where each of the $\beta_i$ is an element in $E^\times$ and $\beta_i \ne \beta_j$ 
    for all $i \ne j$.

    Then, create a set of elements $\set{\gamma_1, \ldots, \gamma_n}$ such that $\gamma_i = \beta_i/\alpha$, then 
    $\gamma_i^n = \frac{\beta_i^n}{\alpha^n} = 1$. Moreover, as $\beta_i \ne \beta_j$ for $i \ne j$, then $\gamma_i \ne \gamma_j$ for 
    $i \ne j$.

    Since there are at least $n$ roots of $1$, which are $\gamma_1, \ldots, \gamma_n$ in the field $E$, then $\mu_n \in E^\times$.
    Therefore, there is an element of order $n$, which is the primitive $n$th root in $E^\times$.
  }

  \qs{}{
    Find $\Gal(E/\mathbb F_9)$ where $E$ denotes the splitting field of $x^{16} - 1$ over $\mathbb F_9$.
  }
  \sol{
    Notice that $f(x) = x^{16} - 1$ is separable as $f'(x) \ne 0$ and $E$ is a splitting field, thus normal,
    therefore, $E/\mathbb F_9$ is galois.

    Consider that the degree of a primitive root of $x^{16} - 1 = (x^8 - 1)(x^8 + 1)$ is 16, then as $\mu_{16}$, the group of roots, 
    is a subgroup of $E^\times$, the splitting field $E$ must be such that $16 \mid \abs{E^\times}$, 
    which, the smallest such $E$ is $\abs{E} = 81$, as $E$ must also be of order $3^n$.
    This means that the splitting field $E$ contains $\mathbb F_{81}$.

    Now, $\mathbb F_{81}$ is the splitting field of 
    \[ x^{81} - x = x(x^{80} - 1) = x(x^{16} - 1)(x^{64} + x^{48} + x^{32} + x^{16} + 1) \]
    Thus, $E = \mathbb F_{81}$, which implies the degree $[E : \mathbb F_9] = 2$.

    As the field extension is galois, then $\abs{\Gal(E/\mathbb F_9)} = 2$.
    Therefore, $\Gal(E/\mathbb F_9) \isom \ZMod[2]$.
  }

  \qs{}{
    Let $\mu_n$ be the group of $n$th root of $1$. Let $r = \lcm(m, n)$ and $s = \gcd(m, n)$. Show that 
    $\Rat(\mu_n)\Rat(\mu_m) = \Rat(\mu_r)$ and $\Rat(\mu_n) \cap \Rat(\mu_m) = \Rat(\mu_s)$.
  }
  \sol{
    Since $n \mid r$, then $\Rat(\mu_n) \subset \Rat(\mu_r)$ and since $m \mid r$, $\Rat(\mu_m) \subset \Rat(\mu_r)$. As 
    two fields are contained in $\Rat(\mu_r)$, the composition, which is the smallest field containing both field,
    must also be contained in $\Rat(\mu_r)$.
    
    Let $\eta_n$ be the primitive $n$th root of unity and $\eta_m$ be the primitive $m$th root. Then, 
    as there exists $a, b$ such that $an + bm = \gcd(n, m)$, then there exist $an + bm \equiv 1 \pmod{\lcm(n, m)}$. 
    Hence, $\eta^{an}\eta^{bm}$ is a primitive $r$th root of unity. Thus, $\mu_r \subset \Rat(\mu_n)\Rat(\mu_m)$, and therefore, 
    $\Rat(\mu_r) \subset \Rat(\mu_n)(\mu_m)$

    Next, as $s \mid n$ and $s \mid m$, then $\Rat(\mu_s) \subset \Rat(\mu_n) \cap \Rat(\mu_m)$ as $x^s - 1 \mid x^n - 1$ and similarly
    for $x^m - 1$.

    Now, as cyclotomic extensions are galois, then 
    \[ \Gal(\Rat(\mu_n)\Rat(\mu_m) / \Rat(\mu_n)) \isom \Gal(\Rat(\mu_n) / \Rat(\mu_n) \cap \Rat(\mu_m)) \]
    which gives that \[ \phi(\lcm(n, m)) = [\Rat(\mu_n)\Rat(\mu_m) : \Rat] = [\Rat(\mu_m): \Rat][\Rat(\mu_n) : \Rat][\Rat(\mu_n) \cap \Rat(\mu_m) : \Rat] \]
    Thus, $[\Rat(\mu_n) \cap \Rat(\mu_m) : \Rat] = \frac{\phi(n)\phi(m)}{\phi(r)}$.
    As \[ \frac{\phi(n)\phi(m)}{\phi(r)} = \frac{n\prod_{p\mid n}(1 - 1/p)\cdot m\prod_{p \mid m}(1 - 1/p)}{\lcm(n, m) \prod_{p \mid \lcm(n ,m)}(1 - 1/p)} = \gcd(n, m)\prod_{p \mid \gcd(n, m)}(1 - 1/p) = \phi(\gcd(n, m))  \]

    Then, $\Rat(\mu_n) \cap \Rat(\mu_m) = \Rat(\mu_d)$
  }

  \qs{}{
    Prove that $\Rat(\mu_n) \subseteq \Rat(\mu_m)$ if and only if $n \mid m$ or $n = 2r$ for some odd divisor $r$ of $m$.
  }
  \sol{
    \twoways{

      Since $\Rat(\mu_n) \subset \Rat(\mu_m)$, then 
      $\Rat(\mu_{\gcd(n, m)}) = \Rat(\mu_n) \cap \Rat(\mu_m) = \Rat(\mu_n)$.
      Let $d = \gcd(n, m)$, then $\phi(d) = \phi(n)$.

      If $d = 1$, then $\phi(d) = \phi(n)$ means $\phi(n) = \phi(1) = 1$. Thus, $n = 1$ or $n = 2$, because if $n \ge 3$, then 
      $\phi(n) \ge \phi(p)$ for some odd prime divisor $p$ of $n$. Thus, $\phi(n) \ge p-1 > 1$.
      In the case that $n = 1$, $n \mid m$.
      In the case that $n = 2$, $n = 2\cdot1$ where $1$ is an odd divisor of $m$.

      Otherwise, $d > 1$ and as $d \mid n$, then it can be written as 
      $d = 2^{a_0}p_1^{a_1}\cdots p_n^{a_n}$ and $n = 2^{b_0}p_1^{b_1}\cdots p_n^{b_n} q$ 
      for some odd number $q$ not divisible by any $p_i$ for odd primes $p_i$ such that 
      $a_i \ge 1$ except for $a_0 \ge 0$ and $b_i \ge a_i$ for all $i$.

      Since $\phi(d) = \phi(n)$, then 
      \[ \phi(2^{a_0})\phi(p_1^{a_1})\cdots\phi(p_n^{a_n}) = \phi(2^{b_0})\phi(p_1^{b_1})\cdots\phi(p_n^{b_n})\phi(q) \]
      as for all any odd prime $p$ and $n \ge 1$, $\phi(p^{n+1}) = p\cdot\phi(p^n)$, then 
      it must be the case that $b_i = a_i$ for all $i \ge 1$ and that $q = 1$.
      Thus, it follows that $\phi(2^a_0) = \phi(2^b_0)$. As $\phi(1) = \phi(2) = 1$ and $\phi(2^{n+1}) = 2\phi(2^n)$ for $n > 1$, then 
      either $a_0 = b_0$ or $a_0 = 0$ and $b_0 = 1$ must hold.

      This means that $d = n$ in the first case, and $2d = n$ for an odd $d$ in the latter case.
      If $d = n$, then $n \mid m$ and otherwise, $n = 2d$ for some $d \mid m$ such that $d$ is odd.

      % If $n = 2r$ for some odd $r$, then $\Rat(\mu_n) = \Rat(\mu_r)$. 
      % Then, it suffices to prove that $\Rat(\mu_n) \subseteq \Rat(\mu_m)$ implies $n \mid m$
      % for any odd $n$, or $n$ such that $4 \mid n$.
      %
      % Let $\bar\mu_m = \mu_m \cup \set{ -x \mid x \in \mu_m }$
      % If $m$ is even, then $\bar\mu_m = \mu_m$ and if $m$ is odd, then $\bar\mu_m = \mu_{2m}$
      % Consider that $\Rat(\mu_m) = \Rat(\bar{\mu_m})$ since $-1 \in \Rat$.
      %
      % Let $m' = m$ if $m$ is even, and $m' = 2m$ otherwise, so that $\bar\mu_m = \mu_{m'}$
      % Then, $\Rat(\mu_n) \subset \Rat(\mu_m) \subset \Rat(\mu_{m'})$
      %
      %
      % If $n \mid m$, then the statement holds true, otherwise, $n \nmid m$ but $n \mid m'$, so it must be the case that $m' = 2m$.
      % As $m' = 2m$, and $n \mid m'$, which $n \nmid m$, then $2 \mid n$ and $\frac{n}{2} \mid m$.
      % This means that $n = 2r$ for some odd divisor $r$ of $m$, hence the statement holds true.
      % \TODO
    } {
      If $n \mid m$, then let $dn = m$. Now, $x^m - 1 = x^{dn} - 1 = (x^n - 1)(x^{(d-1)n} + x^{(d-2)n} + \cdots + 1)$.
      Thus, $x^n - 1 \mid x^m - 1$. This means that any root of $x^n - 1$ is a root of $x^m - 1$, thus $\mu_n \subseteq \mu_m$.
      Therefore, $\Rat(\mu_n) \subseteq \Rat(\mu_m)$.

      Otherwise, if $n = 2r$ but $n \nmid m$ for some odd divisor $r$ of $m$, then $m$ is odd, so $-1 \notin \mu_m$.
      However, $\Rat(\mu_n) \subseteq \Rat(\mu_{2m})$ as $n \mid m$, and $\mu_{2m} = \mu_m \cup \set{-\eta \mid \eta \in \mu_m}$
      As $(-\eta)^{2m} = {(-\eta)^m}^2 = {-1}^2 = 1$.
      Therefore, $\Rat(\mu_{2m}) = \Rat(\mu_{m})$, which gives that $\Rat(\mu_n) \subseteq \Rat(\mu_{m})$.
    }
  }

  \qs{}{
    Find all roots of unity which are contained in $\Rat(\sqrt{-3})$
  }
  \sol{
    As $\Rat(\sqrt{-3})$ is the splitting field of $f(x) = x^2 + 3$, and $\sqrt{-3} \notin \Rat$ then $[\Rat(\sqrt{-3}) : \Rat] = 2$.
    Since the field group $\mu_n$ is cyclic, if $\Rat(\sqrt{-3})$ contains a primitive $n$ root of unity, then it must contain all the 
    $n$th root of unity, which means that it must the splits $x^n - 1$, and therefore have degree $\phi(n)$.

    Notice that $\phi(3) = \phi(4) = \phi(6) = 2$ are the only numbers with this property because for a prime $p > 3$,
    $\phi(p) = p-1 > 2$, which means that for $n > 6$, $\phi(n) > \phi(2)\phi(3) > 2$.

    Consider \[ \frac{\sqrt{-3} + 1}{2}^3 = \frac{\sqrt{-3}^3 + 3\sqrt{-3}^2 + 3\sqrt{-3} + 1}{8} = 1 \]
    with $\frac{\sqrt{-3} + 1}{2} \ne 1$, therefore $\mu_3 \subset \Rat(\sqrt{-3})$. 
    As $\mu_6 = \mu_3 \cup \set{-x \mid x \in \mu_3}$, then $\Rat(\mu_6) = \Rat(\mu_3)$, so $\mu_6 \subset \Rat(\sqrt{-3})$.
    Lastly, as $6$ is the largest integer $n$ in which $\phi(n) = 2$, then there is no other root of unity in $\Rat(\sqrt{-3})$.
  }

  \qs{}{
    Let $F$ be a field of characteristic $p \ne 0$ and let $L/F$ be a field extension.
    Show that if $\alpha \in L$ is separarble over $F$, then $F(\alpha) = F(\alpha^p)$
  }
  \sol{
    Generally, $F(\alpha^p) \subset F(\alpha)$, leaving only to show that $F(\alpha) \subset F(\alpha^p)$.
    Let $m_{\alpha, F}$ be the minimal polynomial of $\alpha$ over $F$ and $m_{\alpha, F(\alpha^p)}$ be the minimal polynomial of 
    $\alpha$ over $F(\alpha^p)$. Then, as $F(\alpha^p)$ is an extension of $F$, then 
    \[ m_{\alpha, F(\alpha^p)} \mid m_{\alpha, F} \]

    From the assumption, $\alpha$ is separable over $F$, thus $m_{\alpha, F}$ is separable. This implies that $m_{\alpha, F(\alpha^p)}$ 
    as it divides a separable polynomial. 
    
    However, notice that $x^p - \alpha^p$ has $\alpha$ as a root, and \[ x^p - \alpha^p = (x-\alpha)^p \]
    This means that $m_{\alpha, F(\alpha^p)} = (x - \alpha)^k$ for some $k < p$.
    However, as $m_{\alpha, F(\alpha^p)}$ is separable, then $m_{\alpha, F(\alpha^p)} = x - \alpha$, otherwise, there is a root with 
    multiplicity greater than $1$, thus not separable.

    As $m_{\alpha, F(\alpha^p)} = x - \alpha$, then $F(\alpha^p, \alpha) = F(\alpha)$ is a field extension of degree 1 over 
    $F(\alpha^p)$. This lead to a conclusion that, $F(\alpha^p) = F(\alpha)$.
  }

  \qs{}{
    Let $A$ be the sum of all elements of a finite field $\mathbb F_q$ and let $B$ be the product of all non-zero elements of 
    $\mathbb F_q$. Compute $A + B$. 
  }
  \sol{
    Let $\chr(\mathbb F_q) = p > 0$.
    It is possible to label the elements of $\mathbb F_q$ as $a_1, \ldots, a_q$ such that $a_q = 0$ and for all $i$, 
    $a_i + a_{q-i} = 0$ as every element in a field has a additive inverse. Then, if $p = 2$, then there will be one element 
    in which $i = q-i$, which is $i = \frac{q}{2}$, and $a_{q/2} + a_{q/2} = 0$ gives that $a_{q/2} = q/2$

    For $p \ne 2$, the elements can be label in another way, which is $m_1, \ldots, m_q$ where $m_q = 0$, $m_1 = 1$, $m_{q-1} = -1$ and
    for other $i$, $m_i \cdot m_{q-i} = 1$. This is because any element of a field, apart from $0$ has an inverse. And as for $p \ne 2$,
    there is only 2 roots of $x^2 - 1$, which are $x = 1$ and $x = -1$, then every element apart from $0, 1, -1$ has 
    inverse that is differred from itself. 

    For $p = 2$, there is only 1 root of $x^2 - 1 = (x - 1)^2$, which is $1$, thus, the 
    other $q-2$ elements apart from $1$ and $0$ has an inverse that is distinct from itself. So the label is set such that 
    $m_i \cdot m_{q-i+1} = 1$ for all $2 \le i < q$ instead.

    If $p \ne 2$, there are $q - 1$ non-zero elements, which is even. Therefore, 
    \[ \sum_{a \in \mathbb F_q} a = \sum_{i = 1}^q a_i = a_1 + a_q + \sum_{i = 2}^{q-1} a_i
    = a_1 + a_q + \sum_{i=2}^{\frac{q}{2}} a_i + a_{q-i} = 1 + 0 + \sum_{i=2}^{\frac{q}{2}} 0 = 1  \]

    \[ \prod_{m \in \mathbb F_q^\times} m = \prod_{i = 1}^{q - 1} m_i = m_1 \cdot m_{q - 1} \cdot \prod_{i = 2}^{q - 2} m_i
    = -1 \cdot \prod_{i = 2}^{\frac{q-1}{2}} m_i \cdot m_{q-i} = - \prod_{i = 2}^{\frac{q-1}{2}} 1 = -1 \]
    So $A + B = 0$.

    Otherwise if $p = 2$, then 
    \[ \sum_{a \in \mathbb F_q} a = \sum_{i = 1}^q a_i = a_q + \sum_{i=1}^{q-1} a_i
    = a_q + a_{q/2} \sum_{i = 1}^{\frac{q}{2} - 1} a_i + a_{q-i} = 0 + \frac{q}{2} + \sum_{i=1}^{\frac{q}{2} - 1} 0 = \frac{q}{2}  \]

    \[ \prod_{m \in \mathbb F_q^\times} m = \prod_{i = 1}^{q-1} m_i = q_1 \prod_{i=2}^{q-1} m_i
    = \prod_{i = 2}^{\frac{q}{2}} m_i \cdot m_{q-i+1} =  \prod_{i=2}^{\frac{q}{2}} 1 = 1  \]
    So, $A + B = \frac{q}{2} + 1$
  }

  \qs{}{
    Assume that $\chr(\mathbb F_q) \ne 2$. Prove that $\abs{\set{x^2 \mid x \in \mathbb F_q}} = (\abs{\mathbb F_q} + 1)/2$.
  }
  \sol{
    Since $\chr(\mathbb F_q) \ne 2$, then $q$ is odd. As there is $q$ elements, then $x^2 = (q-x)^2$ for all $0 \le x < q$. 
    Thus, $\abs{\set{x^2 \mid x \in \mathbb F_q}} \le (\abs{\mathbb F_q} + 1)/2$.

    Now, if consider that the polynomial $x^2 - a$ where $a \in \set{x^2 \mid x \in \mathbb F_q}$ is reducible to 
    $x^2 - a = (x - \alpha)(x + \alpha)$ as $x^2 - a$ is monic. This means that no other element, apart from $\alpha$ and $-\alpha$,
    can be the root of $x^2 - a$, and thus, no other element $\beta$ would satisfy $\beta^2 = a$.

    Thus, each of the element of $\set{ x^2 \mid x \in \mathbb F_q}$ corresponds to 2 mutually exclusive elements of $\mathbb F_q$, 
    namely, $x$ and $-x$ (and no other elements).

    This means that $\abs{\set{x^2 \mid x \in \mathbb F_q}} \ge \abs{\mathbb F_q}/2$
    Hence, $\abs{\set{x^2 \mid x \in \mathbb F_q}} = (\abs{\mathbb F_q} + 1)/2$
  }

\end{document}
