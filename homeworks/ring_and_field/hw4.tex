% chktex-file 44
% chktex-file 8

\documentclass{report}
\usepackage{amsthm}
\usepackage{amsmath}
\usepackage{amssymb}
\usepackage{amssymb}
\usepackage{amsfonts}
\usepackage{xcolor}
\usepackage{tikz}
\usepackage{fancyhdr}
\usepackage{enumerate}
\usepackage{graphicx}
\usepackage[normalem]{ulem}
\usepackage[most,many,breakable]{tcolorbox}
\usepackage[a4paper, top=80pt, foot=25pt, bottom=50pt, left=0.5in, right=0.5in]{geometry}
\usepackage{hyperref, theoremref}
\hypersetup{
	pdftitle={Assignment},
	colorlinks=true, linkcolor=b!90,
	bookmarksnumbered=true,
	bookmarksopen=true
}
\usepackage{nameref}
\usepackage{parskip}
\pagestyle{fancy}

\usepackage[explicit,compact]{titlesec}
\titleformat{\chapter}[block]{\bfseries\huge}{\thechapter. }{\compact}{#1}
        

%%%%%%%%%%%%%%%%%%%%%
%% Defining colors %%
%%%%%%%%%%%%%%%%%%%%%

\definecolor{lr}{RGB}{188, 75, 81}
\definecolor{r}{RGB}{249, 65, 68}
\definecolor{dr}{RGB}{174, 32, 18}
\definecolor{lo}{RGB}{255, 172, 129}
\definecolor{do}{RGB}{202, 103, 2}
\definecolor{o}{RGB}{238, 155, 0}
\definecolor{ly}{RGB}{255, 241, 133}
\definecolor{y}{RGB}{255, 229, 31}
\definecolor{dy}{RGB}{143, 126, 0}
\definecolor{lb}{RGB}{148, 210, 189}
\definecolor{bg}{RGB}{10, 147, 150}
\definecolor{b}{RGB}{39, 125, 161}
\definecolor{db}{RGB}{0, 95, 115}
\definecolor{p}{RGB}{229, 152, 155}
\definecolor{dp}{RGB}{181, 101, 118}
\definecolor{pp}{RGB}{142, 143, 184}
\definecolor{v}{RGB}{109, 89, 122}
\definecolor{lg}{RGB}{144, 190, 109}
\definecolor{g}{RGB}{64, 145, 108}
\definecolor{dg}{RGB}{45, 106, 79}

\colorlet{mysol}{g}
\colorlet{mythm}{lr}
\colorlet{myqst}{db}
\colorlet{myclm}{lb}
\colorlet{mywrong}{r}
\colorlet{mylem}{o}
\colorlet{mydef}{lg}
\colorlet{mycor}{lb}
\colorlet{myrem}{dr}

%%%%%%%%%%%%%%%%%%%%%

\newcommand{\col}[2]{
  \color{#1}#2\color{black}\,
}

\newcommand{\TODO}[1][5cm]{
  \color{red}TODO\color{black}
  \vspace{#1}
}

\newcommand{\wans}[1]{
	\noindent\color{mywrong}\textbf{Wrong answer: }\color{black}
	#1 


}

\newcommand{\wreason}[1]{
	\noindent\color{mywrong}\textbf{Reason: }\color{black}
	#1 

  
}

\newcommand{\sol}[1]{
	\noindent\color{mysol}\textbf{Solution: }\color{black}
	#1


}

\newcommand{\nt}[1]{
  \begin{note}Note: #1\end{note}
}

\newcommand{\ky}[1]{
  \begin{key}#1\end{key}
}

\newcommand{\pf}[1]{
  \begin{myproof}#1\end{myproof}
}

\newcommand{\qs}[3][]{
  \begin{question}{#2}{#1}#3\end{question}
}

\newcommand{\df}[3][]{
  \begin{definition}{#2}{#1}#3\end{definition}
}

\newcommand{\thm}[3][]{
  \begin{theorem}{#2}{#1}#3\end{theorem}
}

\newcommand{\clm}[3][]{
  \begin{claim}{#2}{#1}#3\end{claim} 
}

\newcommand{\lem}[3][]{
  \begin{lemma}{#2}{#1}#3\end{lemma}
}

\newcommand{\cor}[3][]{
  \begin{corollary}{#2}{#1}#3\end{corollary}
}

\newcommand{\rem}[3][]{
  \begin{remark}{#2}{#1}#3\end{remark}
}

\newcommand{\twoways}[2]{
  \leavevmode\\
  ($\Longrightarrow$): 
  \begin{shift}#1\end{shift}
  ($\Longleftarrow$):
  \begin{shift}#2\end{shift} 
}

\newcommand{\nways}[2]{
  \leavevmode\\
  ($#1$): 
  \begin{shift}#2\end{shift}
}

%%%%%%%%%%%%%%%%%%%%%%%%%%%%%% ENVRN

\newenvironment{myproof}[1][\proofname]{%
	\proof[\bfseries #1: ]
}{\endproof}

\tcbuselibrary{theorems,skins,hooks}
\newtcolorbox{shift}
{%
  before upper={\setlength{\parskip}{5pt}},
  blanker,
	breakable,
	width=0.95\textwidth,
  enlarge left by=0.03\textwidth,
}

\tcbuselibrary{theorems,skins,hooks}
\newtcolorbox{key}
{%
	breakable,
	width=0.95\textwidth,
  enlarge left by=0.03\textwidth,
}

\tcbuselibrary{theorems,skins,hooks}
\newtcolorbox{note}
{%
	enhanced,
	breakable,
	colback = white,
	width=\textwidth,
	frame hidden,
	borderline west = {2pt}{0pt}{black},
	sharp corners,
}

\tcbuselibrary{theorems,skins,hooks}
\newtcbtheorem[]{remark}{Remark}
{%
	enhanced,
	breakable,
	colback = white,
	frame hidden,
	boxrule = 0sp,
	borderline west = {2pt}{0pt}{myrem},
	sharp corners,
	detach title,
  before upper={\setlength{\parskip}{5pt}\tcbtitle\par\smallskip},
	coltitle = myrem,
	fonttitle = \bfseries\sffamily,
	description font = \mdseries,
	separator sign none,
	segmentation style={solid, myrem},
}{rem}

\tcbuselibrary{theorems,skins,hooks}
\newtcbtheorem[number within=section]{lemma}{Lemma}
{%
	enhanced,
	breakable,
	colback = white,
	frame hidden,
	boxrule = 0sp,
	borderline west = {2pt}{0pt}{mylem},
	sharp corners,
	detach title,
  before upper={\setlength{\parskip}{5pt}\tcbtitle\par\smallskip},
	coltitle = mylem,
	fonttitle = \bfseries\sffamily,
	description font = \mdseries,
	separator sign none,
	segmentation style={solid, mylem},
}{lem}

\tcbuselibrary{theorems,skins,hooks}
\newtcbtheorem{claim}{Claim}
{%
  parbox=false,
	enhanced,
	breakable,
	colback = white,
	frame hidden,
	boxrule = 0sp,
	borderline west = {2pt}{0pt}{myclm},
	sharp corners,
	detach title,
  before upper={\setlength{\parskip}{5pt}\tcbtitle\par\smallskip},
	coltitle = myclm,
	fonttitle = \bfseries\sffamily,
	description font = \mdseries,
	separator sign none,
	segmentation style={solid, myclm},
}{clm}

\makeatletter
\newtcbtheorem[number within=section, use counter from=lemma]{theorem}{Theorem}{enhanced,
	breakable,
	colback=white,
	colframe=mythm,
	attach boxed title to top left={yshift*=-\tcboxedtitleheight},
	fonttitle=\bfseries,
	title={#2},
	boxed title size=title,
	boxed title style={%
			sharp corners,
			rounded corners=northwest,
			colback=mythm,
			boxrule=0pt,
		},
	underlay boxed title={%
			\path[fill=mythm] (title.south west)--(title.south east)
			to[out=0, in=180] ([xshift=5mm]title.east)--
			(title.center-|frame.east)
			[rounded corners=\kvtcb@arc] |-
			(frame.north) -| cycle;
		},
	#1
}{thm}
\makeatother

\makeatletter
\newtcbtheorem{question}{Question}{enhanced,
	breakable,
	colback=white,
	colframe=myqst,
	attach boxed title to top left={yshift*=-\tcboxedtitleheight},
	fonttitle=\bfseries,
	title={#2},
	boxed title size=title,
	boxed title style={%
			sharp corners,
			rounded corners=northwest,
			colback=myqst,
			boxrule=0pt,
		},
	underlay boxed title={%
			\path[fill=myqst] (title.south west)--(title.south east)
			to[out=0, in=180] ([xshift=5mm]title.east)--
			(title.center-|frame.east)
			[rounded corners=\kvtcb@arc] |-
			(frame.north) -| cycle;
		},
	#1
}{qs}
\makeatother

\makeatletter
\newtcbtheorem[number within=section]{definition}{Definition}{enhanced,
	breakable,
	colback=white,
	colframe=mydef,
	attach boxed title to top left={yshift*=-\tcboxedtitleheight},
	fonttitle=\bfseries,
	title={#2},
	boxed title size=title,
	boxed title style={%
			sharp corners,
			rounded corners=northwest,
			colback=mydef,
			boxrule=0pt,
		},
	underlay boxed title={%
			\path[fill=mydef] (title.south west)--(title.south east)
			to[out=0, in=180] ([xshift=5mm]title.east)--
			(title.center-|frame.east)
			[rounded corners=\kvtcb@arc] |-
			(frame.north) -| cycle;
		},
	#1
}{def}
\makeatother

\makeatletter
\newtcbtheorem[number within=section, use counter from=lemma]{corollary}{Corollary}{enhanced,
	breakable,
	colback=white,
	colframe=mycor,
	attach boxed title to top left={yshift*=-\tcboxedtitleheight},
	fonttitle=\bfseries,
	title={#2},
	boxed title size=title,
	boxed title style={%
			sharp corners,
			rounded corners=northwest,
			colback=mycor,
			boxrule=0pt,
		},
	underlay boxed title={%
			\path[fill=mycor] (title.south west)--(title.south east)
			to[out=0, in=180] ([xshift=5mm]title.east)--
			(title.center-|frame.east)
			[rounded corners=\kvtcb@arc] |-
			(frame.north) -| cycle;
		},
	#1
}{cor}
\makeatother

% Basic
  \DeclareMathOperator{\lcm}{lcm}
  \newcommand{\Real}{\mathbb{R}}
  \newcommand{\Comp}{\mathbb{C}}
  \newcommand{\Nat}{\mathbb{N}}
  \newcommand{\Rat}{\mathbb{Q}}
  \newcommand{\Int}{\mathbb{Z}}
  \newcommand{\set}[1]{\left\{\, #1 \,\right\}}
  \newcommand{\paren}[1]{\left( \; #1 \; \right)}
  \newcommand{\abs}[1]{\left\lvert #1 \right\rvert}
  \newcommand{\ang}[1]{\left\langle #1 \right\rangle}
  \renewcommand{\to}[1][]{\xrightarrow{\text{#1}}}
  \newcommand{\tol}[1][]{\to{$#1$}}
  \newcommand{\curle}{\preccurlyeq}
  \newcommand{\curge}{\succcurlyeq}
  \newcommand{\mapsfrom}{\leftarrow\!\shortmid}

  \newcommand{\mat}[1]{\begin{bmatrix} #1 \end{bmatrix}}
  \newcommand{\pmat}[1]{\begin{pmatrix} #1 \end{pmatrix}}
  \newcommand{\eqs}[1]{\begin{align*} #1 \end{align*}}
  \newcommand{\case}[1]{\begin{cases} #1 \end{cases}}
  

  % Algebra
  \newcommand{\normSg}[0]{\vartriangleleft}
  \newcommand{\ZMod}[1][n]{\mathbb{Z}/#1\mathbb{Z}}
  \newcommand{\isom}{\simeq}
  \newcommand{\mapHom}{\xrightarrow{\text{hom}}}
  \DeclareMathOperator{\Inn}{Inn}
  \DeclareMathOperator{\Aut}{Aut}
  \DeclareMathOperator{\im}{im}
  \DeclareMathOperator{\ord}{ord}
  \DeclareMathOperator{\Gal}{Gal}
  \DeclareMathOperator{\chr}{char}
  \newcommand{\surjto}{\twoheadrightarrow}
  \newcommand{\injto}{\hookrightarrow}

  % Analysis 
  \newcommand{\limty}[1][k]{\lim_{#1\to\infty}}
  \newcommand{\norm}[1]{\left\lVert#1\right\rVert}
  \newcommand{\darrow}{\rightrightarrows}


\fancyhead[L]{Modern Algebra - MAS312}
\fancyhead[R]{\textbf{Touch Sungkawichai} 20210821}

\begin{document}

\qs{}{
  Explain why a field homomorphism is either injective or trivial
}
\sol{
  Let $\phi: F \to E$ be a homomorphism, then $\ker\phi$ is an ideal of $F$. But an ideal of $F$ is either
  $\set{0}$ or $F$ as $F$ is a field. If $\ker\phi = \set{0}$, then $\phi$ is injective. Otherwise, $\ker\phi = F$ means 
  $\phi: f \mapsto 0$, which is that $\phi$ is trivial.
}

\qs{}{
  Let $m$ and $k$ be relatively prime positive integers and let $a \in F$, where $F$ is a field. 
  Show that both polynomials $x^m - a$ and $x^k - a$ are irreducible over $F$ if and only if $x^{mk} - a$ is 
  irreducible over $F$.
}
\sol{
  \twoways{
    Let $x^m - a$ and $x^k - a$ be irreducible and $\alpha$ be a root of $x^{mk} - a$. Therefore, $\alpha^{mk} = a$, which means 
    that $\alpha^{m}$ is a root of $x^k - a$. Now, $\alpha^m \notin F$ therefore $F(\alpha^m)/F$ is an extension with 
    $[F(\alpha^m): F] = k$. This is because the minimal polynomial of $\alpha^m$ is $x^k - a$ which is a degree $k$ polynomial. 
    Similarly, the field extension $F(\alpha^k)/F$ is an extension with $[F(\alpha^k): F] = m$.

    Now, $[F(\alpha^m, \alpha^k): F(\alpha^m)] \le m$ and $[F(\alpha^m, \alpha^k): F(\alpha^k)] \le n$, therefore, 
    \[ [F(\alpha^m, \alpha^k): F] = [F(\alpha^m, \alpha^k): F(\alpha^m)][F(\alpha^m): F] = [F(\alpha^m, \alpha^k): F(\alpha^k)]
    [F(\alpha^k): F] \]
    But since $m$ and $k$ is coprime, then $[F(\alpha^m, \alpha^k): F] = mk$ since it must be divisible by both $m$ and $k$.

    However, consider that $F(\alpha^m, \alpha^k) = F(\alpha)$ as there exist $a, b$ making $am + bk = 1$ as they are coprime which 
    makes $\alpha^{am}\alpha^{bk} = \alpha \in F(\alpha^m, \alpha^k)$.

    Therefore, $[F(\alpha): F] = mk$ which means that the minimal polynomial of $\alpha$ should be of degree $mk$. Therefore, 
    as the minimal polynomial of $\alpha$ must divide $x^{mk} - a$, then it is $x^{mk} - a$.
    This means that $x^{mk} - a$ is irreducible.

  }{
    Assume that $x^m - a$ is reducible. Then, $x^m - a = f(x)g(x)$ where both $f, g$ are not unit.
    Then, consider $x^{mk} - a = {(x^k)}^m - a = f(x^k)g(x^k)$. 
    Notice that $f(x^k)$ cannot be a unit since $f(x^k)$ is not a constant in $F$ because $f(x)$ is at least degree $1$, and so is $g$.
    Therefore, $x^{mk} - a$ is reducible.
  }
}

\qs{}{
  Let $a, b \in E/F$ be nonzero elements. Show that $F(a, b)/F(a^{-1}b^{-1}, a + b)$ is an algebraic extension. 
}
\sol{
  Consider that $(a^{-1}b^{-1})(a + b) = (a^{-1} + b^{-1})$, so $a^{-1} + b^{-1} \in F(a^{-1}b^{-1}, a + b)$. Also, 
  $1 \in F(a^{-1}b^{-1}, a + b)$.

  Consider a polynomial $p(x) = (a^{-1}b^{-1})x^2 - (a^{-1} + b^{-1})x + 1$. It is easy to see that $p(x) \in F(a^{-1}b^{-1}, a + b)[x]$.

  Notice that $(a^{-1}b^{-1})a^2 - (a^{-1} + b^{-1})a + 1 = ab^{-1} - (1 + ab^{-1}) + 1 = 0$, and similarly, 
  $(a^{-1}b^{-1})b^2 - (a^{-1} + b^{-1})b + 1 = 0$. So, $a$ and $b$ are the root of polynomial $p$.

  Since $a$ and $b$ is algebraic over $F(a^{-1}b^{-1}, a + b)$, then $F(a^{-1}b^{-1}, a + b)(a, b)$ is an algebraic extension of 
  $F(a^{-1}b^{-1}, a + b)$. 
  Lastly, it can be shown that $F(a, b) = F(a^{-1}b^{-1}, a + b, a, b)$ as $F(a, b) \subset F(a^{-1}b^{-1}, a + b, a, b)$ trivially and 
  $a^{-1}b^{-1} \in F(a, b)$ and $a + b \in F(a, b)$.

  Therefore, $F(a, b)$ is an algebraic extension of $F(a^{-1}b^{-1}, a + b)$.
}

\qs{}{
  Find the degree of a splitting field of $x^3 - 17$ over $\Rat$.
}
\sol{
  Consider that $x^3 - 17 = \paren{x - \sqrt[3]{17}}\paren{x - \eta\sqrt[3]{17}}\paren{x - \eta^2\sqrt[3]{17}}$ where $\eta$ is the primitive third root
  of $1$. Let $E$ be a splitting field over $\Rat$. Then $E$ must contain $\sqrt[3]{17}$ and $\eta\sqrt[3]{17}$. Therefore, $E$ must
  contain $\eta$. However, if $E$ contains $\sqrt[3]{17}$ and $\eta$, then $E$ contain $\eta\sqrt[3]{17}$ and $\eta^2\sqrt[3]{17}$, 
  which means that $x^3-17$ splits in $E$, thus $E = \Rat(\sqrt[3]{17}, \eta)$ is the splitting field of $x^3 - 17$ over $\Rat$.

  Now, the set $\set{1, \sqrt[3]{17}, \sqrt[3]{17}^2}$ is a basis of $\Rat(\sqrt[3]{17})/\Rat$. This is because
  $\Rat(\sqrt[3]{17}) = \set{a + b\sqrt[3]{17} + c\sqrt[3]{17}^2 \mid a, b, c \in \Rat}$ as $\sqrt[3]{17}^3 = 17 \in \Rat$ and 
  $\sqrt[3]{17}^{-1} = \frac{\sqrt[3]{17}^2}{17}$. So, the set spans $\Rat(\sqrt[3]{17})$.

  Moreover, the set is a linearly independent set. The reason being that $\set{1, \sqrt[3]{17}}$ is linearly independent over $\Rat$ 
  since $\sqrt[3]{17} \notin \Rat$. Then, if $\sqrt[3]{17}^2 = a + b\sqrt[3]{17}$ for some $a, b \in \Rat$, 
  \eqs{ 
    17 = \sqrt[3]{17}^3 &= (\sqrt[3]{17})(a + b\sqrt[3]{17}) \\ 
                        &= b\sqrt[3]{17}^2 + a\sqrt[3]{17} \\ 
                        &= b(a + b\sqrt[3]{17}) + a\sqrt[3]{17} \\ 
                        &= (a + b^2)\sqrt[3]{17} + ba 
  } 
  which means that $\sqrt[3]{17} \in \Rat$. This implication creates contradiction, so the set must be linearly
  independent.

  Next, since $\eta = e^{\frac{2i\pi}{3}}$ is the primitive third root,
  the set $\set{1, \eta}$ is a basis of $\Rat(\sqrt[3]{17}, \eta)/\Rat(\sqrt[3]{17})$. This is because is field is 
  spans by $\set{1, \eta, \eta^2}$ as $\eta^3 = 1$ and $\eta^{-1} = \eta^2$. However, $\eta^2 = -\eta - 1$. 
  Moreover, $\eta \in \Comp-\Real$, so $\set{1, \eta}$ is linearly independent, thus, the set is a basis for the field.

  Since $[\Rat(\sqrt[3]{17}, \eta): \Rat(\sqrt[3]{17})] = 2$ and $[\Rat(\sqrt[3]{17}) : \Rat] = 3$, then $[E : \Rat] = 6$.
  So, the degree of a splitting field of $x^3 - 17$ over $\Rat$ is $6$.
}

\qs{}{
  Let $\xi \in \Comp$ be a primitive $n$th root of unity. Prove that $\Rat(\xi)$ is a splitting field of $x^n-1$ over $\Rat$.
}
\sol{
  Notice that $\xi = e^{\frac{2i\pi}{n}}$ is a primitive $n$th root of unity because $\xi^{n} = 1$ and $\xi^{i} \ne \xi^{j}$ for 
  $i, j \in \set{0, \cdots, n-1}$ such that $i \ne j$.

  Next, since $(\xi^i)^n - 1 = (\xi^n)^i - 1 = 1 - 1 = 0$, then $(x - \xi^i)$ divides $(x^n - 1)$. Moreover, since $\xi^i \ne \xi^j$,
  then $(x - \xi^0)(x - \xi^1)\cdots(x - \xi^{n-1})$ divides $(x^n-1)$. But both polynomial have the same degree, 
  so it leads to concluding that 
  \[ (x^n - 1) = (x - \xi^0)(x - \xi^1)\cdots(x - \xi^{n-1}) \]

  Now, as $\set{\xi^0, \cdots, \xi^{n-1}}$ is the set of all root of $x^n - 1$, it follows that $\Rat(\xi^0, \cdots, \xi^{n-1})$ is the
  smallest field containing all root of $x^n - 1$. Therefore, it is the splitting field of $x^n - 1$.

  Lastly, since $\xi^i \in \Rat(\xi)$ for all $i \in \set{0, \cdots, n-1}$, $\Rat(\xi^0, \cdots, \xi_{n-1}) = \Rat(\xi)$
}

\qs{}{
  Let $f(x) = x^6 - 5x^3 - 2$ be a polynomial in $\Rat[x]$. Find the splitting field $E$ of $f(x)$ over $\Rat$. 
  Compute $[E: \Rat]$.
}
\sol{
  Notice that 
  \[ x^6 - 5x^3 - 2  = \paren{x^3 + \frac{5 + \sqrt{33}}{2}}\paren{x^3 + \frac{5 - \sqrt{33}}{2}} \]  
  And $x^3 + \alpha = \paren{x + \eta\sqrt[3]{\alpha}}\paren{x + \eta^2\sqrt[3]{\alpha}}\paren{x + \eta^3\sqrt[3]{\alpha}}$ where 
  $\eta$ is a primitive third root.
  
  For simplicity, let denote $\alpha = \sqrt[3]{\frac{5 + \sqrt{33}}{2}}$ and $\alpha' = \sqrt[3]{\frac{5 - \sqrt{33}}{2}}$.
  Then, the roots of polynomials are $\alpha, \alpha\eta, \alpha\eta^2, \alpha', \alpha'\eta, \alpha'\eta^2$.

  If a field contains all roots, then it must contains $\alpha$ and $\alpha\eta$, thus it must contains $\eta$.
  Consider the field $E = \Rat(\alpha, \alpha', \eta)$. Then, $E$ contains 
  $\alpha, \alpha\eta, \alpha\eta^2, \alpha', \alpha'\eta, \alpha'\eta^2$. Therefore, it is the smallest field containing all roots of 
  the polynomial. Thus, it is the splitting field.

  The splitting field is $\Rat\paren{\sqrt[3]{\frac{5 + \sqrt{33}}{2}}, \sqrt[3]{\frac{5 - \sqrt{33}}{2}}, \eta}$

  To compute the degree, first consider $[\Rat(\sqrt{33}): \Rat]$. The degree of that extension is $2$ since the basis of the 
  vector space is $\set{1, \sqrt{33}}$ since it spans the space by definition, 
  (as $\sqrt{33}^2 \in \Rat$, $\sqrt{33}^{-1} = \sqrt{33}/33$).

  Next, consider $\left[ \Rat\paren{\sqrt[3]{\frac{5 + \sqrt{33}}{2}}} : \Rat(\sqrt{33}) \right]$. 
  Notice that $\set{1, \alpha}$ is linearly independent as $(a + b\sqrt{33}) \ne \alpha$ for any $a, b \in \Rat$.
  Since $\set{1, \alpha}$ is linearly independent, it can be shown that $\set{1, \alpha, \alpha^2}$ is also linearly independent,
  because otherwise, if $\alpha^2 = a\alpha + b$ for some $a, b \in \Rat(\sqrt{33})$, then 
  \eqs{ \alpha^3 &= a\alpha^2 + b\alpha \\ 
                 &= a(a\alpha + b) + b\alpha \\ 
                 &= (a^2 + b)\alpha + ab
               }
  which contradicts that ${1, \alpha}$ is linearly independent.
  Moreover, thes set $\set{1, \alpha, \alpha^2}$ spans the space since $\alpha^{-1} = \alpha^2/\alpha^3$ and 
  $\alpha^3 \in \Rat(\sqrt{33})$. Therefore, the degree of the extension is $3$.

  Next, consider the extension $\left[  \Rat\paren{\alpha, \alpha'} : \Rat\paren{\alpha}  \right]$.
  Notice that the set $\set{1, \alpha'}$ is linearly independent, and similarly to the above proof, $\set{1, \alpha', \alpha'^2}$ is a 
  linearly independent set. Moreover, it spans the space since $\alpha'^{-1} = \alpha'^2/\alpha'^3$ and $\alpha'^3 \in \Rat(\alpha)$.
  Therefore, the degree of the extension is $3$.

  Lastly, the extension $\left[ \Rat(\alpha, \alpha', \eta) : \Rat(\alpha, \alpha') \right]$ is $2$ since $\set{1, \eta}$ is clearly 
  independent because $\eta \in \Comp - \Real$ and $\Rat(\alpha, \alpha') \subset \Real$. Moreover, the set spans the space because 
  $\eta^{-1} = \eta^2 = -\eta - 1$ since it is the primitive third root.

  Therefore, the degree of \[ [E : \Rat] = [\Rat(\alpha, \alpha', \eta) : \Rat(\alpha, \alpha')][\Rat(\alpha, \alpha') : \Rat(\alpha)][\Rat(\alpha): \Rat(\sqrt{33})][\Rat(\sqrt{33}) : \Rat] \]
  which equates to $2 \cdot 3 \cdot 3 \cdot 2$, which is $36$.

}

\qs{}{
  Show that the field extension $\Rat(\sqrt{3} + \sqrt{7})$ over $\Rat$ is normal.
}
\sol{
  Since $3$ and $7$ are prime, $\Rat(\sqrt{3} + \sqrt{7}) = \Rat(\sqrt{3}, \sqrt{7})$.
  Consider the minimal polynomial, $m_{\sqrt{3}}(x) = x^2 - 3$ and $m_{\sqrt{7}}(x) = x^2 - 7$. The polynomials are minimal since 
  $\sqrt{3}$ and $\sqrt{7}$ is not rational, thus the minimal polynomial cannot be linear.

  Now, a field extension is normal if and only if it splits the product of all minimal polynomials, thus, $\Rat(\sqrt{3}, \sqrt{7})$ 
  is normal if and only if $(x^2-3)(x^2-7)$ splits. Which they split since
  \[ (x^2 - 3)(x^2 - 7) = (x - \sqrt{3})(x + \sqrt{3})(x - \sqrt{7})(x + \sqrt{7}) \]
  Since $\pm \sqrt{3}, \pm\sqrt{7} \in \Rat(\sqrt{3}, \sqrt{7})$, then the field is normal over $\Rat$.

  As $\Rat(\sqrt{3} + \sqrt{7}) = \Rat(\sqrt{3}, \sqrt{7})$, the field $\Rat(\sqrt{3} + \sqrt{7})$ is normal over $\Rat$.
}

\qs{}{
  Let $f$ be an irreducible polynomial over a field $F$. Prove that if $E/F$ is normal, then $f$ factors into a product
  of irreducible polynomials of the same degree over $E$.
}
\sol{
  Let $E/F$ be normal so that for any extension $L/E$ and morphism $\phi: E \to L$ with $\phi \mid_F = id_F$, $\phi(E) = E$. 
  Let $f$ be an irreducible in $F$ such that it is $f = g_1 \cdots g_k$ over $E$. Then, denote $\alpha_i$ as a root of $g_i$.

  Consider an extension $\phi: E \to L$ of $id_F$ such that $\phi(\alpha_1) = \alpha_i$. Note that the extension is a well-defined
  homomorphism since $\alpha_i \not\in F$ and is the only point not in $F$ that the image is specified. Since the field $E$ is normal, 
  $\phi(E) = E$, therefore 
  \[ g_1(\alpha_1) = 0 = \phi(g_1)(\phi(\alpha_1)) = \phi(g_1)(\alpha_i) \]
  So, $g_i \mid \phi(g_1)$. But $\phi(g_1)$ is irreducible, so $g_i \sim \phi(g_1)$, which is that $\deg(g_1) = \deg(g_i)$. 

  Since $i$ is arbitrary, for any $i$, $\deg(g_1) = \deg(g_1)$, so the degree of every factor of $f$ is equal.
}

\qs{}{
  Let $\alpha = \sqrt{3 + \sqrt{3}}$. Find the normal closure of $\Rat(\alpha)/\Rat$. 
}
\sol{
  Consider 
  \[ f(x) = x^4 - 6x^2 + 6 = \paren{x - \sqrt{3 - \sqrt{3}}}\paren{x - \sqrt{3 + \sqrt{3}}}
  \paren{x + \sqrt{3 - \sqrt{3}}}\paren{x + \sqrt{3 + \sqrt{3}}} \]

  Then, any product of 1 linear factor is not in $\Rat[x]$. Any product of 2 linear factors results the constant term being either 
  $\pm 3 \pm \sqrt{3}$ or $\pm \sqrt{3 + \sqrt{3}}\sqrt{3 - \sqrt{3}} = \pm \sqrt{6}$ which is not in $\Rat$.
  The product of 3 linear factors contain the constant term of $\pm \sqrt{3 \pm \sqrt{3}}\sqrt{6} = \pm 3\sqrt{2 \pm 6\sqrt{3}}$ 
  which is not in $\Rat$. So, $f(x)$ is the minimal polynomial of $\alpha$.

  Let $\bar\alpha = \sqrt{3 - \sqrt{3}}$.
  As $\Rat(\alpha, \bar\alpha)$ is a splitting field of of $m_\alpha$, then 
  $\Rat(\alpha, \bar\alpha)/\Rat(\alpha)$ is the normal closure of $\Rat(\alpha)/\Rat$.
}

\qs{}{
  Find a normal closure of $\Rat(\sqrt[4]{11})/\Rat$
}
\sol{
  Consider that $m_{\sqrt[4]{11}}$ must divide $f(x) = x^4 - 11$ since $f(\sqrt[4]{11}) = 0$. 
  Now, \[x^4-11 = (x^2 - \sqrt{11})(x^2 + \sqrt{11}) = (x - \sqrt[4]{11})(x + \sqrt[4]{11})(x - i\sqrt[4]{11})(x + i\sqrt[4]{11}) \]
  It can be seen that $m_{\sqrt[4]{11}} = f$ since any combinations of 3 or less factors of the splits will result in the constant term 
  being $\pm\sqrt[4]{11}^3$ or $\pm i \sqrt[4]{11}^3$ which is not an element of $\Rat$.

  Since $\pm i\sqrt[4]{11} \notin \Rat$, then it is clear that $\Rat(\sqrt[4]{11})$ is not normal. 
  Moreover, the normal closure should split $m_{\sqrt[4]{11}}$, so the normal closure must contain $i\sqrt[4]{11}$. 

  Now, if $N/\Rat(\sqrt[4]{11})/\Rat$ contains $i\sqrt[4]{11}$, then it contains $-i\sqrt[4]{11}$ and $-\sqrt[4]{11}$ by the property of 
  field. Thus, $N = \Rat(\sqrt[4]{11}, i\sqrt[4]{11})$ splits $m_{\sqrt[4]{11}}$.
  
  Therefore, $\Rat(\sqrt[4]{11}, i\sqrt[4]{11})/\Rat(\sqrt[4]{11})$ is a normal closure of $\Rat(\sqrt[4]{11})/\Rat$
}
\end{document}
