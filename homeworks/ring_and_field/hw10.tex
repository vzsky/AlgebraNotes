% chktex-file 44
% chktex-file 8

\documentclass{report}
\usepackage{amsthm}
\usepackage{amsmath}
\usepackage{amssymb}
\usepackage{amssymb}
\usepackage{amsfonts}
\usepackage{xcolor}
\usepackage{tikz}
\usepackage{fancyhdr}
\usepackage{enumerate}
\usepackage{graphicx}
\usepackage[normalem]{ulem}
\usepackage[most,many,breakable]{tcolorbox}
\usepackage[a4paper, top=80pt, foot=25pt, bottom=50pt, left=0.5in, right=0.5in]{geometry}
\usepackage{hyperref, theoremref}
\hypersetup{
	pdftitle={Assignment},
	colorlinks=true, linkcolor=b!90,
	bookmarksnumbered=true,
	bookmarksopen=true
}
\usepackage{nameref}
\usepackage{parskip}
\pagestyle{fancy}

\usepackage[explicit,compact]{titlesec}
\titleformat{\chapter}[block]{\bfseries\huge}{\thechapter. }{\compact}{#1}
        

%%%%%%%%%%%%%%%%%%%%%
%% Defining colors %%
%%%%%%%%%%%%%%%%%%%%%

\definecolor{lr}{RGB}{188, 75, 81}
\definecolor{r}{RGB}{249, 65, 68}
\definecolor{dr}{RGB}{174, 32, 18}
\definecolor{lo}{RGB}{255, 172, 129}
\definecolor{do}{RGB}{202, 103, 2}
\definecolor{o}{RGB}{238, 155, 0}
\definecolor{ly}{RGB}{255, 241, 133}
\definecolor{y}{RGB}{255, 229, 31}
\definecolor{dy}{RGB}{143, 126, 0}
\definecolor{lb}{RGB}{148, 210, 189}
\definecolor{bg}{RGB}{10, 147, 150}
\definecolor{b}{RGB}{39, 125, 161}
\definecolor{db}{RGB}{0, 95, 115}
\definecolor{p}{RGB}{229, 152, 155}
\definecolor{dp}{RGB}{181, 101, 118}
\definecolor{pp}{RGB}{142, 143, 184}
\definecolor{v}{RGB}{109, 89, 122}
\definecolor{lg}{RGB}{144, 190, 109}
\definecolor{g}{RGB}{64, 145, 108}
\definecolor{dg}{RGB}{45, 106, 79}

\colorlet{mysol}{g}
\colorlet{mythm}{lr}
\colorlet{myqst}{db}
\colorlet{myclm}{lb}
\colorlet{mywrong}{r}
\colorlet{mylem}{o}
\colorlet{mydef}{lg}
\colorlet{mycor}{lb}
\colorlet{myrem}{dr}

%%%%%%%%%%%%%%%%%%%%%

\newcommand{\col}[2]{
  \color{#1}#2\color{black}\,
}

\newcommand{\TODO}[1][5cm]{
  \color{red}TODO\color{black}
  \vspace{#1}
}

\newcommand{\wans}[1]{
	\noindent\color{mywrong}\textbf{Wrong answer: }\color{black}
	#1 


}

\newcommand{\wreason}[1]{
	\noindent\color{mywrong}\textbf{Reason: }\color{black}
	#1 

  
}

\newcommand{\sol}[1]{
	\noindent\color{mysol}\textbf{Solution: }\color{black}
	#1


}

\newcommand{\nt}[1]{
  \begin{note}Note: #1\end{note}
}

\newcommand{\ky}[1]{
  \begin{key}#1\end{key}
}

\newcommand{\pf}[1]{
  \begin{myproof}#1\end{myproof}
}

\newcommand{\qs}[3][]{
  \begin{question}{#2}{#1}#3\end{question}
}

\newcommand{\df}[3][]{
  \begin{definition}{#2}{#1}#3\end{definition}
}

\newcommand{\thm}[3][]{
  \begin{theorem}{#2}{#1}#3\end{theorem}
}

\newcommand{\clm}[3][]{
  \begin{claim}{#2}{#1}#3\end{claim} 
}

\newcommand{\lem}[3][]{
  \begin{lemma}{#2}{#1}#3\end{lemma}
}

\newcommand{\cor}[3][]{
  \begin{corollary}{#2}{#1}#3\end{corollary}
}

\newcommand{\rem}[3][]{
  \begin{remark}{#2}{#1}#3\end{remark}
}

\newcommand{\twoways}[2]{
  \leavevmode\\
  ($\Longrightarrow$): 
  \begin{shift}#1\end{shift}
  ($\Longleftarrow$):
  \begin{shift}#2\end{shift} 
}

\newcommand{\nways}[2]{
  \leavevmode\\
  ($#1$): 
  \begin{shift}#2\end{shift}
}

%%%%%%%%%%%%%%%%%%%%%%%%%%%%%% ENVRN

\newenvironment{myproof}[1][\proofname]{%
	\proof[\bfseries #1: ]
}{\endproof}

\tcbuselibrary{theorems,skins,hooks}
\newtcolorbox{shift}
{%
  before upper={\setlength{\parskip}{5pt}},
  blanker,
	breakable,
	width=0.95\textwidth,
  enlarge left by=0.03\textwidth,
}

\tcbuselibrary{theorems,skins,hooks}
\newtcolorbox{key}
{%
	breakable,
	width=0.95\textwidth,
  enlarge left by=0.03\textwidth,
}

\tcbuselibrary{theorems,skins,hooks}
\newtcolorbox{note}
{%
	enhanced,
	breakable,
	colback = white,
	width=\textwidth,
	frame hidden,
	borderline west = {2pt}{0pt}{black},
	sharp corners,
}

\tcbuselibrary{theorems,skins,hooks}
\newtcbtheorem[]{remark}{Remark}
{%
	enhanced,
	breakable,
	colback = white,
	frame hidden,
	boxrule = 0sp,
	borderline west = {2pt}{0pt}{myrem},
	sharp corners,
	detach title,
  before upper={\setlength{\parskip}{5pt}\tcbtitle\par\smallskip},
	coltitle = myrem,
	fonttitle = \bfseries\sffamily,
	description font = \mdseries,
	separator sign none,
	segmentation style={solid, myrem},
}{rem}

\tcbuselibrary{theorems,skins,hooks}
\newtcbtheorem[number within=section]{lemma}{Lemma}
{%
	enhanced,
	breakable,
	colback = white,
	frame hidden,
	boxrule = 0sp,
	borderline west = {2pt}{0pt}{mylem},
	sharp corners,
	detach title,
  before upper={\setlength{\parskip}{5pt}\tcbtitle\par\smallskip},
	coltitle = mylem,
	fonttitle = \bfseries\sffamily,
	description font = \mdseries,
	separator sign none,
	segmentation style={solid, mylem},
}{lem}

\tcbuselibrary{theorems,skins,hooks}
\newtcbtheorem{claim}{Claim}
{%
  parbox=false,
	enhanced,
	breakable,
	colback = white,
	frame hidden,
	boxrule = 0sp,
	borderline west = {2pt}{0pt}{myclm},
	sharp corners,
	detach title,
  before upper={\setlength{\parskip}{5pt}\tcbtitle\par\smallskip},
	coltitle = myclm,
	fonttitle = \bfseries\sffamily,
	description font = \mdseries,
	separator sign none,
	segmentation style={solid, myclm},
}{clm}

\makeatletter
\newtcbtheorem[number within=section, use counter from=lemma]{theorem}{Theorem}{enhanced,
	breakable,
	colback=white,
	colframe=mythm,
	attach boxed title to top left={yshift*=-\tcboxedtitleheight},
	fonttitle=\bfseries,
	title={#2},
	boxed title size=title,
	boxed title style={%
			sharp corners,
			rounded corners=northwest,
			colback=mythm,
			boxrule=0pt,
		},
	underlay boxed title={%
			\path[fill=mythm] (title.south west)--(title.south east)
			to[out=0, in=180] ([xshift=5mm]title.east)--
			(title.center-|frame.east)
			[rounded corners=\kvtcb@arc] |-
			(frame.north) -| cycle;
		},
	#1
}{thm}
\makeatother

\makeatletter
\newtcbtheorem{question}{Question}{enhanced,
	breakable,
	colback=white,
	colframe=myqst,
	attach boxed title to top left={yshift*=-\tcboxedtitleheight},
	fonttitle=\bfseries,
	title={#2},
	boxed title size=title,
	boxed title style={%
			sharp corners,
			rounded corners=northwest,
			colback=myqst,
			boxrule=0pt,
		},
	underlay boxed title={%
			\path[fill=myqst] (title.south west)--(title.south east)
			to[out=0, in=180] ([xshift=5mm]title.east)--
			(title.center-|frame.east)
			[rounded corners=\kvtcb@arc] |-
			(frame.north) -| cycle;
		},
	#1
}{qs}
\makeatother

\makeatletter
\newtcbtheorem[number within=section]{definition}{Definition}{enhanced,
	breakable,
	colback=white,
	colframe=mydef,
	attach boxed title to top left={yshift*=-\tcboxedtitleheight},
	fonttitle=\bfseries,
	title={#2},
	boxed title size=title,
	boxed title style={%
			sharp corners,
			rounded corners=northwest,
			colback=mydef,
			boxrule=0pt,
		},
	underlay boxed title={%
			\path[fill=mydef] (title.south west)--(title.south east)
			to[out=0, in=180] ([xshift=5mm]title.east)--
			(title.center-|frame.east)
			[rounded corners=\kvtcb@arc] |-
			(frame.north) -| cycle;
		},
	#1
}{def}
\makeatother

\makeatletter
\newtcbtheorem[number within=section, use counter from=lemma]{corollary}{Corollary}{enhanced,
	breakable,
	colback=white,
	colframe=mycor,
	attach boxed title to top left={yshift*=-\tcboxedtitleheight},
	fonttitle=\bfseries,
	title={#2},
	boxed title size=title,
	boxed title style={%
			sharp corners,
			rounded corners=northwest,
			colback=mycor,
			boxrule=0pt,
		},
	underlay boxed title={%
			\path[fill=mycor] (title.south west)--(title.south east)
			to[out=0, in=180] ([xshift=5mm]title.east)--
			(title.center-|frame.east)
			[rounded corners=\kvtcb@arc] |-
			(frame.north) -| cycle;
		},
	#1
}{cor}
\makeatother

% Basic
  \DeclareMathOperator{\lcm}{lcm}
  \newcommand{\Real}{\mathbb{R}}
  \newcommand{\Comp}{\mathbb{C}}
  \newcommand{\Nat}{\mathbb{N}}
  \newcommand{\Rat}{\mathbb{Q}}
  \newcommand{\Int}{\mathbb{Z}}
  \newcommand{\set}[1]{\left\{\, #1 \,\right\}}
  \newcommand{\paren}[1]{\left( \; #1 \; \right)}
  \newcommand{\abs}[1]{\left\lvert #1 \right\rvert}
  \newcommand{\ang}[1]{\left\langle #1 \right\rangle}
  \renewcommand{\to}[1][]{\xrightarrow{\text{#1}}}
  \newcommand{\tol}[1][]{\to{$#1$}}
  \newcommand{\curle}{\preccurlyeq}
  \newcommand{\curge}{\succcurlyeq}
  \newcommand{\mapsfrom}{\leftarrow\!\shortmid}

  \newcommand{\mat}[1]{\begin{bmatrix} #1 \end{bmatrix}}
  \newcommand{\pmat}[1]{\begin{pmatrix} #1 \end{pmatrix}}
  \newcommand{\eqs}[1]{\begin{align*} #1 \end{align*}}
  \newcommand{\case}[1]{\begin{cases} #1 \end{cases}}
  

  % Algebra
  \newcommand{\normSg}[0]{\vartriangleleft}
  \newcommand{\ZMod}[1][n]{\mathbb{Z}/#1\mathbb{Z}}
  \newcommand{\isom}{\simeq}
  \newcommand{\mapHom}{\xrightarrow{\text{hom}}}
  \DeclareMathOperator{\Inn}{Inn}
  \DeclareMathOperator{\Aut}{Aut}
  \DeclareMathOperator{\im}{im}
  \DeclareMathOperator{\ord}{ord}
  \DeclareMathOperator{\Gal}{Gal}
  \DeclareMathOperator{\chr}{char}
  \newcommand{\surjto}{\twoheadrightarrow}
  \newcommand{\injto}{\hookrightarrow}

  % Analysis 
  \newcommand{\limty}[1][k]{\lim_{#1\to\infty}}
  \newcommand{\norm}[1]{\left\lVert#1\right\rVert}
  \newcommand{\darrow}{\rightrightarrows}


\fancyhead[L]{Modern Algebra 2 - MAS312: Homework 10}
\fancyhead[R]{\textbf{Touch Sungkawichai} 20210821}

\begin{document}

\qs{}{
  Let $F = \Rat(\eta)$, where $\eta = \cos(2\pi/7) + i\sin(2\pi/7) \in \Comp$. Assume that $E/F$ is a Galois extension of degree 7.
  Show that there exists $\alpha \in E$ such that $\alpha^7 \in F$ and $E = F(\alpha)$.
}
\sol{
  As $\eta = e^{i2\pi/7}$, then let $\mu = \set{\eta, \eta^2, \cdots, \eta^6, 1}$.
  It is easy to check that $\mu$ is a set of solutions of $x^7 - 1$. 
  As $F = \Rat(\eta)$, then $\mu \subset F$.

  Now, assuming that $E/F$ is galois of degree 7, then $\Gal(E/F) \isom \ZMod[7]$ as it is the only group with order 7.
  As a kummer extension, $E/F$ is cyclic of order 7 if and only it is a splitting field of $x^7 - a$ for some $a \in F$.
  Therefore, there exists a root of $x^7 - a$, say $\alpha$ in the field $E$ such that $\alpha^7 = a \in F$.

  Then, consider that the set of all roots of $x^7 - a$ is $\set{\alpha, \alpha\eta, \cdots, \alpha\eta^6}$. 
  Thus, $E = F(\alpha)$ as $\eta \in F$.
}

\qs{}{
  Let $G$ be a finite group. Prove that there exists a polynomial $f$ over a field $F$ such that $\Gal(f) \isom G$.
}
\sol{
  Let $G$ be a finite group of order $n$, then there is an embedding of $G$ to $S_n$, 
  ie. $G$ is isomorphic to a subgroup of $S_n$. If there is a galois field extension $E/F$ such that $\Gal(E/F) \isom S_n$, then 
  by the galois correspondence, there must be a field extension $E/K/F$ such that $\Gal(E/K) \isom G$. Since $E/K$ is normal, 
  then $E$ must be a splitting field of some $f$ over field $K$. Hence, it is left to find a galois field extension 
  $E/F$ such that $\Gal(E/F) \isom S_n$.

  Let $x_1, x_2, \ldots, x_n$ be $n$ variables and let 
  \[ s_1 = x_1 + x_2 + \cdots + x_n \]
  \[ s_2 = x_1x_2 + x_1x_3 + \cdots + x_{n-1}x_n\]
  \[ \cdots \]
  \[ s_n = x_1 \cdot x_2  \cdots  x_n \]

  Then, notice that \[ (x - x_1)(x - x_2)\cdots (x - x_n) = x^n - s_1x^{n-1} + \cdots \pm s_n \] is a polynomial in 
  $\Rat(s_1, s_2, \ldots, s_n)[x]$
  So, $\Rat(x_1, x_2, \ldots, x_n)$ is the splitting field of $f$ over $\Rat(s_1, s_2, \ldots, s_n)$, and as $x_i$ are distinct, 
  $E = \Rat(x_1, x_2, \ldots, x_n)$ is galois over $F = \Rat(s_1, s_2, \ldots, s_n)$.

  Let $\sigma \in S_n$ be any permutation on the set of roots, then $\sigma$ is an automorphism of $E$ that fixes $F$ as 
  every $s_i$ is fixed by a permutation of roots. Moreover, $\Gal(E/F) < S_n$ since the degree of $f$ is $n$. Thus, 
  it must be the case that $\Gal(E/F) \isom S_n$, which finishes the proof.

}

\qs{}{
  Let $p$ be a prime integer and let $H$ be a subgroup of $S_p$ containing a $p$-cycle and a transposition. Prove that $H = S_p$.
}
\newcommand{\e}[1]{(#1)^\sigma}
\sol{
  To avoid using confusing notation, the cycle notation which was normall $(1, 2, \ldots, n)$ be be written as $[1 \; 2 \; \cdots \; n]$ instead.

  Without loss of generality, let the given transposition shuffles $1$ and $2$ and let $\bar{\sigma}$ be the $p$ cycle.
  Then, there is some $q < p$ such that $\bar{\sigma}^q(1) = 2$. Then, let $\sigma = \bar{\sigma}^q$

  Now, as $p$ is a prime, $\gcd(q, p) = 1$, therefore, $\sigma$ is of order $p$, which means that $\sigma$ is also a $p$-cycle.

  Then, $\sigma [1 \; 2] \sigma^{-1}$ is a transposition that shuffles $\sigma(1)$ and $\sigma(2)$.
  Now, as $\sigma(1) = 2$, then it maps $2$ to $\sigma(2)$.
  By this construction, it is possible to create $n$ transpositions, which are 
  \[ [\sigma^{p-1}(2) \; 2], [2 \; \sigma(2)], \ldots, [\sigma^{p-2}(2) \; \sigma^{p-1}(2)] \]
  Since $\sigma$ is cyclic, then it is of degree $p-1$, thus listed transpositions are pairwise distinct.

  Write $\e{n}$ instead $\sigma^{n}(2)$ for $0 \le n < m < p$ for brevity, so that the constructed transposition are 
  \[ [\e{0} \; \e{1}], [\e{1} \; \e{2}], \ldots, [\e{p-2} \; \e{p-1}], [\e{p-1} \; \e{0}] \]
  

  Then, let $[a, b]$ be an arbitrary transposition in $S_p$, then $a = \sigma^n(2)$ and $b = \sigma^m(2)$. for some $n$ and $m$.
  Assume without loss of generality that $0 \le n < m < p$. Thus, $a = \e{n}$ and $b = \e{m}$.

  Then, 
  \[ [\e{n} \; \cdots \; \e{m}] = [\e{n} \; \e{n+1}][\e{n+1} \; \e{n+2}] \cdots [\e{m-1} \; \e{m}] \]
  And
  \[ [\e{n} \; \e{m}] = [\e{n} \; \cdots \; \e{m}] [\e{m-1} \; \e{m}] [\e{n} \; \cdots \; \e{m}]^{-1} \]

  Thus, all transpositions are generated by a transposition and a $p$-cycle.
  Now, as all transpositions are the generator of symmetric group as a cycle 
  \[[a_1 \; a_2 \; \cdots \; a_k] = [a_1 \; a_2] [a_2 \; a_3] \cdots [a_{k-1} \; a_k]\]
  then, a transposition and a $p$-cycle generate $S_p$.

}

\qs{}{
  Let $f$ be a polynomial of degree 3 over $\Rat$. Prove that if $\Gal(f) = \ZMod[3]$, then $f$ has exactly three real roots.
}
\sol{
  Notice that $f$ can have a maximum of 3 roots. 
  As when $x$ approaches $\infty$, $f(x)$ approaches $\infty$ and as $x$ approaches $-\infty$, then $f(x)$ approaches $\infty$, then 
  $f(x)$ must cross the $x$-axis at least once. This means that $x$ has at least one real root.

  Assuming for contradiction that $f$ has a non-real complex root.
  Then, there must be at least $2$ non-real complex roots, otherwise, the product of all roots, 
  which is the constant term in the minimal polynomial will be non-real, thus non-rational.

  First, assuming that $f$ is irreducible, then $f$ contain exactly 2 non-real complex roots, 
  which must be $x$ and $\bar{x}$, a complex and its complex 
  conjugation. As $f$ is irreducible, then $\Gal(f)$ acts on the roots transitively, 
  therefore, $\Gal(f) \isom S_3$ since a transposition and a cycle generate the symmetric group.

  Then, if $f$ is reducible, then $f(x) = m(x - a)(x^2 - bx - c)$ such that $x^2 - bx - c$ has no real roots
  Therefore $\Gal(f) = \Gal(x^2 - bx - c)$ is of degree 2.

  So, the same conclusion that $\Gal(f) \ne \ZMod[3]$ is reached, thus the statement holds by contraposition.
}

\qs{}{
  Determine $\Gal(f)$ of $f(x) = x^3 + 4x + 2$ over $\Rat$
}
\sol{
  Firstly, the polynomial $f(x)$ is irreducible by the eisenstein criterion with $p=2$. Therefore, the group 
  $G = \Gal(f)$ must acts transitively on the set of 3 roots. Thus $G \isom A_3$ or $G \isom S_3$.

  Consider that the discriminant of the polynomial is $-4(4^3) -27(2^2) = -364$ is not a square, then $G$ is not a subgroup of $A_3$.
  Thus, $G \isom S_3$.
}

\qs{}{
  Let $f$ be a polynomial of degree 3 over $\Rat$. Find all possible $\Gal(f)$.
}
\sol{
  If $f$ is irreducible, then $G = \Gal(f)$ acts transitively. Thus $G$ is a transitive subgroup of $S_3$.
  In this case, either $G \isom S_3$ or $G \isom A_3$.
  It can also be shown that $G \isom A_3$ if and only if the discriminant, $D(f)$, is a square.

  Otherwise, $f$ is reducible. If $f$ is a product of one linear and one irreducible quadratic, then the galois group 
  $G$ must isomorphic to $S_2$, as it is the product of the group of the linear polynomial, which is $\set{e}$, and the group of 
  the quadratic part, which is $S_2$, as it is the only transitive subgroup of $S_2$.

  Lastly, if $f$ is a product of three linear polynomials, then all roots are a member of $F$, which is that $G \isom \set{e}$.

  To show that all of the cases are possible, consider 
  \[ \Gal(x^3-2) \isom S_3 \]
  as it is irreducible by eisenstein and have non-squared discriminant.
  \[ \Gal(x^3 - 3x + 1) \isom A_3 \] 
  as $x^3 - x + 1$ is irreducible over $\mathbb F_2$, so it $x^3 - 3x + 1$ is irreducible over $\Rat$. 
  Moreover, the discriminant is $D(x^3 - 3x + 1) = -4\cdot(-3)^3 - 27\cdot1^2 = 81 = 9^2$ is a square.
  \[ \Gal(x^3-1) \isom \ZMod[2] \]
  as the splitting field of $x^3 - 1$ is of degree $\phi(3) = 2$ over $\Rat$.
  \[ \Gal(x^3-x) \isom \set{e} \]
  as $x^3 - x = x(x-1)(x+1)$.
}

\qs{}{
  Let $f \in \Rat[x]$ be an irreducible polynomial of degree 4 with $\Gal(f) = S_4$. 
  Show that there is no nontrivial intermediate 
  field between $\Rat(\alpha)$ and $\Rat$ where $\alpha$ is a root of $f$.
}
\sol{
  Let $E$ be the splitting field of $f$ over $\Rat$.
  Let $\alpha_1, \ldots, \alpha_4$ be roots of $f$ such that $\alpha = \alpha_1$ 
  then $G = \Gal(E/\Rat(\alpha))$ is a subgroup of $S_4$ that fixes $\alpha_1$ and only $\alpha_1$.
  This implies that $G$ must be a transitive subgroup of $S_4$, which is either $S_3$ or $A_3$.

  If $G \isom S_3$, then there is no other proper subgroup of $S_4$ that contains $S_3$, 
  thus, there is no nontrivial intermediate subfield by the galois correspondence theorem.
  This is true because if there is proper subgroup of $S_4$ containing $S_3$, then it must have order $12$ as $6 \mid 12$ 
  and $12 \mid 24$ is the only number satisfying this properties.
  However, the only group of order $12$ in $S_4$ is $A_4$, and $S_3$ is not contained by $A_4$.

  If $G \isom A_3$, then $G$ is cyclic, thus $\Rat(\alpha)$ must be a splitting field of $x^3 - a$ for some $a \in \Rat$.
  So, $\Rat(\alpha)$ must be galois over $\Rat$. However, as $A_3$ is not a normal subgroup of $S_4$, $\Rat(\alpha)/\Rat$ 
  cannot be normal. This yield a contradiction.
}

\qs{}{
  Prove the following statements.
  \begin{enumerate}[a.]
    \item If $L/F$ is a radical extension and $\sigma: L \to K$ is a field homomorphism, then $\sigma(L)/\sigma(F)$ is a radical 
      extension
    \item Let $K/L_i/F$ be fields with $1 \le i \le m$. If each $L_i/F$ is a radical extension, then $L_1\cdots L_m/F$ is a radical 
      extension
  \end{enumerate}
}
\sol{
  \begin{enumerate}[a.]
    \item If $L/F$ is a radical extension, then assume WLOG that it is an $n$-radical for some $n$.
      Then, \[ F \subset F(\alpha_1) \subset \cdots \subset F(\alpha_1, \ldots, \alpha_m) = L \] 
      where $\alpha_i^n \in F(\alpha_1, \cdots, \alpha_{i-1})$ and 
      for $i > 1$ and $\alpha_1^n \in F$. As $\sigma$ is a field homomorphism, then for any field $E$ and $\beta$
      \[ \sigma(E(\beta)) = \sigma(E)(\sigma(\beta)) \]
      which means \[ \sigma(F) \subset \sigma(F)(\sigma(\alpha_1)) \subset \cdots 
      \subset \sigma(F)(\sigma(\alpha_1), \ldots, \sigma(\alpha_m)) = \sigma(L) \]
      From here, $\sigma(\alpha_1)^n = \sigma(\alpha_1^n) \in \sigma(F)$ since $\alpha_1 \in F$.
      Moreover, for all $i > 1$, $\sigma(\alpha_i)^n = \sigma(\alpha_i^n) \in \sigma(F(\alpha_1, \ldots, \alpha_{i-1}))$ by similar 
      logic.

      Thus, $\sigma(L)/\sigma(F)$ is a radical extension.

    \item Firstly, consider when $m = 2$. 
      Since $L_2$ is radical, then let $L_2$ be $k$ radical, without loss of generality. So,
      $L_2 = F(\alpha_1, \ldots, \alpha_m)$ for $\alpha_i \in L_2$, such that there are chains 
      \[ F \subset F(\alpha_1) \subset \cdots \subset F(\alpha_1, \ldots, \alpha_n) = L_2 \] 
      with $\alpha_1^k \in F$ and for$i>1$, $\alpha_i^k \in F(\alpha_1, \ldots, \alpha_{i-1})$.

      As \[L_1L_2 = L_1(\beta_1, \ldots, \beta_m)\] by definition. Then, consider a chain 
      \[ L_1 \subset L_1(\alpha_1) \subset \cdots \subset L_1(\alpha_1, \ldots, \alpha_n) = L_1L_2 \]
      then, $\alpha_1^k \in F \subset L_1$ and for $i > 1$, $\alpha_i^k \in F(\alpha_1, \ldots, \alpha_{i-1}) 
      \subset L_1(\alpha_1, \ldots, \alpha_{i-1})$. Thus, $L_1L_2$ is $k$-radical over $L_1$.

      As $L_1L_2/L_1$ and $L_1/F$ are both radical, then by the transitivity, $L_1L_2/F$ is also radical.
      Hence, the statement is proved for $m = 2$.

      Now, for $m > 2$, assume for induction that $L_1L_2\cdots L_{m-1}$ is radical over $F$ under the given condition.
      Then, let $K_1 = L_1L_2\cdots L_{m-1}$ and $K_2 = L_m$ so that the conditions for the case $m=2$ are satisfied.
      By the proof, $K_1K_2$ is radical. However, as $K_1K_2 = L_1\cdots L_{m}$, then $L_1\cdots L_m$ is radical over $F$.
      Hence, the statement was proven by induction.
  \end{enumerate}
}

\qs{}{
  Show that the polynomial $f(x) = 2x^5 - 5x^4 + 5$ over $\Rat$ is not solvable by radicals.
}
\sol{
  Notice that $f'(x) = 10x^4 - 20x^3$ has just two solutions, which are $x = 0$ and $x = 2$. 
  It can be checked that $f(0) = 5 > 0$ and $f(2) = -11 < 0$. With $f(x)$ being a degree 5 polynomial, 
  as $x \to \infty$, $f(x)$ approaches $\infty$ and as $x \to -\infty$, $f(x)$ approaches $-\infty$. This means that the graph of 
  $f(x)$ must intercept the $x$-axis exactly 3 times by the intermediate value theorem.

  Since $f(x)$ has 5 roots and exactly 3 are real roots, then $f(x)$ has exactly 2 non-real complex roots.
  Moreover, $f(x)$ is irreducible by the eisenstein criterion at $p = 5$. Thus $G = \Gal(f)$ is a transitive subgroup of $S_5$.
  With the presence of tranposition automorphism that map the complex root to its conjugate, and the fact that $G$ is a transitive 
  subgroup thus contain a $5$-cycle, the only possibility is that $G \isom S_5$ as a tranposition and $5$-cycle generates $S_5$.

  As $S_5$ is not solvable, $f$ is not solvable by radicals according to the galois theorem.
}

\qs{}{
  Determine $\Gal(f)$ of $f(x) = x^4 + 4x^2 + 2$ over $\Rat$.
}
\sol{
  Let $E$ be a splitting field of $f$ over $\Rat$ and consider that 
  \eqs{ f(x) = x^4 + 4x^2 + 2 &= \paren{x^2 + 2 - \sqrt{2}}\paren{x^2 + 2 + \sqrt{2}} \\ 
                              &= \paren{x - \sqrt{2 - \sqrt{2}}}\paren{x + \sqrt{2 - \sqrt{2}}}\paren{x - \sqrt{2 + \sqrt{2}}}\paren{x + \sqrt{2 + \sqrt{2}}} 
  }

  Let $\alpha_1 = \sqrt{2 - \sqrt{2}},\; \alpha_2 = -\sqrt{2 - \sqrt{2}},\; \alpha_3 = \sqrt{2 + \sqrt{2}},\; 
  \alpha_4 = -\sqrt{2 + \sqrt{2}}$ be the four roots of $f$.
                            
  And let 
  \begin{align*}
    \beta_1 &= \alpha_1\alpha_2 + \alpha_3\alpha_4 = (-(2 - \sqrt{2})) + (-(2 + \sqrt{2})) = -4 \\
    \beta_2 &= \alpha_1\alpha_3 + \alpha_2\alpha_4 = (\sqrt{2}) + (\sqrt{2}) = 2\sqrt{2} \\
    \beta_3 &= \alpha_1\alpha_4 + \alpha_2\alpha_3 = (-\sqrt{2}) + (-\sqrt{2}) = -2\sqrt{2} 
  \end{align*}

  Thus, $K = F(\beta_1, \beta_2, \beta_3)$ is galois over $\Rat$ and 
  let $G = \Gal(f)$ so that $K = E^{G \cap V}$. As $K/\Rat$ and $E/\Rat$ are galois, then 
  \[ m = \frac{\abs{G}}{\abs{G \cap V}} = [K: \Rat] = [\Rat(\sqrt{2}): \Rat] = 2 \]
  
  Then, notice that $f$ is reducible over $K$ as shown above, therefore, $G \isom \ZMod[4]$.
}

\end{document}
