% chktex-file 44
% chktex-file 8

\documentclass{report}
\usepackage{amsthm}
\usepackage{amsmath}
\usepackage{amssymb}
\usepackage{amssymb}
\usepackage{amsfonts}
\usepackage{xcolor}
\usepackage{tikz}
\usepackage{fancyhdr}
\usepackage{enumerate}
\usepackage{graphicx}
\usepackage[normalem]{ulem}
\usepackage[most,many,breakable]{tcolorbox}
\usepackage[a4paper, top=80pt, foot=25pt, bottom=50pt, left=0.5in, right=0.5in]{geometry}
\usepackage{hyperref, theoremref}
\hypersetup{
	pdftitle={Assignment},
	colorlinks=true, linkcolor=b!90,
	bookmarksnumbered=true,
	bookmarksopen=true
}
\usepackage{nameref}
\usepackage{parskip}
\pagestyle{fancy}

\usepackage[explicit,compact]{titlesec}
\titleformat{\chapter}[block]{\bfseries\huge}{\thechapter. }{\compact}{#1}
        

%%%%%%%%%%%%%%%%%%%%%
%% Defining colors %%
%%%%%%%%%%%%%%%%%%%%%

\definecolor{lr}{RGB}{188, 75, 81}
\definecolor{r}{RGB}{249, 65, 68}
\definecolor{dr}{RGB}{174, 32, 18}
\definecolor{lo}{RGB}{255, 172, 129}
\definecolor{do}{RGB}{202, 103, 2}
\definecolor{o}{RGB}{238, 155, 0}
\definecolor{ly}{RGB}{255, 241, 133}
\definecolor{y}{RGB}{255, 229, 31}
\definecolor{dy}{RGB}{143, 126, 0}
\definecolor{lb}{RGB}{148, 210, 189}
\definecolor{bg}{RGB}{10, 147, 150}
\definecolor{b}{RGB}{39, 125, 161}
\definecolor{db}{RGB}{0, 95, 115}
\definecolor{p}{RGB}{229, 152, 155}
\definecolor{dp}{RGB}{181, 101, 118}
\definecolor{pp}{RGB}{142, 143, 184}
\definecolor{v}{RGB}{109, 89, 122}
\definecolor{lg}{RGB}{144, 190, 109}
\definecolor{g}{RGB}{64, 145, 108}
\definecolor{dg}{RGB}{45, 106, 79}

\colorlet{mysol}{g}
\colorlet{mythm}{lr}
\colorlet{myqst}{db}
\colorlet{myclm}{lb}
\colorlet{mywrong}{r}
\colorlet{mylem}{o}
\colorlet{mydef}{lg}
\colorlet{mycor}{lb}
\colorlet{myrem}{dr}

%%%%%%%%%%%%%%%%%%%%%

\newcommand{\col}[2]{
  \color{#1}#2\color{black}\,
}

\newcommand{\TODO}[1][5cm]{
  \color{red}TODO\color{black}
  \vspace{#1}
}

\newcommand{\wans}[1]{
	\noindent\color{mywrong}\textbf{Wrong answer: }\color{black}
	#1 


}

\newcommand{\wreason}[1]{
	\noindent\color{mywrong}\textbf{Reason: }\color{black}
	#1 

  
}

\newcommand{\sol}[1]{
	\noindent\color{mysol}\textbf{Solution: }\color{black}
	#1


}

\newcommand{\nt}[1]{
  \begin{note}Note: #1\end{note}
}

\newcommand{\ky}[1]{
  \begin{key}#1\end{key}
}

\newcommand{\pf}[1]{
  \begin{myproof}#1\end{myproof}
}

\newcommand{\qs}[3][]{
  \begin{question}{#2}{#1}#3\end{question}
}

\newcommand{\df}[3][]{
  \begin{definition}{#2}{#1}#3\end{definition}
}

\newcommand{\thm}[3][]{
  \begin{theorem}{#2}{#1}#3\end{theorem}
}

\newcommand{\clm}[3][]{
  \begin{claim}{#2}{#1}#3\end{claim} 
}

\newcommand{\lem}[3][]{
  \begin{lemma}{#2}{#1}#3\end{lemma}
}

\newcommand{\cor}[3][]{
  \begin{corollary}{#2}{#1}#3\end{corollary}
}

\newcommand{\rem}[3][]{
  \begin{remark}{#2}{#1}#3\end{remark}
}

\newcommand{\twoways}[2]{
  \leavevmode\\
  ($\Longrightarrow$): 
  \begin{shift}#1\end{shift}
  ($\Longleftarrow$):
  \begin{shift}#2\end{shift} 
}

\newcommand{\nways}[2]{
  \leavevmode\\
  ($#1$): 
  \begin{shift}#2\end{shift}
}

%%%%%%%%%%%%%%%%%%%%%%%%%%%%%% ENVRN

\newenvironment{myproof}[1][\proofname]{%
	\proof[\bfseries #1: ]
}{\endproof}

\tcbuselibrary{theorems,skins,hooks}
\newtcolorbox{shift}
{%
  before upper={\setlength{\parskip}{5pt}},
  blanker,
	breakable,
	width=0.95\textwidth,
  enlarge left by=0.03\textwidth,
}

\tcbuselibrary{theorems,skins,hooks}
\newtcolorbox{key}
{%
	breakable,
	width=0.95\textwidth,
  enlarge left by=0.03\textwidth,
}

\tcbuselibrary{theorems,skins,hooks}
\newtcolorbox{note}
{%
	enhanced,
	breakable,
	colback = white,
	width=\textwidth,
	frame hidden,
	borderline west = {2pt}{0pt}{black},
	sharp corners,
}

\tcbuselibrary{theorems,skins,hooks}
\newtcbtheorem[]{remark}{Remark}
{%
	enhanced,
	breakable,
	colback = white,
	frame hidden,
	boxrule = 0sp,
	borderline west = {2pt}{0pt}{myrem},
	sharp corners,
	detach title,
  before upper={\setlength{\parskip}{5pt}\tcbtitle\par\smallskip},
	coltitle = myrem,
	fonttitle = \bfseries\sffamily,
	description font = \mdseries,
	separator sign none,
	segmentation style={solid, myrem},
}{rem}

\tcbuselibrary{theorems,skins,hooks}
\newtcbtheorem[number within=section]{lemma}{Lemma}
{%
	enhanced,
	breakable,
	colback = white,
	frame hidden,
	boxrule = 0sp,
	borderline west = {2pt}{0pt}{mylem},
	sharp corners,
	detach title,
  before upper={\setlength{\parskip}{5pt}\tcbtitle\par\smallskip},
	coltitle = mylem,
	fonttitle = \bfseries\sffamily,
	description font = \mdseries,
	separator sign none,
	segmentation style={solid, mylem},
}{lem}

\tcbuselibrary{theorems,skins,hooks}
\newtcbtheorem{claim}{Claim}
{%
  parbox=false,
	enhanced,
	breakable,
	colback = white,
	frame hidden,
	boxrule = 0sp,
	borderline west = {2pt}{0pt}{myclm},
	sharp corners,
	detach title,
  before upper={\setlength{\parskip}{5pt}\tcbtitle\par\smallskip},
	coltitle = myclm,
	fonttitle = \bfseries\sffamily,
	description font = \mdseries,
	separator sign none,
	segmentation style={solid, myclm},
}{clm}

\makeatletter
\newtcbtheorem[number within=section, use counter from=lemma]{theorem}{Theorem}{enhanced,
	breakable,
	colback=white,
	colframe=mythm,
	attach boxed title to top left={yshift*=-\tcboxedtitleheight},
	fonttitle=\bfseries,
	title={#2},
	boxed title size=title,
	boxed title style={%
			sharp corners,
			rounded corners=northwest,
			colback=mythm,
			boxrule=0pt,
		},
	underlay boxed title={%
			\path[fill=mythm] (title.south west)--(title.south east)
			to[out=0, in=180] ([xshift=5mm]title.east)--
			(title.center-|frame.east)
			[rounded corners=\kvtcb@arc] |-
			(frame.north) -| cycle;
		},
	#1
}{thm}
\makeatother

\makeatletter
\newtcbtheorem{question}{Question}{enhanced,
	breakable,
	colback=white,
	colframe=myqst,
	attach boxed title to top left={yshift*=-\tcboxedtitleheight},
	fonttitle=\bfseries,
	title={#2},
	boxed title size=title,
	boxed title style={%
			sharp corners,
			rounded corners=northwest,
			colback=myqst,
			boxrule=0pt,
		},
	underlay boxed title={%
			\path[fill=myqst] (title.south west)--(title.south east)
			to[out=0, in=180] ([xshift=5mm]title.east)--
			(title.center-|frame.east)
			[rounded corners=\kvtcb@arc] |-
			(frame.north) -| cycle;
		},
	#1
}{qs}
\makeatother

\makeatletter
\newtcbtheorem[number within=section]{definition}{Definition}{enhanced,
	breakable,
	colback=white,
	colframe=mydef,
	attach boxed title to top left={yshift*=-\tcboxedtitleheight},
	fonttitle=\bfseries,
	title={#2},
	boxed title size=title,
	boxed title style={%
			sharp corners,
			rounded corners=northwest,
			colback=mydef,
			boxrule=0pt,
		},
	underlay boxed title={%
			\path[fill=mydef] (title.south west)--(title.south east)
			to[out=0, in=180] ([xshift=5mm]title.east)--
			(title.center-|frame.east)
			[rounded corners=\kvtcb@arc] |-
			(frame.north) -| cycle;
		},
	#1
}{def}
\makeatother

\makeatletter
\newtcbtheorem[number within=section, use counter from=lemma]{corollary}{Corollary}{enhanced,
	breakable,
	colback=white,
	colframe=mycor,
	attach boxed title to top left={yshift*=-\tcboxedtitleheight},
	fonttitle=\bfseries,
	title={#2},
	boxed title size=title,
	boxed title style={%
			sharp corners,
			rounded corners=northwest,
			colback=mycor,
			boxrule=0pt,
		},
	underlay boxed title={%
			\path[fill=mycor] (title.south west)--(title.south east)
			to[out=0, in=180] ([xshift=5mm]title.east)--
			(title.center-|frame.east)
			[rounded corners=\kvtcb@arc] |-
			(frame.north) -| cycle;
		},
	#1
}{cor}
\makeatother

% Basic
  \DeclareMathOperator{\lcm}{lcm}
  \newcommand{\Real}{\mathbb{R}}
  \newcommand{\Comp}{\mathbb{C}}
  \newcommand{\Nat}{\mathbb{N}}
  \newcommand{\Rat}{\mathbb{Q}}
  \newcommand{\Int}{\mathbb{Z}}
  \newcommand{\set}[1]{\left\{\, #1 \,\right\}}
  \newcommand{\paren}[1]{\left( \; #1 \; \right)}
  \newcommand{\abs}[1]{\left\lvert #1 \right\rvert}
  \newcommand{\ang}[1]{\left\langle #1 \right\rangle}
  \renewcommand{\to}[1][]{\xrightarrow{\text{#1}}}
  \newcommand{\tol}[1][]{\to{$#1$}}
  \newcommand{\curle}{\preccurlyeq}
  \newcommand{\curge}{\succcurlyeq}
  \newcommand{\mapsfrom}{\leftarrow\!\shortmid}

  \newcommand{\mat}[1]{\begin{bmatrix} #1 \end{bmatrix}}
  \newcommand{\pmat}[1]{\begin{pmatrix} #1 \end{pmatrix}}
  \newcommand{\eqs}[1]{\begin{align*} #1 \end{align*}}
  \newcommand{\case}[1]{\begin{cases} #1 \end{cases}}
  

  % Algebra
  \newcommand{\normSg}[0]{\vartriangleleft}
  \newcommand{\ZMod}[1][n]{\mathbb{Z}/#1\mathbb{Z}}
  \newcommand{\isom}{\simeq}
  \newcommand{\mapHom}{\xrightarrow{\text{hom}}}
  \DeclareMathOperator{\Inn}{Inn}
  \DeclareMathOperator{\Aut}{Aut}
  \DeclareMathOperator{\im}{im}
  \DeclareMathOperator{\ord}{ord}
  \DeclareMathOperator{\Gal}{Gal}
  \DeclareMathOperator{\chr}{char}
  \newcommand{\surjto}{\twoheadrightarrow}
  \newcommand{\injto}{\hookrightarrow}

  % Analysis 
  \newcommand{\limty}[1][k]{\lim_{#1\to\infty}}
  \newcommand{\norm}[1]{\left\lVert#1\right\rVert}
  \newcommand{\darrow}{\rightrightarrows}


\fancyhead[L]{Modern Algebra 2 - MAS312: HW7}
\fancyhead[R]{\textbf{Touch Sungkawichai} 20210821}

\begin{document}

\qs{}{
  Give an example of a finite field extension that is not generated by a single element.
}
\sol{
  Consider $E = \mathbb F_p(X, Y)$ to be a field of rational functions with two variables and $F = \mathbb F_p(X^p, Y^p)$ so that $E/F$. 
  Since $T^p - X^p = 0$, $X \in E$ is a of degree at most $p$ over $F$ and similarly, as $T^p - Y^p = 0$, $Y$ is of degree at most  
  $p$ over $F$. This means that $E/F$ is finite, therefore $F$ should be a field of rational functions of two variables. 
  Otherwise, if $X^p$ or $Y^p$ is algebraic, then $X$ or $Y$ respectively will be algebraic, contradicting the assumption.

  Next, notice that $f(T) = T^p - X^p$ and $X^p$ is irreducible in $\mathbb F_p(Y^p)[X^p]$. 
  This means $f(T)$ is irreducible over $F$ by the Eisenstein criterion. Moreover $f(X) = 0$, therefore, the degree of 
  $\mathbb F_p(X, Y^p)$ over $F$ is $p$. Then considering $g(P) = T^p - Y^p$ in $\mathbb F_p(X)[Y^p]$ gives that $\mathbb F_p(X, Y)$ is 
  a degree $p$ extension over $\mathbb F_p(X, Y^p)$ using similar reasoning.

  Now, let $\alpha$ be an arbitrary element of $E$, which is that 
  \[ \alpha = \alpha_{0, 0} + \alpha_{1, 0}X + \alpha_{0, 1}Y + \cdots + \alpha_{n, m}X^nY^m \] 
  where $\alpha_{i, j} \in \mathbb F_p$.

  Then, \[ \alpha^p = \alpha_{0, 0}^p + \alpha_{1, 0}^pX^p + \cdots +  \alpha_{n, m}^p(X^p)^n(Y^p)^m \] is an element in $F$.
  Thus, $\alpha^p \in F$, which means that $\alpha$ is the root of some polynomial $T^p - \alpha^p$ over $F$.
  Since $\alpha$ can be at most degree $p$, then $E \ne F(\alpha)$ for any $\alpha \in E$. Thus, $E/F$ is finite generated by a single 
  element.
} 

\qs{}{
  Let $\alpha = 1 + \sqrt[3]{2} + \sqrt[3]{4}$. Determine whether or not $\Rat(\alpha)/\Rat$ is normal.
}
\sol{
  Consider that \[ \alpha^2 = 1 + \sqrt[3]{4} + 2\sqrt[3]{2} + 2\sqrt[3]{4} + 2\sqrt[3]{2} + 4 = 5 + 4\sqrt[3]{2} + 3\sqrt[3]{2} \]
  so $\alpha^2 - 3\alpha -2 = \sqrt[3]{2}$, which means 
  \[ \alpha^3 - 3\alpha^2 - 2\alpha = 2 + \sqrt[3]{2} + \sqrt[3]{4} \]
  Then, $\alpha^3 - 3\alpha^2 - 3\alpha - 1 = 0$. Thus, $f(x) = x^3 -3x^2 - 3x - 1$ has $\alpha$ as a root.

  Moreover, consider that $f(x + 1) = x^3 - 6x - 6$ where $3$ is irreducible in $\Int$ dividing $6$ but $3^2 \nmid 6$. Then, the 
  Eisenstein criterion applies. So, $f(x+1)$ and thus $f(x)$ is irreducible.

  Now, notice that 
  \[f(x + \alpha) = x^3 + (3\alpha -3) x^2 + (3\alpha^2 - 6\alpha - 3)x + f(\alpha) = x^3 + 3(\alpha-1)x^2 + 3((\alpha-1)^2-2)x \]
  Consider the root $\beta$ of $f$, for if $\beta \ne \alpha$, then $x \ne 0$ in above equation, which gives 
  \[ \beta^2 + 3(\alpha - 1)\beta + 3((\alpha - 1)^2 - 2) = 0 \]
  Now, as $\alpha - 1 = \sqrt[3]{2} + \sqrt[3]{4}$ and 
  $(\sqrt[3]{4} + 2\sqrt[3]{2})^2 = 2\sqrt[3]{2} + 4\sqrt[3]{4} + 4 > 7 + (\sqrt[3]{4} + 2\sqrt[3]{2})$ 
  because $\sqrt[3]{x} > 1$ for any $x > 1$.

  With the monotonicity of the polynomial function $h(x) = x^2-x$ for $x > 1$, it follows that $\sqrt[3]{4} + 2\sqrt[3]{2} > 4$

  Thus,
  \[ (\alpha - 1)^2 = \sqrt[3]{4} + 2\sqrt[3]{2} + 4 > 8 \] 
  So, $(3(\alpha - 1))^2 - 4\cdot3((\alpha - 1)^2 - 2) = -3(\alpha - 1)^2 + 24 < 0$. This means that $\beta$ must be a complex number.
  However, $\Rat(\alpha)$ is the smallest field generated by $\alpha$, a real number. Thus, since $\Real$ is a field and 
  $\Rat(\alpha) \subset \Real$, $\beta \notin \Rat(\alpha)$. This means that $\Rat(\alpha)$ is not normal since it does not split $f(x)$.
}

\qs{}{
  Find the Galois group of $\Rat(\sqrt{2}, \sqrt{7}, \sqrt{19})/\Rat$
}
\sol{
  Notice that the degree $[\Rat(\sqrt{2}, \sqrt{7}, \sqrt{19}): \Rat]$ is at most 
  $[\Rat(\sqrt{2}): \Rat][\Rat(\sqrt{7}):\Rat][\Rat(\sqrt{19}): \Rat] = 8$ So the galois group $G$ of the extension must be at most 
  order $8$.

  Considering $\Rat$-automorphisms, it must send $\sqrt{2} \to \pm \sqrt{2}$, $\sqrt{7}\to\pm\sqrt{7}$ and $\sqrt{19}\to\pm\sqrt{19}$.
  since it must preserve the root of $f(x) = x^2 - 2$, $g(x) = x^2 - 7$, and $h(x) = x^2 - 19$.

  Let 
  \[ \sigma_2: \sqrt{2} \mapsto -\sqrt{2}, \sqrt{7} \mapsto \sqrt{7}, \text{ and } \sqrt{19} \mapsto \sqrt{19} \]
  \[ \sigma_7: \sqrt{2} \mapsto \sqrt{2}, \sqrt{7} \mapsto -\sqrt{7}, \text{ and } \sqrt{19} \mapsto \sqrt{19} \]
  \[ \sigma_{19}: \sqrt{2} \mapsto \sqrt{2}, \sqrt{7} \mapsto \sqrt{7}, \text{ and } \sqrt{19} \mapsto -\sqrt{19} \]

  Then all of them are $\Rat$-automorphisms.

  Moreover, the compositions of them are also $\Rat$-automorphisms, and the composition of them, in this case is commutative.
  This is because $\sigma_2$ permutes only $\sqrt{2}$ and $-\sqrt{2}$, and similarly for $\sigma_7$ and $\sigma_{19}$.
  It also means that they are of degree 2.

  Thus, there are total of $8$ $\Rat$-automorphisms, which are $id, \sigma_2, \sigma_7, \sigma_{19}, \sigma_2\circ\sigma_7, 
  \sigma_2\circ\sigma_{19}, \sigma_7\circ\sigma_{19}, $ and $\sigma_2\circ\sigma_7\circ\sigma_{19}$.

  The group is isomorphic to $(\ZMod[2])\times(\ZMod[2])\times(\ZMod[2])$ by the isomorphism 
  \[ \psi: \sigma_2 \mapsto (1, 0, 0), \sigma_7 \mapsto (0, 1, 0), \text{ and } \sigma_{19} \mapsto (0, 0, 1) \]
}

\qs{}{
  Let $E$ be a splitting field of $x^4 + 3x^2 + 1$ over $\Rat$. Determine $\Gal(E/\Rat)$
}
\sol{
  The galois group $G = \Gal(E/\Rat)$ should be a subgroup of $S_4$ as $f(x) = x^4 + 3x^2 + 1$ is of degree $4$.

  Firstly, $f(x)$ is irreducible over $\Rat$, which equivalent to that it is irreducible over $\Int$ as $f(x)$ is monic, thus primitive.
  Now, if $f(x)$ is reducible to $P(x)Q(x)$ over $\Int$, then one of the divisor of $f(x)$ must be of degree not more than 2 and monic 
  since f(x) is monic.

  If $f(x) = P(x)Q(x)$, then $f(x) = P(x)Q(x)$ modulo $2$. which means that $f_2(x)$ is reducible over $\mathbb F_2$, where 
  $f_2(x) = x^4 + x^2 + 1 \equiv f(x) \pmod{2}$.
  But since $\gcd(x^4 + x^2 + 1, x^{2^2} - x) = \gcd(x^4 + x^2 + 1, x^3 - 1) = \gcd(x^3 + x^2 + 1, x^3 + 1) = \gcd(x^2, x^3 + 1) = 1$, 
  then, $f_2(x)$ is irreducible over $\mathbb F_2$, which means that $f(x)$ is irreducible over $\Rat$.

  Now, notice that when letting $\alpha = \sqrt{\frac{3 + \sqrt{5}}{2}}$ and $\bar\alpha = \sqrt{\frac{3 - \sqrt{5}}{2}}$
  \[ f(x) = (x^2 + \alpha^2)(x^2 + \bar\alpha^2) = (x + i\alpha)(x - i\alpha)(x + i\bar\alpha)(x - i\bar\alpha) \]
  And $\alpha \cdot \bar\alpha = \sqrt{ \frac{\paren{3 + \sqrt{5}}\paren{3 - \sqrt{5}}}{4} } = \sqrt{\frac{9 - 5}{4}} = 1$
  which means $\bar\alpha = 1/\alpha$.
  So, $E = \Rat(i\alpha)$.

  Since $f(x)$ is irreducible, then $[\Rat(i\alpha): \Rat] = 4$.
  Thus, the order of the galois group $G = \Gal(E/\Rat)$ is 4.

  Now, let consider two $\Rat$-automorphisms $\phi: i\alpha \to -i\alpha$ and $\psi: i\alpha \to i\bar\alpha$. 
  Then, $\phi^2(i\alpha) = -\phi(i\alpha) = i\alpha$. Since $\phi^2$ fixes the generator of $E$, it is $id$.
  Moreover, $\psi^2(i\alpha) = \psi(i\bar\alpha) = -\psi(1/i\alpha) = -1/(i\bar\alpha) = i\alpha$. Therefore, 
  $\psi^2 = id$. Since $\psi \ne \phi$ but both are of order 2, then $G \isom K_4$, as it is the only group with the 
  properties.
}

\qs{}{
  Let $E = \Rat(\sqrt[3]{13}, \eta)/\Rat$ where $\eta$ is a primitive $3$rd root of $1$. Determine $\Gal(E/\Rat)$
}
\sol{
  Notice that the splitting field of $f(x) = x^3 - 13 = (x - \sqrt[3]{13})(x - \sqrt[3]{13}\eta)(x - \sqrt[3]{13}\eta^2)$ is $E$.
  Thus, the galois group $G = \Gal(E/\Rat)$ is a subgroup of $S_3$ and is of degree $6$.

  This is because $E/\Rat$ is Galois (as it is a splitting field, thus normal, of a separable field, as $\Rat$ is perfect), and the 
  degree $[E: \Rat] = [E: \Rat(\sqrt[3]{13})][\Rat(\sqrt[3]{13}): \Rat]$ where the first term is $2$ as $\eta^2 + \eta + 1 = 0$ and 
  $\eta \notin \Rat(\sqrt[3]{13})$ as $\Rat(\sqrt[3]{13}) \subset \Real$ but $\eta \notin \Real$. And the second term is $3$ because 
  the polynomial $f(x) = x^3 - 13$ is irreducible over $\Rat$ by the eisenstein criterion with $f(\sqrt[3]{13}) = 0$.

  Therefore, $G \isom S_3$ as $\abs{S_3} = 6$.
}

\qs{}{
  Let $E$ be a splitting field of $x^4 - 2$ over $\Rat$. Compute $\Gal(E/\Rat)$
}
\sol{
  Notice that $f(x) = x^4 - 2$ is irreducible over $\Rat$ by the eisenstein criterion.
  \[ f(x) = x^4 - 2 = \paren{x^2 - \sqrt{2}} \paren{x^2 - \sqrt{2}} 
  = \paren{x - \sqrt[4]{2}} \paren{x + \sqrt[4]{2}} \paren{x - i\sqrt[4]{2}} \paren{x + i\sqrt[4]{2}} \]
  Then, it is evident that $E = \Rat(\sqrt[4]{2}, i)$
  
  As $E/\Rat$ is a splitting field over $\Rat$, which is a perfect field, then it is normal and separable, thus galois.

  Now, $[E: \Rat] = [E: \Rat(\sqrt[4]{2})] [\Rat(\sqrt[4]{2}): \Rat]$, where the first term is $2$ as the polynomial $x^2 + 1$ is 
  satisfied by $i$, and $i \notin \Rat(\sqrt[4]{2})$ and $i \notin \Real$ but $\Rat(\sqrt[4]{2}) \subset \Real$. The second term is 
  $[\Rat(\sqrt[4]{2}): \Rat] = 4$ as $f(x)$ is irreducible and $f(\sqrt[4]{2}) = 0$.
  Therefore, $[E: \Rat] = 8 = \abs{G}$.

  As $G = \Gal(E/\Rat)$ is a splitting field of $f(x)$, then $G \subset S_4$

  Notice that $\abs{S_4} = 24 = 8\cdot3$, so the subgroup of order $8$ of $S_4$ is unique, and since 
  \[ \set{id, (1 2 3 4), (1 3) (2 4), (1 4 3 2), (1 3), (2 4), (1 4)(2 3), (1 2)(3 4)} \] 
  is a subgroup of 
  $S_4$, and it is isomorphic to $D_8$ with $(1 2 3 4) \mapsto r$ and $(1 3) \mapsto f$.

  Thus, the galois group $\Gal(E/\Rat) \isom D_8$
}

\qs{}{
  Let $\eta$ be a primitive $3$rd root of $1$ and $E = \Rat(\sqrt{3}, \sqrt{11}, \eta)$. Find $\Gal(E/\Rat)$
}
\sol{
  Firstly, $E$ is normal if and only if it splits the minimal polynomial of $\sqrt{3}$, $\sqrt{11}$, and $\eta$, which it does
  as 
  \begin{itemize}
    \item $m_{\sqrt{3}} \mid x^2 - 3 = (x - \sqrt{3})(x + \sqrt{3})$ 
    \item $m_{\sqrt{11}} \mid x^2 - 11 = (x - \sqrt{11})(x + \sqrt{11})$ 
    \item $m_\eta \mid x^2 + x + 1 = (x - \eta)(x - \eta^2)$ 
  \end{itemize}

  Since $E$ is normal, and $\Rat$ is perfect, then $E/\Rat$ is galois.

  Now, \[ [E: \Rat] = [E: \Rat(\sqrt{3}, \sqrt{11})] [\Rat(\sqrt{3}, \sqrt{11}) : \Rat(\sqrt{3})] [\Rat(\sqrt{3}) : \Rat] \] 
  
  Firstly, $[\Rat(\sqrt{3}) : \Rat] = 2$ since $x^2 - 3$ is irreducible over $\Rat$ by the Eisenstein criterion. 

  Next, $[\Rat(\sqrt{3}, \sqrt{11}): \Rat] = 2$ since the minimal polynomial of $\sqrt{11}$ over $\Rat(\sqrt{3})$ must divide 
  $x^2 - 11$, and $\sqrt{11} \notin \Rat(\sqrt{3})$. This is because if it is, $\sqrt{11} = a + b\sqrt{3}$ for some $a, b \in \Rat$ as 
  $\sqrt{3}^2 = 3 \in \Rat$. Now, squaring both sides gives $11 = a^2 + 3b^2 + 2ab\sqrt{3}$, so $a = 0$ or $b = 0$.
  If $a = 0$, then $11 = 3b^2$ is imposible as $3 \nmid 11$, and if $b = 0$, $11 = a^2$ is not possible as $\sqrt{11} \notin \Rat$. This 
  is because $x^2 - 11$ is irreducible over $\Rat$ by the Eisenstein criterion.

  Then, $[E: \Rat(\sqrt{3}, \sqrt{11})] = 2$ as $x^2 + x + 1$ is satisfied by $\eta$ and $\eta \notin \Real$ but 
  $\Rat(\sqrt{3}, \sqrt{11}) \subset \Real$. So, $\eta \notin \Rat(\sqrt{3}, \sqrt{11})$.
  
  Therefore, $[E: \Rat] = 8 = \abs{\Gal(E/\Rat)}$.

  As the $\Rat$-automorphism must fix the root of $x^2 - 3$, $x^2 - 11$, and $x^2 + x + 1$, then it must send 
  $\sqrt{3} \mapsto \pm \sqrt{3}$, $\sqrt{11} \mapsto \pm \sqrt{11}$, and $\eta \mapsto \eta$ or $\eta \mapsto \eta^2$. 

  Let 
  \[ \sigma_3: \sqrt{3} \mapsto -\sqrt{3}, \sqrt{11} \mapsto \sqrt{11}, \text{ and } \eta \mapsto \eta \]
  \[ \sigma_{11}: \sqrt{3} \mapsto \sqrt{3}, \sqrt{11} \mapsto -\sqrt{11}, \text{ and } \eta \mapsto \eta \]
  \[ \sigma_{\eta}: \sqrt{3} \mapsto \sqrt{3}, \sqrt{11} \mapsto \sqrt{11}, \text{ and } \eta \mapsto \eta^2 \]
  Then, each of $\sigma_3$, $\sigma_{11}$, and $\sigma_\eta$ is of order 2, and the composition are always commutative.
  Notice that the set $\set{id, \sigma_3, \sigma_{11}, \sigma_\eta, \sigma_3\circ\sigma_{11}, \sigma_3\circ\sigma_\eta, 
  \sigma_{11} \circ\sigma_\eta, \sigma_3\circ\sigma_{11}\circ\sigma_{\eta} }$ is a group of order $8$ containing $\Rat$-automorphisms of 
  $E$, thus $G = \Gal(E/\Rat)$ must be that group.

  Moreover, as every elements in the group is of order 2, then $G \isom (\ZMod[2])\times(\ZMod[2])\times(\ZMod[2])$
}

\qs{}{
  Let $\alpha$ be a root of $x^6 + 3$ and let $E = \Rat(\alpha)$. Show that $E/\Rat$ is Galois and determine $\Gal(E/\Rat)$.
}
\sol{
  Firstly, notice that $f(x) = x^6 + 3$ is irreducible over $\Rat$ by the Eisenstein criterion. Then, if $\alpha$ is a root 
  of $f(x)$, it must follows, that $\alpha, \alpha\eta$, $\alpha\eta^2$, $-\alpha$, $-\alpha\eta$, $-\alpha\eta^2$ are all of the 
  roots of $f$.
  This is because $f$ is of degree $6$, and $1, \eta, \eta^2, -1, -\eta, -\eta^2$  are pairwise distinct with $\eta^6 = (-1)^6 = 1$
  , which is because $\eta = e^{i\frac{\pi}{3}}$.

  Now, as $\alpha^6 + 3 = 0$, so let $x^2 + x + 1 = 0 = \alpha^6 + 3$. Solving for $x$ gives 
  \eqs{
    x^2 + x - (\alpha^6 - 2) &= 0 \\ 
    x &= \frac{-1 \pm \sqrt{1 + 4\alpha^6 + 8}}{2} \\ 
      &= \frac{-1 \pm \sqrt{9 + 3\alpha^6}}{2} \\ 
      &= \frac{-1 \pm \alpha^3}{2}
  }
  Disregarding one of the solutions, it is possible to assigh $\bar\eta = \frac{\alpha^3 - 1}{2}$ so that 
  \[ \bar\eta^2 + \bar\eta + 1 = \frac{\alpha^6 - 2\alpha^3 + 1}{4} + \frac{\alpha^3 - 1}{2} + 1 
  = \frac{1}{4}(\alpha^6 - 2\alpha^3 + 1 + 2\alpha^3 - 2 + 4) = 0 \]
  Morover, $\bar\eta \ne 1$ as otherwise the minimal polynomial of $\alpha$ must be of degree 3 contradicting the irreducibility of $f$.

  Thus, $\bar\eta$ is a primitive third root, which means that 
  \[ \set{\alpha, \alpha\eta, \alpha\eta^2, -\alpha, -\alpha\eta, -\alpha\eta^2} =  
      \set{\alpha, \alpha\bar\eta, \alpha\bar\eta^2, -\alpha, -\alpha\bar\eta, -\alpha\bar\eta^2} \]

  Now, as $\bar\eta \in \Rat(\alpha)$, it follows that $\Rat(\alpha)$ contains all the roots of $f$, so it is the splitting field of $f$.
  So, $[E: \Rat] = \deg(f) = 6$.

  Then, let $G = \Gal(E/\Rat)$ is of order $6$.
  Consider two $\Rat$-automorphisms of $E$ which are $\phi$ and $\psi$ such that $\phi(\alpha) = -\alpha$ 
  and $\psi(\alpha) = \alpha\bar\eta$. 
  Then, 
  \eqs{ \phi\circ\psi(\alpha) &= \phi(\alpha\bar\eta)\\ 
                              &= -\alpha\cdot\phi\paren{\frac{\alpha^3 - 1}{2}}\\ 
                              &= -\alpha\paren{\frac{-\alpha^3 - 1}{2}}\\ 
                              &= \alpha\paren{\frac{\alpha^3 - 1}{2} + 1} \\
                              &= \alpha(\bar\eta + 1)
                            }
  and 
  \eqs{ \psi\circ\phi(\alpha) &= \psi(-\alpha) \\ 
                              &= -\psi(\alpha) \\ 
                              &= -\alpha\bar\eta
                            }

  Now, as $\alpha(\bar\eta + 1) - - \alpha\bar\eta = \alpha\bar\eta^2 + \alpha\bar\eta = \alpha\bar\eta(\bar\eta^2) = \alpha \ne 0$, 
  then $\phi\circ\psi \ne \psi\circ\phi$. 

  Thus $G$ is not abelian. Therefore $G \isom D_6$, as it is the unique non-abeian group of order $6$.
}

\qs{}{
  Let $E$ be a splitting field of $x^4 + 1$ over $\Rat$. Find $\Gal(E/\Rat)$.
}
\sol{
  Notice that $f(x) = x^4 + 1$ is irreducible because if it is not, then $x^4 + 1$ should be reducible over $\mathbb F_2$. This is 
  because if $f(x) = P(x)Q(x)$ over $\Rat$, then $\bar{f}(x) = \bar{P}(x)\bar{Q}(x)$, where $f = \bar f, P = \bar P, Q = \bar Q \pmod 2$.
  As $f$ is monic, this also means that one of the irreducible divisors of $\bar f$ is a monic.

  However, $\gcd(x^4 + 1, x^{2^2} - x) = \gcd(x + 1, x^3 - 1) = \gcd(x + 1, x - 1) = 1$, which contradicts the existence of such divisors, 
  so $f(x)$ must be irreducible.

  Since \[ f(x) = x^4 + 1 = (x^2 - i)(x^2 + i) = (x - \sqrt{i})(x + \sqrt{i})(x - i\sqrt{i})(x + i\sqrt{i}) \]
  Notice that $-\sqrt{i} = \sqrt{i}^5$, $i\sqrt{i} = \sqrt{i}^3$, and $-i\sqrt{i} = \sqrt{i}^7$.
  Then, clearly, 
  $E = \Rat(\sqrt{i})$. Therefore, $[E : \Rat] = 4$.

  This means that $G = \Gal(E/\Rat)$ is a subgroup of order 4. 


  Let $\phi$ be a $\Rat$-automorphism of $E$ that sends $\sqrt{i}$ to $i\sqrt{i}$.
  Then, \[ \phi^2(\sqrt{i}^k) = \phi((i\sqrt{i})^k) = \phi(\sqrt{i}^{3k}) = (i\sqrt{i})^{3k} = \sqrt{i}^{9k} = \sqrt{i}^k \]
  As $\phi^2$ fixes $\Rat$ and all the roots of $f(x)$, it fixes $E$. So, $\phi^2 = id$.
  

  Now, let $\psi$ be another $\Rat$-automorphism of $E$ that sends $\sqrt{i}$ to $-\sqrt{i}$.
  Then, \[ \psi^2(\sqrt{i}^k) = \psi((-1)^k\sqrt{i}^k) = (-1)^k\psi(\sqrt{i}) = (-1)^{2k}\sqrt{i}^k = \sqrt{i}^k \]
  Again, $\psi^2$ fixes $\Rat$ and all roots of $E$, so it is the identity. 

  As $\phi \ne \psi$ because $\phi(\sqrt{i}) \ne \psi(\sqrt{i})$, there are at least two elements of $G$ of order 2. However, there is 
  only one group with these properties, which is $K_4$, so $G \isom K_4 \isom (\ZMod[2])\times(\ZMod[2])$. This is because there are
  two groups of order $4$, which are $C_4$ and $K_4$, but $C_4$ contains only one element with order 2.
}

\qs{}{
  Let $E/F$ be a Galois extension such that $[E: F]$ is even. Prove that there exists a subfield $K$ of $E$ 
  containing $F$ such that $[E: K] = 2$
}
\sol{
  Let $G = \Gal(E/F)$ so that $\abs{G} = 2n$ for some $n$. 
  By the Sylow's theorems, there is a subgroup of order of order 2, let $H < G$ with $\abs{H} = 2$. 

  Then, consider the fixed point of $H$, which is \[ K = \set{ a \in E \mid \sigma(a) = a \forall \sigma \in H } \]
  Notice that $H < G = \Gal(E/F)$, thus all elements of $H$ fixes $F$. Hence, $F \subset K$.
  Also, $K$ is a field because $\sigma \in H$ is an automorphism. If $\sigma$ fixes $k, l \in K$, then it must also fix 
  $k^{-1}, kl, k - l$, as $\sigma(1) = \sigma(k)\sigma(k^{-1})$ and the rest due to the properties of homomorphism.

  By the Galois correspondence theorem, $\Gal(E/K) = H$, therefore, $[E : K] = \abs{H} = 2$.
}

\end{document}
