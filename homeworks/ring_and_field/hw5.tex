% chktex-file 44
% chktex-file 8

\documentclass{report}
\usepackage{amsthm}
\usepackage{amsmath}
\usepackage{amssymb}
\usepackage{amssymb}
\usepackage{amsfonts}
\usepackage{xcolor}
\usepackage{tikz}
\usepackage{fancyhdr}
\usepackage{enumerate}
\usepackage{graphicx}
\usepackage[normalem]{ulem}
\usepackage[most,many,breakable]{tcolorbox}
\usepackage[a4paper, top=80pt, foot=25pt, bottom=50pt, left=0.5in, right=0.5in]{geometry}
\usepackage{hyperref, theoremref}
\hypersetup{
	pdftitle={Assignment},
	colorlinks=true, linkcolor=b!90,
	bookmarksnumbered=true,
	bookmarksopen=true
}
\usepackage{nameref}
\usepackage{parskip}
\pagestyle{fancy}

\usepackage[explicit,compact]{titlesec}
\titleformat{\chapter}[block]{\bfseries\huge}{\thechapter. }{\compact}{#1}
        

%%%%%%%%%%%%%%%%%%%%%
%% Defining colors %%
%%%%%%%%%%%%%%%%%%%%%

\definecolor{lr}{RGB}{188, 75, 81}
\definecolor{r}{RGB}{249, 65, 68}
\definecolor{dr}{RGB}{174, 32, 18}
\definecolor{lo}{RGB}{255, 172, 129}
\definecolor{do}{RGB}{202, 103, 2}
\definecolor{o}{RGB}{238, 155, 0}
\definecolor{ly}{RGB}{255, 241, 133}
\definecolor{y}{RGB}{255, 229, 31}
\definecolor{dy}{RGB}{143, 126, 0}
\definecolor{lb}{RGB}{148, 210, 189}
\definecolor{bg}{RGB}{10, 147, 150}
\definecolor{b}{RGB}{39, 125, 161}
\definecolor{db}{RGB}{0, 95, 115}
\definecolor{p}{RGB}{229, 152, 155}
\definecolor{dp}{RGB}{181, 101, 118}
\definecolor{pp}{RGB}{142, 143, 184}
\definecolor{v}{RGB}{109, 89, 122}
\definecolor{lg}{RGB}{144, 190, 109}
\definecolor{g}{RGB}{64, 145, 108}
\definecolor{dg}{RGB}{45, 106, 79}

\colorlet{mysol}{g}
\colorlet{mythm}{lr}
\colorlet{myqst}{db}
\colorlet{myclm}{lb}
\colorlet{mywrong}{r}
\colorlet{mylem}{o}
\colorlet{mydef}{lg}
\colorlet{mycor}{lb}
\colorlet{myrem}{dr}

%%%%%%%%%%%%%%%%%%%%%

\newcommand{\col}[2]{
  \color{#1}#2\color{black}\,
}

\newcommand{\TODO}[1][5cm]{
  \color{red}TODO\color{black}
  \vspace{#1}
}

\newcommand{\wans}[1]{
	\noindent\color{mywrong}\textbf{Wrong answer: }\color{black}
	#1 


}

\newcommand{\wreason}[1]{
	\noindent\color{mywrong}\textbf{Reason: }\color{black}
	#1 

  
}

\newcommand{\sol}[1]{
	\noindent\color{mysol}\textbf{Solution: }\color{black}
	#1


}

\newcommand{\nt}[1]{
  \begin{note}Note: #1\end{note}
}

\newcommand{\ky}[1]{
  \begin{key}#1\end{key}
}

\newcommand{\pf}[1]{
  \begin{myproof}#1\end{myproof}
}

\newcommand{\qs}[3][]{
  \begin{question}{#2}{#1}#3\end{question}
}

\newcommand{\df}[3][]{
  \begin{definition}{#2}{#1}#3\end{definition}
}

\newcommand{\thm}[3][]{
  \begin{theorem}{#2}{#1}#3\end{theorem}
}

\newcommand{\clm}[3][]{
  \begin{claim}{#2}{#1}#3\end{claim} 
}

\newcommand{\lem}[3][]{
  \begin{lemma}{#2}{#1}#3\end{lemma}
}

\newcommand{\cor}[3][]{
  \begin{corollary}{#2}{#1}#3\end{corollary}
}

\newcommand{\rem}[3][]{
  \begin{remark}{#2}{#1}#3\end{remark}
}

\newcommand{\twoways}[2]{
  \leavevmode\\
  ($\Longrightarrow$): 
  \begin{shift}#1\end{shift}
  ($\Longleftarrow$):
  \begin{shift}#2\end{shift} 
}

\newcommand{\nways}[2]{
  \leavevmode\\
  ($#1$): 
  \begin{shift}#2\end{shift}
}

%%%%%%%%%%%%%%%%%%%%%%%%%%%%%% ENVRN

\newenvironment{myproof}[1][\proofname]{%
	\proof[\bfseries #1: ]
}{\endproof}

\tcbuselibrary{theorems,skins,hooks}
\newtcolorbox{shift}
{%
  before upper={\setlength{\parskip}{5pt}},
  blanker,
	breakable,
	width=0.95\textwidth,
  enlarge left by=0.03\textwidth,
}

\tcbuselibrary{theorems,skins,hooks}
\newtcolorbox{key}
{%
	breakable,
	width=0.95\textwidth,
  enlarge left by=0.03\textwidth,
}

\tcbuselibrary{theorems,skins,hooks}
\newtcolorbox{note}
{%
	enhanced,
	breakable,
	colback = white,
	width=\textwidth,
	frame hidden,
	borderline west = {2pt}{0pt}{black},
	sharp corners,
}

\tcbuselibrary{theorems,skins,hooks}
\newtcbtheorem[]{remark}{Remark}
{%
	enhanced,
	breakable,
	colback = white,
	frame hidden,
	boxrule = 0sp,
	borderline west = {2pt}{0pt}{myrem},
	sharp corners,
	detach title,
  before upper={\setlength{\parskip}{5pt}\tcbtitle\par\smallskip},
	coltitle = myrem,
	fonttitle = \bfseries\sffamily,
	description font = \mdseries,
	separator sign none,
	segmentation style={solid, myrem},
}{rem}

\tcbuselibrary{theorems,skins,hooks}
\newtcbtheorem[number within=section]{lemma}{Lemma}
{%
	enhanced,
	breakable,
	colback = white,
	frame hidden,
	boxrule = 0sp,
	borderline west = {2pt}{0pt}{mylem},
	sharp corners,
	detach title,
  before upper={\setlength{\parskip}{5pt}\tcbtitle\par\smallskip},
	coltitle = mylem,
	fonttitle = \bfseries\sffamily,
	description font = \mdseries,
	separator sign none,
	segmentation style={solid, mylem},
}{lem}

\tcbuselibrary{theorems,skins,hooks}
\newtcbtheorem{claim}{Claim}
{%
  parbox=false,
	enhanced,
	breakable,
	colback = white,
	frame hidden,
	boxrule = 0sp,
	borderline west = {2pt}{0pt}{myclm},
	sharp corners,
	detach title,
  before upper={\setlength{\parskip}{5pt}\tcbtitle\par\smallskip},
	coltitle = myclm,
	fonttitle = \bfseries\sffamily,
	description font = \mdseries,
	separator sign none,
	segmentation style={solid, myclm},
}{clm}

\makeatletter
\newtcbtheorem[number within=section, use counter from=lemma]{theorem}{Theorem}{enhanced,
	breakable,
	colback=white,
	colframe=mythm,
	attach boxed title to top left={yshift*=-\tcboxedtitleheight},
	fonttitle=\bfseries,
	title={#2},
	boxed title size=title,
	boxed title style={%
			sharp corners,
			rounded corners=northwest,
			colback=mythm,
			boxrule=0pt,
		},
	underlay boxed title={%
			\path[fill=mythm] (title.south west)--(title.south east)
			to[out=0, in=180] ([xshift=5mm]title.east)--
			(title.center-|frame.east)
			[rounded corners=\kvtcb@arc] |-
			(frame.north) -| cycle;
		},
	#1
}{thm}
\makeatother

\makeatletter
\newtcbtheorem{question}{Question}{enhanced,
	breakable,
	colback=white,
	colframe=myqst,
	attach boxed title to top left={yshift*=-\tcboxedtitleheight},
	fonttitle=\bfseries,
	title={#2},
	boxed title size=title,
	boxed title style={%
			sharp corners,
			rounded corners=northwest,
			colback=myqst,
			boxrule=0pt,
		},
	underlay boxed title={%
			\path[fill=myqst] (title.south west)--(title.south east)
			to[out=0, in=180] ([xshift=5mm]title.east)--
			(title.center-|frame.east)
			[rounded corners=\kvtcb@arc] |-
			(frame.north) -| cycle;
		},
	#1
}{qs}
\makeatother

\makeatletter
\newtcbtheorem[number within=section]{definition}{Definition}{enhanced,
	breakable,
	colback=white,
	colframe=mydef,
	attach boxed title to top left={yshift*=-\tcboxedtitleheight},
	fonttitle=\bfseries,
	title={#2},
	boxed title size=title,
	boxed title style={%
			sharp corners,
			rounded corners=northwest,
			colback=mydef,
			boxrule=0pt,
		},
	underlay boxed title={%
			\path[fill=mydef] (title.south west)--(title.south east)
			to[out=0, in=180] ([xshift=5mm]title.east)--
			(title.center-|frame.east)
			[rounded corners=\kvtcb@arc] |-
			(frame.north) -| cycle;
		},
	#1
}{def}
\makeatother

\makeatletter
\newtcbtheorem[number within=section, use counter from=lemma]{corollary}{Corollary}{enhanced,
	breakable,
	colback=white,
	colframe=mycor,
	attach boxed title to top left={yshift*=-\tcboxedtitleheight},
	fonttitle=\bfseries,
	title={#2},
	boxed title size=title,
	boxed title style={%
			sharp corners,
			rounded corners=northwest,
			colback=mycor,
			boxrule=0pt,
		},
	underlay boxed title={%
			\path[fill=mycor] (title.south west)--(title.south east)
			to[out=0, in=180] ([xshift=5mm]title.east)--
			(title.center-|frame.east)
			[rounded corners=\kvtcb@arc] |-
			(frame.north) -| cycle;
		},
	#1
}{cor}
\makeatother

% Basic
  \DeclareMathOperator{\lcm}{lcm}
  \newcommand{\Real}{\mathbb{R}}
  \newcommand{\Comp}{\mathbb{C}}
  \newcommand{\Nat}{\mathbb{N}}
  \newcommand{\Rat}{\mathbb{Q}}
  \newcommand{\Int}{\mathbb{Z}}
  \newcommand{\set}[1]{\left\{\, #1 \,\right\}}
  \newcommand{\paren}[1]{\left( \; #1 \; \right)}
  \newcommand{\abs}[1]{\left\lvert #1 \right\rvert}
  \newcommand{\ang}[1]{\left\langle #1 \right\rangle}
  \renewcommand{\to}[1][]{\xrightarrow{\text{#1}}}
  \newcommand{\tol}[1][]{\to{$#1$}}
  \newcommand{\curle}{\preccurlyeq}
  \newcommand{\curge}{\succcurlyeq}
  \newcommand{\mapsfrom}{\leftarrow\!\shortmid}

  \newcommand{\mat}[1]{\begin{bmatrix} #1 \end{bmatrix}}
  \newcommand{\pmat}[1]{\begin{pmatrix} #1 \end{pmatrix}}
  \newcommand{\eqs}[1]{\begin{align*} #1 \end{align*}}
  \newcommand{\case}[1]{\begin{cases} #1 \end{cases}}
  

  % Algebra
  \newcommand{\normSg}[0]{\vartriangleleft}
  \newcommand{\ZMod}[1][n]{\mathbb{Z}/#1\mathbb{Z}}
  \newcommand{\isom}{\simeq}
  \newcommand{\mapHom}{\xrightarrow{\text{hom}}}
  \DeclareMathOperator{\Inn}{Inn}
  \DeclareMathOperator{\Aut}{Aut}
  \DeclareMathOperator{\im}{im}
  \DeclareMathOperator{\ord}{ord}
  \DeclareMathOperator{\Gal}{Gal}
  \DeclareMathOperator{\chr}{char}
  \newcommand{\surjto}{\twoheadrightarrow}
  \newcommand{\injto}{\hookrightarrow}

  % Analysis 
  \newcommand{\limty}[1][k]{\lim_{#1\to\infty}}
  \newcommand{\norm}[1]{\left\lVert#1\right\rVert}
  \newcommand{\darrow}{\rightrightarrows}


\fancyhead[L]{Modern Algebra 2 - MAS312}
\fancyhead[R]{\textbf{Touch Sungkawichai} 20210821}

\newcommand{\F}{\mathbb{F}}

\begin{document}

\qs{}{
  Construct the field of $9$ elements. Write out the addition and multiplication tables.
} 
\sol{
  Consider the set $\set{0, 1, 2, i, 1 + i, 2 + i, 2i, 2 + i, 2 + 2i}$ of 9 elements with the following tables.
  \begin{center}
    \begin{tabular}{ |c|c c c c c c c c c| } 
      \hline
      $+$   & 0    & 1    & 2    & i    & 1+i  & 2+i  & 2i   & 1+2i & 2+2i \\ 
      \hline
      0     & 0    & 1    & 2    & i    & 1+i  & 2+i  & 2i   & 1+2i & 2+2i \\ 
      1     & 1    & 2    & 0    & 1+i  & 2+i  & i    & 1+2i & 2+2i & 2i   \\ 
      2     & 2    & 0    & 1    & 2+i  & i    & 1+i  & 2+2i & 2i   & 1+2i \\ 
      i     & i    & 1+i  & 2+i  & 2i   & 1+2i & 2+2i & 0    & 1    & 2    \\ 
      1+i   & 1+i  & 2+i  & i    & 1+2i & 2+2i & 2i   & 1    & 2    & 0    \\ 
      2+i   & 2+i  & i    & 1+i  & 2+2i & 2i   & 1+2i & 2    & 0    & 1    \\ 
      2i    & 2i   & 1+2i & 2+2i & 0    & 1    & 2    & i    & 1+i  & 2+i  \\ 
      1+2i  & 1+2i & 2+2i & 2i   & 1    & 2    & 0    & 1+i  & 2+i  & i    \\ 
      2+2i  & 2+2i & 2i   & 1+2i & 2    & 0    & 1    & 2+i  & i    & 1+i  \\ 
      \hline
    \end{tabular}
  \end{center}
  and 
  \begin{center}
    \begin{tabular}{ |c|c c c c c c c c c| } 
      \hline
      $\times$ & 0 & 1    & 2    & i    & 1+i  & 2+i  & 2i   & 1+2i & 2+2i \\ 
      \hline
      0        & 0 & 0    & 0    & 0    & 0    & 0    & 0    & 0    & 0    \\ 
      1        & 0 & 1    & 2    & i    & 1+i  & 2+i  & 2i   & 1+2i & 2+2i \\ 
      2        & 0 & 2    & 1    & 2i   & 2+2i & 1+2i & i    & 2+i  & 1+i  \\ 
      i        & 0 & i    & 2i   & 2    & 2+i  & 2+2i & 1    & 1+i  & 1+2i \\ 
      1+i      & 0 & 1+i  & 2+2i & 2+i  & 2i   & 1    & 1+2i & 2    & i    \\ 
      2+i      & 0 & 2+i  & 1+2i & 2+2i & 1    & i    & 1+i  & 2i   & 2    \\ 
      2i       & 0 & 2i   & i    & 1    & 1+2i & 1+i  & 2    & 2+2i & 2+i  \\ 
      1+2i     & 0 & 1+2i & 2+i  & 1+i  & 2    & 2i   & 2+2i & i    & 1    \\ 
      2+2i     & 0 & 2+2i & 1+i  & 1+2i & i    & 2    & 2+i  & 1    & 2i   \\ 
      \hline
    \end{tabular}
  \end{center}
  Since the field is unique up to isomorphism, the field of $9$ elements is as described in the table.
}

\qs{}{
  Determine whether or not two fields $\F_3[x]/(x^2 - 2)$ and $\F_3[x]/(x^2 - 2x - 1)$ are isomorphic. If they are isomorphic, find
  an isomorphism.
}
\sol{
  Notice that since there are $3$ irreducible linear polynomials over $\F_3$ which are $x$, $x - 1$, and $x+1$, and 
  $x^3 - x = x(x+1)(x-1)$
  Next, since \[ \gcd(x^2 - 2, x^3 - x) = \gcd(x^2 - 2, x^2 - 1) = 1 \] and 
 \[\gcd(x^2 - 2x - 1, x^3 - x) = \gcd(x^2 - 2x - 1, x^2 - 1) = \gcd(x^2 - 2x - 1, 2x) = \gcd(x^2 - 1, 2x) = 1 \]
  it follows that, $x^2 - 2$ and $x^2 - 2x - 1$ are both irreducible, thus 
  \[\F_{3^2} \isom \F_3[x]/(x^2 - 2) \isom \F_3[x]/(x^2 - 2x - 1) \]

  Note that since the polynomials are irreducible, these two are fields. 
  
  Now, for the isomorphism,
  consider that for any $f(x) \in \F_3[x]$, there exists $r(x)$ such that $f(x) = q(x)(x^2 - 2x - 1) + r(x)$ 
  where $\deg(r) \le 1$ by the euclidean algorithm. Thus, for any $f(x) \in \F_3[x]/(x^2 - 2x - 1)$, $f(x) = r(x)$
  for some linear or constant $r(x)$. 
  Moreover, there exists $r'(x) \in \F_3[x]/(x^2 - 2)$ such that $\phi(r') = r$ when 
  \[ \phi: \F_3[x]/(x^2 - 1) \injto \F_3[x] \surjto \F_3[x]/(x^2 - 2x - 1) \] by the natural embedings.
  
  And if $\phi(r)(x) = 0 \in \F_3[x]/(x^2 - 2x - 1)$, then the corresponding polynomial in $\F_3[x]$ (in the middle step of $\phi$) 
  should only be $q(x)(x^2 - 2x - 1)$ for some $q(x)$. Then, as the $\F_3[x]/(x^2 - 2) \injto \F_3[x]$ is the natural embeding, 
  we have that $r = 0$, as there is no polynomial of degree greater than 1 in $\F_3[x]/(x^2 - 2)$ and the natural embeding 
  preserves degree.

  Therefore, $\phi$ is injective and surjective, thus it is an isomorphism.
}

\qs{}{
  Let $\F_q$ be a finite field and let $n$ be a positive integer. Show that there exists an irreducible polynomial over $\F_q$ of 
  degree $n$.
}
\sol{
  \clm[fieldpower]{Existence of $\F_{q^n}$}{
    $\F_q$ must have characteristic $p$ for some prime $p$ as it is finite. Thus, $\F_p \subset \F_q$. 
    Then, the degree $[\F_q : \F_p] = k$ for some integer, so $q = p^k$. Therefore, there exists a field $\F_{q^n} = \F_{p^{kn}}$. 
  }

  Since there is such field, consider $E = \F_{q^n}$ and that $[E : \F_q] = n$ and $E$ is a finite extension, 
  thus $E = \F_q(\alpha)$ for some $\alpha \in E$.
  But since the degree of $[E : \F_q] = n$, then the minimal polynomial $m_\alpha \in \F_q[x]$ is of degree $n$.
  Since $m_\alpha$ is minimal, it is irreducible. 
}

\qs{}{
  Find a splitting field of $x^6 - 3$ over $\F_7$ and the degree of the splitting field. 
}
\sol{
  Notice that if there is a linear or
  quadratic irreducible element that divides $x^6 - 3$, then it must divides $x^{7^2} - x$ since $x^{7^2} - x$ is 
  the product of all irreducible polynomial degree $1$ and $2$.

  Note that over a field of characteristic 7,  \[ (x^6 - 3)^7 = {x^6}^7 - 3^7 \] and
  \[ ((x^6 - 3)^7 + 3^7)(x^6 - 3) = (x^{42})(x^6 - 3) = (x^{48} - 3x^{42}) \]
  Then, if something divides $x^6 - 3$ and $x^{7^2} - x$, then it must divides $\gcd(x^6 - 3, x^{49} - x)$.
  But \eqs{ 
    \gcd(x^6 - 3, x^{49} - x) &= \gcd(x^6 - 3, x^{48} - 1) \\  
                              &= \gcd(x^6 - 3, x^{48} - 1 - x^{48} + 3x^{42}) \\ 
                              &= \gcd(x^6 - 3, 3x^{42} - 1 - 3(x^{42} - 3^7) \\ 
                              &= \gcd(x^6 - 3, 3^8 - 1)\\ 
                              &= 1
                            }
  Thus, there is none.

  Next, if there is a cubic irreducible polynomial dividing $x^6 - 3$, then it must divides $x^{7^3} - x$ since $x^{7^3} - x$ is the 
  product of all irreducible polynomial degree dividing $3$.

  Note that $7^3 = 343$, 
  \[ ((x^6 - 3)^7)^7 = (x^{42} - 3^7)^7 = (x^{294} - 3^{49}) \] 
  and \[ (x^{48})(x^6 - 3)^{49} = (x^{48})(x^{294} - 3^{49}) = x^{342} - 3^{49}x^{48} \]
  So, \eqs{
    \gcd(x^6 - 3, x^{343} - x) &= \gcd(x^6 - 3, x^{342} - 1) \\ 
                               &= \gcd(x^6 - 3, x^{342} - 1 - x^{342} + 3^{49}x^{48}) \\ 
                               &= \gcd(x^6 - 3, 3^{49}(x^{48} - 1) + 3^{49} - 1) \\
                               &= \gcd(x^6 - 3, 3^{49}(3^8 - 1) + 3^{49} - 1) \\ 
                               &= 1
  } 
  Thus, there is none. 

  If there is no irreducible divisor of degree less than 4, there is no irreducible divisor. Thus, $x^6 - 3$ is irreducible.
  Let $E$ be the splitting field of $f$ over $\F_7$. Then since $\F_7 \subset E$, $E = \F_{7^k}$ for some $k$. If $k \le 5$, it is 
  already shown that $x^6 - 3$ is irreducible, thus, does not divide ${x^7}^k - x$ which is the product of irreducible degree dividing 
  $k$. As $\F_{7^k}$ is the splitting field of ${x^7}^k - x$, then it is not the splitting field of $f$. 

  However, $x^6 - 3$ divides ${x^7}^6 - x$ since it is the product of all irreducible polynomials degree dividing $6$.
  So, $\F_{7^6}$ splits ${x^7}^6 - x$, thus it splits $f$.
  Therefore, the spliting field of $f$ over $\F_7$ is $\F_{7^6}$, which gives that $[\F_{7^6}: \F_{7}] = 6$.
}

\qs{}{
  Let $f \in \F_q[x]$. Show that if $f$ is irreducible, then $f$ divides $x^{q^{\deg(f)}} - x$.
}
\sol{
  Let $\alpha$ be a root of $f$, then $[\F_q(\alpha) : \F_q] = \deg(f) = n$.
  As $q = p^k$ and there exists a field $\F_{q^n} = \F_{p^{nk}}$ (as per claim~\ref{clm:fieldpower}). Then, 
  $\F_q(\alpha) = \F_{q^n} = \F_{p^{nk}}$.

  Now, $\F_{p^{nk}}$ is the splitting field of $x^{p^{nk}} - x$ over $\F_p$ and $\alpha$ is an element in the splitting field 
  with $\alpha \notin \F_q$. (because $\F_q(\alpha) = \F_{q^n}$). Therefore, $\alpha$ is a root of $x^{q^n}$. 
  Hence, it follows that $f \mid x^{q^{n}} - x$.
}

\qs{}{
  Let $p$ be a prime integer. Find the smallest integer $n$ such that $\F_{p^n}$ contains two subfields isomorphic to $\F_{p^r}$ and 
  $\F_{p^s}$.
}
\sol{
  Notice that $\F_{p^r} \subset \F_{p^n}$ if and only if $r \mid n$ and similarly for $s$.
  If $n = \lcm(r, s)$ be the smallest integer that is divisible by $r$ and $s$, then, 
  $\F_{p^r} \subset \F_{p^n}$ and $\F_{p^s} \subset \F_{p^n}$. 
  And by definition, $\lcm(r, s)$ is the least number, thus $n = \lcm(r, s)$.
} 

\qs{}{
  Prove that every finite extension of a finite field is normal.
}
\sol{
  Let $F = \F_q$ be an arbitrary finite field and $E/F$ be a finite extension with $[E:F] = n$. 
  Then, $q = p^k$ and that there exists $\F_{q^n}$ as from claim~\ref{clm:fieldpower}. 

  Since $[E : F] = [\F_{q^n} : F]$ is finite over finite field $F$, then $\abs{E} = \abs{\F_{q^n}}$, which means that they are isomorphic
  due to the uniqueness of finite fields.

  Now, as $\F_{q^n}$ is the splitting field of $x^{p^{nk}} - x$ over $\F_p$, then it is normal over $\F_p$. 
  Moreover, ad $\F_q$ is an extension of $\F_p$, then $\F_{q^n}$ is also normal over $\F_q$.
  Lastly, as $E \isom \F_{q^n}$, $E$ is normal over $\F_q = F$.
}

\qs{}{
  Let $F$ be a field of $\text{char}(F) = p$. Prove that the quotient field of the polynomial ring $F[x]$ over $F(x^p)$ is normal.
}
\sol{
  Consider that $x^p$ is transcendental in $F$ because if not, then there is a polynomial $a_0 + a_1x^p + \cdots + a_nx^{pn} = 0$ where 
  $a_i \in F$. But that means $x$ is also algebraic over $F$, which contradicts that $F[x]$ is a polynomial ring.

  Therefore, $F[x^p]$ is a polynomial ring. Consider that $x^p$ is irreducible $F[x^p]$, so the Eisenstein criterion applies for the 
  $f(t) = t^p - x^p$ in $F[x^p][t]$. Hence, $f(t)$ is irreducible over $F(x^p)$. Since $x$ is a root of the polynomial, then 
  $[F(x) : F(x^p)] = p$ as $f$ is the minimal polynomial of $x$ and $F(x) = F(x^p)(x)$.
  Note also that $p$ is a prime integer.

  Next, notice that $f(t) = t^p - x^p = (t - x)^p$ over any field $F$ of characteristic $p$. Therefore, $f(t)$ splits over $F(x)$.
  Moreover, if there is another field $F(x)/E/F(x^p)$, then $[E : F(x^p)] = 1$, which is $E = F(x^p)$ or $[F(x) : E] = 1$, which is 
  that $E = F(x)$. Therefore, $F(x)$ is the splitting field of $f(t)$ over $F(x^p)$. Therefore, $F(x)$, the quotient 
  field of $F[x]$, is normal over $F(x^p)$.
}

\qs{}{
  Show that the polynomial $x^4 + 1$ is not irreducible over any field of nonzero characteristic.
}
\sol{
  For $p = 2$, consider that $1 + 1 = 0$.
  This implies that \[(x + 1)^4 = (x^2 + x + x + 1)^2 = (x^2 + 1)^2 = (x^4 + x^2 + x^2 + 1) = (x^4 + 1)\]
  which means $(x^4+1)$ is not irreducible over any field of characteristic 2. 

  Otherwise $p$ is odd. Then there are 4 cases for $p$, which is $p \equiv 1, 3, 5, 7 \pmod{8}$.

  For the case that $p \equiv 1 \text{ or } 7 \pmod{8}$, there exist a number $r$ such that $r^2 \equiv 2 \pmod{p}$.
  In other words, in the field with characteristic $p \equiv \pm 1 \pmod{8}$, there is an element $r$ such that $r \cdot r = 2$.
  Then, as \[ (x^2 - rx + 1)(x^2 + rx + 1) = (x^2 + 1)^2 - (rx)^2 = x^4 + 2x^2 + 1 - r^2x^2 = x^4 + 1 \]
  the polynomial is reducible.

  Lastly, if $p \equiv 3 \text{ or } 5 \pmod{8}$, there exist a number $r$ such that $r^2 \equiv -2 \pmod{p}$, which means that 
  in the field of characteristic $p \equiv \pm 3 \pmod{8}$, there must be an element $r$ such that $r \cdot r = -2$.
  Then, similarly, \[ (x^2 - rx - 1)(x^2 + rx - 1) = (x^2 - 1)^2 - (rx^2) = x^4 - 2x^2 + 1 - r^2x^2 = x^4 + 1 \].
  This implies that the polynomial is not irreducible.

  Thus, the polynomial $x^4 + 1$ is not irreducible over any field of non-zero characteristic.
}

\qs{}{
  Let $F$ be a field. Show that if $a \in F\backslash F^p$ for a prime $p$, then $x^p - a$ is an irreducible polynomial over $F$.
}
\sol{
  Consider that $F^p$ is the set $\set{ x^p \mid x \in F }$, then let $a \in F$ and assume that $x^p - a$ is reducible. 
  There must be some polynomial $g \in F[x]$ with $\deg(g) = k$ such that $k < r$ and $g \mid f$.
  Let $E$ be the splitting field of $g$ over $F$, so that \[ g(x) = (x - \alpha_1)(x - \alpha_2) \cdots (x - \alpha_k) \]
  As $g \in F[x]$, then $g_0$, the constant term of $g$ must be an element of $F$, therefore, 
  \[ \alpha_1\alpha_2\cdots\alpha_k = g_0 \in F \]

  Since $\alpha_i$ is a root of $g$, then it is of $f$, so $f(\alpha_i) = 0$ for any $i$. This means that ${\alpha_i}^p = a$ for all $i$.
  Next, consider that \[ a^k = {\alpha_1}^p{\alpha_2}^p\cdots{\alpha_k}^p = {(\alpha_1\alpha_2\cdots\alpha_k)}^p = {g_0}^p \]

  Since $k < p$ and $p$ is prime, then there exists integer $n, m$ making $nk + mp = 1$.
  From $a^k = {g_0}^p$, it could be infer that \[ a = a^{nk + mp} = a^{nk}a^{mp} = g_0^{np}a^{mp} = {(g_0^na^m)}^p \]
  Since $g_0^na^m \in F$, then $a \in F^p$. 

  Hence, by contraposition, if $a \in F \backslash F^p$ for a prime $p$, then $x^p - a$ is an irreducible polynomial over $F$.
}

\end{document}
