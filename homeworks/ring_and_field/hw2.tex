% chktex-file 44
% chktex-file 8

\documentclass{report}
\usepackage{amsthm}
\usepackage{amsmath}
\usepackage{amssymb}
\usepackage{amssymb}
\usepackage{amsfonts}
\usepackage{xcolor}
\usepackage{tikz}
\usepackage{fancyhdr}
\usepackage{enumerate}
\usepackage{graphicx}
\usepackage[normalem]{ulem}
\usepackage[most,many,breakable]{tcolorbox}
\usepackage[a4paper, top=80pt, foot=25pt, bottom=50pt, left=0.5in, right=0.5in]{geometry}
\usepackage{hyperref, theoremref}
\hypersetup{
	pdftitle={Assignment},
	colorlinks=true, linkcolor=b!90,
	bookmarksnumbered=true,
	bookmarksopen=true
}
\usepackage{nameref}
\usepackage{parskip}
\pagestyle{fancy}

\usepackage[explicit,compact]{titlesec}
\titleformat{\chapter}[block]{\bfseries\huge}{\thechapter. }{\compact}{#1}
        

%%%%%%%%%%%%%%%%%%%%%
%% Defining colors %%
%%%%%%%%%%%%%%%%%%%%%

\definecolor{lr}{RGB}{188, 75, 81}
\definecolor{r}{RGB}{249, 65, 68}
\definecolor{dr}{RGB}{174, 32, 18}
\definecolor{lo}{RGB}{255, 172, 129}
\definecolor{do}{RGB}{202, 103, 2}
\definecolor{o}{RGB}{238, 155, 0}
\definecolor{ly}{RGB}{255, 241, 133}
\definecolor{y}{RGB}{255, 229, 31}
\definecolor{dy}{RGB}{143, 126, 0}
\definecolor{lb}{RGB}{148, 210, 189}
\definecolor{bg}{RGB}{10, 147, 150}
\definecolor{b}{RGB}{39, 125, 161}
\definecolor{db}{RGB}{0, 95, 115}
\definecolor{p}{RGB}{229, 152, 155}
\definecolor{dp}{RGB}{181, 101, 118}
\definecolor{pp}{RGB}{142, 143, 184}
\definecolor{v}{RGB}{109, 89, 122}
\definecolor{lg}{RGB}{144, 190, 109}
\definecolor{g}{RGB}{64, 145, 108}
\definecolor{dg}{RGB}{45, 106, 79}

\colorlet{mysol}{g}
\colorlet{mythm}{lr}
\colorlet{myqst}{db}
\colorlet{myclm}{lb}
\colorlet{mywrong}{r}
\colorlet{mylem}{o}
\colorlet{mydef}{lg}
\colorlet{mycor}{lb}
\colorlet{myrem}{dr}

%%%%%%%%%%%%%%%%%%%%%

\newcommand{\col}[2]{
  \color{#1}#2\color{black}\,
}

\newcommand{\TODO}[1][5cm]{
  \color{red}TODO\color{black}
  \vspace{#1}
}

\newcommand{\wans}[1]{
	\noindent\color{mywrong}\textbf{Wrong answer: }\color{black}
	#1 


}

\newcommand{\wreason}[1]{
	\noindent\color{mywrong}\textbf{Reason: }\color{black}
	#1 

  
}

\newcommand{\sol}[1]{
	\noindent\color{mysol}\textbf{Solution: }\color{black}
	#1


}

\newcommand{\nt}[1]{
  \begin{note}Note: #1\end{note}
}

\newcommand{\ky}[1]{
  \begin{key}#1\end{key}
}

\newcommand{\pf}[1]{
  \begin{myproof}#1\end{myproof}
}

\newcommand{\qs}[3][]{
  \begin{question}{#2}{#1}#3\end{question}
}

\newcommand{\df}[3][]{
  \begin{definition}{#2}{#1}#3\end{definition}
}

\newcommand{\thm}[3][]{
  \begin{theorem}{#2}{#1}#3\end{theorem}
}

\newcommand{\clm}[3][]{
  \begin{claim}{#2}{#1}#3\end{claim} 
}

\newcommand{\lem}[3][]{
  \begin{lemma}{#2}{#1}#3\end{lemma}
}

\newcommand{\cor}[3][]{
  \begin{corollary}{#2}{#1}#3\end{corollary}
}

\newcommand{\rem}[3][]{
  \begin{remark}{#2}{#1}#3\end{remark}
}

\newcommand{\twoways}[2]{
  \leavevmode\\
  ($\Longrightarrow$): 
  \begin{shift}#1\end{shift}
  ($\Longleftarrow$):
  \begin{shift}#2\end{shift} 
}

\newcommand{\nways}[2]{
  \leavevmode\\
  ($#1$): 
  \begin{shift}#2\end{shift}
}

%%%%%%%%%%%%%%%%%%%%%%%%%%%%%% ENVRN

\newenvironment{myproof}[1][\proofname]{%
	\proof[\bfseries #1: ]
}{\endproof}

\tcbuselibrary{theorems,skins,hooks}
\newtcolorbox{shift}
{%
  before upper={\setlength{\parskip}{5pt}},
  blanker,
	breakable,
	width=0.95\textwidth,
  enlarge left by=0.03\textwidth,
}

\tcbuselibrary{theorems,skins,hooks}
\newtcolorbox{key}
{%
	breakable,
	width=0.95\textwidth,
  enlarge left by=0.03\textwidth,
}

\tcbuselibrary{theorems,skins,hooks}
\newtcolorbox{note}
{%
	enhanced,
	breakable,
	colback = white,
	width=\textwidth,
	frame hidden,
	borderline west = {2pt}{0pt}{black},
	sharp corners,
}

\tcbuselibrary{theorems,skins,hooks}
\newtcbtheorem[]{remark}{Remark}
{%
	enhanced,
	breakable,
	colback = white,
	frame hidden,
	boxrule = 0sp,
	borderline west = {2pt}{0pt}{myrem},
	sharp corners,
	detach title,
  before upper={\setlength{\parskip}{5pt}\tcbtitle\par\smallskip},
	coltitle = myrem,
	fonttitle = \bfseries\sffamily,
	description font = \mdseries,
	separator sign none,
	segmentation style={solid, myrem},
}{rem}

\tcbuselibrary{theorems,skins,hooks}
\newtcbtheorem[number within=section]{lemma}{Lemma}
{%
	enhanced,
	breakable,
	colback = white,
	frame hidden,
	boxrule = 0sp,
	borderline west = {2pt}{0pt}{mylem},
	sharp corners,
	detach title,
  before upper={\setlength{\parskip}{5pt}\tcbtitle\par\smallskip},
	coltitle = mylem,
	fonttitle = \bfseries\sffamily,
	description font = \mdseries,
	separator sign none,
	segmentation style={solid, mylem},
}{lem}

\tcbuselibrary{theorems,skins,hooks}
\newtcbtheorem{claim}{Claim}
{%
  parbox=false,
	enhanced,
	breakable,
	colback = white,
	frame hidden,
	boxrule = 0sp,
	borderline west = {2pt}{0pt}{myclm},
	sharp corners,
	detach title,
  before upper={\setlength{\parskip}{5pt}\tcbtitle\par\smallskip},
	coltitle = myclm,
	fonttitle = \bfseries\sffamily,
	description font = \mdseries,
	separator sign none,
	segmentation style={solid, myclm},
}{clm}

\makeatletter
\newtcbtheorem[number within=section, use counter from=lemma]{theorem}{Theorem}{enhanced,
	breakable,
	colback=white,
	colframe=mythm,
	attach boxed title to top left={yshift*=-\tcboxedtitleheight},
	fonttitle=\bfseries,
	title={#2},
	boxed title size=title,
	boxed title style={%
			sharp corners,
			rounded corners=northwest,
			colback=mythm,
			boxrule=0pt,
		},
	underlay boxed title={%
			\path[fill=mythm] (title.south west)--(title.south east)
			to[out=0, in=180] ([xshift=5mm]title.east)--
			(title.center-|frame.east)
			[rounded corners=\kvtcb@arc] |-
			(frame.north) -| cycle;
		},
	#1
}{thm}
\makeatother

\makeatletter
\newtcbtheorem{question}{Question}{enhanced,
	breakable,
	colback=white,
	colframe=myqst,
	attach boxed title to top left={yshift*=-\tcboxedtitleheight},
	fonttitle=\bfseries,
	title={#2},
	boxed title size=title,
	boxed title style={%
			sharp corners,
			rounded corners=northwest,
			colback=myqst,
			boxrule=0pt,
		},
	underlay boxed title={%
			\path[fill=myqst] (title.south west)--(title.south east)
			to[out=0, in=180] ([xshift=5mm]title.east)--
			(title.center-|frame.east)
			[rounded corners=\kvtcb@arc] |-
			(frame.north) -| cycle;
		},
	#1
}{qs}
\makeatother

\makeatletter
\newtcbtheorem[number within=section]{definition}{Definition}{enhanced,
	breakable,
	colback=white,
	colframe=mydef,
	attach boxed title to top left={yshift*=-\tcboxedtitleheight},
	fonttitle=\bfseries,
	title={#2},
	boxed title size=title,
	boxed title style={%
			sharp corners,
			rounded corners=northwest,
			colback=mydef,
			boxrule=0pt,
		},
	underlay boxed title={%
			\path[fill=mydef] (title.south west)--(title.south east)
			to[out=0, in=180] ([xshift=5mm]title.east)--
			(title.center-|frame.east)
			[rounded corners=\kvtcb@arc] |-
			(frame.north) -| cycle;
		},
	#1
}{def}
\makeatother

\makeatletter
\newtcbtheorem[number within=section, use counter from=lemma]{corollary}{Corollary}{enhanced,
	breakable,
	colback=white,
	colframe=mycor,
	attach boxed title to top left={yshift*=-\tcboxedtitleheight},
	fonttitle=\bfseries,
	title={#2},
	boxed title size=title,
	boxed title style={%
			sharp corners,
			rounded corners=northwest,
			colback=mycor,
			boxrule=0pt,
		},
	underlay boxed title={%
			\path[fill=mycor] (title.south west)--(title.south east)
			to[out=0, in=180] ([xshift=5mm]title.east)--
			(title.center-|frame.east)
			[rounded corners=\kvtcb@arc] |-
			(frame.north) -| cycle;
		},
	#1
}{cor}
\makeatother

% Basic
  \DeclareMathOperator{\lcm}{lcm}
  \newcommand{\Real}{\mathbb{R}}
  \newcommand{\Comp}{\mathbb{C}}
  \newcommand{\Nat}{\mathbb{N}}
  \newcommand{\Rat}{\mathbb{Q}}
  \newcommand{\Int}{\mathbb{Z}}
  \newcommand{\set}[1]{\left\{\, #1 \,\right\}}
  \newcommand{\paren}[1]{\left( \; #1 \; \right)}
  \newcommand{\abs}[1]{\left\lvert #1 \right\rvert}
  \newcommand{\ang}[1]{\left\langle #1 \right\rangle}
  \renewcommand{\to}[1][]{\xrightarrow{\text{#1}}}
  \newcommand{\tol}[1][]{\to{$#1$}}
  \newcommand{\curle}{\preccurlyeq}
  \newcommand{\curge}{\succcurlyeq}
  \newcommand{\mapsfrom}{\leftarrow\!\shortmid}

  \newcommand{\mat}[1]{\begin{bmatrix} #1 \end{bmatrix}}
  \newcommand{\pmat}[1]{\begin{pmatrix} #1 \end{pmatrix}}
  \newcommand{\eqs}[1]{\begin{align*} #1 \end{align*}}
  \newcommand{\case}[1]{\begin{cases} #1 \end{cases}}
  

  % Algebra
  \newcommand{\normSg}[0]{\vartriangleleft}
  \newcommand{\ZMod}[1][n]{\mathbb{Z}/#1\mathbb{Z}}
  \newcommand{\isom}{\simeq}
  \newcommand{\mapHom}{\xrightarrow{\text{hom}}}
  \DeclareMathOperator{\Inn}{Inn}
  \DeclareMathOperator{\Aut}{Aut}
  \DeclareMathOperator{\im}{im}
  \DeclareMathOperator{\ord}{ord}
  \DeclareMathOperator{\Gal}{Gal}
  \DeclareMathOperator{\chr}{char}
  \newcommand{\surjto}{\twoheadrightarrow}
  \newcommand{\injto}{\hookrightarrow}

  % Analysis 
  \newcommand{\limty}[1][k]{\lim_{#1\to\infty}}
  \newcommand{\norm}[1]{\left\lVert#1\right\rVert}
  \newcommand{\darrow}{\rightrightarrows}


\fancyhead[L]{Modern Algebra II MAS312}
\fancyhead[R]{\textbf{Touch Sungkawichai} 20210821}

\begin{document}
  \qs{}{
    Let $R$ be a domain which is not a field. Prove that $R[x]$ is not a principal ideal doamin.
  }
  \sol{
    Since $R$ is not a field, there is a non-zero $a$ such that $a \not\in R^\times$. 
    Let $I$ be the ideal generated by $a$ and $x$. If $I$ is principal, then $I = bR[x]$ for some $b \in R[x]$.
    Let $b = b_x x + b_0$, then consider $x \in I$, so $x = f(b_x x + b_0)$ for some $f = \sum_{i=0}^k f_ix^i$. 
    Since $a$ is non-zero, $b_0$ is non-zero. $R$ is a domain, $f_0$ must be $0$. 
    But then $f_i$ for $1 \le i \le k$ must be $0$ since $f_i x^i (b_x x)$ must be $0$.
    This means that $f = 0$, but $0(b_x x + b_0) \ne x$. Thus $I$ cannot be principal.
  }

  \qs{}{
    Let $S$ be a multiplicative subset of a principal ideal domain $R$. Show that the localization $S^{-1}R$ is also a principal domain.
  }
  \sol{
    Let $J$ be an ideal $\set{\frac{a/s}}$ in $S^{-1}R$. If $\frac{a}{s} \in J$, then $\frac{a}{s}\cdot\frac{s}{1} = \frac{a}{1} \in J$, 
    because $J$ is an ideal. Then, let $I = \set{a \mid \frac{a}{1} \in J} \subset R$. Then, $0 \in I$. Moreover, if $a, b \in I$ and 
    $r \in R$, then $\frac{a}{1}, \frac{b}{1} \in J$, and $\frac{r}{1} \in S^{-1}R$. 
    Therefore, $\frac{ab}{1}, \frac{a-b}{1}, \frac{ar}{1} \in J$, which means that $ab, a-b, ar \in I$. Thus, $I$ is an ideal of $R$.

    Now, assume that $I$ is principal, and $I = aR$ for some element $a \in R$.
    Then, for any element $\frac{b}{s} \in J$, it is true that $\frac{b}{s} = \frac{b}{1}\frac{1}{s}$ where $\frac{1}{s} \in S^{-1}R$.
    Thus, if $\frac{b}{1} \in J$, then $\frac{b}{s} \in J$. However, if since $b \in I$, $b = ax$ for some element $x \in R$. 
    Which means that $\frac{b}{1} = \frac{a}{1}\frac{x}{1}$ for some $\frac{x}{1} \in S^{-1}R$. Thus, any element of $J$ is generated by 
    $\frac{a}{1}$. So, $S^{-1}R$ is a principal ideal domain.
  }

  \qs{}{
    Show that $R = \set{a + b\theta \mid \theta = \frac{1 + \sqrt{-19}}{2}, a, b \in \Int }$ is not a Euclidean domain. 
    (Hint: First, show that $N(x) = x\bar x = a^2 + 5b^2 + ab$ for $x = a + b\theta$ and $R^\times = \set{\pm 1}$. Assume that 
    $R$ is a Euclidean domain with $\phi$. Choose $a \in R$ such that $\phi(a)$ is the smallest integer in 
    $\set{ \phi(x) \mid x \ne 0, x \not\in R^\times}$. Show that there exist no $q$ and $r$ such that $2 = aq + r$ with $r = 0$ or 
    $\phi(r) < \phi(a)$) 
  }
  \sol{
    Firstly, consider that 
    \[ \theta^2 = \paren{ \frac{1 + \sqrt{-19}}{2} }^2 = \frac{1 + 2\sqrt{-19} -19}{4} = \frac{-9 + \sqrt{-19}}{2} = \theta  - 5 \]
    Then, for $x = a + b\theta$, let $N(x) = a^2 + 5b^2 + ab$, so that $N$ is multiplicative.
    Let $y = c + d\theta$, then 
    \eqs{
      N(xy) &= N((a + b\theta)(c + d\theta)) \\ 
            &= N((ac - 5bd) + (ad + cb + bd)\theta) \\ 
            &= (ac - 5bd)^2 + 5(ad + cb + bd)^2 + (ac - 5bd)(ad + cb + bd) \\ 
            &= a^2c^2 - 10abcd + 25b^2d^2 + 5a^2d^2 + 5c^2b^2 + 5b^2d^2 + 10abcd + 10b^2cd + 10abd^2  \\ 
            &\;\;\;\; + a^2cd + abc^2 - 5abd^2 - 5b^2cd - 5b^2d^2 \\
            &= a^2c^2 + abcd + 25b^2d^2 + 5a^2d^2 + 5c^2b^2 + 5b^2cd + 5abd^2 + a^2cd + abc^2 \\ 
            &= a^2c^2 + 5b^2c^2  +abc^2 + 5a^2d^2 + 25b^2d^2 + 5abc^2 + a^2cd + 5b^2cd + abcd \\
            &= (a^2 + 5b^2 + ab)(c^2 + 5d^2 + cd) \\
            &= N(x)N(y)
    }
    Note also that for any $0 \ne x \in R$, it follows that $N(x) \ge 1$ since 
    \[ N(x) = a^2 + 5b^2 + ab = \paren{a + \frac{b}{2}}^2 + 19\paren{\frac{b}{2}}^2 \ge 1 \]

    Moreover, $N(1) = 1$, so if $x \in R^\times$, then $N(x) = 1$. 
    Now, if $x = a + b\theta$, then $b > 0$ implies $N(x) > 1$, and if $b = 0$, $N(x) = a^2 = 1$ only for $a = \pm 1$. 
    So the only solutions for $N(x) = 1$ are $x = 1$ and $x = -1$. Then, it is easy to check that $-1 \cdots -1 = 1$, 
    thus $-1 \in R^\times$. Therefore, $R^\times = \set{\pm 1}$

    Assume for contradiction that $R$ is a Euclidean Domain with $\phi$. Then let $0 \ne a \not\in R^\times$ be the element with 
    $\phi(a)$ being smallest in the set $\set{ \phi(x) \mid x \ne 0, x \not\in R^\times }$. 

    Note for future usage that there is no element $x \in R$ for which $N(x) = 2$ or $N(x) = 3$. 
    This is due to the fact that if that there is, then let that element be $a + b\theta$. 
    Now, $(a + b/2)^2 + 19(b/2)^2 \le 3$, so $b = 0$ otherwise $19(b/2)^2 > 3$. but then there is no $a^2 = 2$
    and no $a^2 = 3$ for $a \in \Int$, thus a contradiction.

    Now, let $2 = aq + r$. If $r = 0$, then $2 = aq$. This could happen only if $N(a) = 2$ or $N(a) = 4$ since $N(2) = 4$.
    However, there is no element with $N(x) = 2$, so $N(a) = 4$. Now, consider $\theta = aq' + r'$. 
    Since $N(\theta) = 5$, then $r' \ne 0$ because $N(a) \nmid N(\theta)$.
    But because of the minimality of $a$, $r$ must be a unit. Now, $N(\theta + 1) = 7$ and $N(\theta - 1) = 5$ are both prime.
    This means that $N(a) \nmid N(\theta + 1)$ and $N(a) \nmid N(\theta - 1)$, so $\theta \ne aq' + r'$ for any $q', r' \in R$.

    In the other case, if $r \ne 0$, then $r \in R^\times$ is forced as otherwise $\phi(r) \ge \phi(a)$.
    So either $1 = aq$ or $3 = aq$. However, $a$ is not a unit, thus $1 \ne aq$ for any $q \in R$.
    If $3 = aq$, then $9 = N(3) = N(a)N(q)$, which is that $N(a) = 3$ or $N(a) = 9$. However, there is no element with $N(x) = 3$. 
    So it must be the case that $N(a) = 9$.
    But then, consider $\theta = aq' + r'$. Notice that as $N(\theta) = 5$, then $r' \ne 0$, which means that $r' \in R^\times$.
    But $N(a) \nmid N(\theta + 1)$ and $N(a) \nmid N(\theta - 1)$, so $\theta \ne aq' + r'$ for any $q', r' \in R$. 

    This contradiction showed that $R$ is not a euclidean domain.
  }

  \qs{}{
    Let $R$ be a domain. Show that $R$ is a unique factorization domain if and only if every irreducible element of $R$ is prime
    and $R$ satisfies ACC on principal ideals.
  }
  \sol{
    \twoways{
      If $R$ is a unique factorization domain, then every irreducible element of $R$ is prime. 
      Consider if $p$ is an irreducible element and $p \mid xy$ for some $x, y \in R$, then $xy = pz$ for some $z \in R$. 
      Now, as $R$ is a unique factorization domain, write $x = ua_1\cdots a_n$, $y=vb_1\cdots b_m$, and 
      $z = wc_1\cdots c_k$ for unit $u, v, w$ and irreducible elements $a_1, \ldots, a_n, b_1, \ldots, b_m, c_1, \ldots, c_k$.
      Then, \[ (uvw^{-1})a_1\ldots a_n b_1 \ldots b_m = p c_1 \cdots c_k \]
      So by uniqueness of factorization, $p \sim a_i$ or $p \sim b_i$ for some $i$, which means either $p \mid x$ or $p \mid y$.

      Next, let $C$ be a chain of principal ideal $I_1 \subset \cdots$, and let $I_i = a_iR$ for some element $a_i \in R$.
      By the property of UFD, $a_1 = u_1c_1 \cdots c_n$ for irreducible element. Then, as $aR \subset bR$ means $b \mid a$, then 
      $a_n | a_1$ for any $n > 1$, but then $a_n = u_nc_1\cdots c_{k_n}$ for some $0 \le k_n \le n$.
      Note that $k_n = 0$ means $a_n = u_n$

      Then, let $K = \set{k_n \mid a_n = u_nc_1 \cdots c_{k_n}, \; \forall n}$. Since $K \subset {0, 1, \cdots, n}$, there is a minimal 
      element by the well ordering principal, let that element be $k$, and $a_m = u_mc_1 \cdots c_k$.
      Now, if there is some $m' > m$ such that $I_{m'} \ne I_m$, then $I_m \subsetneq I_{m'}$, which means that
      $m' \mid m$ and $m \nmid m'$. This means that $a_{m'} = u_{m'}c_1 \cdots c_{k'}$ for some $k' < k$. This is a contradiction.
      Hence, $I_m = I_{m+1} = \cdots$ terminates the chain $C$ finitely. 
    }{ 
      Let $S = \set{aR \mid a \ne 0, a \notin R^\times, a \text{ is not a product of irreducibles} }$.
      If $S \ne \emptyset$, 
      then there is a maximal element of $S$ because every chain $C \subset S$ terminates finitely. Let $bR$ be a maximal element of 
      $S$ for some $b \in R$. Then, $bR \in S$, so $b$ is not irreducible, so $b = xy$ for some nonunit $x, y$.
      This means that $bR \subsetneq xR$ and $bR \subsetneq yR$, as if $bR = xR$, then $y$ is a unit, 
      and similar logic prevents $bR = yR$.

      By the maximality of $bR$, $x$ and $y$ must be a product of irreducibles. Therefore, $b$ is a product of irreducible, which gives
      contradiction. Therefore, $S = \emptyset$, which means that $S$ is a factorization domain.

      Now, assume that $uc_1\cdots c_n = vd_1 \cdots d_m$ where $c_1, \ldots, c_n, d_1, \ldots, d_m$ are irreducible, then,
      they are also prime by assumption. 
      Now, consider that $c_n$ divides $d_i$ for some $i$ as they are prime, then, assume without loss of generality that $c_n \mid d_m$.
      Then, $uc_1 \cdots c_n = vwd_1 \cdots d_{m-1}c_n$, which is that $u c_1 \cdots c_{n-1} = vwd_1 \cdots d_{m-1}$. 
      By induction hypothesis, $c_1 \cdots c_{n-1}$ is a unique factorization, thus $c_1 \cdots c_n$ is a unique factorization. 
      Notice that basic case that $u c_1 = v d_1$ is unique by definition of irreducibility.
      Therefore, the domain is a UFD.
    }
  }

  \qs{}{
    Show that if the polynomial ring $R[x]$ is Noetherian, then so is $R$.
  }
  \sol{
    Let $I$ be any ideal, then let $I$ be generated by $G = \set{\alpha, \ldots}$ then $I' = \ang{G \cup \set{x}}$ is an ideal of $R[x]$.
    This is because for any $f \in R[x]$ and $i \in I'$, 
    \[ fi = f_0i + \sum_{j = 1}^n f_jix^j = f_0i + x\paren{\sum_{j=1}^n f_jix^{j-1}} = f_0i_0 + 
    x\paren{\sum_{j=1}^m i_jx^{j-1}f_0 + \sum_{j=1}^n f_jix^{j-1}} \in I' \]
    as $i = i_0 + \sum_{j=1}^m i_jx^j$, $x \in I'$, and $i_0 \in I$.

    So $I'$ is finitely generated, thus $\set{\alpha, \ldots, x}$ is finite, which means that $\set{\alpha, \ldots}$ is finite.
  }

  \qs{}{
    Give an example of a Noetherian ring that is not a unique factorization domain.
  }
  \sol{
    Consider $R = \Int[\sqrt{-5}] = \set{a + b\sqrt{-5} \mid a, b \in \Int}$. Then $R$ is generated by $1$ and $\sqrt{-5}$. 
    Therefore, $R$ is Noetherian since it is finitely generated. However, $6 = 2 \times 3 = (1 + \sqrt{-5})(1 - \sqrt{-5})$ with 
    $2$, $3$, $(1 + \sqrt{-5})$, $(1 - \sqrt{-5})$ be all irreducible.
  }

  \qs{}{
    An integral domain in which every nonzero nonunit can be factored into irreducibles is called a factorization domain.
    Give an example of a domain that is not a factorization domain.
  }
  \sol{
    Let $R = \ang{\set{\sum_{i=0}a_i{b_i^{r_i}} \mid a_i \in \Int, b_i \in \Nat, r_i \in \Rat^+}}$.
    In other words, $R$ is generated by the set above.

    Then $R$ is a ring by construction, which is that the result of addition, subtraction, multiplication is a part of $R$.

    Note that $R \subset \set{\sum_{i=0}a_i{b_i^{r_i}} \mid a_i \in \Int, b_i \in \Nat, r_i \in \Real^+}$.

    And since for $r_i \in \Rat^+$, $b_i^{r_i}$ is algebraic in $\Real/\Int$, it follows that
    $R \subset \set{a \mid a \text{ is algebraic in } \Real/\Int}$ , which is a field, then $R$ must contain no zero divisor,
    thus, $R$ is a domain.
  
    Notice that the unit of the ring are $1$ and $-1$ as for any $r \in R$, the exponent $r_i$ in $\sum_{i=0}a_i{b_i^{r_i}}$ is always 
    a positive real number.

    Let $r$ be irreducible, then $r \ne 1$, then $r = \sqrt{r} \sqrt{r}$ with $\sqrt{r} \notin R^\times$. Thus, $r$ is not irreducible.
    Therefore, any element is not a product of irreducible elements.
  }

  \qs{}{
    Let $R$ be a Noetherian ring. Show that the ring $R[[x]]$ of formal power series is Noetherian. 
  }
  \sol{
    Assume for contradiction that $R[[x]]$ is not Noetherian, thus there is a chain $ I_1 \subsetneq I_2 \subsetneq \cdots $
    that does not terminate. 
    Now, consider $\phi: R[[x]] \to R$ given by $f = 0 + 0 + \cdots + a_nx^n + \cdots \mapsto a_n$ where $n$ is the lowest term with 
    non-zero coefficient. 
    Then, $\phi(I_i)$ is an ideal, since if $a \in \phi(I_i)$ and $b \in R$. Then there is $f = ax^n + \cdots, g = bx^m + \cdots$ 
    with lowest coefficient $a$ and $b$, where $f$ and $g$ is in $I_i$ and $R[[x]]$ respectively.
    So, there is polynomial $fg \in I_i$ which has lowest coefficient $ab$. Thus, $ab \in \phi(I_i)$.
    Also, if $a, b \in \phi(I_i)$, then there is $f = ax^n + \cdots, g = bx^m + \cdots$ in $I_i$. Then, let $n > m$ without loss of 
    generality. It follows that $f - gx^{n-m} = {a-b}x^n + \cdots$ is a polynomial in $I_i$. Thus, $a-b \in \phi(I_i)$.

    Now, let $J_i = \phi(I_i)$. Since $I_1 \subsetneq I_2 \subsetneq \cdots $, it follows that $J_1 \subseteq J_2 \subseteq \cdots$
    But if chain $J_i$ terminates at $J_n$, then $I_i$ would also have to terminate. This is because 
    \[ \phi(I_n) = \phi(I_{n+1}) = \cdots \]

    Now, let $f \in I_{k+1} - I_k$ contains the lowest non-zero term $a_nx^n$, then $fx^m \in I_n$ for some $m$ otherwise 
    $\phi(I_k) \ne \phi(I_{k+1})$. Thus, for each of the finite generator of $J_n$, the polynomial with lowest nonzero coefficient
    being that generator has finite degree of lowest nonzero term. Finite sum of finite is finite. Thus, there can be only finitely many 
    $I_n = I_{n+1} = \cdots = I_{n+k}$, so the chain $I$ must stabilize at $I_{n+k}$.

    By contradiction, $R[[x]]$ must be Noetherian.
  }

  \qs{}{
    Let $\phi: R \to S$ be a ring homomorphism of commutative rings. Show that if $R$ is Noetherian, then so is $\phi(R)$ 
  }
  \sol{
    Let $\phi: R \surjto S$ be a surjective homomorphism induced by $\phi': R \to S'$ with $S = \im \phi'$.
    Let $I$ be an ideal of $S$, and $J = \phi^{-1}(I)$ then for $x, y \in J$ and $r in R$, it follows that
    $\phi(x), \phi(y) \in J$ and $\phi(r) \in S$. Thus, $\phi(xy), \phi(x - y), \phi(rx)$ are elements of $I$. 
    Thus, $xy, x-y, rx$ are elements of $J$. Thus, a preimage of an ideal is an ideal.

    Now, let proceed by contraposition. 
    If $S$ is not Noetherian, there exists a chain $I_1 \subsetneq I_2 \subsetneq \cdots$ that does not terminate.
    As if $x \in \phi^{-1}(I_i)$, then $\phi(x) \in I_i$, which means $\phi(x) \in I_j$ for every $j > i$. So, $x \in \phi^{-1}(I_j)$.
    Moreover, if there is an element $y \notin I_i$ but $y \in I_{i+1}$, then there must be an element $x \in R$ such that $\phi(x) = y$. 
    However, as $y \notin I_i$, $x \notin \phi^{-1}(I_i)$ but $x \in \phi^{-1}(I_{i+1})$. Thus, the chain is strict.
    Then, \[ \phi^{-1}(I_1) \subsetneq \phi^{-1}(I_2) \subsetneq \cdots \] is a chain of ideal that does not terminate, which means that 
    $R$ must also be non-noetherian. 

    By contraposition, if $R$ is noetherian, then $S$ must be noetherian.
  }

  \qs{}{
    \begin{enumerate}[a]
      \item Show that if $R$ is a domain, then so is $R[x]$
      \item Let $F$ be a field. Show that there exist infinitely many monic, irreducible polynomial in $F[x]$.
    \end{enumerate}
  }
  \sol{
    \begin{enumerate}[a]
      \item Let $R$ be a domain, then $R[x] = \set{ \sum_{i=1}^n r_ix^i \mid r_i \in R}$. 
        Then for some $r = \sum_{i=1}^n r_ix^i$ and $s = \sum_{i=1}^m s_ix^i$, the product 
        \eqs{ 
          rs &= \paren{\sum_{i=1}^n r_ix^i }\paren{\sum_{i=1}^m s_ix^i}\\
             &= \sum_{i=1}^n\sum_{j=1}^m r_is_j x^{i+j} \\ 
             &= \sum_{j=1}^m\sum_{i=1}^n s_jr_i x^{j+i} \\ 
             &= \paren{\sum_{i=1}^m s_ix^i}\paren{\sum_{i=1}^n r_ix^i} \\ 
             &= sr
        }
        Thus, $R$ is commutative.

        Now consider that $0 = rs$, then 
        \[ 0 = \sum_{i=1}^n \sum_{j=1}^m r_is_j x^{i+j} = \sum_{i=1}^{n+m} \sum_{j = 0}^{i} r_{j}s_{i-j}x^{i} \]
        
        Then if $r \ne 0$, there exists some $r_i \ne 0$. Then $\sum_{j=0}^i r_is_{i-j} = 0$ 
        which is that $s_{0}, \ldots, s_{i}$ are all $0$. Now assume that there is $j > i$ such that $s_j \ne 0$. 
        By the same argument, $r_0, \ldots, r_j$ must be all zero, but that means $r_i = 0$, contradicting that $r_i \ne r_0$.
        So, there must be no $j$ such that $s_j \ne 0$.
        Thus, $s = 0$. Therefore, $R$ has no zero divisor, and $R$ is a domain. 
        
      \item Since $F$ is a field, then $F[x]$ is a principal ideal domain, and therefore, it is a unique factorization domain.
        If $F$ is an infinite field, then the set $\set{(x + a) \mid a \in F}$ is a set of monic irreducible elements. Thus, there
        are infinitely many monic irreducible polynomial.
        Otherwise, $F$ is a finite field.
        In this case, assume for contradiction that there are finitely many irreducible polynomial $f_1, \ldots, f_n$. 
        Then, $f = f_1 \cdots f_n + 1$ is an element of $F[x]$, which is a UFD. So, $f = f_i g$ for some $1 \le i \le n$ and polynomial 
        $g \in F[x]$. But as $f_i \mid f_1 \cdots f_n$, it must follow that $f_i \mid 1$, which contradict that $f_i$ is irreducible,
        thus a non-unit.

        Therefore, there are infinitely many irreducible polynomial.
        Now, since $F[x]^\times = F^\times = F - \set{0}$. There are finite, say $k$, unit in the field, namely $I_1, \ldots, I_k$.
        If there is a finite number of monic irreducible polynomial, $m_1, \ldots, m_n$, then all irreducible are 
        \[ I_1m_1, \ldots, I_1m_n, I_2m_1, \ldots, I_2m_n, \ldots, I_km_n \]
        Since if $I_1 f$ for any non-unit $f$ results obviously to a reducible element.
        As the result contradicts with the fact that there are infinitely many irreducible, then there must be also infinitely 
        many monic irreducible polynomial in $F[x]$.
    \end{enumerate}
  }

\end{document}
