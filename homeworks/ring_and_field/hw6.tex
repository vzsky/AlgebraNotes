% chktex-file 44
% chktex-file 8

\documentclass{report}
\usepackage{amsthm}
\usepackage{amsmath}
\usepackage{amssymb}
\usepackage{amssymb}
\usepackage{amsfonts}
\usepackage{xcolor}
\usepackage{tikz}
\usepackage{fancyhdr}
\usepackage{enumerate}
\usepackage{graphicx}
\usepackage[normalem]{ulem}
\usepackage[most,many,breakable]{tcolorbox}
\usepackage[a4paper, top=80pt, foot=25pt, bottom=50pt, left=0.5in, right=0.5in]{geometry}
\usepackage{hyperref, theoremref}
\hypersetup{
	pdftitle={Assignment},
	colorlinks=true, linkcolor=b!90,
	bookmarksnumbered=true,
	bookmarksopen=true
}
\usepackage{nameref}
\usepackage{parskip}
\pagestyle{fancy}

\usepackage[explicit,compact]{titlesec}
\titleformat{\chapter}[block]{\bfseries\huge}{\thechapter. }{\compact}{#1}
        

%%%%%%%%%%%%%%%%%%%%%
%% Defining colors %%
%%%%%%%%%%%%%%%%%%%%%

\definecolor{lr}{RGB}{188, 75, 81}
\definecolor{r}{RGB}{249, 65, 68}
\definecolor{dr}{RGB}{174, 32, 18}
\definecolor{lo}{RGB}{255, 172, 129}
\definecolor{do}{RGB}{202, 103, 2}
\definecolor{o}{RGB}{238, 155, 0}
\definecolor{ly}{RGB}{255, 241, 133}
\definecolor{y}{RGB}{255, 229, 31}
\definecolor{dy}{RGB}{143, 126, 0}
\definecolor{lb}{RGB}{148, 210, 189}
\definecolor{bg}{RGB}{10, 147, 150}
\definecolor{b}{RGB}{39, 125, 161}
\definecolor{db}{RGB}{0, 95, 115}
\definecolor{p}{RGB}{229, 152, 155}
\definecolor{dp}{RGB}{181, 101, 118}
\definecolor{pp}{RGB}{142, 143, 184}
\definecolor{v}{RGB}{109, 89, 122}
\definecolor{lg}{RGB}{144, 190, 109}
\definecolor{g}{RGB}{64, 145, 108}
\definecolor{dg}{RGB}{45, 106, 79}

\colorlet{mysol}{g}
\colorlet{mythm}{lr}
\colorlet{myqst}{db}
\colorlet{myclm}{lb}
\colorlet{mywrong}{r}
\colorlet{mylem}{o}
\colorlet{mydef}{lg}
\colorlet{mycor}{lb}
\colorlet{myrem}{dr}

%%%%%%%%%%%%%%%%%%%%%

\newcommand{\col}[2]{
  \color{#1}#2\color{black}\,
}

\newcommand{\TODO}[1][5cm]{
  \color{red}TODO\color{black}
  \vspace{#1}
}

\newcommand{\wans}[1]{
	\noindent\color{mywrong}\textbf{Wrong answer: }\color{black}
	#1 


}

\newcommand{\wreason}[1]{
	\noindent\color{mywrong}\textbf{Reason: }\color{black}
	#1 

  
}

\newcommand{\sol}[1]{
	\noindent\color{mysol}\textbf{Solution: }\color{black}
	#1


}

\newcommand{\nt}[1]{
  \begin{note}Note: #1\end{note}
}

\newcommand{\ky}[1]{
  \begin{key}#1\end{key}
}

\newcommand{\pf}[1]{
  \begin{myproof}#1\end{myproof}
}

\newcommand{\qs}[3][]{
  \begin{question}{#2}{#1}#3\end{question}
}

\newcommand{\df}[3][]{
  \begin{definition}{#2}{#1}#3\end{definition}
}

\newcommand{\thm}[3][]{
  \begin{theorem}{#2}{#1}#3\end{theorem}
}

\newcommand{\clm}[3][]{
  \begin{claim}{#2}{#1}#3\end{claim} 
}

\newcommand{\lem}[3][]{
  \begin{lemma}{#2}{#1}#3\end{lemma}
}

\newcommand{\cor}[3][]{
  \begin{corollary}{#2}{#1}#3\end{corollary}
}

\newcommand{\rem}[3][]{
  \begin{remark}{#2}{#1}#3\end{remark}
}

\newcommand{\twoways}[2]{
  \leavevmode\\
  ($\Longrightarrow$): 
  \begin{shift}#1\end{shift}
  ($\Longleftarrow$):
  \begin{shift}#2\end{shift} 
}

\newcommand{\nways}[2]{
  \leavevmode\\
  ($#1$): 
  \begin{shift}#2\end{shift}
}

%%%%%%%%%%%%%%%%%%%%%%%%%%%%%% ENVRN

\newenvironment{myproof}[1][\proofname]{%
	\proof[\bfseries #1: ]
}{\endproof}

\tcbuselibrary{theorems,skins,hooks}
\newtcolorbox{shift}
{%
  before upper={\setlength{\parskip}{5pt}},
  blanker,
	breakable,
	width=0.95\textwidth,
  enlarge left by=0.03\textwidth,
}

\tcbuselibrary{theorems,skins,hooks}
\newtcolorbox{key}
{%
	breakable,
	width=0.95\textwidth,
  enlarge left by=0.03\textwidth,
}

\tcbuselibrary{theorems,skins,hooks}
\newtcolorbox{note}
{%
	enhanced,
	breakable,
	colback = white,
	width=\textwidth,
	frame hidden,
	borderline west = {2pt}{0pt}{black},
	sharp corners,
}

\tcbuselibrary{theorems,skins,hooks}
\newtcbtheorem[]{remark}{Remark}
{%
	enhanced,
	breakable,
	colback = white,
	frame hidden,
	boxrule = 0sp,
	borderline west = {2pt}{0pt}{myrem},
	sharp corners,
	detach title,
  before upper={\setlength{\parskip}{5pt}\tcbtitle\par\smallskip},
	coltitle = myrem,
	fonttitle = \bfseries\sffamily,
	description font = \mdseries,
	separator sign none,
	segmentation style={solid, myrem},
}{rem}

\tcbuselibrary{theorems,skins,hooks}
\newtcbtheorem[number within=section]{lemma}{Lemma}
{%
	enhanced,
	breakable,
	colback = white,
	frame hidden,
	boxrule = 0sp,
	borderline west = {2pt}{0pt}{mylem},
	sharp corners,
	detach title,
  before upper={\setlength{\parskip}{5pt}\tcbtitle\par\smallskip},
	coltitle = mylem,
	fonttitle = \bfseries\sffamily,
	description font = \mdseries,
	separator sign none,
	segmentation style={solid, mylem},
}{lem}

\tcbuselibrary{theorems,skins,hooks}
\newtcbtheorem{claim}{Claim}
{%
  parbox=false,
	enhanced,
	breakable,
	colback = white,
	frame hidden,
	boxrule = 0sp,
	borderline west = {2pt}{0pt}{myclm},
	sharp corners,
	detach title,
  before upper={\setlength{\parskip}{5pt}\tcbtitle\par\smallskip},
	coltitle = myclm,
	fonttitle = \bfseries\sffamily,
	description font = \mdseries,
	separator sign none,
	segmentation style={solid, myclm},
}{clm}

\makeatletter
\newtcbtheorem[number within=section, use counter from=lemma]{theorem}{Theorem}{enhanced,
	breakable,
	colback=white,
	colframe=mythm,
	attach boxed title to top left={yshift*=-\tcboxedtitleheight},
	fonttitle=\bfseries,
	title={#2},
	boxed title size=title,
	boxed title style={%
			sharp corners,
			rounded corners=northwest,
			colback=mythm,
			boxrule=0pt,
		},
	underlay boxed title={%
			\path[fill=mythm] (title.south west)--(title.south east)
			to[out=0, in=180] ([xshift=5mm]title.east)--
			(title.center-|frame.east)
			[rounded corners=\kvtcb@arc] |-
			(frame.north) -| cycle;
		},
	#1
}{thm}
\makeatother

\makeatletter
\newtcbtheorem{question}{Question}{enhanced,
	breakable,
	colback=white,
	colframe=myqst,
	attach boxed title to top left={yshift*=-\tcboxedtitleheight},
	fonttitle=\bfseries,
	title={#2},
	boxed title size=title,
	boxed title style={%
			sharp corners,
			rounded corners=northwest,
			colback=myqst,
			boxrule=0pt,
		},
	underlay boxed title={%
			\path[fill=myqst] (title.south west)--(title.south east)
			to[out=0, in=180] ([xshift=5mm]title.east)--
			(title.center-|frame.east)
			[rounded corners=\kvtcb@arc] |-
			(frame.north) -| cycle;
		},
	#1
}{qs}
\makeatother

\makeatletter
\newtcbtheorem[number within=section]{definition}{Definition}{enhanced,
	breakable,
	colback=white,
	colframe=mydef,
	attach boxed title to top left={yshift*=-\tcboxedtitleheight},
	fonttitle=\bfseries,
	title={#2},
	boxed title size=title,
	boxed title style={%
			sharp corners,
			rounded corners=northwest,
			colback=mydef,
			boxrule=0pt,
		},
	underlay boxed title={%
			\path[fill=mydef] (title.south west)--(title.south east)
			to[out=0, in=180] ([xshift=5mm]title.east)--
			(title.center-|frame.east)
			[rounded corners=\kvtcb@arc] |-
			(frame.north) -| cycle;
		},
	#1
}{def}
\makeatother

\makeatletter
\newtcbtheorem[number within=section, use counter from=lemma]{corollary}{Corollary}{enhanced,
	breakable,
	colback=white,
	colframe=mycor,
	attach boxed title to top left={yshift*=-\tcboxedtitleheight},
	fonttitle=\bfseries,
	title={#2},
	boxed title size=title,
	boxed title style={%
			sharp corners,
			rounded corners=northwest,
			colback=mycor,
			boxrule=0pt,
		},
	underlay boxed title={%
			\path[fill=mycor] (title.south west)--(title.south east)
			to[out=0, in=180] ([xshift=5mm]title.east)--
			(title.center-|frame.east)
			[rounded corners=\kvtcb@arc] |-
			(frame.north) -| cycle;
		},
	#1
}{cor}
\makeatother

% Basic
  \DeclareMathOperator{\lcm}{lcm}
  \newcommand{\Real}{\mathbb{R}}
  \newcommand{\Comp}{\mathbb{C}}
  \newcommand{\Nat}{\mathbb{N}}
  \newcommand{\Rat}{\mathbb{Q}}
  \newcommand{\Int}{\mathbb{Z}}
  \newcommand{\set}[1]{\left\{\, #1 \,\right\}}
  \newcommand{\paren}[1]{\left( \; #1 \; \right)}
  \newcommand{\abs}[1]{\left\lvert #1 \right\rvert}
  \newcommand{\ang}[1]{\left\langle #1 \right\rangle}
  \renewcommand{\to}[1][]{\xrightarrow{\text{#1}}}
  \newcommand{\tol}[1][]{\to{$#1$}}
  \newcommand{\curle}{\preccurlyeq}
  \newcommand{\curge}{\succcurlyeq}
  \newcommand{\mapsfrom}{\leftarrow\!\shortmid}

  \newcommand{\mat}[1]{\begin{bmatrix} #1 \end{bmatrix}}
  \newcommand{\pmat}[1]{\begin{pmatrix} #1 \end{pmatrix}}
  \newcommand{\eqs}[1]{\begin{align*} #1 \end{align*}}
  \newcommand{\case}[1]{\begin{cases} #1 \end{cases}}
  

  % Algebra
  \newcommand{\normSg}[0]{\vartriangleleft}
  \newcommand{\ZMod}[1][n]{\mathbb{Z}/#1\mathbb{Z}}
  \newcommand{\isom}{\simeq}
  \newcommand{\mapHom}{\xrightarrow{\text{hom}}}
  \DeclareMathOperator{\Inn}{Inn}
  \DeclareMathOperator{\Aut}{Aut}
  \DeclareMathOperator{\im}{im}
  \DeclareMathOperator{\ord}{ord}
  \DeclareMathOperator{\Gal}{Gal}
  \DeclareMathOperator{\chr}{char}
  \newcommand{\surjto}{\twoheadrightarrow}
  \newcommand{\injto}{\hookrightarrow}

  % Analysis 
  \newcommand{\limty}[1][k]{\lim_{#1\to\infty}}
  \newcommand{\norm}[1]{\left\lVert#1\right\rVert}
  \newcommand{\darrow}{\rightrightarrows}


\fancyhead[L]{Modern Algebra 2 MAS312}
\fancyhead[R]{\textbf{Touch Sungkawichai} 20210821}

\begin{document}

\qs{}{
  Let $F$ be a field of $\chr(F) = p$. Show that if $\gcd([K: F], p) = 1$, then $K/F$ is separable.
}
\sol{
  Let $K_s = \set{\alpha \mid \alpha \text{ is separable over } F}$, then notice that by problem 10, $K/K_s$ is purely inseparable.
  Then as $K/F$ is finite, $K/K_s$ is finite, but by problem 8, $[K: K_s]$ must be of $p$-power, so let $[K: K_s] = p^n$
  Now, as $[K: F] = [K: K_s][K_s: F]$, it follows that $p^n \mid [K: F]$. However, as $\gcd([K: F], p) = 1$, then $n = 0$.
  Therefore, $[K: F] = [K_s: F]$, thus $K/F$ must be separable.
}

\qs{}{
  Let $f \in F[x]$ be a nonzero polynomial over a field $F$. Show that the polynomial $f/\gcd(f, f')$ is separable, 
  where $f'$ denotes the derivative of $f$.
}
\sol{
  If $f' = 0$, then $\gcd(f, f') = f$, and $f/\gcd(f, f') = 1$ is separable. Now assume $f' \ne 0$
  Consider a field $K/F$ such that $f$ and $f'$ are split in $K$, then let $\alpha$ be arbitrary root of $f$ and $n$ be the maximal 
  number such that $(x - \alpha)^n \mid f$. Then, $f(x) = (x - \alpha)^n g(x)$ for some polynomial $g(x)$ over $F$. So, 
  \[ 0 \ne f'(x) = n(x - \alpha)^{n-1}g(x) + (x - \alpha)^n g'(x) = (x - \alpha)^{n-1}(ng(x) - (x - \alpha)g'(x)) \]
  As $(x - \alpha)^{n-1} \mid f'(x)$, then $(x - \alpha)^{n-1} \mid \gcd(f, f')$. Hence, $\alpha$ is a root of $f/\gcd(f, f')$ with 
  multiplicity at most 1.
  Therefore, $f/\gcd(f, f')$ is separable.
}

\qs{}{
  Show that a field $F$ is perfect if and only if any finite extension of $F$ is separable. 
}
\sol{
  \twoways{
    If $\chr(F) = 0$, let $f$ be an irreducible polynomial over $F$. 
    Then, $f' \ne 0$ since if $\deg(f) > 1$, $\deg(f') \ge 1$ and if $\deg(f) = 1$, then $f = ax + b$, so 
    $f' = a \ne 0$. Therefore, as $\deg(f') = \deg(f) - 1$ and $f' \ne 0$, and $f$ is irreducible, it follows that $\gcd(f, f') = 1$.
    As $f' \ne 0$, $f$ is separable. Therefore, all irreducible polynomial in $F[x]$ is separable.

    If $\chr(F) = p$, then $f' = 0$ means $f = a_nx^{np} + \cdots + a_1x^p + a_0$ since otherwise, if $f$ contains some term $bx^k$ 
    where $p$ does not divide $k$, then $f'$ contains $bkx^{k-1}$ where $bk \ne 0$. However, as $F$ is perfect, for every $a_i$ there 
    exists $b_i$ such that $b_i^p = a_i$. Therefore,
    \[ f = a_nx^{np} + \cdots + a_0 = b_n^px^{np} + \cdots + b_0^p = (b_nx^n + \cdots + b_0)^p \]
    Thus, $f$ is not irreducible. Therefore, all irreducible polynomial in $F[x]$ is separable.

    As all irreducible polynomial in $F$ is separable, let $E/F$ be a finite extension, thus algebraic.
    Then for any $\alpha \in E$, there exists $m_\alpha$ such that it is monic irreducible and $m_\alpha(\alpha) = 0$.
    Then, since $m_\alpha$ is irreducible, it is separable, thus $\alpha$ is separable.
  }{
    Assuming that all finite extension of $F$ is separable, let $f$ be an irreducible polynomial over $F$ and $\alpha$ is a root of $f$.
    Then, choose $E = F(\alpha)$ so that $[E: F] = \deg(f)$ is finite, thus $E/F$ is separable. Therefore $\alpha$ is separable, 
    so $f$ separable. 

    Assuming that $F$ is not perfect, which is that $\chr(F) = p$ and $F \ne F^p$. Choose $a \in F - F^p$ and construct a polynomial
    $f(x) = x^p - a$. Then, consider $b$ in the splitting field of $f$ over $F$ such that $b$ is the root of $f$. Then, $b^p = a$. 
    Therefore, $f(x) = x^p - a = (x - b)^p$, which means that $f$ is not separable.

    Next, choose $E = F(b)$ so that $[E: F] \le \deg(f) = p$ is finite, therefore, $E$ is separable by assumption. This means that 
    $b$ is separable, which is that the minimal polynomial of $b$, say $m_b$ must divides $f$ and be separable. This leaves the only 
    possibility of $m_b = (x - b)$ over $F$. So, $b \in F$. However, if $b \in F$, 
    then $a = b^p \in F^p$ contradicting that $a \in F - F^p$.
  }
}

\qs{}{
  Show that every finite extension of a perfect field is perfect.
}
\sol{
  Let $F$ be a perfect field and $E/F$ a finite extension. Consider a finite extension $K$ over $E$ so that $K/E/F$ is an extension.
  Then, $K/F$ is also a finite extension by the tower law. As $F$ is perfect, $K$ is separable. This was shown in the previous problem.
  As $K$ is an arbitrary finite extension of $E$, then all finite extension of $E$ is separable. So, $E$ is perfect. This was also 
  shown in the previous problem.
}

\qs{}{
  Let $F$ be a finite field and of characteristic 2. Show that if $E/F$ is a separable extension of degree 2, 
  then $E = F(\alpha)$ for some $\alpha \in E$ such that $\alpha^2 + \alpha \in F$.
}
\sol{
  As $E/F$ is degree $2$, then $E = F(\alpha)$ for some $\alpha \in E$. Moreover, there exists irreducible polynomial $f$ of degree 2 
  such that $f(\alpha) = 0$. Since $E/F$ is separable, then $\alpha$ must be separable, so $f$ must be separable. This mean that 
  $ f = (x - \alpha)(x - \beta) = x^2 - (\alpha + \beta) + \alpha\beta $ for some $\beta \ne \alpha$. Thus, $\alpha\beta \in F$.
  Consider $e = f(\alpha\beta) = \alpha(\beta - 1)\beta(\alpha - 1)$ as an element in $F$. 
  Then, \[\frac{e}{\beta^2 - \beta} = \alpha(\alpha - 1) = \alpha^2 - \alpha = \alpha^2 + \alpha - 2\alpha = \alpha^2 + \alpha \]
  So, $\alpha^2 + \alpha \in F$.
} 

\qs{}{
  Let $F$ be a finite field and let $E = F(\alpha, \beta)/F$ be a finite extension.
  Show that if $F(\alpha) \cap F(\beta) = F$, then $E = F(\alpha + \beta)$
}
\sol{
  Let $F = \mathbb{F}_q$ for $q = p^r$ for some prime $p$. Then, $F(\alpha) \isom \mathbb F_{q^n}$ and $F(\beta) \isom \mathbb F_{q^m}$.
  Notice that $E = F(\alpha, \beta)$ is finite, thus any subfield of $E$ of certain order is unique since the unique field 
  of order $q^l$ is the field of $\set{x \mid x^{q^l} - x = 0}$, which there are exactly $q^l$ solutions.
  Then assume $d \mid n$ and $d \mid m$ for $d > 1$. This means that $\mathbb F_{q^d} \subset \mathbb F_{q^n}$ and 
  $\mathbb F_{q^d} \subset \mathbb F_{q^m}$. As $F(\alpha) \subset E$ and $F(\beta) \subset E$, 
  the field $\mathbb F_{q^d}$ is the same, thus $F(\alpha) \cap F(\beta) = \mathbb F_{q^d} \ne F$.
  Therefore, it is acheived that $\gcd(n, m) = 1$.

  Now, consider the minimal polynomial $f$ of $\alpha + \beta$. Notice since $x^{q^d} - x$ is the product of all monic irreducible 
  polynomial over $\mathbb F_q$, then if the degree of $f$ is $k$, then $x^{q^k} - x$ is satisfied by $\alpha + \beta$. Moreover, 
  since $f$ is minimal, $x^{q^d} - x$ should not be satisfied by $\alpha + \beta$ for any $d < k$.

  Assume that $(\alpha + \beta)^{q^d} - (\alpha + \beta) = 0$, then $(\alpha + \beta)^{q^d} = \alpha + \beta$.
  Then, \[ \alpha + \beta = (\alpha + \beta)^{q^{nd}} = \alpha^{q^{nd}} + \beta^{q^{nd}} = \alpha + \beta^{q^{nd}} \]
  because $\alpha^{p^{n}} = \alpha$
  Thus, $\beta = \beta^{q^{nd}}$. But since $\beta$ is of degree $m$ with $\gcd(n, m) = 1$, then $m \mid d$.
  Similarly, the same process for $(\alpha + \beta)^{q^{md}} = \alpha + \beta$ shows that $n \mid d$. Thus, $d$ is at least $nm$.
  Moreover, $x^{q^{nm}} - x$ is satisfy by $\alpha + \beta$ as $\alpha + \beta$ is a member of $F(\alpha, \beta)$, which is of degree 
  at most $nm$ over $F$. This proves that $[F(\alpha + \beta): F] = nm$ and $[F(\alpha, \beta): F] = nm$, thus,
  $F(\alpha + \beta) = F(\alpha, \beta)$.
}

\qs{}{
  Let $E/F$ be a finite extension and $E_s = \set{\alpha \in E \mid \alpha \text{ is separable over }F}$. 
  Prove that $E_s$ is a subfield of $E$ containing $F$.
}
\sol{
  Consider that for any $a \in F$, the polynomial $(x - a)$ over $F$ is irreducible and separable, thus $a \in F$ is separable over $F$. 
  So $F \subset E_s$.

  Note that for any set $S$ of separable elements, $F(S)$ is separable, consider the proof by induction.
  If $\abs{S} = 1$, then let $E_s = \set{\alpha}$. It is clear that $F(\alpha)/F$ is separable as $\alpha$ is separable.
  Then, assume that the result holds for $\abs{S} = n$. Then consider $\alpha \notin S$ such that $\alpha$ is separable.
  Since $F(S)$ is separable over $F$ and $F(S)(\set{\alpha})$ is separable over $F(S)$ as shown above 
  (because $\abs{\set{\alpha}} = 1$). Then, $F(S + \set{\alpha})$ is separable over $F$.
  Therefore, by induction,  $F(S)$ is separable.

  Next, notice that $E_s$ is a set of separable elements and $F(E_s)$ is the smallest field containing $E_s$, 
  the smallest field containing all separable elements.
  As $E_s$ is a set of separable elements, then $F(E_s)$ is separable, so every element of $F(E_s)$ is separable over $F$, 
  which is that $F(E_s) \subset E_s$ by definition of $E_s$.
  In other words, $E_s = F(E_s)$ is a subfield.
}

\qs{}{
  Let $E/F$ be a field of $\chr(F) = p$ and let $E/F$ be a finite extension.
  An element $\alpha \in E$ is called purely inseparable over $F$ if $\alpha^{p^n} \in F$ for some nonnegative integer $n$.
  We say that $E/F$ is purely inseparable if all elements in $E$ are purely inseparable over $F$. 
  Show that $E/F$ is purely inseparable if and only if $E/L$ and $L/F$ are purely inseparable. Deduce that a finite purely inseparable
  extension has $p$-power degree. 
}
\sol{
  \twoways{
    Consider $E/L$ and $L/F$. If $\alpha$ is any element in $L$, then $\alpha \in E$, so $\alpha$ is purely inseparable over $F$ as 
    $E/F$ is purely inseparable. Thus, $L/F$ is purely inseparable. 
    Next, If $\alpha \in E$ is any element, then $\alpha^{p^n} \in F \subset L$ for some nonnegative $n$. Thus, $\alpha$ is 
    purely inseparable over $L$. So, $E/L$ is purely inseparable.
  }{
    Let $\alpha$ be any element in $E$, then as $E/L$ is purely inseparable, $\alpha^{p^n} \in L$ for some nonnegative $n$.
    Then as $L/F$ is purely inseparable, $(\alpha^{p^n})^{p^m} \in F$ for some nonnegative $m$. 
    But as $$(\alpha^{p^n})^{p^m} = \alpha^{p^n\cdot p^m} = \alpha^{p^{n + m}}$$ where $n + m$ is a nonnegative integer, then, 
    $E/F$ is inseparable.
  }
  
  If $E = F$ is purely inseparable, then $[E: F] = 1 = p^0$
  Next, if $E/F$ is a finite purely inseparable extension with $E \ne F$ then for any element $\alpha \in E-F$ there is a polynomial 
  $f(x) = (x^{p^n} - \alpha^{p^n})$ over
  $F$ such that $f(\alpha) = 0$. Thus the minimal polynomial of $\alpha$ must divides $f$. 
  As $f(x) = (x^{p^n} - \alpha^{p^n})$ over $F$ of characteristic $p$, then 
  \[ f(x) = (x - \alpha)^{p^n} = (x^p - \alpha^p)^{p^{n-1}} = \cdots = (x^{p^n} - \alpha^{p^n}) \]
  Note that any divisor of $f(x)$ must be in the above form, as for $k$ that is not divisible by $p$, 
  \[ (x - \alpha)^{kp^n} = (x^{p^n} - \alpha^{p^n})^k = x^{p^nk} \pm (k\alpha^{p^n}) x^{p^n(k-1)} + \cdots + \alpha^{p^nk} \] 
  If $\alpha^{p^n} \in F$, then $(x - \alpha)^{p^n}$ is already less degree than $(x - \alpha)^{kp^n}$ otherwise,
  $k\alpha^{p^n} \notin F$ since $\alpha^{p^n} \notin F$ but $0 \ne k \in F$. 

  Therefore, the minimal polynomial of $\alpha$ must be of degree $p^{k}$ for some integer $k$. Since $E = F(\beta)$ for some 
  element $\beta \in E$, the degree $[E: F] = \deg(m_\beta) = p^k$ for some integer $k$. 
}

\qs{}{
  Let $E/F$ be a finite extension of a field $F$ of $\chr(F) = p$. Prove that if $\alpha \in E$ is separable and purely inseparable 
  over $F$, then $\alpha \in F$. 
}
\sol{
  Let $\alpha \in E$ be separable and purely inseparable. Then $\alpha^{p^n} \in F$ for some nonnegative integer $n$.
  Then, consider the polynomial $f(x) = x^{p^n} - \alpha^{p^n}$ over $F$. Since $f(\alpha) = 0$, the minimal polynomial of $\alpha$ must 
  divide $f$.

  However as $F$ is of characteristic $p$, \[ f = (x^{p^n} - \alpha^{p^n}) = (x - \alpha)^{p^n} \]
  Since $\alpha$ is separable over $F$, then $m_\alpha = (x - \alpha)$ must hold. Since $m_\alpha$ is the minimal polynomial over $F$, 
  then $-\alpha \in F$, which is that $\alpha \in F$.
} 

\qs{}{
  Let $E/F$ be a finite extension of a field $F$ of $\chr(F) = p$. Show that $E/E_s$ is purely inseparable, where $E_s$ is the subfield
  defined in problem 7.
}
\sol{
  Let $\alpha$ be any element in $E$. If $\alpha$ is separable, then choose $\alpha = \alpha^{p^0}$ is separable, thus $\alpha$ is 
  purely inseparable. Otherwise, let $f$ be the minimal polynomial of $\alpha$. Note that $f$ is irreducible and non-zero.
  As $\alpha$ is inseparable, $\gcd(f, f') \ne 1$, thus $f' = 0$ (because otherwise, $f \mid f'$).

  In problem 3, and in fact, in class, it was shown that if $\chr(F) = p$, $f' = 0$ implies
  \[ f(x) = a_nx^{np} + \cdots + a_1x^p + a_0 = g(x^{p^k}) \] for some non-zero polynomial $g$ over $F$ and some largest integer $k$
  such that $p^k \mid np$ for all $n$ such that $a_n \ne 0$. This is due to the fact that if there is a term $a_mx^m$ with non-zero $a_m$
  and $p \nmid m$ in $f(x)$, then $f'(x) \ne 0$ as it contains the term $ma_m$ where $m \ne 0$.

  By choosing $g(x)$ in that way, there exists some term with non-zero coefficient $x^m$ with $p \nmid m$ (because otherwise $k$ is not 
  maximized) in $g(x)$. Therefore, $g' \ne 0$. If $g$ is reducible, say $g(x) = h_1(x)h_2(x)$, then 
  \[ f(x) = g(x^{p^k}) = h_1(x^{p^k})h_2(x^{p^k}) \] contradicting that $f$ irreducible, thus $g$ is irreducible.

  Since $\deg(g') < \deg(g)$ and $g$ is irreducible, it follows that $\gcd(g, g') = 1$, because otherwise, $g \mid g'$.
  Now, as $\gcd(g, g') = 1$, $g$ must be separable.

  Now, notice that $g(\alpha^{p^k}) = f(\alpha) = 0$, so the minimal polynomial of $\alpha^{p^k}$ must divide $g$.
  However, as $g$ is separable, $\alpha^{p^k}$ must be separable.
  This means that $\alpha^{p^k} \in E_s$, thus $\alpha$ is purely inseparable over $E_s$.
  As $\alpha$ is arbitrary, $E/E_s$ is purely inseparable.
}

\end{document}
