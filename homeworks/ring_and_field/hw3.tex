% chktex-file 44
% chktex-file 8

\documentclass{report}
\usepackage{amsthm}
\usepackage{amsmath}
\usepackage{amssymb}
\usepackage{amssymb}
\usepackage{amsfonts}
\usepackage{xcolor}
\usepackage{tikz}
\usepackage{fancyhdr}
\usepackage{enumerate}
\usepackage{graphicx}
\usepackage[normalem]{ulem}
\usepackage[most,many,breakable]{tcolorbox}
\usepackage[a4paper, top=80pt, foot=25pt, bottom=50pt, left=0.5in, right=0.5in]{geometry}
\usepackage{hyperref, theoremref}
\hypersetup{
	pdftitle={Assignment},
	colorlinks=true, linkcolor=b!90,
	bookmarksnumbered=true,
	bookmarksopen=true
}
\usepackage{nameref}
\usepackage{parskip}
\pagestyle{fancy}

\usepackage[explicit,compact]{titlesec}
\titleformat{\chapter}[block]{\bfseries\huge}{\thechapter. }{\compact}{#1}
        

%%%%%%%%%%%%%%%%%%%%%
%% Defining colors %%
%%%%%%%%%%%%%%%%%%%%%

\definecolor{lr}{RGB}{188, 75, 81}
\definecolor{r}{RGB}{249, 65, 68}
\definecolor{dr}{RGB}{174, 32, 18}
\definecolor{lo}{RGB}{255, 172, 129}
\definecolor{do}{RGB}{202, 103, 2}
\definecolor{o}{RGB}{238, 155, 0}
\definecolor{ly}{RGB}{255, 241, 133}
\definecolor{y}{RGB}{255, 229, 31}
\definecolor{dy}{RGB}{143, 126, 0}
\definecolor{lb}{RGB}{148, 210, 189}
\definecolor{bg}{RGB}{10, 147, 150}
\definecolor{b}{RGB}{39, 125, 161}
\definecolor{db}{RGB}{0, 95, 115}
\definecolor{p}{RGB}{229, 152, 155}
\definecolor{dp}{RGB}{181, 101, 118}
\definecolor{pp}{RGB}{142, 143, 184}
\definecolor{v}{RGB}{109, 89, 122}
\definecolor{lg}{RGB}{144, 190, 109}
\definecolor{g}{RGB}{64, 145, 108}
\definecolor{dg}{RGB}{45, 106, 79}

\colorlet{mysol}{g}
\colorlet{mythm}{lr}
\colorlet{myqst}{db}
\colorlet{myclm}{lb}
\colorlet{mywrong}{r}
\colorlet{mylem}{o}
\colorlet{mydef}{lg}
\colorlet{mycor}{lb}
\colorlet{myrem}{dr}

%%%%%%%%%%%%%%%%%%%%%

\newcommand{\col}[2]{
  \color{#1}#2\color{black}\,
}

\newcommand{\TODO}[1][5cm]{
  \color{red}TODO\color{black}
  \vspace{#1}
}

\newcommand{\wans}[1]{
	\noindent\color{mywrong}\textbf{Wrong answer: }\color{black}
	#1 


}

\newcommand{\wreason}[1]{
	\noindent\color{mywrong}\textbf{Reason: }\color{black}
	#1 

  
}

\newcommand{\sol}[1]{
	\noindent\color{mysol}\textbf{Solution: }\color{black}
	#1


}

\newcommand{\nt}[1]{
  \begin{note}Note: #1\end{note}
}

\newcommand{\ky}[1]{
  \begin{key}#1\end{key}
}

\newcommand{\pf}[1]{
  \begin{myproof}#1\end{myproof}
}

\newcommand{\qs}[3][]{
  \begin{question}{#2}{#1}#3\end{question}
}

\newcommand{\df}[3][]{
  \begin{definition}{#2}{#1}#3\end{definition}
}

\newcommand{\thm}[3][]{
  \begin{theorem}{#2}{#1}#3\end{theorem}
}

\newcommand{\clm}[3][]{
  \begin{claim}{#2}{#1}#3\end{claim} 
}

\newcommand{\lem}[3][]{
  \begin{lemma}{#2}{#1}#3\end{lemma}
}

\newcommand{\cor}[3][]{
  \begin{corollary}{#2}{#1}#3\end{corollary}
}

\newcommand{\rem}[3][]{
  \begin{remark}{#2}{#1}#3\end{remark}
}

\newcommand{\twoways}[2]{
  \leavevmode\\
  ($\Longrightarrow$): 
  \begin{shift}#1\end{shift}
  ($\Longleftarrow$):
  \begin{shift}#2\end{shift} 
}

\newcommand{\nways}[2]{
  \leavevmode\\
  ($#1$): 
  \begin{shift}#2\end{shift}
}

%%%%%%%%%%%%%%%%%%%%%%%%%%%%%% ENVRN

\newenvironment{myproof}[1][\proofname]{%
	\proof[\bfseries #1: ]
}{\endproof}

\tcbuselibrary{theorems,skins,hooks}
\newtcolorbox{shift}
{%
  before upper={\setlength{\parskip}{5pt}},
  blanker,
	breakable,
	width=0.95\textwidth,
  enlarge left by=0.03\textwidth,
}

\tcbuselibrary{theorems,skins,hooks}
\newtcolorbox{key}
{%
	breakable,
	width=0.95\textwidth,
  enlarge left by=0.03\textwidth,
}

\tcbuselibrary{theorems,skins,hooks}
\newtcolorbox{note}
{%
	enhanced,
	breakable,
	colback = white,
	width=\textwidth,
	frame hidden,
	borderline west = {2pt}{0pt}{black},
	sharp corners,
}

\tcbuselibrary{theorems,skins,hooks}
\newtcbtheorem[]{remark}{Remark}
{%
	enhanced,
	breakable,
	colback = white,
	frame hidden,
	boxrule = 0sp,
	borderline west = {2pt}{0pt}{myrem},
	sharp corners,
	detach title,
  before upper={\setlength{\parskip}{5pt}\tcbtitle\par\smallskip},
	coltitle = myrem,
	fonttitle = \bfseries\sffamily,
	description font = \mdseries,
	separator sign none,
	segmentation style={solid, myrem},
}{rem}

\tcbuselibrary{theorems,skins,hooks}
\newtcbtheorem[number within=section]{lemma}{Lemma}
{%
	enhanced,
	breakable,
	colback = white,
	frame hidden,
	boxrule = 0sp,
	borderline west = {2pt}{0pt}{mylem},
	sharp corners,
	detach title,
  before upper={\setlength{\parskip}{5pt}\tcbtitle\par\smallskip},
	coltitle = mylem,
	fonttitle = \bfseries\sffamily,
	description font = \mdseries,
	separator sign none,
	segmentation style={solid, mylem},
}{lem}

\tcbuselibrary{theorems,skins,hooks}
\newtcbtheorem{claim}{Claim}
{%
  parbox=false,
	enhanced,
	breakable,
	colback = white,
	frame hidden,
	boxrule = 0sp,
	borderline west = {2pt}{0pt}{myclm},
	sharp corners,
	detach title,
  before upper={\setlength{\parskip}{5pt}\tcbtitle\par\smallskip},
	coltitle = myclm,
	fonttitle = \bfseries\sffamily,
	description font = \mdseries,
	separator sign none,
	segmentation style={solid, myclm},
}{clm}

\makeatletter
\newtcbtheorem[number within=section, use counter from=lemma]{theorem}{Theorem}{enhanced,
	breakable,
	colback=white,
	colframe=mythm,
	attach boxed title to top left={yshift*=-\tcboxedtitleheight},
	fonttitle=\bfseries,
	title={#2},
	boxed title size=title,
	boxed title style={%
			sharp corners,
			rounded corners=northwest,
			colback=mythm,
			boxrule=0pt,
		},
	underlay boxed title={%
			\path[fill=mythm] (title.south west)--(title.south east)
			to[out=0, in=180] ([xshift=5mm]title.east)--
			(title.center-|frame.east)
			[rounded corners=\kvtcb@arc] |-
			(frame.north) -| cycle;
		},
	#1
}{thm}
\makeatother

\makeatletter
\newtcbtheorem{question}{Question}{enhanced,
	breakable,
	colback=white,
	colframe=myqst,
	attach boxed title to top left={yshift*=-\tcboxedtitleheight},
	fonttitle=\bfseries,
	title={#2},
	boxed title size=title,
	boxed title style={%
			sharp corners,
			rounded corners=northwest,
			colback=myqst,
			boxrule=0pt,
		},
	underlay boxed title={%
			\path[fill=myqst] (title.south west)--(title.south east)
			to[out=0, in=180] ([xshift=5mm]title.east)--
			(title.center-|frame.east)
			[rounded corners=\kvtcb@arc] |-
			(frame.north) -| cycle;
		},
	#1
}{qs}
\makeatother

\makeatletter
\newtcbtheorem[number within=section]{definition}{Definition}{enhanced,
	breakable,
	colback=white,
	colframe=mydef,
	attach boxed title to top left={yshift*=-\tcboxedtitleheight},
	fonttitle=\bfseries,
	title={#2},
	boxed title size=title,
	boxed title style={%
			sharp corners,
			rounded corners=northwest,
			colback=mydef,
			boxrule=0pt,
		},
	underlay boxed title={%
			\path[fill=mydef] (title.south west)--(title.south east)
			to[out=0, in=180] ([xshift=5mm]title.east)--
			(title.center-|frame.east)
			[rounded corners=\kvtcb@arc] |-
			(frame.north) -| cycle;
		},
	#1
}{def}
\makeatother

\makeatletter
\newtcbtheorem[number within=section, use counter from=lemma]{corollary}{Corollary}{enhanced,
	breakable,
	colback=white,
	colframe=mycor,
	attach boxed title to top left={yshift*=-\tcboxedtitleheight},
	fonttitle=\bfseries,
	title={#2},
	boxed title size=title,
	boxed title style={%
			sharp corners,
			rounded corners=northwest,
			colback=mycor,
			boxrule=0pt,
		},
	underlay boxed title={%
			\path[fill=mycor] (title.south west)--(title.south east)
			to[out=0, in=180] ([xshift=5mm]title.east)--
			(title.center-|frame.east)
			[rounded corners=\kvtcb@arc] |-
			(frame.north) -| cycle;
		},
	#1
}{cor}
\makeatother

% Basic
  \DeclareMathOperator{\lcm}{lcm}
  \newcommand{\Real}{\mathbb{R}}
  \newcommand{\Comp}{\mathbb{C}}
  \newcommand{\Nat}{\mathbb{N}}
  \newcommand{\Rat}{\mathbb{Q}}
  \newcommand{\Int}{\mathbb{Z}}
  \newcommand{\set}[1]{\left\{\, #1 \,\right\}}
  \newcommand{\paren}[1]{\left( \; #1 \; \right)}
  \newcommand{\abs}[1]{\left\lvert #1 \right\rvert}
  \newcommand{\ang}[1]{\left\langle #1 \right\rangle}
  \renewcommand{\to}[1][]{\xrightarrow{\text{#1}}}
  \newcommand{\tol}[1][]{\to{$#1$}}
  \newcommand{\curle}{\preccurlyeq}
  \newcommand{\curge}{\succcurlyeq}
  \newcommand{\mapsfrom}{\leftarrow\!\shortmid}

  \newcommand{\mat}[1]{\begin{bmatrix} #1 \end{bmatrix}}
  \newcommand{\pmat}[1]{\begin{pmatrix} #1 \end{pmatrix}}
  \newcommand{\eqs}[1]{\begin{align*} #1 \end{align*}}
  \newcommand{\case}[1]{\begin{cases} #1 \end{cases}}
  

  % Algebra
  \newcommand{\normSg}[0]{\vartriangleleft}
  \newcommand{\ZMod}[1][n]{\mathbb{Z}/#1\mathbb{Z}}
  \newcommand{\isom}{\simeq}
  \newcommand{\mapHom}{\xrightarrow{\text{hom}}}
  \DeclareMathOperator{\Inn}{Inn}
  \DeclareMathOperator{\Aut}{Aut}
  \DeclareMathOperator{\im}{im}
  \DeclareMathOperator{\ord}{ord}
  \DeclareMathOperator{\Gal}{Gal}
  \DeclareMathOperator{\chr}{char}
  \newcommand{\surjto}{\twoheadrightarrow}
  \newcommand{\injto}{\hookrightarrow}

  % Analysis 
  \newcommand{\limty}[1][k]{\lim_{#1\to\infty}}
  \newcommand{\norm}[1]{\left\lVert#1\right\rVert}
  \newcommand{\darrow}{\rightrightarrows}


\fancyhead[L]{Modern Algebra 2 - MAS312}
\fancyhead[R]{\textbf{Touch Sungkawichai} 20210821}

\begin{document}
  \qs{}{
    Let $F[x]$ be a polynomial ring over a field $F$ and let $f(x) \in F[x]$. Show that the ideal generated by $f(x)$ is maximal if and 
    only if $f(x)$ is irreducible.
  }
  \sol{
    \twoways{
      If $(f(x))$ is a maximal ideal, then it is also a prime ideal, which means $f(x)$ is prime. Therefore, $f(x)$ is irreducible.
    }{
      Because $F$ is a field, $F[x]$ is a PID, so if $f(x)$ is irreducible, then $f(x)$ is prime. This means that $(f(x))$ is a prime 
      ideal. But as $F[x]$ is a PID, a prime ideal is a maximal ideal. 
    }
  }

  \qs{}{
    Prove that $(x^n - 1)/(x - 1)$ is irreducible in $\Int[x]$ if and only if $n$ is a prime integer.
  }
  \sol{
    Firstly, notice that \[ \frac{x^n - 1}{x - 1} = x^{n-1} + x^{n-2} + x^{n-3} + \cdots + 1 \] 
    and that $(\Int[x])^\times = \set{\pm 1}$
    \twoways{
      The proof use contraposition. 
      For $n = 1$, $(x^n - 1)/(x-1) = 1$ is a unit. Thus, it is not irreducible.

      If $n \ne 1$ is not a prime integer, then let $ab = n$ for some integer $a, b$ which is not $1$.
      Then \[ (x^{a-1} + x^{a-2} + \cdots + 1)(x^{(b-1)a} + x^{(b-2)a} + \cdots + 1) = (x^{n-1} + \cdots + 1) \]
      Since both are not a unit, then the polynomial is not irreducible.
    }{
      Let $n$ be prime.
      Since 
      \eqs{ \frac{x^n - 1}{x-1} &= \frac{((x-1)+1)^n - 1}{x-1} \\ 
                                &= \frac{(x-1)^n + \pmat{n \\ n-1}(x-1)^{n-1} + \cdots + \pmat{n \\ 0}1 - 1}{x-1} \\
                                &= (x-1)^{n-1} + \pmat{n \\ n-1}(x-1)^{n-1} + \cdots + \pmat{n \\ 1}
      }

      Then, as $n \mid \pmat{n \\ i}$ for all $1 \le i \le n-1$ and $n^2 \nmid \pmat{n \\ 1}$, the eisenstein criterion applies.
      So, $\frac{x^n-1}{x-1}$ is irreducible in $\Rat[x-1]$, thus it is irreducible in $\Rat[x]$.
      Moreover, as $\frac{x^n - 1}{x-1} = x^{n-1} + x^{n-2} + \cdots + 1$, then it is monic, thus primitive.
      So, $\frac{x^n-1}{x-1}$ is irreducible in $\Int[x]$.
    }
  }

  \qs{}{
    Let $R$ be a UFD and let $f \in R[x]$ be a primitive polynomial. Show that if a non-constant polynomial $g$ divides $f$, then $g$ 
    is also primitive.
  }
  \sol{
    If $g \mid f$, then there is some polynomial $h$ such that $gh = f$. So, as $f$ is primitive, then $C(g)C(h) = R$.
    Now, assume that $g$ is not primitive, then $C(g) = aR$ for some $a \notin R^{\times}$. 
    But $aRC(h) = R$ is not possible as $1 \not \in aRC(h)$ no matter what polynomial is $h$. 
    Thus, by contradiction, $g$ must be primitive.
  }

  \qs{}{
    Show that $x^4 + yx + 5y + 2y^2x^2 \in \Comp[x, y]$ is irreducible.
  }
  \sol{
    Notice that $y$ is irreducible in the field $\Comp[y]$ because if $y = ab$ for some $a, b \in \Comp[y]$. It must be the case that 
    $\deg{y} = \deg{a} + \deg{b}$. But since degree is non-negative, $\deg{a} = 0$ without loss of generality. Then, $a \ne 0$, otherwise
    $y = ab = 0$. So $a \in \Comp - \set{0} = \Comp^{\times}$. 

    Rewriting the polynomial gives $f = x^4 + (2y^2)x^2 + (y)x + (5y) \in \Comp[y][x]$. Then, $y \in \Comp[y]$ is irreducible such that 
    $y \mid 5y$, $y \mid y$, $y \mid 2y^2$, $y \nmid 1$, and $y^2 \nmid 5y$. Therefore, by the eisenstein criterion, 
    $f$ is irreducible over $(\Comp[x, y] - \set{0})^{-1}\Comp[x, y]$, which is the quotient field of $\Comp[x, y]$. However, as $f$ 
    is monic, it is primitive. Therefore, $f$ is also irreducible in $\Comp[x, y]$ since it is primitive and irreducible 
    in the quotient field.
  }
  
  \qs{}{
    Let $f = a_nx^n + a_{n-1}x^{n-1} + \cdots + a_0 \in \Rat[x]$. Let $f'$ denote the derivative of $f$, i.e., 
    $f' = a_nnx^{n-1} + a_{n-1}(n-1)x^{n-2} + \cdots + a_1$. Show that $f$ is divisible by the square of a non-constant polynomial in 
    $\Rat[x]$ if and only if $f$ and $f'$ are not relatively prime.
  }
  \sol{
    Since $\Rat$ is a field, then $\Rat[x]$ is a PID and UFD. 
    \twoways{
      If $f$ is divisible by a square of non-constant polynomial, $g^2$. Then $f = g^2h$ for some $h$.
      Now, by the product rule of derivative, $f' = (g^2)'h + g^2h' = 2gg'h + g^2h' = g(2g'h + gh')$. 
      Therefore, a non-constant, which means non-unit $g$ divides both $f$ and $f'$. Thus, $f$ and $f'$ is not-relatively prime.

      Note that if $g$ is constant, then $g^2$ is constant, which is not the case, so $g$ is non-constant.
    }{
      If $f$ and $f'$ is not relatively prime, then let $g$ be an irreducible, thus non-constant polynomial such that 
      $g \mid f$ and $g \mid f'$. Then, as $g \mid f$, let $f = gh$ for some polynomial $h$. So, $f' = gh' + g'h$ by the product rule.
      However, as $g \mid f$ and $g \mid gh'$, it must be the case that $g \mid g'h$. But $g \nmid g'$ since $\deg{g} > \deg{g'}$. 
      But since $\Rat[x]$ is a PID, then $g$ is prime, which means that $g \mid h$.

      Now, since $g \mid h$, then $h = gr$ for some polynomial $r$. This means that  $f = gh = ggr$. Therefore, $f$ is divisible by 
      a square for non-constant polynomial.
    }
  }

  \qs{}{
    Let $E/F$ be a field extension and let $a, b \in E$. Show that if $[F(a): F]$ and $[F(b): F]$ are relatively prime, then 
    $[F(a, b): F] = [F(a) : F][F(b): F]$
  }
  \sol{
    If $a$ or $b$ is transcendental, then $[F(a, b):F]$ is not finite, and either $[F(a) : F]$ or $[F(b) : F]$ is not finite, thus, 
    $[F(a, b) : F] = [F(a): F][F(b): F]$. 

    Now, consider the case where both $a, b$ are algebraic element over $F$.

    Since $F(a, b) \isom F[a, b] \isom F[a][b] \isom F(a)[b]$, then $F(a, b)$ is a field extension of $F(a)$.
    This means that \[ [F(a, b): F] = [F(a, b): F(a)][F(a) : F] = [F(a, b): F(b)][F(b): F] \]

    But $[F(b): F]$ divides $[F(a, b): F] = [F(a, b): F(a)][F(a): F]$ while being relatively prime to $[F(a): F]$ means that 
    $[F(b): F] \mid [F(a, b): F(a)]$. 

    However, $[F(a, b): F(a)] \le [F(b), F]$ as if $f$ is a minimal 
    polynomial of $b$ over $F$, then the minimal polynomial of $b$ over $F(a)$ would divide $f$ since $f(b) = 0$ over $F(a)$.

    Therefore, $[F(a, b): F(a)] = [F(b), F]$. Which means that $[F(a, b): F] = [F(a, b), F(a)][F(a), F] = [F(a), F][F(b), F]$.
  }
  
  \qs{}{
    Find the minimal polynomial of $\sqrt{p} + \sqrt{q}$ over $\Rat$ where $p \ne q$ are prime integers. 
    Show also that $\Rat(\sqrt{p} + \sqrt{q}) = \Rat(\sqrt{p}, \sqrt{q})$.
  }
  \sol{
    Let $x = \sqrt{p} + \sqrt{q}$, then $x^2 = p + 2\sqrt{pq} + q$. 
    Now, $x^2 - (p + q) = 2\sqrt{pq}$, so 
    \eqs{
      x^4 - 2(p+q)x^2 + (p+q)^2 = 4pq \\ 
      x^4 - 2(p+q)x^2 + (p-q)^2 = 0 
    }
    So, let $f = x^4 - 2(p+q)x^2 + (p-q)^2$, 
    then $m_{\sqrt{p} + \sqrt{q}, \Rat} \mid f$. However, $x^4 - 2(p+q)x^2 + (p-q)^2$ does not factor in $\Rat$ because it is factored 
    as \[ (x^2 - (p+q) + 2\sqrt{pq})(x^2 - (p+q) - 2\sqrt{pq}) \] which is then factored as 
    \[(x - \sqrt{p + q - 2\sqrt{pq}})(x + \sqrt{p + q - 2\sqrt{pq}})(x - \sqrt{p+q+2\sqrt{pq}})(x + \sqrt{p+q+2\sqrt{pq}}) \in \Real\]
    But as $\Real$ is an extension over $\Rat$, and any combination of the factors resulting in a polynomial in $\Real[x]$ but not 
    in $\Rat[x]$. The combination of the factors is in $\Rat[x]$ only when all the factors are multiplied.
    This means that $f$ is not a product of polynomial in $\Rat[x]$, thus it is irreducible, thus, $m_{\sqrt{p} + \sqrt{q}, \Rat} = f$.

    Now, consider that $\Rat(\sqrt{p} + \sqrt{q}) \subset \Rat(\sqrt{p}, \sqrt{q})$ obviously because $r + s(\sqrt{p} + \sqrt{q})$ for 
    any $r, s \in \Rat$ is $r + s\sqrt{p} + s\sqrt{q}$, which is an element of $\Rat(\sqrt{p}, \sqrt{q})$.

    Next, consider that $\sqrt{p} + \sqrt{q} \in \Rat(\sqrt{p} + \sqrt{q})$. 
    Since $(\sqrt{p} - \sqrt{q})(\sqrt{p} + \sqrt{q}) = p + q$. 
    Then, $\sqrt{p} - \sqrt{q} = \frac{p + q}{\sqrt{p} + \sqrt{q}} \in \Rat(\sqrt{p} + \sqrt{q})$ because it is 
    a field (by closure of field). 
    Therefore, $\sqrt{p} = 1/2((\sqrt{p} - \sqrt{q}) + (\sqrt{p} + \sqrt{q})) \in \Rat(\sqrt{p} + \sqrt{q})$.
    Then, $\sqrt{p}$ and $\sqrt{q}$ is a member of $\Rat(\sqrt{p} + \sqrt{q})$. 
    Therefore, $\Rat(\sqrt{p}, \sqrt{q}) = \Rat(\sqrt{p} + \sqrt{q})$
  }

  \qs{}{
    Let $m, n$ be natural numbers. Determine when two fields $\Rat(\sqrt{n})$ and $\Rat(\sqrt{m})$ are isomorphic.
  }
  \sol{
    \newcommand{\Q}[1]{\Rat(\sqrt{#1})}
    Notice that if $n$ and $m$ are a square of integers, including zero and one, then $\sqrt{n}$ and $\sqrt{m}$ are integer, 
    thus, $\sqrt{n}, \sqrt{m} \in \Rat$.
    Therefore, $\Q{n} \isom \Q{m} \isom \Rat$.

    Otherwise $\sqrt{n}$ is algebraic in $\Real/\Rat$ as it is a root of polynomial $x^2 - n = 0$, which has degree 2.
    Since $\sqrt{n} \notin \Rat$, then $x^2 - n$ is the minimal polynomial of $\sqrt{n}$. And $\Q{n}/\Rat$ is a vector space of 
    dimension 2, which means that $\set{1, \sqrt{n}}$ is a basis of the vector space as it is a linearly independent set.

    If there is an isomorphism $\phi: \Q{n} \isom \Q{m}$, then $\phi(1) = 1$ as it is the mutiplicative identity in each field, 
    but then $\phi(r) = r$ for every $r \in \Rat$ as $\phi(r) = \phi(a/b) = \phi(a)\phi(b)^{-1}$ for $a, b \in \Int$ and
    $\phi(a) = \phi(1) + \cdots + \phi(1) = 1 + \cdots + 1 = a$ for any positive integer $a$. Moreover, $\phi(-a) = -\phi(a) = -a$ 
    since $0 = \phi(0) = \phi(a - a) = \phi(a) + \phi(-a)$.
    Now, $\phi(\sqrt{n})\phi(\sqrt{n}) = \phi(n) = n$, thus, $\phi(\sqrt{n}) = \sqrt{n}$ or $\phi(\sqrt{n}) = -\sqrt{n}$ only. 

    Therefore, if $n = r^2m$ for some $0 \ne r \in \Rat$,
    \[ \Q{n} = \set{x + y\sqrt{n} \mid x, y \in \Rat} = \set{x + ry\sqrt{n} \mid x, y \in \Rat} 
    = \set{x + y\sqrt{m} \mid x,y \in \Rat } = \Q{m} \]
    Together with identity map, $\Q{n} \isom \Q{m}$

    And if $n \ne r^2m$ for any $0 \ne r \in \Rat$, then $\sqrt{n} \in \Q{n}$ but $\sqrt{n} \notin \Q{m}$ since 
    $\sqrt{n} \ne r\sqrt{m}$ for any $r$. So $-\sqrt{n} \notin \Q{m}$. Thus, there cannot be any isomorphism between 
    $\Q{n}$ and $\Q{m}$. 
    
    To conclude, for $n \ne 0, m \ne 0$, $\Q{n} \isom \Q{m}$ if and only if $n = r^2m$ for some $r \in \Rat$, 
    and $\Q{n} \isom \Q{0}$ if and only if $\Q{n} \isom \Q{1}$
  }

  \qs{}{
    Let $E/F$ be a field extension and let $a \in E$. Prove that if $f(a)$ is algebraic over $F$ for a non-constant polynomial 
    $f \in F[x]$, then $a$ is algebraic over $F$ as well. 
  }
  \sol{
    If $f(a)$ is algebraic, then there exists $0 \ne g \in F[x]$ such that $f(a)$ is the root of $g$. This means that $g(f(a)) = 0$. 

    As $g$ is non-zero, let $g(x) = g_nx^n + \cdots + g_0$ where $g_n \ne 0$ and as $f$ is non-constant, 
    then let $f(x) = f_mx^m + \cdots + f_0$ where $f_m \ne 0$

    Then, \eqs{ g\circ f(x) &= g_nf(x)^n + \cdots + g_0 \\ 
                            &= g_n (f_mx^m + \cdots + f_0)^n + \cdots + g_1(f_mx^m + \cdots + f_0) + g_0 \\ 
                            &= g_nf_m^nx^{mn} + \cdots + g_nf_0^n + \cdots + g_1f_mx^m + \cdots + g_1f_0 + g_0 }
    which asserts that $g\circ f$ is non-constant as $g_nf_m \ne 0$, thus it is non-zero.

    Moreover, as $g, f \in F[x]$, then $g \circ f \in F[x]$ as shown in the above equations.
    Since $g \circ f \in F[x]$ is non-zero but $g \circ f(a) = 0$, then $a$ is algebraic. 
  }

  \qs{}{
    Let $F(x)$ be the quotient field of the polynomial ring $F[x]$ over a field $F$. Find the degree $[F(x) : F(\frac{x^2}{x-1})$
  }
  \sol{
    \newcommand{\x}{\frac{x^2}{x-1}}
    Let $p$ be the polynomial $y^2 - \frac{x^2}{x-1}y-1$ in $F(\x)[y]$. Now, by eisenstein criterion, $\x$ is irreducible in $F[\x]$
    and satisfies the condition of einsenstein. Thus, $p$ is irreducible in $F(\x)[y]$. Moreover it is not hard to see that
    $y = x$ is a root of $p$, because $x^2 - \x(x-1) = 0$.

    Therefore, the minimal polynomial of $x$ in $F(\x)$ is $p$, and since $\deg(p) = 2$,
    the degree $[F(x): F(\x)]$ is $2$.
  }
\end{document}
