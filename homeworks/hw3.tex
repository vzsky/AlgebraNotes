% chktex-file 44
% chktex-file 8

\documentclass{report}
\usepackage{amsthm}
\usepackage{amsmath}
\usepackage{amssymb}
\usepackage{amssymb}
\usepackage{amsfonts}
\usepackage{xcolor}
\usepackage{tikz}
\usepackage{fancyhdr}
\usepackage{enumerate}
\usepackage{graphicx}
\usepackage[normalem]{ulem}
\usepackage[most,many,breakable]{tcolorbox}
\usepackage[a4paper, top=80pt, foot=25pt, bottom=50pt, left=0.5in, right=0.5in]{geometry}
\usepackage{hyperref, theoremref}
\hypersetup{
	pdftitle={Assignment},
	colorlinks=true, linkcolor=b!90,
	bookmarksnumbered=true,
	bookmarksopen=true
}
\usepackage{nameref}
\usepackage{parskip}
\pagestyle{fancy}

\usepackage[explicit,compact]{titlesec}
\titleformat{\chapter}[block]{\bfseries\huge}{\thechapter. }{\compact}{#1}
        

%%%%%%%%%%%%%%%%%%%%%
%% Defining colors %%
%%%%%%%%%%%%%%%%%%%%%

\definecolor{lr}{RGB}{188, 75, 81}
\definecolor{r}{RGB}{249, 65, 68}
\definecolor{dr}{RGB}{174, 32, 18}
\definecolor{lo}{RGB}{255, 172, 129}
\definecolor{do}{RGB}{202, 103, 2}
\definecolor{o}{RGB}{238, 155, 0}
\definecolor{ly}{RGB}{255, 241, 133}
\definecolor{y}{RGB}{255, 229, 31}
\definecolor{dy}{RGB}{143, 126, 0}
\definecolor{lb}{RGB}{148, 210, 189}
\definecolor{bg}{RGB}{10, 147, 150}
\definecolor{b}{RGB}{39, 125, 161}
\definecolor{db}{RGB}{0, 95, 115}
\definecolor{p}{RGB}{229, 152, 155}
\definecolor{dp}{RGB}{181, 101, 118}
\definecolor{pp}{RGB}{142, 143, 184}
\definecolor{v}{RGB}{109, 89, 122}
\definecolor{lg}{RGB}{144, 190, 109}
\definecolor{g}{RGB}{64, 145, 108}
\definecolor{dg}{RGB}{45, 106, 79}

\colorlet{mysol}{g}
\colorlet{mythm}{lr}
\colorlet{myqst}{db}
\colorlet{myclm}{lb}
\colorlet{mywrong}{r}
\colorlet{mylem}{o}
\colorlet{mydef}{lg}
\colorlet{mycor}{lb}
\colorlet{myrem}{dr}

%%%%%%%%%%%%%%%%%%%%%

\newcommand{\col}[2]{
  \color{#1}#2\color{black}\,
}

\newcommand{\TODO}[1][5cm]{
  \color{red}TODO\color{black}
  \vspace{#1}
}

\newcommand{\wans}[1]{
	\noindent\color{mywrong}\textbf{Wrong answer: }\color{black}
	#1 


}

\newcommand{\wreason}[1]{
	\noindent\color{mywrong}\textbf{Reason: }\color{black}
	#1 

  
}

\newcommand{\sol}[1]{
	\noindent\color{mysol}\textbf{Solution: }\color{black}
	#1


}

\newcommand{\nt}[1]{
  \begin{note}Note: #1\end{note}
}

\newcommand{\ky}[1]{
  \begin{key}#1\end{key}
}

\newcommand{\pf}[1]{
  \begin{myproof}#1\end{myproof}
}

\newcommand{\qs}[3][]{
  \begin{question}{#2}{#1}#3\end{question}
}

\newcommand{\df}[3][]{
  \begin{definition}{#2}{#1}#3\end{definition}
}

\newcommand{\thm}[3][]{
  \begin{theorem}{#2}{#1}#3\end{theorem}
}

\newcommand{\clm}[3][]{
  \begin{claim}{#2}{#1}#3\end{claim} 
}

\newcommand{\lem}[3][]{
  \begin{lemma}{#2}{#1}#3\end{lemma}
}

\newcommand{\cor}[3][]{
  \begin{corollary}{#2}{#1}#3\end{corollary}
}

\newcommand{\rem}[3][]{
  \begin{remark}{#2}{#1}#3\end{remark}
}

\newcommand{\twoways}[2]{
  \leavevmode\\
  ($\Longrightarrow$): 
  \begin{shift}#1\end{shift}
  ($\Longleftarrow$):
  \begin{shift}#2\end{shift} 
}

\newcommand{\nways}[2]{
  \leavevmode\\
  ($#1$): 
  \begin{shift}#2\end{shift}
}

%%%%%%%%%%%%%%%%%%%%%%%%%%%%%% ENVRN

\newenvironment{myproof}[1][\proofname]{%
	\proof[\bfseries #1: ]
}{\endproof}

\tcbuselibrary{theorems,skins,hooks}
\newtcolorbox{shift}
{%
  before upper={\setlength{\parskip}{5pt}},
  blanker,
	breakable,
	width=0.95\textwidth,
  enlarge left by=0.03\textwidth,
}

\tcbuselibrary{theorems,skins,hooks}
\newtcolorbox{key}
{%
	breakable,
	width=0.95\textwidth,
  enlarge left by=0.03\textwidth,
}

\tcbuselibrary{theorems,skins,hooks}
\newtcolorbox{note}
{%
	enhanced,
	breakable,
	colback = white,
	width=\textwidth,
	frame hidden,
	borderline west = {2pt}{0pt}{black},
	sharp corners,
}

\tcbuselibrary{theorems,skins,hooks}
\newtcbtheorem[]{remark}{Remark}
{%
	enhanced,
	breakable,
	colback = white,
	frame hidden,
	boxrule = 0sp,
	borderline west = {2pt}{0pt}{myrem},
	sharp corners,
	detach title,
  before upper={\setlength{\parskip}{5pt}\tcbtitle\par\smallskip},
	coltitle = myrem,
	fonttitle = \bfseries\sffamily,
	description font = \mdseries,
	separator sign none,
	segmentation style={solid, myrem},
}{rem}

\tcbuselibrary{theorems,skins,hooks}
\newtcbtheorem[number within=section]{lemma}{Lemma}
{%
	enhanced,
	breakable,
	colback = white,
	frame hidden,
	boxrule = 0sp,
	borderline west = {2pt}{0pt}{mylem},
	sharp corners,
	detach title,
  before upper={\setlength{\parskip}{5pt}\tcbtitle\par\smallskip},
	coltitle = mylem,
	fonttitle = \bfseries\sffamily,
	description font = \mdseries,
	separator sign none,
	segmentation style={solid, mylem},
}{lem}

\tcbuselibrary{theorems,skins,hooks}
\newtcbtheorem{claim}{Claim}
{%
  parbox=false,
	enhanced,
	breakable,
	colback = white,
	frame hidden,
	boxrule = 0sp,
	borderline west = {2pt}{0pt}{myclm},
	sharp corners,
	detach title,
  before upper={\setlength{\parskip}{5pt}\tcbtitle\par\smallskip},
	coltitle = myclm,
	fonttitle = \bfseries\sffamily,
	description font = \mdseries,
	separator sign none,
	segmentation style={solid, myclm},
}{clm}

\makeatletter
\newtcbtheorem[number within=section, use counter from=lemma]{theorem}{Theorem}{enhanced,
	breakable,
	colback=white,
	colframe=mythm,
	attach boxed title to top left={yshift*=-\tcboxedtitleheight},
	fonttitle=\bfseries,
	title={#2},
	boxed title size=title,
	boxed title style={%
			sharp corners,
			rounded corners=northwest,
			colback=mythm,
			boxrule=0pt,
		},
	underlay boxed title={%
			\path[fill=mythm] (title.south west)--(title.south east)
			to[out=0, in=180] ([xshift=5mm]title.east)--
			(title.center-|frame.east)
			[rounded corners=\kvtcb@arc] |-
			(frame.north) -| cycle;
		},
	#1
}{thm}
\makeatother

\makeatletter
\newtcbtheorem{question}{Question}{enhanced,
	breakable,
	colback=white,
	colframe=myqst,
	attach boxed title to top left={yshift*=-\tcboxedtitleheight},
	fonttitle=\bfseries,
	title={#2},
	boxed title size=title,
	boxed title style={%
			sharp corners,
			rounded corners=northwest,
			colback=myqst,
			boxrule=0pt,
		},
	underlay boxed title={%
			\path[fill=myqst] (title.south west)--(title.south east)
			to[out=0, in=180] ([xshift=5mm]title.east)--
			(title.center-|frame.east)
			[rounded corners=\kvtcb@arc] |-
			(frame.north) -| cycle;
		},
	#1
}{qs}
\makeatother

\makeatletter
\newtcbtheorem[number within=section]{definition}{Definition}{enhanced,
	breakable,
	colback=white,
	colframe=mydef,
	attach boxed title to top left={yshift*=-\tcboxedtitleheight},
	fonttitle=\bfseries,
	title={#2},
	boxed title size=title,
	boxed title style={%
			sharp corners,
			rounded corners=northwest,
			colback=mydef,
			boxrule=0pt,
		},
	underlay boxed title={%
			\path[fill=mydef] (title.south west)--(title.south east)
			to[out=0, in=180] ([xshift=5mm]title.east)--
			(title.center-|frame.east)
			[rounded corners=\kvtcb@arc] |-
			(frame.north) -| cycle;
		},
	#1
}{def}
\makeatother

\makeatletter
\newtcbtheorem[number within=section, use counter from=lemma]{corollary}{Corollary}{enhanced,
	breakable,
	colback=white,
	colframe=mycor,
	attach boxed title to top left={yshift*=-\tcboxedtitleheight},
	fonttitle=\bfseries,
	title={#2},
	boxed title size=title,
	boxed title style={%
			sharp corners,
			rounded corners=northwest,
			colback=mycor,
			boxrule=0pt,
		},
	underlay boxed title={%
			\path[fill=mycor] (title.south west)--(title.south east)
			to[out=0, in=180] ([xshift=5mm]title.east)--
			(title.center-|frame.east)
			[rounded corners=\kvtcb@arc] |-
			(frame.north) -| cycle;
		},
	#1
}{cor}
\makeatother

% Basic
  \DeclareMathOperator{\lcm}{lcm}
  \newcommand{\Real}{\mathbb{R}}
  \newcommand{\Comp}{\mathbb{C}}
  \newcommand{\Nat}{\mathbb{N}}
  \newcommand{\Rat}{\mathbb{Q}}
  \newcommand{\Int}{\mathbb{Z}}
  \newcommand{\set}[1]{\left\{\, #1 \,\right\}}
  \newcommand{\paren}[1]{\left( \; #1 \; \right)}
  \newcommand{\abs}[1]{\left\lvert #1 \right\rvert}
  \newcommand{\ang}[1]{\left\langle #1 \right\rangle}
  \renewcommand{\to}[1][]{\xrightarrow{\text{#1}}}
  \newcommand{\tol}[1][]{\to{$#1$}}
  \newcommand{\curle}{\preccurlyeq}
  \newcommand{\curge}{\succcurlyeq}
  \newcommand{\mapsfrom}{\leftarrow\!\shortmid}

  \newcommand{\mat}[1]{\begin{bmatrix} #1 \end{bmatrix}}
  \newcommand{\pmat}[1]{\begin{pmatrix} #1 \end{pmatrix}}
  \newcommand{\eqs}[1]{\begin{align*} #1 \end{align*}}
  \newcommand{\case}[1]{\begin{cases} #1 \end{cases}}
  

  % Algebra
  \newcommand{\normSg}[0]{\vartriangleleft}
  \newcommand{\ZMod}[1][n]{\mathbb{Z}/#1\mathbb{Z}}
  \newcommand{\isom}{\simeq}
  \newcommand{\mapHom}{\xrightarrow{\text{hom}}}
  \DeclareMathOperator{\Inn}{Inn}
  \DeclareMathOperator{\Aut}{Aut}
  \DeclareMathOperator{\im}{im}
  \DeclareMathOperator{\ord}{ord}
  \DeclareMathOperator{\Gal}{Gal}
  \DeclareMathOperator{\chr}{char}
  \newcommand{\surjto}{\twoheadrightarrow}
  \newcommand{\injto}{\hookrightarrow}

  % Analysis 
  \newcommand{\limty}[1][k]{\lim_{#1\to\infty}}
  \newcommand{\norm}[1]{\left\lVert#1\right\rVert}
  \newcommand{\darrow}{\rightrightarrows}


\fancyhead[L]{HW 3 - Modern Algebra MAS311}
\fancyhead[R]{\textbf{Touch Sungkawichai} 20210821}

\begin{document}
  \qs{}{
    Prove that $\ZMod[p] \times \ZMod[q] \times \ZMod[r]$ is cyclic if and only if 
    $p, q, r$ are pairwise relatively prime.
  }  
  \sol{
    Denote $\ZMod[p] \times \ZMod[q] \times \ZMod[r]$ as $G$. 
    \twoways{
      If $G$ is cyclic, then there is an element $g$ that generates $G$, or $\ang{g} = G$. 
      Let $g = ({[g_1]}_p, {[g_2]}_q, {[g_3]}_r)$.
      Assume that $p, q, r$ are not pairwise relatively prime.
      Let say that $\gcd(p, q) = d > 1$ without loss of generality. Then there is $a, b \in \Int$ such that $ap + bq = d$
      Then consider all value $k\in\Nat$ such that $(dk)g_1 \equiv 0 \mod p$, and $(dk)g_2 \equiv 1 \mod q$. 
      Then \begin{align*} 
          dk(g_1 + g_2) - 1 &= \alpha p + \beta q
      \\  (ap + bq)k(g_1+g_2) - \alpha p - \beta q &= 1
      \\  (a(g_1+g_2) - \alpha)k p + (b(g_1+g_2) - \beta)k q &= 1
      \end{align*}
      So, $d = 1$ contradicts that $p$ and $q$ are not relatively prime.

      Hence, $G$ is cyclic implies $p, q, r$ are pairwise relatively prime.
    }{
      If $p, q, r$ is pairwise relatively prime, then consider any $g = (g_1, g_2, g_3) \in G$.
      Notice that $g^n = (g_1^n, g_2^n, g_3^n)$, and 
      \[ g_1^{kp + k'} = g_1^{k'}, g_2^{kq+k'} = g_2^{k'}, g_3^{kr+k'} = g_3^{k'} \]
      But a $k \in \Nat$ satisfying the following exists by the chinese remainder theorem.
      \begin{gather*} k \equiv k_1 \mod p \\ k \equiv k_2 \mod q \\ k \equiv k_3 \mod r \end{gather*}
      Hence, $\forall h\in G, \exists n \in \Nat \quad g^n = h$, proving that $G$ is cyclic.
    }
  }  
  \qs{}{
    Prove that a group $G$ is abelian if and only if the map $f: G \to G$ 
    given by $f(g) = g^{-1}$ for all $g \in G$ is a homomorphism.
  }
  \sol{
    \twoways{
      If the group is abelian, then for all $g, h \in G$, $gh = hg$. 
      So \[ g^{-1}h^{-1} = h^{-1}g^{-1} = {(gh)}^{-1} \]
      Hence, a function $f(g) = g^{-1}$ preserves the binary operator, as $f(g)f(h) = f(gh)$, and thus, is a homomorphism.
    }{
      If $f$ is a homomorphism, then $\forall g, h \in G \quad f(g)f(h) = f(gh)$. Which means that 
      \[ g^{-1}h^{-1} = {(gh)}^{-1} = h^{-1}g^{-1} \]
      But since $g^{-1} \in G$ for any $g$, then the above relation shows that every element is commutative.
      Hence, the group is abelian.
    }
  } 

  \qs{}{
    Show that the map $f: \Int \to \ZMod$ defined by $f(a) = {[a]}_n$ is an epimorphism 
    with $\ker f = \set{mn \mid m \in \Int}$ 
  }
  \sol{
    The map $f$ is well-defined since for $a = b$, $f(a) = {[a]}_n = {[b]}_n = f(b)$ as $b = a + nk$ for some $n \in \Nat$.
    Consider that $\ZMod = \set{{[0]}_n, \ldots, {[n-1]}_n}$, then we know that $f(a) = {[a]}_n$ for all integer $0 \le a < n$.
    Therefore, $f$ is surjective, hence, an epimorphism. 
    
    Moreover, for $f(x) = {[0]}_n$, we get that $n \vert x$, so $\ker f = \set{x \mid n \vert x} = \set{mn \mid m \in \Int}$
  } 
  
  \qs{}{
    Prove that every finitely generated subgroup of the additive group $\Rat$ is cyclic.
    Exhibit a proper subgroup of the additive group $\Rat$ that is not cyclic.
  }
  \sol{ 
    Note that I use the convention notation $g^n$ instead of $ng$ despite the group being additive.

    Let $S$ be a subgroup of the additive group $\Rat$ that is generated by $g_1, \ldots, g_n$. 
    Then we can choose $g_0$ such that
    \[ g_1 = g_0^{k_1}, g_2 = g_0^{k_2}, \ldots, g_n = g_0^{k_n} \quad \exists k_1, \ldots, k_n \]
    (for example, by taking $g_0 = \frac{1}{g'_1 \times \cdots \times g'_n}$ where $g'_i$ denotes the denominator of $g_i$
    
    Since $\forall h \in S, \quad h = g_1^{m_1}g_2^{m_2}\cdots g_n^{m_n}$, 
    then \[ h = g_0^{k_1m_1}\cdots g_0^{k_n m_n} = g_0^{k_1m_1 + \cdots + k_n m_n} \]
    Hence, the group is generated by $g_0$. Thus the group is cyclic. 

    For a proper subgroup of the additive group $\Rat$ that is not cyclic, consider the group 
    \[ C = \set{\frac{a}{2^b} \mid a, b \in \Int} \] 
    Firstly, note that $C$ is a subgroup, since for all $\frac{a}{2^b}, \frac{c}{2^d} \in C$, 
    the element $\frac{a2^d + b2^c}{2^{b+d}} \in C$.
    Moreover, for every $a \in C$, $-a \in C$. 

    However, $\frac{1}{3} \not\in C$ as $3 = 2^{x}$ has no integer solution. 

    Lastly, $C$ is not generated by a single element. To show this, assume otherwise that $C = \ang{g}$. 
    Then $g \ne 0$, as if $g = 0$, $g^n = 0$. Moreover, $\frac{g}{2} \in C$ but is not in $\ang{g}$. 
    This shows that $C$ is not cyclic.
  }

  \qs{}{
    Let $G$ be a cyclic group of order $n$ and let $d$ be a divisor of $n$. 
    Show that $G$ has exactly one subgroup order $d$.
  }
  \sol{
    \clm[subcyiscy]{
      All finite subgroups of a cyclic group is cylic.
      \pf{ Since a cyclic group $G$ is generated by a single element, $g$, then each element 
      in the subgroup $S \le G$ is in $G$, which means that it must be some power of $g$. 
      Then, let $S = \set{g^{a_1}, \ldots, g^{a_n}}$, then it is possible to choose the smallest possitive $a_i$, so let $b$ 
      be that element.
      With the division algorithm, we know that for some $a$, $g^{a} = g^{mb + c}$ for some integer $m$ and $0 \le c < b$. 
      The closure of the group asserts that $g^{c}$ must be an element of $S$, but $0 \le c < b$, so $c = 0$. 

      Hence, $g^b$ generates $S$.
      }
    }
    From claim~\ref{clm:subcyiscy}, a subgroup of order $d$ must be cyclic. And since every finite 
    cyclic group of order $n$ is isomorphic to the group $\ZMod[n]$, then we will show that the subgroup 
    of order $d$ of $\ZMod[n]$ is unique. This generalize naturally to every cyclic groups.

    Let $\frac{n}{d} = k \in \Nat$, and consider ${[1]}_n$ is an element of order $n$ in $G$.
    Then, there is an element $k{[1]}_n$ denoted by ${[k]}_n$ such that $\ang{{[k]}_n}$ is a subgroup of order $d$.

    Now, any subgroup $S$ of order $d$ must be generated by one element, that is of order $d$, 
    as an element of order $p$ will always generate a group with $p$ distinct elements.

    Therefore, $\ang{{[k]}_n} = \ang{{[m]}_n}$ for some element $m$ with order $d$.
    But since ${[m]}_n = m{[1]}_n$, then $dm{[1]}_n = {[0]}_n$. So, $n \vert dm$. 
    Hence, $k(\frac{dm}{n}) = m$, so $(\frac{dm}{n}){[k]}_n = {[m]}_n$. 
    
    Lastly, ${[m]}_n \in \ang{{[k]}_n}$ implies that $S$ must be unique.
  }

  \qs{}{
    Find the center of the group $SL_2(\Real)$
  }
  \sol{
    Consider 2 matrices $A = \begin{bmatrix}a & b \\ c & d\end{bmatrix}$, 
    and $B = \begin{bmatrix} x & y \\ z & w \end{bmatrix}$, then 

    \[ AB = \begin{bmatrix} ax + bz & ay + bw \\ cx + dz & cy + dw \end{bmatrix} \text{ and }
    BA = \begin{bmatrix} ax + cy & bx + dy \\ az + cw & bz + dw \end{bmatrix} \]
    
    So, $AB = BA$ when \begin{align*}
      bz &= cy \\
      ay + bw &= bx + dy \\
      cx + dz &= az + cw \\
    \end{align*}
    Since $bz = cy$ must holds for any value of $b$ and $c$, then $y = z = 0$ is the only possible option.
    Then it follows that $x = w$. 

    Therefore, $B = \begin{bmatrix} w & 0 \\ 0 & w \end{bmatrix}$ commutes with any matrix $A \in SL_2(\Real)$.
    But since $B \in SL_2(\Real)$, Then $\det B = 1$, which means that $w = \pm 1$. Therefore, the center of $SL_2(\Real)$
    is $\set{I_2, -I_2}$.  
  }

  \qs{}{
    Let $f: G \to H$ be a homomorphism and $g \in G$. Assume that $\abs{g}$ and $\abs{f(g)}$ are finite.
    Show that $\abs{f(g)}$ divides $\abs{g}$
  }
  \sol{
    Let the order of $g$ be $n$. 
    Since a homomorphism must preserve the binary operator, then \[f(1) = f(g^n) = {f(g)}^n = 1\] 
    Therefore, if the order of $f(g)$ be $d$, then there exist integer $k$ such that ${f(g)}^{dk} = {f(g)}^n = 1$. 
    So, $d$ divides $n$.
  }

  \newcommand{\mulZpnG}{(\ZMod[(p^n-1)])^\times}
  \qs{}{
    Let $p$ be a prime and let $n$ be a positive integer. Find the order of $[p]$ in the multiplicative group
    $\mulZpnG$ and deduce that $n \vert \varphi(p^n-1)$, where $\varphi$ denotes Euler's function.
  }
  \sol{
    let the order of $[p]$ be $d$, then $p^d \equiv 1 \mod p^n-1$, but since for $d < n$, $p^d < p^n$, 
    then $d = n$ is the smallest solution, and thus is the order of $[p]$. 
    
    Moreover, $\mulZpnG = \set{ m < p^n-1 \mid \gcd(m, p^n-1) = 1}$. So $\abs{\mulZpnG} = \varphi(p^n-1)$ by definition.
    Lastly, by lagrange theorem, the order of subgroup divides the order of the group, and $\ang{[p]}$ is a cyclic subgroup of $\mulZpnG$,
    therefore, $n \vert \varphi(p^n-1)$
  }

  \qs{}{
    Show that the quaternion group $Q_8$ and the dihedral group $D_8$ are not isomorphic. 
    Show also that $Q_8$ is not isomorphic to a subgroup of $S_n$ for any $n \le 7$
  }
  \sol{
    Consider that there is an element $r$ in $D_8$ such that $r \ne 1$, $r^2 \ne 1$, $r^3 \ne 1$ and $r^4 = 1$.
    But in $Q_8 = \set{\pm 1, \pm i, \pm j, \pm k}$, if $q \in Q_8$, $q^2 = 1$. 
    Hence, there cannot be an isomorphism between $D_8$ and $Q_8$. 

    Note that $S_n$ is a subgroup of $S_{n+1}$. Hence, we will only show that $Q_8$ is not a subgroup of $S_7$, 
    as if it is not a subgroup of $S_7$, then it is not a subgroup of a subgroup of $S_7$.
    
    Now, assume that there is a isomorphism between $Q_8$ and a subgroup of $S_7$. The assumption is equivalent to having
    an injective homomorphism from $Q_8$ to $S_7$. Which means that there must be a group action between $Q_8$ and $A$ where
    $A$ is a set with 7 elements, such that $\abs{set{g \mid g \cdot x = x}} = 1$, followed from the fact that the homomorphism
    must be injective.

    This follows from the derivation of the equivalence of homomorphism and group action.
    The construction of the equivalence of $\varphi$ to a group action asserts the the kernel of $\varphi$ consists of all 
    the element $g \in G$ such that $g \cdot x = x$.

    From the assumption, we construct the operator $(\cdot)$. Firstly $1 \cdot x = x$ and for $g \ne 1$, $g \cdot x \ne x$.
    Let there be an element $a \in A$. Then we know that \[ -1 \cdot a = b \quad \exists b \ne a \].
    Moreover \[i \cdot a = -i \cdot i\cdot i \cdot a = -i \cdot -1 \cdot a = -i \cdot b = c \] for an element $c$ that is 
    pairwise distinct from $a$ and $b$. 
    Next, \[j \cdot c = j \cdot i \cdot a = j \cdot -i \cdot b = d \] for an element $d$ that is pairwise distinct 
    from $a, b, c$, since $ji \ne 1$ and $j(-i) \ne 1$.
    Next, \[ -1 \cdot d = -j \cdot c = -k \cdot b = k \cdot a = e \] for some element $e$ that is distinct from $a, b, c, d$. 
    Then we can have, 
    \[ i \cdot e = -i \cdot d = k \cdot c = j \cdot b = -j \cdot a = f \] 
    for some element $f$ that is distinct from $a, b, c, d, e$
    Then \[ -k \cdot f = -j \cdot e = j \cdot d = -1 \cdot c = i \cdot b = -i \cdot a = g \] 
    for some $g$ that is distinct from $a, b, c, d, e, f$. Making the set $A$ contain exactly 7 elements. 

    However, since \[ -k \cdot g = -1 \cdot f = -i \cdot e = i \cdot d = k \cdot c = -j \cdot b = j \cdot a = h \] 
    such that $h$ is pairwise distinct from the prior 7 elements, this means that $A$ contains more than 7 elements.

    Thus, it can be concluded that there is no isomorphism from $Q_8$ to a subgroup of $S_7$, by contradiction. And lastly, 
    there must be no isomorphism from $Q_8$ to a subgroup of $S_n$ for any $n < 8$. 
  }

  \qs{}{
    Find all homomorphism from $\ZMod$ to $\Rat$. 
  }
  \sol{
    Consider a homomorphism $\varphi: \ZMod \to \Rat$, then 
    \[ \varphi({[0]}_n) = \varphi(n{[1]}_n) = n\varphi({[1]}_n) = 0 \]
    but for $q \in \Rat$, $nq = 0$ if and only if $q = 0$.
    Therefore, there is only one homomorphism from $\ZMod$ to $\Rat$ which is $\forall x \; \varphi(x) = 0$.
  }

\end{document}
