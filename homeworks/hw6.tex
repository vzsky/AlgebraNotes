% chktex-file 44
% chktex-file 8

\documentclass{report}
\usepackage{amsthm}
\usepackage{amsmath}
\usepackage{amssymb}
\usepackage{amssymb}
\usepackage{amsfonts}
\usepackage{xcolor}
\usepackage{tikz}
\usepackage{fancyhdr}
\usepackage{enumerate}
\usepackage{graphicx}
\usepackage[normalem]{ulem}
\usepackage[most,many,breakable]{tcolorbox}
\usepackage[a4paper, top=80pt, foot=25pt, bottom=50pt, left=0.5in, right=0.5in]{geometry}
\usepackage{hyperref, theoremref}
\hypersetup{
	pdftitle={Assignment},
	colorlinks=true, linkcolor=b!90,
	bookmarksnumbered=true,
	bookmarksopen=true
}
\usepackage{nameref}
\usepackage{parskip}
\pagestyle{fancy}

\usepackage[explicit,compact]{titlesec}
\titleformat{\chapter}[block]{\bfseries\huge}{\thechapter. }{\compact}{#1}
        

%%%%%%%%%%%%%%%%%%%%%
%% Defining colors %%
%%%%%%%%%%%%%%%%%%%%%

\definecolor{lr}{RGB}{188, 75, 81}
\definecolor{r}{RGB}{249, 65, 68}
\definecolor{dr}{RGB}{174, 32, 18}
\definecolor{lo}{RGB}{255, 172, 129}
\definecolor{do}{RGB}{202, 103, 2}
\definecolor{o}{RGB}{238, 155, 0}
\definecolor{ly}{RGB}{255, 241, 133}
\definecolor{y}{RGB}{255, 229, 31}
\definecolor{dy}{RGB}{143, 126, 0}
\definecolor{lb}{RGB}{148, 210, 189}
\definecolor{bg}{RGB}{10, 147, 150}
\definecolor{b}{RGB}{39, 125, 161}
\definecolor{db}{RGB}{0, 95, 115}
\definecolor{p}{RGB}{229, 152, 155}
\definecolor{dp}{RGB}{181, 101, 118}
\definecolor{pp}{RGB}{142, 143, 184}
\definecolor{v}{RGB}{109, 89, 122}
\definecolor{lg}{RGB}{144, 190, 109}
\definecolor{g}{RGB}{64, 145, 108}
\definecolor{dg}{RGB}{45, 106, 79}

\colorlet{mysol}{g}
\colorlet{mythm}{lr}
\colorlet{myqst}{db}
\colorlet{myclm}{lb}
\colorlet{mywrong}{r}
\colorlet{mylem}{o}
\colorlet{mydef}{lg}
\colorlet{mycor}{lb}
\colorlet{myrem}{dr}

%%%%%%%%%%%%%%%%%%%%%

\newcommand{\col}[2]{
  \color{#1}#2\color{black}\,
}

\newcommand{\TODO}[1][5cm]{
  \color{red}TODO\color{black}
  \vspace{#1}
}

\newcommand{\wans}[1]{
	\noindent\color{mywrong}\textbf{Wrong answer: }\color{black}
	#1 


}

\newcommand{\wreason}[1]{
	\noindent\color{mywrong}\textbf{Reason: }\color{black}
	#1 

  
}

\newcommand{\sol}[1]{
	\noindent\color{mysol}\textbf{Solution: }\color{black}
	#1


}

\newcommand{\nt}[1]{
  \begin{note}Note: #1\end{note}
}

\newcommand{\ky}[1]{
  \begin{key}#1\end{key}
}

\newcommand{\pf}[1]{
  \begin{myproof}#1\end{myproof}
}

\newcommand{\qs}[3][]{
  \begin{question}{#2}{#1}#3\end{question}
}

\newcommand{\df}[3][]{
  \begin{definition}{#2}{#1}#3\end{definition}
}

\newcommand{\thm}[3][]{
  \begin{theorem}{#2}{#1}#3\end{theorem}
}

\newcommand{\clm}[3][]{
  \begin{claim}{#2}{#1}#3\end{claim} 
}

\newcommand{\lem}[3][]{
  \begin{lemma}{#2}{#1}#3\end{lemma}
}

\newcommand{\cor}[3][]{
  \begin{corollary}{#2}{#1}#3\end{corollary}
}

\newcommand{\rem}[3][]{
  \begin{remark}{#2}{#1}#3\end{remark}
}

\newcommand{\twoways}[2]{
  \leavevmode\\
  ($\Longrightarrow$): 
  \begin{shift}#1\end{shift}
  ($\Longleftarrow$):
  \begin{shift}#2\end{shift} 
}

\newcommand{\nways}[2]{
  \leavevmode\\
  ($#1$): 
  \begin{shift}#2\end{shift}
}

%%%%%%%%%%%%%%%%%%%%%%%%%%%%%% ENVRN

\newenvironment{myproof}[1][\proofname]{%
	\proof[\bfseries #1: ]
}{\endproof}

\tcbuselibrary{theorems,skins,hooks}
\newtcolorbox{shift}
{%
  before upper={\setlength{\parskip}{5pt}},
  blanker,
	breakable,
	width=0.95\textwidth,
  enlarge left by=0.03\textwidth,
}

\tcbuselibrary{theorems,skins,hooks}
\newtcolorbox{key}
{%
	breakable,
	width=0.95\textwidth,
  enlarge left by=0.03\textwidth,
}

\tcbuselibrary{theorems,skins,hooks}
\newtcolorbox{note}
{%
	enhanced,
	breakable,
	colback = white,
	width=\textwidth,
	frame hidden,
	borderline west = {2pt}{0pt}{black},
	sharp corners,
}

\tcbuselibrary{theorems,skins,hooks}
\newtcbtheorem[]{remark}{Remark}
{%
	enhanced,
	breakable,
	colback = white,
	frame hidden,
	boxrule = 0sp,
	borderline west = {2pt}{0pt}{myrem},
	sharp corners,
	detach title,
  before upper={\setlength{\parskip}{5pt}\tcbtitle\par\smallskip},
	coltitle = myrem,
	fonttitle = \bfseries\sffamily,
	description font = \mdseries,
	separator sign none,
	segmentation style={solid, myrem},
}{rem}

\tcbuselibrary{theorems,skins,hooks}
\newtcbtheorem[number within=section]{lemma}{Lemma}
{%
	enhanced,
	breakable,
	colback = white,
	frame hidden,
	boxrule = 0sp,
	borderline west = {2pt}{0pt}{mylem},
	sharp corners,
	detach title,
  before upper={\setlength{\parskip}{5pt}\tcbtitle\par\smallskip},
	coltitle = mylem,
	fonttitle = \bfseries\sffamily,
	description font = \mdseries,
	separator sign none,
	segmentation style={solid, mylem},
}{lem}

\tcbuselibrary{theorems,skins,hooks}
\newtcbtheorem{claim}{Claim}
{%
  parbox=false,
	enhanced,
	breakable,
	colback = white,
	frame hidden,
	boxrule = 0sp,
	borderline west = {2pt}{0pt}{myclm},
	sharp corners,
	detach title,
  before upper={\setlength{\parskip}{5pt}\tcbtitle\par\smallskip},
	coltitle = myclm,
	fonttitle = \bfseries\sffamily,
	description font = \mdseries,
	separator sign none,
	segmentation style={solid, myclm},
}{clm}

\makeatletter
\newtcbtheorem[number within=section, use counter from=lemma]{theorem}{Theorem}{enhanced,
	breakable,
	colback=white,
	colframe=mythm,
	attach boxed title to top left={yshift*=-\tcboxedtitleheight},
	fonttitle=\bfseries,
	title={#2},
	boxed title size=title,
	boxed title style={%
			sharp corners,
			rounded corners=northwest,
			colback=mythm,
			boxrule=0pt,
		},
	underlay boxed title={%
			\path[fill=mythm] (title.south west)--(title.south east)
			to[out=0, in=180] ([xshift=5mm]title.east)--
			(title.center-|frame.east)
			[rounded corners=\kvtcb@arc] |-
			(frame.north) -| cycle;
		},
	#1
}{thm}
\makeatother

\makeatletter
\newtcbtheorem{question}{Question}{enhanced,
	breakable,
	colback=white,
	colframe=myqst,
	attach boxed title to top left={yshift*=-\tcboxedtitleheight},
	fonttitle=\bfseries,
	title={#2},
	boxed title size=title,
	boxed title style={%
			sharp corners,
			rounded corners=northwest,
			colback=myqst,
			boxrule=0pt,
		},
	underlay boxed title={%
			\path[fill=myqst] (title.south west)--(title.south east)
			to[out=0, in=180] ([xshift=5mm]title.east)--
			(title.center-|frame.east)
			[rounded corners=\kvtcb@arc] |-
			(frame.north) -| cycle;
		},
	#1
}{qs}
\makeatother

\makeatletter
\newtcbtheorem[number within=section]{definition}{Definition}{enhanced,
	breakable,
	colback=white,
	colframe=mydef,
	attach boxed title to top left={yshift*=-\tcboxedtitleheight},
	fonttitle=\bfseries,
	title={#2},
	boxed title size=title,
	boxed title style={%
			sharp corners,
			rounded corners=northwest,
			colback=mydef,
			boxrule=0pt,
		},
	underlay boxed title={%
			\path[fill=mydef] (title.south west)--(title.south east)
			to[out=0, in=180] ([xshift=5mm]title.east)--
			(title.center-|frame.east)
			[rounded corners=\kvtcb@arc] |-
			(frame.north) -| cycle;
		},
	#1
}{def}
\makeatother

\makeatletter
\newtcbtheorem[number within=section, use counter from=lemma]{corollary}{Corollary}{enhanced,
	breakable,
	colback=white,
	colframe=mycor,
	attach boxed title to top left={yshift*=-\tcboxedtitleheight},
	fonttitle=\bfseries,
	title={#2},
	boxed title size=title,
	boxed title style={%
			sharp corners,
			rounded corners=northwest,
			colback=mycor,
			boxrule=0pt,
		},
	underlay boxed title={%
			\path[fill=mycor] (title.south west)--(title.south east)
			to[out=0, in=180] ([xshift=5mm]title.east)--
			(title.center-|frame.east)
			[rounded corners=\kvtcb@arc] |-
			(frame.north) -| cycle;
		},
	#1
}{cor}
\makeatother

% Basic
  \DeclareMathOperator{\lcm}{lcm}
  \newcommand{\Real}{\mathbb{R}}
  \newcommand{\Comp}{\mathbb{C}}
  \newcommand{\Nat}{\mathbb{N}}
  \newcommand{\Rat}{\mathbb{Q}}
  \newcommand{\Int}{\mathbb{Z}}
  \newcommand{\set}[1]{\left\{\, #1 \,\right\}}
  \newcommand{\paren}[1]{\left( \; #1 \; \right)}
  \newcommand{\abs}[1]{\left\lvert #1 \right\rvert}
  \newcommand{\ang}[1]{\left\langle #1 \right\rangle}
  \renewcommand{\to}[1][]{\xrightarrow{\text{#1}}}
  \newcommand{\tol}[1][]{\to{$#1$}}
  \newcommand{\curle}{\preccurlyeq}
  \newcommand{\curge}{\succcurlyeq}
  \newcommand{\mapsfrom}{\leftarrow\!\shortmid}

  \newcommand{\mat}[1]{\begin{bmatrix} #1 \end{bmatrix}}
  \newcommand{\pmat}[1]{\begin{pmatrix} #1 \end{pmatrix}}
  \newcommand{\eqs}[1]{\begin{align*} #1 \end{align*}}
  \newcommand{\case}[1]{\begin{cases} #1 \end{cases}}
  

  % Algebra
  \newcommand{\normSg}[0]{\vartriangleleft}
  \newcommand{\ZMod}[1][n]{\mathbb{Z}/#1\mathbb{Z}}
  \newcommand{\isom}{\simeq}
  \newcommand{\mapHom}{\xrightarrow{\text{hom}}}
  \DeclareMathOperator{\Inn}{Inn}
  \DeclareMathOperator{\Aut}{Aut}
  \DeclareMathOperator{\im}{im}
  \DeclareMathOperator{\ord}{ord}
  \DeclareMathOperator{\Gal}{Gal}
  \DeclareMathOperator{\chr}{char}
  \newcommand{\surjto}{\twoheadrightarrow}
  \newcommand{\injto}{\hookrightarrow}

  % Analysis 
  \newcommand{\limty}[1][k]{\lim_{#1\to\infty}}
  \newcommand{\norm}[1]{\left\lVert#1\right\rVert}
  \newcommand{\darrow}{\rightrightarrows}


\fancyhead[L]{HW 6 - Modern Algebra MAS311}
\fancyhead[R]{\textbf{Touch Sungkawichai} 20210821}

\begin{document}
  \qs{}{
    Show that $\Aut(\ZMod[4]) \isom \ZMod[2]$ and $\Aut(D_8) \isom D_8$
  }
  \sol{
    For $\Aut(\ZMod[4])$, consider that for a automorphism $\sigma$ on $\ZMod[4]$, $\sigma([0]) = [0]$ and 
    $\sigma([2]) = [2]$ or $[0]$ since $\sigma(2[2]) = 2\sigma([2]) = [0]$ and $2x = [0]$ has only two solutions in
    $\ZMod[4]$. But if $\sigma([2]) = [0]$ then $\sigma$ will not be an isomorphism, as it will not be surjevtive.
    Thus, $\sigma([2]) = [2]$. 
    Now, we can check that $\sigma = id$ and $\sigma = (1 3)$ are both automorphism.
    Firstly, $id$ is trivially an isomorphism, and for $(1 3)$, we can see that $\ker(1 3) = \set{0}$ and $\im(1 3) = 
    \set{1, 2, 3, 4}$.

    Since there are only $2$ automorphisms of $\ZMod[4]$, 
    then $\Aut(\ZMod[4]) \isom \ZMod[2]$ by the uniqueness of group of order $2$. 

    For $D_8$, consider an automorphism $\sigma$ on $D_8$, we have that $\sigma(r) = r$ or $\sigma(r) = r^3$ since an 
    automorphism must preserve the order of the element in the group. Moreover, for we have that $\sigma(f) \ne r^2$
    since if that is the case, then $\sigma(fr) = \sigma(f)\sigma(r) = r^2r^3$ is an element of order $4$ while $fr$ is 
    of order 2.

    Now, for $\sigma(r) = r'$ and $\sigma(f) = f'$, we have $\sigma(f^ir^j) = \sigma(f^i)\sigma(r^j) = f'^ir'^j$ to 
    preserve the homomorphism. Since, $r' = r$ or $r' = r^{-1}$ and $f' = fr^i$ for some $i$, we have $f'^ir'^j$ 
    uniquely represent an element of $D_8$. Because 
    \begin{align*}
      f^ir^j &= f^ir^j \\ 
      f^ir^j &= f^i(r^3)^{-j} \\ 
      f^ir^j &= (fr)^i r^{j-i} \\ 
      f^ir^j &= (fr)^i (r^3)^{i-j} \\ 
      f^ir^j &= (fr^2)^i r^{j-2i} \\ 
      f^ir^j &= (fr^2)^i (r^3)^{2i+j} \\ 
      f^ir^j &= (fr^3)^i (r^3)^{j-3i} \\ 
      f^ir^j &= (fr^3)^i (r^3)^{3i-j} 
    \end{align*}
    
    Therefore, we have show that there is exactly 8 automorphisms on $D_8$. We will then show that the automorphisms 
    mentioned formed a group under composition, $\circ$. 

    We define $\sigma_{r', f'}$ to be the automorphism that maps $r$ to $r'$ and $f$ to $f'$. Then, the composition 
    of the automorphsim follows that following table.  

    Note that in this table, we have $\sigma_a \times \sigma_b = \sigma_b \circ \sigma_a$

    \begin{tabular}{|c|c c c c c c c c|}
      \hline
      $\times $&$ \sigma_{r, f} $&$\sigma_{r, fr} $&$\sigma_{r,fr^2}$&$\sigma_{r,fr^3}$&$\sigma_{r^3, f}$&$\sigma_{r^3, fr}$&$\sigma_{r^3, fr^2}$&$\sigma_{r^3, fr^3} $\\ 
      \hline
      $\sigma_{r, f}    $&$ \sigma_{r, f} $&$\sigma_{r, fr}$&$\sigma_{r,fr^2}$&$\sigma_{r,fr^3}   $&$\sigma_{r^3, f}$&$\sigma_{r^3, fr}$&$\sigma_{r^3, fr^2}$&$\sigma_{r^3, fr^3}$\\ 
      $\sigma_{r, fr}   $&$\sigma_{r, fr}$&$\sigma_{r,fr^2}$&$\sigma_{r,fr^3}$&$ \sigma_{r, f}    $&$\sigma_{r^3, fr^3}$&$\sigma_{r^3, f}$&$\sigma_{r^3, fr}$&$\sigma_{r^3, fr^2}$\\     
      $\sigma_{r,fr^2}  $&$\sigma_{r,fr^2}$&$\sigma_{r,fr^3}$&$ \sigma_{r, f} $&$\sigma_{r, fr}   $&$\sigma_{r^3, fr^2}$&$\sigma_{r^3, fr^3}$&$\sigma_{r^3, f}$&$\sigma_{r^3, fr}$\\ 
      $\sigma_{r,fr^3}  $&$\sigma_{r,fr^3}$&$ \sigma_{r, f} $&$\sigma_{r, fr}$&$\sigma_{r,fr^2}   $&$\sigma_{r^3, fr}$&$\sigma_{r^3, fr^2}$&$\sigma_{r^3, fr^3}$&$\sigma_{r^3, f}$\\ 
      $\sigma_{r^3, f}    $&$\sigma_{r^3, f}$&$\sigma_{r^3, fr}$&$\sigma_{r^3, fr^2}$&$\sigma_{r^3, fr^3}$&$ \sigma_{r, f} $&$\sigma_{r, fr}$&$\sigma_{r,fr^2}$&$\sigma_{r,fr^3}   $\\
      $\sigma_{r^3, fr}   $&$\sigma_{r^3, fr}$&$\sigma_{r^3, fr^2}$&$\sigma_{r^3, fr^3}$&$\sigma_{r^3, f}$&$\sigma_{r,fr^3}$&$ \sigma_{r, f} $&$\sigma_{r, fr}$&$\sigma_{r,fr^2}   $\\ 
      $\sigma_{r^3, fr^2} $&$\sigma_{r^3, fr^2}$&$\sigma_{r^3, fr^3}$&$\sigma_{r^3, f}$&$\sigma_{r^3, fr}$&$\sigma_{r,fr^2}$&$\sigma_{r,fr^3}$&$ \sigma_{r, f} $&$\sigma_{r, fr}   $\\
      $\sigma_{r^3, fr^3} $&$\sigma_{r^3, fr^3}$&$\sigma_{r^3, f}$&$\sigma_{r^3, fr}$&$\sigma_{r^3, fr^2}$&$\sigma_{r, fr}$&$\sigma_{r,fr^2}$&$\sigma_{r,fr^3}$&$ \sigma_{r, f}    $\\ 
      \hline
    \end{tabular}

    Which is identical to the table of $D_8$ as in here.

    \begin{tabular}{|c|c c c c c c c c|}
      \hline
      $\times $&$ 1 $&$ r  $&$ r^2 $&$ r^3 $&$ f $&$ fr $&$ fr^2 $&$ fr^3  $\\ 
      \hline
      $1    $&$ 1 $&$ r $&$ r^2 $&$ r^3    $&$ f $&$ fr $&$ fr^2 $&$ fr^3 $\\ 
      $r    $&$ r $&$ r^2 $&$ r^3 $&$ 1    $&$ fr^3 $&$ f $&$ fr $&$ fr^2 $\\     
      $r^2  $&$ r^2 $&$ r^3 $&$ 1 $&$ r    $&$ fr^2 $&$ fr^3 $&$ f $&$ fr $\\ 
      $r^3  $&$ r^3 $&$ 1 $&$ r $&$ r^2    $&$ fr $&$ fr^2 $&$ fr^3 $&$ f $\\ 
      $f    $&$ f $&$ fr $&$ fr^2 $&$ fr^3 $&$ 1 $&$ r $&$ r^2 $&$ r^3    $\\
      $fr   $&$ fr $&$ fr^2 $&$ fr^3 $&$ f $&$ r^3 $&$ 1 $&$ r $&$ r^2    $\\ 
      $fr^2 $&$ fr^2 $&$ fr^3 $&$ f $&$ fr $&$ r^2 $&$ r^3 $&$ 1 $&$ r    $\\
      $fr^3 $&$ fr^3 $&$ f $&$ fr $&$ fr^2 $&$ r $&$ r^2 $&$ r^3 $&$ 1    $\\ 
      \hline
    \end{tabular}
  }

  \qs{}{
    Determine that inner automorphism groups $\Inn(\Int)$ and $\Inn(\ZMod[n])$
  }
  \sol{
    Since $\Int$ and $\ZMod[n]$ are both cyclic, then they are abelian. 
    Therefore, $Z(\Int) = \Int$ and $ Z(\ZMod[n]) = \ZMod[n]$. Thus, $\Int/Z(\Int) \isom \Inn(\Int)$
    and $\ZMod[n] \isom \Inn(\ZMod[n])$
    Which is $\Inn(\Int) \isom \set{e}$ and $\Inn(\ZMod[n]) \isom \set{e}$
  } 

  \qs{}{
    Let $H$ be a subgroup of $G$. Show that the centralizer $C_G(H)$ of $H$ in $G$ is a normal subgroup
    of $N_G(H)$. Show also that the homomorphism $c:G \to \Aut(G)$ given by conjugation induces an
    injective homomorphism $N_G(H)/C_G(H) \to \Aut(H)$.
  }
  \sol{
    Since $C_G(H) = \set{g \in G \mid gh = hg \; \forall h \in H}$, and $N_G(H) = \set{g \in G \mid gH = Hg}$. 
    Now, consider $g \in C_G(H)$, $\forall h \in H, \; gh = hg$, which means that $gH = Hg$, thus, 
    $C_G(H) \subseteq N_G(H)$.

    Moreover, we know that $C_G(H)$ is a subgroup of $N_G(H)$ since $C_G(H)$ is a subgroup of $G$. 
    Now, for any element $n \in N_G(H)$, we have 
    \[ nC_G(H)n^{-1} = n\set{ghg^{-1} \mid \forall h \in H, \forall g \in G}n^{-1} = \set{ngh(ng)^{-1} \mid \forall 
    h \in H, \forall g \in G} = C_G(H) \] since $ng \in G$ by the closure.
    Therefore, $C_G(H) \normSg N_G(H)$
  
    Consider the homomorphism $c$ as $c : g \mapsto \sigma_g$ for which $\sigma_g$ is an automorphism in $G$ such that 
    $\sigma_g: h \mapsto ghg^{-1}$. Now, by restricting the domain of $c$ to $N_G(H)$, we have the homomorphism
    $c: N_G(H) \to \Aut(H)$ since $c(h) = \sigma_h$ and $\sigma_h(x) = hxh^{-1} \in H$ for all $x \in H$ since $h$ is 
    an element of the normalizer of $H$. 

    Now, consider $\ker c = \set{h \mid c(h) = \sigma_h = id}$. Now, if $\sigma_h(x) = hxh^{-1} = x$ for all $x \in H$,
    then $h$ must be an element of the centralizer of $H$ by definition (as $hx = xh$ for all $x$). So, the kernel of 
    the homomorphism $c$ is $\set{h \mid \sigma_h = id} = C_G(H)$.

    Thus, we have another homomorphism $c': N_G(H)/C_G(H) \to \Aut(H)$ induced by the homomorphism $c$, 
    given by $c': hC_G(H) \mapsto c(h)$. 
    where the kernel $\ker c' = \set{C_G(H)}$. So, $c'$ is an injective homomorphism.
  } 

  \qs{}{
    Let $P$ be a subgroup of $S_p$ of order $p$, where $p$ denotes a prime integer. Prove that 
    $\abs{N_{S_p}(P)} = p(p-1)$ and $N_{S_p}(P)/C_{S_p}(P) \isom \Aut(P)$
  }
  \sol{
    Consider the group $S_p$ and the corresponding subgroup $P$ of order $p$. Then, there is $(p-1)!$ cycles of length
    $p$ since each cycle can be written in the form of $(1 \; \sigma(1) \; \sigma(2) \; \sigma(p-1))$. Since there is 
    $(p-1)!$ possible cycle that can be written in that form, as it is a permutation of $p-1$ objects. 
    Moreover, each subgroup of order $p$ corresponds to $p-1$ cycles of length $p$. This is due the fact that 
    $\sigma, \sigma^2, \ldots, \sigma^p-1$ generates the same group as $\gcd(k, p) = 1 \forall 1 < k < p$. 

    Now, note that the largest power of $p$ that divides $p!$ is $p^1 = p$, so the order of sylow $p$ subgroup is $p$.
    Consider that there is $(p-2)!$ subgroup of order $p$, we know that the conjugation of a sylow $p$ subgroup is 
    another sylow $p$ subgroup, thus, the size of orbit of $P$ by conjugation on $G$ is $(p-2)!$. This means that the 
    size of the stabilizer of $P$, which is the normalizer of $P$ is $p/(p-2)! = p(p-1)$.

    Next, we use similar argument from the last problem to create $c: N_{S_p}(P) \to \Aut(P)$ in the same way. We have 
    that for any $\sigma \in \Aut(P)$, $\sigma$ is uniquely identify by a single point as $P$ is cyclic. 
    If we define $\sigma_g$ as $\sigma_g : h \mapsto ghg^{-1}$, then $\sigma_g$ are all the possible automorphism of $P$. 
    This is because the orbit of a conjugacy action of $N_G(P)$ on $P$ defined as $h \mapsto ghg^{-1}$ must be of 
    size $p$ for $h$ of order $p$. 

    This means that the homomorphism $c$ is an epimorphism, and thus, by the proof provided in the last problem, the 
    induced homomorphism $c': N_{S_p}(P)/C_{S_p}(P) \to \Aut(P)$ is an injective homomorphism, but since $c$ is 
    surjective, then it is an isomorphism.
  }

  \qs{}{
    Let $H$ be a subgroup of $G$ and let $H$ act on $G/H$ by translation. Find the orbits, stabilizers, and 
    fixed points of the action.
  }
  \sol{
    Let $H$ acts on $G/H$, then consider $G/H = \set{H, g_1H, g_2H, \cdots}$, for some $g_1, \; g_2, \; \cdots$.

    The orbits of the group action is the set $\set{hgH \forall h \in H, g \in G}$

    The stabilizers of the group action is the set $\set{h \in H \mid hgH = gH}$, which is the set 
    $\set{h \in H \mid g^{-1}hg \in H}$ 

    The fixed points of the action is the point $gH$ for which $hgH = gH$. 
    Thus the set of fixed point is the set 
    \[ \set{gH \in G/H \mid hgH = gH} = \set{gH \in G/H \mid g^{-1}hg \in H} \]
  } 

  \qs{}{
    Let $G$ act on a set $X$. Show that if $x, x' \in X$ satisfy $gx = x'$ for some $g \in G$ then 
    $G_{x'} = gG_xg^{-1}$
  }
  \sol{
    Let $g$ be the element such that $gx = x'$, then we show that if $g' \in G_x'$ then we know that $g'x' = x' = gx$.
    Therefore, $g'gx = gx$, thus, $g^{-1}g'g x = x$, so $g' \in g^{-1}G_x{g^{-1}}^{-1}$. 

    On the other hand, if $g' \in g^{-1}G_x{g^{-1}}^{-1}$, we have that $g^{-1}g'gx = x$, so $g'gx = gx$, 
    thus $g'x' = x'$. Therefore, we have $g' \in G_{x'}$

    So, $G_{x'} = g^{-1}G_x{g^{-1}}^{-1}$
  } 

  \qs{}{
    Let $H < G$. Show that if the center of $G$ contains $H$ and the quotient group $G/H$ is cylic, then $G$ is abelian.
  }
  \sol{
    Firstly, if the center of $G$ contains $H$, then $H$ must be abelian, and thus $H$ is a normal subgroup of $G$.
    If $G/H$ is cylic, then let $G/H$ be generated by $gH$, which means that $G =\set{g, g^2, \cdots}H$. 
    Now for any $a, b \in G$, we have that $a = g^nh$ and $b = g^mh'$ for some integer $n, m$ and some $h, h' \in H$. 
    which means that \[ ab = g^nhg^mh' = g^ng^mhh' = g^mg^nh'h = g^mh'g^nh = ba \] Thus, $G$ is abelian.
  } 

  \qs{}{
    Let $G$ be a $p$-group. Show that $G$ has a normal subgroup of order $p$.
  }
  \sol{
    Since there is only 1 element of order 1, which is the unique identity element in $G$ of order $p^n$ for some $n$. 
    Moreover, there is no element of order $k$ for $1 < k < p$ since $k \not\vert p$. Therefore, there must be an element
    of order $p$ in the group $G$, since there is at least 2 elements of $G$.

    Then let that element be $g$. We can generate a subgroup $\ang{g}$ which has order $p$ since it contains exactly $p$
    elements of the form $g^1, \; g^2, \; \ldots,\; g^p$. Now, since $\ang{g}$ is cyclic, then $\ang{g}$ is abelian. 
    Therefore, $\ang{g}$ is a normal subgroup of $G$ for which $\abs{\ang{g}} = p$.
  }

  \qs{}{
    Let $H$ be a subgroup of a finite group $G$. Prove that if $G = \bigcup_{x \in G} xHx^{-1}$ then $H = G$. 
  }
  \sol{
    Assume that $H \ne G$ to prove using the contraposition. 

    Firstly, let $N_G(H)$ be the normalizer of $H$ in $G$, which is $N_G(H) = \set{ g \mid gHg^{-1} = H}$.
    So we get that $\set{gN_G(H)}$ totally partition $G$. Thus, $G$ can be uniquely represented as $gn$ for 
    $g \in G/N_G(H)$ and $n \in N_G(H)$. Now, for $gn \in G$, the conjugated subgroup 
    $gnH(gn)^{-1} = gnHn^{-1}g^{-1} = gHg^{-1}$. Therefore, there is at most $\abs{G/N_G(H)} = [G:N_G(H)]$ conjugated
    subgroup of $H$.

    Now, since there is $\abs{H}$ elements in each of $\set{gHg^{-1}}$, therefore, the union 
    \[ \abs{ \bigcup_{g \in G} gHg^{-1} } < \sum_{g \in G} \abs{gHg^{-1}} \le \abs{H}[G:N_G(H)] = G \]
    Since the identity element is contained in every conjugated subgroup $gHg^{-1}$. 
    From this, it is acheived that $\bigcup_{g \in G} gHg^{-1} \ne G$.

    Hence, the statement is proved by the contraposition.
  } 

  \qs{}{
    Let $H$ be a nontrivial normal subgroup of a $p$-group $G$. Show that $H \cap Z(G)$ is nontrivial.
  }
  \sol{
    Let consider an action $\cdot$ of a group $G$ that acts on $H$ by conjugation. Then the action is well define since 
    $g \cdot h = ghg^{-1} \in H$ as $H$ is a normal subgroup of $G$. Then, the orbit of $h$ is 
    $Gh = \set{ghg^{-1} \mid g \in G}$ so any orbit of size one $Gh = \set{h} = \set{ghg^{-1}}$ implies that 
    $gh = hg$, so it is in the center of $G$ by definition. 

    Now, by the orbit stabilizer theorem, we get that $\abs{Gh} = [G:G_h]$. So, $\abs{Gh}$ must divides $p^n$ as 
    $[G:G_h] = \abs{G}/\abs{G_h}$ divides $p^n$.
    There is at least one orbit of size one, which is $\abs{Ge} = \abs{\set{e}} = 1$, 
    Consider if there is no other orbit of size $1$, then $\abs{H} = p^m = 1 + pk$ for some $m,k$, as
    the orbits of size greater than one must have the size that divides $p^n$, which must be a multiple of $p$. 
    This means that there must be at least $p$
    elements of $H$ that its orbit is of size one, which is the element of $H$ that is in the center of $G$.

    Since $p > 1$, $H \cap Z(G)$ is nontrivial as $\abs{H \cap Z(G)} > p > 1$
  } 

\end{document}
