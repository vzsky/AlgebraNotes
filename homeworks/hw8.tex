% chktex-file 44
% chktex-file 8

\documentclass{report}
\usepackage{amsthm}
\usepackage{amsmath}
\usepackage{amssymb}
\usepackage{amssymb}
\usepackage{amsfonts}
\usepackage{xcolor}
\usepackage{tikz}
\usepackage{fancyhdr}
\usepackage{enumerate}
\usepackage{graphicx}
\usepackage[normalem]{ulem}
\usepackage[most,many,breakable]{tcolorbox}
\usepackage[a4paper, top=80pt, foot=25pt, bottom=50pt, left=0.5in, right=0.5in]{geometry}
\usepackage{hyperref, theoremref}
\hypersetup{
	pdftitle={Assignment},
	colorlinks=true, linkcolor=b!90,
	bookmarksnumbered=true,
	bookmarksopen=true
}
\usepackage{nameref}
\usepackage{parskip}
\pagestyle{fancy}

\usepackage[explicit,compact]{titlesec}
\titleformat{\chapter}[block]{\bfseries\huge}{\thechapter. }{\compact}{#1}
        

%%%%%%%%%%%%%%%%%%%%%
%% Defining colors %%
%%%%%%%%%%%%%%%%%%%%%

\definecolor{lr}{RGB}{188, 75, 81}
\definecolor{r}{RGB}{249, 65, 68}
\definecolor{dr}{RGB}{174, 32, 18}
\definecolor{lo}{RGB}{255, 172, 129}
\definecolor{do}{RGB}{202, 103, 2}
\definecolor{o}{RGB}{238, 155, 0}
\definecolor{ly}{RGB}{255, 241, 133}
\definecolor{y}{RGB}{255, 229, 31}
\definecolor{dy}{RGB}{143, 126, 0}
\definecolor{lb}{RGB}{148, 210, 189}
\definecolor{bg}{RGB}{10, 147, 150}
\definecolor{b}{RGB}{39, 125, 161}
\definecolor{db}{RGB}{0, 95, 115}
\definecolor{p}{RGB}{229, 152, 155}
\definecolor{dp}{RGB}{181, 101, 118}
\definecolor{pp}{RGB}{142, 143, 184}
\definecolor{v}{RGB}{109, 89, 122}
\definecolor{lg}{RGB}{144, 190, 109}
\definecolor{g}{RGB}{64, 145, 108}
\definecolor{dg}{RGB}{45, 106, 79}

\colorlet{mysol}{g}
\colorlet{mythm}{lr}
\colorlet{myqst}{db}
\colorlet{myclm}{lb}
\colorlet{mywrong}{r}
\colorlet{mylem}{o}
\colorlet{mydef}{lg}
\colorlet{mycor}{lb}
\colorlet{myrem}{dr}

%%%%%%%%%%%%%%%%%%%%%

\newcommand{\col}[2]{
  \color{#1}#2\color{black}\,
}

\newcommand{\TODO}[1][5cm]{
  \color{red}TODO\color{black}
  \vspace{#1}
}

\newcommand{\wans}[1]{
	\noindent\color{mywrong}\textbf{Wrong answer: }\color{black}
	#1 


}

\newcommand{\wreason}[1]{
	\noindent\color{mywrong}\textbf{Reason: }\color{black}
	#1 

  
}

\newcommand{\sol}[1]{
	\noindent\color{mysol}\textbf{Solution: }\color{black}
	#1


}

\newcommand{\nt}[1]{
  \begin{note}Note: #1\end{note}
}

\newcommand{\ky}[1]{
  \begin{key}#1\end{key}
}

\newcommand{\pf}[1]{
  \begin{myproof}#1\end{myproof}
}

\newcommand{\qs}[3][]{
  \begin{question}{#2}{#1}#3\end{question}
}

\newcommand{\df}[3][]{
  \begin{definition}{#2}{#1}#3\end{definition}
}

\newcommand{\thm}[3][]{
  \begin{theorem}{#2}{#1}#3\end{theorem}
}

\newcommand{\clm}[3][]{
  \begin{claim}{#2}{#1}#3\end{claim} 
}

\newcommand{\lem}[3][]{
  \begin{lemma}{#2}{#1}#3\end{lemma}
}

\newcommand{\cor}[3][]{
  \begin{corollary}{#2}{#1}#3\end{corollary}
}

\newcommand{\rem}[3][]{
  \begin{remark}{#2}{#1}#3\end{remark}
}

\newcommand{\twoways}[2]{
  \leavevmode\\
  ($\Longrightarrow$): 
  \begin{shift}#1\end{shift}
  ($\Longleftarrow$):
  \begin{shift}#2\end{shift} 
}

\newcommand{\nways}[2]{
  \leavevmode\\
  ($#1$): 
  \begin{shift}#2\end{shift}
}

%%%%%%%%%%%%%%%%%%%%%%%%%%%%%% ENVRN

\newenvironment{myproof}[1][\proofname]{%
	\proof[\bfseries #1: ]
}{\endproof}

\tcbuselibrary{theorems,skins,hooks}
\newtcolorbox{shift}
{%
  before upper={\setlength{\parskip}{5pt}},
  blanker,
	breakable,
	width=0.95\textwidth,
  enlarge left by=0.03\textwidth,
}

\tcbuselibrary{theorems,skins,hooks}
\newtcolorbox{key}
{%
	breakable,
	width=0.95\textwidth,
  enlarge left by=0.03\textwidth,
}

\tcbuselibrary{theorems,skins,hooks}
\newtcolorbox{note}
{%
	enhanced,
	breakable,
	colback = white,
	width=\textwidth,
	frame hidden,
	borderline west = {2pt}{0pt}{black},
	sharp corners,
}

\tcbuselibrary{theorems,skins,hooks}
\newtcbtheorem[]{remark}{Remark}
{%
	enhanced,
	breakable,
	colback = white,
	frame hidden,
	boxrule = 0sp,
	borderline west = {2pt}{0pt}{myrem},
	sharp corners,
	detach title,
  before upper={\setlength{\parskip}{5pt}\tcbtitle\par\smallskip},
	coltitle = myrem,
	fonttitle = \bfseries\sffamily,
	description font = \mdseries,
	separator sign none,
	segmentation style={solid, myrem},
}{rem}

\tcbuselibrary{theorems,skins,hooks}
\newtcbtheorem[number within=section]{lemma}{Lemma}
{%
	enhanced,
	breakable,
	colback = white,
	frame hidden,
	boxrule = 0sp,
	borderline west = {2pt}{0pt}{mylem},
	sharp corners,
	detach title,
  before upper={\setlength{\parskip}{5pt}\tcbtitle\par\smallskip},
	coltitle = mylem,
	fonttitle = \bfseries\sffamily,
	description font = \mdseries,
	separator sign none,
	segmentation style={solid, mylem},
}{lem}

\tcbuselibrary{theorems,skins,hooks}
\newtcbtheorem{claim}{Claim}
{%
  parbox=false,
	enhanced,
	breakable,
	colback = white,
	frame hidden,
	boxrule = 0sp,
	borderline west = {2pt}{0pt}{myclm},
	sharp corners,
	detach title,
  before upper={\setlength{\parskip}{5pt}\tcbtitle\par\smallskip},
	coltitle = myclm,
	fonttitle = \bfseries\sffamily,
	description font = \mdseries,
	separator sign none,
	segmentation style={solid, myclm},
}{clm}

\makeatletter
\newtcbtheorem[number within=section, use counter from=lemma]{theorem}{Theorem}{enhanced,
	breakable,
	colback=white,
	colframe=mythm,
	attach boxed title to top left={yshift*=-\tcboxedtitleheight},
	fonttitle=\bfseries,
	title={#2},
	boxed title size=title,
	boxed title style={%
			sharp corners,
			rounded corners=northwest,
			colback=mythm,
			boxrule=0pt,
		},
	underlay boxed title={%
			\path[fill=mythm] (title.south west)--(title.south east)
			to[out=0, in=180] ([xshift=5mm]title.east)--
			(title.center-|frame.east)
			[rounded corners=\kvtcb@arc] |-
			(frame.north) -| cycle;
		},
	#1
}{thm}
\makeatother

\makeatletter
\newtcbtheorem{question}{Question}{enhanced,
	breakable,
	colback=white,
	colframe=myqst,
	attach boxed title to top left={yshift*=-\tcboxedtitleheight},
	fonttitle=\bfseries,
	title={#2},
	boxed title size=title,
	boxed title style={%
			sharp corners,
			rounded corners=northwest,
			colback=myqst,
			boxrule=0pt,
		},
	underlay boxed title={%
			\path[fill=myqst] (title.south west)--(title.south east)
			to[out=0, in=180] ([xshift=5mm]title.east)--
			(title.center-|frame.east)
			[rounded corners=\kvtcb@arc] |-
			(frame.north) -| cycle;
		},
	#1
}{qs}
\makeatother

\makeatletter
\newtcbtheorem[number within=section]{definition}{Definition}{enhanced,
	breakable,
	colback=white,
	colframe=mydef,
	attach boxed title to top left={yshift*=-\tcboxedtitleheight},
	fonttitle=\bfseries,
	title={#2},
	boxed title size=title,
	boxed title style={%
			sharp corners,
			rounded corners=northwest,
			colback=mydef,
			boxrule=0pt,
		},
	underlay boxed title={%
			\path[fill=mydef] (title.south west)--(title.south east)
			to[out=0, in=180] ([xshift=5mm]title.east)--
			(title.center-|frame.east)
			[rounded corners=\kvtcb@arc] |-
			(frame.north) -| cycle;
		},
	#1
}{def}
\makeatother

\makeatletter
\newtcbtheorem[number within=section, use counter from=lemma]{corollary}{Corollary}{enhanced,
	breakable,
	colback=white,
	colframe=mycor,
	attach boxed title to top left={yshift*=-\tcboxedtitleheight},
	fonttitle=\bfseries,
	title={#2},
	boxed title size=title,
	boxed title style={%
			sharp corners,
			rounded corners=northwest,
			colback=mycor,
			boxrule=0pt,
		},
	underlay boxed title={%
			\path[fill=mycor] (title.south west)--(title.south east)
			to[out=0, in=180] ([xshift=5mm]title.east)--
			(title.center-|frame.east)
			[rounded corners=\kvtcb@arc] |-
			(frame.north) -| cycle;
		},
	#1
}{cor}
\makeatother

% Basic
  \DeclareMathOperator{\lcm}{lcm}
  \newcommand{\Real}{\mathbb{R}}
  \newcommand{\Comp}{\mathbb{C}}
  \newcommand{\Nat}{\mathbb{N}}
  \newcommand{\Rat}{\mathbb{Q}}
  \newcommand{\Int}{\mathbb{Z}}
  \newcommand{\set}[1]{\left\{\, #1 \,\right\}}
  \newcommand{\paren}[1]{\left( \; #1 \; \right)}
  \newcommand{\abs}[1]{\left\lvert #1 \right\rvert}
  \newcommand{\ang}[1]{\left\langle #1 \right\rangle}
  \renewcommand{\to}[1][]{\xrightarrow{\text{#1}}}
  \newcommand{\tol}[1][]{\to{$#1$}}
  \newcommand{\curle}{\preccurlyeq}
  \newcommand{\curge}{\succcurlyeq}
  \newcommand{\mapsfrom}{\leftarrow\!\shortmid}

  \newcommand{\mat}[1]{\begin{bmatrix} #1 \end{bmatrix}}
  \newcommand{\pmat}[1]{\begin{pmatrix} #1 \end{pmatrix}}
  \newcommand{\eqs}[1]{\begin{align*} #1 \end{align*}}
  \newcommand{\case}[1]{\begin{cases} #1 \end{cases}}
  

  % Algebra
  \newcommand{\normSg}[0]{\vartriangleleft}
  \newcommand{\ZMod}[1][n]{\mathbb{Z}/#1\mathbb{Z}}
  \newcommand{\isom}{\simeq}
  \newcommand{\mapHom}{\xrightarrow{\text{hom}}}
  \DeclareMathOperator{\Inn}{Inn}
  \DeclareMathOperator{\Aut}{Aut}
  \DeclareMathOperator{\im}{im}
  \DeclareMathOperator{\ord}{ord}
  \DeclareMathOperator{\Gal}{Gal}
  \DeclareMathOperator{\chr}{char}
  \newcommand{\surjto}{\twoheadrightarrow}
  \newcommand{\injto}{\hookrightarrow}

  % Analysis 
  \newcommand{\limty}[1][k]{\lim_{#1\to\infty}}
  \newcommand{\norm}[1]{\left\lVert#1\right\rVert}
  \newcommand{\darrow}{\rightrightarrows}


\fancyhead[L]{HW 8 - Modern Algebra MAS311}
\fancyhead[R]{\textbf{Touch Sungkawichai} 20210821}

\begin{document}

  \qs{}{
    Let $[G,G]$ be the subgroup of $G$ generated by all commutators. 
    Prove that $[G,G]$ is normal in $G$ and $G/[G,G]$ is abelian.
  }
  \sol{
    By definition, $[G,G]$ is the subgroup $\ang{\set{xyx^{-1}y^{-1} \mid x,y\in G}}$. Consider any $g \in G$, 
    and any element $h = [x_1,y_1]\cdots[x_k,y_k]$ in $[G,G]$, it follows that, 
    \begin{align*}
      ghg^{-1} = g ([x_1,y_1]\cdots[x_k,y_k]) g^{-1} &= g[x_1,y_1]g^{-1} \; g[x_2,y_2]g^{-1} \cdots g[x_k,y_k]g^{-1} \\
                                          &= gx_1y_1x_1^{-1}y_1^{-1}g^{-1} \cdots gx_ky_kx_k^{-1}y_k^{-1}g^{-1} \\ 
                                          &= gx_1g^{-1}gy_1g^{-1}gx_1^{-1}g^{-1}gy_1^{-1}g^{-1} \cdots 
                                          gx_kg^{-1}gy_kg^{-1}gx_k^{-1}g^{-1}gy_k^{-1}g^{-1} \\
                                          &= [gx_1g^{-1}, gy_1g^{-1}] \cdots [gx_kg^{-1}, gy_kg^{-1}] \\
    \end{align*}
    But as $gx_ig^{-1}$ and $gy_ig^{-1}$ is a member of $G$ for any $i$ by closure, it follows that for any $g \in G$, 
    $ghg^{-1}$ is a member of $[G,G]$ by definition. Thus, $g[G,G]g^{-1} \subseteq [G,G]$, but consider $ghg^{-1} = e$ if and 
    only if $h = e$, thus $[G,G] \subseteq g[G,G]g^{-1}$, which means that $[G,G]$ is a normal subgroup of $G$.

    Next, consider the quotient set $G/[G,G] = \set{g[G,G] \mid g \in G}$, and let $g[G,G]$ and $h[G,G]$ be the element of 
    $G/[G,G]$. Consider that $g[G,G] \cdot h[G,G] = gh[G,G]$. 
    As $h^{-1}, g^{-1} \in G$, then $[h^{-1}, g^{-1}] \in [G,G]$, it follows that 
    $gh[h^{-1}, g^{-1}] \in gh[G,G]$ but \[ gh[h^{-1}, g^{-1}] = ghh^{-1}g^{-1}hg = hg \]
    Moreover, since $hg \in hg[G,G]$ and the coset are mutually exclusive, it follows that $gh[G,G] = hg[G,G]$. 
    Thus, $G/[G,G]$ is abelian.
  }

  \qs{}{
    Determine all conjugacy classes in $S_5$.
  } 
  \sol{
    \clm[conjcyc]{
      Two element $\sigma$ and $\sigma'$ of $S_n$ are in the same conjugacy class if they share the same cycle structure.
      \pf{
        Let $\sigma = (a_{11}\cdots a_{1n_1}) \cdots (a_{r1}\cdots a_{rn_r})$ be an element of $S_n$ 
        where each cycle $(a_{i1} \cdots a_{in_i})$ is pairwise disjoint to the other cycle. 
        Then for any $\tau \in S_n$, 
        \[ \tau \sigma \tau^{-1} 
          = \tau (a_{11}\cdots a_{1n_1})\cdots(a_{r1} \cdots a_{rn_r}) \tau^{-1}
          = \tau (a_{11} \cdots a_{1n_1}) \tau^{-1} \cdots \tau (a_{r1} \cdots a_{rn_r}) \tau^{-1} \]
        Which maps 
        $\tau(a_{ij}) \to[$\tau^{-1}$] a_{ij} \to[$\sigma$] a_{i(j+1)} \to[$\tau$] \tau(a_{i(j+1)})$ for each $i$. 
        Thus, \[ \tau\sigma\tau^{-1} = (\tau(a_{11}) \cdots \tau(a_{1n_1}))\cdots(\tau(a_{r1})\cdots \tau(a_{rn_r})) \]

        Moreover, it is possible to choose $\tau$ so that the resulting $\tau\sigma\tau^{-1}$ is any cycle of that cycle type,
        as $\tau \in S_n$.
      }
    }
    
    From the claim, it follows that the conjugacy class of $S_5$ is determined by the cycle structure of each type. 
    The first class being the identity, with cycle (1)(2)(3)(4)(5). 
    Next, the class of all transposition. (ab)(c)(d)(e). The class of all three-cycle, (abc)(d)(e). 
    The class of all four-cycle, (abcd)(e). The class of all five-cycle (abcde). The class of all two two-cycles, (ab)(cd)(e).
    And lastly, the class of all two-and-three-cycles (ab)(cde). 

    Starting with the class of all transposition, there is $\frac{5!}{2!3!} = 10$ ways to choose 2 elements, thus, there is 10
    transpositions in the class. 

    For the class of three cycles, there is $\frac{5!}{3!2!} = 10$ ways to choose 3 elements, but the reversal gives another
    cycle, thus, totaling of 20 cycles in the class.

    For the class of four cycles, there is $5$ ways to choose 4 elements, but for each way, there is $3!$ permutation that gives 
    different cycles, thus, there is a total of 30 cycles in the class.

    For the class of five cycles, all elements can permutes in $4! = 24$ ways, each gives a different cycle, thus there is 24
    cycles in the class.

    Next, for the class of two two-cycles, there is $\frac{5!}{2!2!1!} = 30$ to choose the elements. 
    Since the each cycle is counted twice, the number of cycle in this class is $\frac{30}{2} = 15$ cycles. 

    Lastly, for the class of two-and-three cycles, there is $\frac{5!}{2!3!} = 10$ ways to form 2 and 3 groups, and each one 
    generate two different cycles, by the reversal of the three cycle. Thus, there is a total of 20 cycles in the group.

    To sum up, the class equation of $S_5$ is $5! = 120 = 1 + 10 + 20 + 30 + 24 + 15 + 20$, and the conjugacy class of $S_5$ is 
    as shown above.
  }

  \qs{}{
    Show that $A_n$ has trivial center for $n \ge 4$.
  }
  \sol{
    For $n = 4$, the center of $A_4$ is the set $\set{z \mid zgz^{-1} = g \forall g \in A_4}$. 
    Consider the conjugacy class of $A_4$. As the conjugation preserves the cycle structure by a proof similar to
    claim~\ref{clm:conjcyc}, the class of $A_4$ are the identity class, class of $(a b c)$, and the class of $(a b)(c d)$.
    
    The class of $(a b c)$ has $\frac{4!}{3!1!} \times 2$ cycles and the class of $(a b)(c d)$ has $\frac{4!}{2!2! \cdot 2}$
    cycles.

    One can verify the correctness of the conjugation class by considering the class equation 
    \[ \abs{A_4} = 12 = 1 + 4\times2 + 3 \]. 
    Thus, there is only one element in the center.
    Therefore, the center is trivial.
    So, $A_4$ has trivial center.

    For $n \ge 5$, the group $A_n$ is simple. Since the center of $A_n$ is normal to $A_n$, then $Z(A_n) = \set{e}$ or 
    $Z(A_n) = A_n$. However, $A_n$ is not abelian. Therefore, $Z(A_n) \ne A_n$. Hence, it follows that $Z(A_n) = \set{e}$.
  } 

  \qs{}{
    Show that $A_n$ is the only nontrivial normal subgroup of $S_n$ for $n \ge 5$.
  }
  \sol{
    Consider a non-trivial normal subgroup $N$ of $S_n$ and assume that $N \ne A_n$. 
    It follows that $N \cap A_n$ is a normal subgroup of $A_n$. However, as $A_n$ is simple, $N \cap A_n = \set{e}$ or $A_n$.

    If $N \cap A_n = A_n$, then $A_n < N < S_n$. But it is impossible since $\abs{A_n} = \frac{\abs{S_n}}{2}$. Which means that 
    there is no integer value for $\abs{N}$ that would obey lagrange's theorem, given that $N \ne S_n$ and $N \ne A_n$.

    If $N \cap A_n = \set{e}$, then by the second isomorphism theorem, $NA_n/A_n \isom N/A_n\cap N \isom N$.
    However, as $N \ne A_n$ with $N \not< A_n$ (as $A_n$ is simple), and $[S_n: A_n] = 2$, then $NA_n = S_n$. 
    So, $S_n/A_n \isom \ZMod[2] \isom N$

    Thus, $\abs{N} = 2$. Notice that a subgroup of $S_n$ with order 2 are a subgroup containing just one transposition (and 
    identity). And since for any transposition $(a b)$, there is $c$ pairwise distinct to $a$ and $b$ such that 
    \[ (a c)(a b)(a c)^{-1} = (a c)(b a c) = (b c) \ne (a b) \]
    Thus, such group $N$ with order 2 cannot be normal. 

    Therefore, $A_n$ is the only normal subgroup of $S_n$ for $n \ge 5$.
  } 

  \qs{}{
    Find all integers $n$ such that $A_n$ is perfect, ie. $A_n = [A_n, A_n]$.
  }
  \sol{
    Consider $A_2$ is a trivial subgroup, thus $A_2 = [A_2, A_2]$ trivially.

    For $A_3$, $\abs{A_3} = 3$, thus $A_3 \isom \ZMod[3]$. Therefore, $A_3$ is abelian, hence for any $x, y \in A_3$, the 
    commutator $[x,y] = xyx^{-1}y^{-1} = xx^{-1}yy^{-1} = e$. Making $[A_3, A_3] = \set{e} \ne A_3$

    For $A_4$, consider that a subgroup $N = \set{e, (12)(34), (13)(24), (23)(14)}$ is a subgroup of $A_4$. This can 
    be shown using a multiplication table, or consider that the inverse of each element is itself, and that the product of 
    each pair is, without loss of generality, $(a b)(c d) \times (a c)(b d) = (a d)(b c)$. Therefore, $N$ is a subgroup of $A_4$.
    Moreover, since conjugation preserves the cycle structure as shown in claim~\ref{clm:conjcyc} and that 
    there is only $\frac{4!}{2!2!2} = 3$ cycles of this type in $S_4$, the group $N$ is a normal 
    subgroup of $A_4$. Now, consider that $A_4/N$ is a group with order $12/4 = 3$, thus is a cyclic, which implies that 
    the group is abelian. Thus, the group $N$ must contains $[A_4, A_4]$ as it is the smallest group in which the quotient is 
    abelian. Thus, $[A_4, A_4] \ne A_4$. 

    For $n \ge 5$, it was shown in problem 1, $[A_n, A_n] \normSg A_n$. However, the only normal subgroup of $A_n$ 
    is the trivial subgroup $\set{e}$ and $A_n$ as $A_n$ is simple. Therefore, $[A_n, A_n] = A_n$ or $[A_n, A_n] = \set{e}$ 
    However, if $[A_n, A_n] = \set{e}$ then $A_n/[A_n, A_n]$ is isomorphic to $A_n$ which is not abelian, contradicts that 
    the quotient of the commutator group must always be abelian. Therefore, $A_n = [A_n, A_n]$.

    Therefore, $A_n$ is perfect for $n = 2$ and $n \ge 5$.
  } 

  \qs{}{
    Let $N = \set{e, (12)(34), (13)(24), (14)(23)}$ be a subgroup of $S_4$. Show that $N$ is a normal subgroup of $S_4$ 
    and the factor group $S_4/N$ is isomorphic to $S_3$. 
  }
  \sol{
    As shown in claim~\ref{clm:conjcyc}, conjugation preserves the cycle structure. Notice that there is only 
    $\frac{4!}{2!2!2} = 3$ possible permutations of this cycle type that are in $S_4$,
    which are all the element of $N$. Therefore, the conjugation $gNg^{-1} = N$ for any $g$. 
    This showed that $N$ is a normal subgroup of $S_4$.

    Now, notice that an element $g \in S_4$ is in one of the following form $e = id, (a b), (a b c), (a b c d), (a b)(c d)$
    For the element in the form of $(a b c)$, then it can be written as $(a b)(b c)$, $(b c)(c a)$, or equivalently, $(c a)(a b)$
    And if it is in the form $(a b c d)$, then it can be written as $(a b)(b c)(c d)$, $(b c)(c d)(d a)$, $(c d)(d a)(a b)$
    or equivalently, $(d a)(a b)(b c)$
    Therefore, an element of $S_4$ can be written as $\tau = \sigma (x 4)$ where $\sigma$ is a permutation that fixes $4$.

    As $N = \set{e, (12)(34), (13)(24), (23)(14)}$, then if 
    \begin{gather*}
      \tau = \sigma (1 4) \implies \tau N = \sigma' N \text{ where } \sigma' = \sigma (2 3) \\ 
      \tau = \sigma (2 4) \implies \tau N = \sigma' N \text{ where } \sigma' = \sigma (1 3) \\ 
      \tau = \sigma (3 4) \implies \tau N = \sigma' N \text{ where } \sigma' = \sigma (1 2) \\ 
      \text{otherwise }, \tau N = \sigma' N \text{ where } \sigma' = \tau
    \end{gather*}
    So there is an isomorphism from $S_4/N \to S_3$ via $\tau N \mapsto \sigma'$ defined above 
  } 

  \qs{}{
    Let $H$ be a non-trivial subgroup of a finite group $G$ such that $\abs{G}$ does not divide $[G:H]!$. 
    Show that $H$ contains a non-trivial normal subgroup of $G$. In particular, $G$ is not simple.
  }
  \sol{
    Let $x = [G:H]$.
    Consider an $G$ action on a set $X = G/H$ given by $g \cdot xH = gxH$. 
    Then $e \cdot xH = exH = xH$, and $g \cdot g' \cdot xH = gg'xH = (gg') \cdot xH$, thus the action is well defined.
    So, there must be a homomorphism from $\phi: G \to S_{x}$ corresponds to the action. Thus, by the first isomorphism theorem, 
    $G/\ker\phi \isom \im\phi$ and $\im\phi \vert S_x$, implying that $\abs{G} / \abs{\ker\phi} \vert x!$. 
    However, $\abs{G} \not\vert x!$ implying that $\abs{\ker\phi} \ne 1$. 
    Hence, $\ker\phi$ is a non-trivial subgroup of $G$.

    Now, consider that $\ker\phi$ is the set $\set{g \mid g \cdot xH = xH \forall xH \in G/H}$, which means that $g \in \ker\phi$
    must have the property that $gH = H$, since the coset partition the group $G$, then for $x \not\in H$, $xH \ne H$ as $xH$ 
    contain an element $x$ which is not in $H$. So, $\ker\phi \subseteq H$.
  }

  \qs{}{
    Let $p$ be a prime, Show that all groups of order $8p$ and $48$ are not simple.
  }
  \sol{
    \clm[ppowernotsimp]{
      A group of order $p^m$ is not simple, for $m > 1$. 
      \pf{
        Let $G$ be a group of order $p^m$, then if $G$ is abelian, then there exists an element of order $p$, therefore, 
        there is a subgroup of order $p$. However, the subgroup must be normal as $G$ is abelian, thus, $G$ is not simple.
        Otherwise, if $G$ is not abelian, then by the class equation, $\abs{G} = p^m = \abs{Z(G)} + \sum \abs{G_x}$ with 
        $\abs{G_x} \vert \abs{G}$ and $\abs{G_x} > 1$. This means that $\abs{Z(G)} > 1$ since otherwise, $p^m = 1 + p(k)$ for 
        some $k$. As $G$ is not abelian, $Z(G) \ne G$. Therefore, $Z(G)$ is a normal subgroup of $G$ making $G$ non-simple. 
      }
    }
    \clm[2m3notsimp]{
      A group of order $2^m \cdot 3$ is not simple. For $m \ge 2$. 
      \pf {
        For $\abs{G} = 2^m \cdot 3$, Let $P$ be a sylow 2-subgroup of $G$, then the number of subgroup is 
        $n_2$ where $n_2 \vert 3$ and $n_2 \equiv 1 \pmod2$. Thus, $n_2 = 1, 3$ are the only possible option. 
        If $n_2 = 3$, then consider a conjugation action of $G$ on the set $X$ of all sylow 2-subgroup. 
        Since $\abs{X} = 3$, the action corresponds to a homomorphism $\varphi: G \to S_3$. 
        But as $\abs{G} = 2^m \cdot 3 > 2 \cdot 3 = 6 = \abs{S_3}$, then $\abs{\ker\varphi} > 1$ and $\ker\varphi \ne G$ as the 
        action is not trivial, ie. there are 3 subgroups. Thus, $\ker\varphi$ is a normal subgroup of $G$, which 
        makes $G$ non-simple.
      }
    }

    For a group of order $48$, $\abs{G} = 48 = 2^4 3$. The group $G$ cannot be simple by claim~\ref{clm:2m3notsimp}

    For $p = 2$, $\abs{G} = 16 = 2^4$ is not simple by claim~\ref{clm:ppowernotsimp}

    For $p = 3$, $\abs{G} = 2^3 3$. The group $G$ cannot be simple by claim~\ref{clm:2m3notsimp}

    For $p = 7$, Let $P$ be a sylow 7-subgroup of a group $G$ with order $8p$, then, the number of subgroup is $n_p \vert 8$
    with $n_p \equiv 1 \pmod7$, which is either $n_p = 1$ or $n_p = 8$. If $n_p = 1$,, then $P$ is a normal suubgroup of $G$, 
    making $G$ simple, but if $n_p = 8$, then there must be $8 \cdot 6 = 48$ elements of order 7, leaving with $56 - 48 = 8$
    elements, which is enough for only one sylow 2-subgroup of $G$. As sylow 2-subgroup of $G$ is unique, then it is normal. 
    Hence, $G$ is not simple.

    For other $p$, Let $P$ be a sylow $p$-subgrouup of $G$ with order $8p$, then the number of subgroup is $n_p \vert 8$ 
    with $n_p \equiv 1 \pmod{p}$, thus $n_p = kp + 1$ for some $k$. Notice that $kp + 1 = 2$ is impossible, $kp + 1 = 4$ is 
    possible only when $p = 3$, $kp + 1 = 8$ is possible only for $p = 7$. Thus, for $p \ne 2$, $p \ne 3$ and $p \ne 7$, the 
    number of sylow $p$-subgroup is 1. Thus, the sylow subgroup is a normal subgroup of $G$ making $G$ non-simple.

  } 

  \qs{}{
    Let $p$ be a prime and let $m \ge 1$ be an integer. Prove that all group of order $2p^m$ and $4p^m$ are not simple.
  }
  \sol{
    If $p = 2$, then $2p^m = p^{m+1}$ and $4p^m = p^{m+2}$. Either way, the group must be non-simple by
    claim~\ref{clm:ppowernotsimp}. 

    Otherwise, for a group $G$ of order $2p^m$, consider the number of a sylow $p$-subgroup of $G$. Then, $n_p \vert 2$ and 
    $n_p \equiv 1 \pmod{p}$. Thus, $n_p = 1$ must hold. So, the sylow $p$ subgroup is a normal subgroup of $G$, which proves 
    that group $G$ cannot be simple.

    For a group $G$ of order $4p^m$, if $p = 3$ and $m = 1$, then the group of order $4\cdot3^1 = 2^2 3$ is not simple by 
    claim~\ref{clm:2m3notsimp}. 

    For the case of group $G$ with order $4p^m$ where $p = 3$ and $m > 1$, consider that the number of sylow 3-subgroup of $G$
    as $n_3$. From $n_3 \vert p$ and $n_3 \equiv 1 \pmod{p}$, $n_3 = 1$ or $4$ must hold. If $n_3 = 1$, then the sylow subgroup 
    is unique, thus normal, hence $G$ is non-simple. If $n_3 = 4$, then consider the conjugation action of $G$ on the set 
    of all sylow 3-subgroup of $G$. We have that $\abs{G} \ge 4 \times 3^2 = 36$. Thus, the corresponding homomorphism
    $\varphi: G \to S_4$ must have non-trivial kernel. This is because $\abs{S_4} = 24 < 36 \le \abs{G}$. Thus, the kernel 
    $\ker\varphi$ is normal in $G$, again, note that $\ker\varphi \ne G$ as the action is non-trivial.
    So $G$ is non-simple.

    Otherwise, for a group $G$ of order $4p^m$, consider the number of a sylow $p$-subgroup of $G$. Then, $n_p \vert 4$ and 
    $n_p \equiv 1 \pmod{p}$. Thus, $n_p = 1$ for $p > 3$. Thus, a group of order $4p^m$ must be non-simple.

  }

  \qs{}{
    Prove that if there exists a chain of subgroups $G_1 \le G_2 \le \cdots \le G$ such that $G = \bigcup_{i=1}^\infty G_i$
    and each $G_i$ is simple, then $G$ is simple.
  }
  \sol{
    Assume $N$ to be a normal subgroup of $G$, then since $G_n < G$ for all $n \in \Nat$, it is acheived that 
    $N \cap G_n \normSg G_n$. But as $G_n$ is simple, then $N \cap G_n = \set{e}$ or $G_n$.

    If there exists integer $n$ such that $N \cap G_n = G_n$. Then $G_n \subseteq N \cap G_{n+1}$, Thus
    $N \cap G_{n+1} = G_{n+1}$ as $N \cap G_{n+1} \ne \set{e}$. This prove that $N \cap G_k = G_k$ for any $k \ge n$ by 
    induction. So, for any $k \ge n$, $G_k < N$. Thus, $\bigcup_{i=k}^\infty G_i = \bigcup_{i=1}^\infty < N$, which is $G < N$.
    But as $N$ is a normal subgroup of $G$, then $N = G$

    If there does not exist such $n$, then there is no $N \cap G_n = G_n$, so for all $n$, $N \cap G_n = \set{e}$.
    Consider that $\bigcup_{i=1}^\infty N \cap G_i = \set{e}$. But then 
    $\bigcup_{i=1}^\infty N \cap G_i = N \cap \bigcup_{i=1}^\infty G_i = N \cap G = \set{e}$
    As $N \normSg G$, then $N = \set{e}$
  } 

\end{document}
