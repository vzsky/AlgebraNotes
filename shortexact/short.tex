% chktex-file 44
% chktex-file 8

\documentclass{article}
\usepackage{amsthm}
\usepackage{amsmath}
\usepackage{amssymb}
\usepackage{amssymb}
\usepackage{amsfonts}
\usepackage{xcolor}
\usepackage{tikz}
\usepackage{fancyhdr}
\usepackage{enumerate}
\usepackage{graphicx}
\usepackage[normalem]{ulem}
\usepackage[most,many,breakable]{tcolorbox}
\usepackage[a4paper, top=80pt, foot=25pt, bottom=50pt, left=0.5in, right=0.5in]{geometry}
\usepackage{hyperref, theoremref}
\hypersetup{
	pdftitle={Assignment},
	colorlinks=true, linkcolor=b!90,
	bookmarksnumbered=true,
	bookmarksopen=true
}
\usepackage{nameref}
\usepackage{parskip}
\pagestyle{fancy}

\usepackage[explicit,compact]{titlesec}
\titleformat{\chapter}[block]{\bfseries\huge}{\thechapter. }{\compact}{#1}
        

%%%%%%%%%%%%%%%%%%%%%
%% Defining colors %%
%%%%%%%%%%%%%%%%%%%%%

\definecolor{lr}{RGB}{188, 75, 81}
\definecolor{r}{RGB}{249, 65, 68}
\definecolor{dr}{RGB}{174, 32, 18}
\definecolor{lo}{RGB}{255, 172, 129}
\definecolor{do}{RGB}{202, 103, 2}
\definecolor{o}{RGB}{238, 155, 0}
\definecolor{ly}{RGB}{255, 241, 133}
\definecolor{y}{RGB}{255, 229, 31}
\definecolor{dy}{RGB}{143, 126, 0}
\definecolor{lb}{RGB}{148, 210, 189}
\definecolor{bg}{RGB}{10, 147, 150}
\definecolor{b}{RGB}{39, 125, 161}
\definecolor{db}{RGB}{0, 95, 115}
\definecolor{p}{RGB}{229, 152, 155}
\definecolor{dp}{RGB}{181, 101, 118}
\definecolor{pp}{RGB}{142, 143, 184}
\definecolor{v}{RGB}{109, 89, 122}
\definecolor{lg}{RGB}{144, 190, 109}
\definecolor{g}{RGB}{64, 145, 108}
\definecolor{dg}{RGB}{45, 106, 79}

\colorlet{mysol}{g}
\colorlet{mythm}{lr}
\colorlet{myqst}{db}
\colorlet{myclm}{lb}
\colorlet{mywrong}{r}
\colorlet{mylem}{o}
\colorlet{mydef}{lg}
\colorlet{mycor}{lb}
\colorlet{myrem}{dr}

%%%%%%%%%%%%%%%%%%%%%

\newcommand{\col}[2]{
  \color{#1}#2\color{black}\,
}

\newcommand{\TODO}[1][5cm]{
  \color{red}TODO\color{black}
  \vspace{#1}
}

\newcommand{\wans}[1]{
	\noindent\color{mywrong}\textbf{Wrong answer: }\color{black}
	#1 


}

\newcommand{\wreason}[1]{
	\noindent\color{mywrong}\textbf{Reason: }\color{black}
	#1 

  
}

\newcommand{\sol}[1]{
	\noindent\color{mysol}\textbf{Solution: }\color{black}
	#1


}

\newcommand{\nt}[1]{
  \begin{note}Note: #1\end{note}
}

\newcommand{\ky}[1]{
  \begin{key}#1\end{key}
}

\newcommand{\pf}[1]{
  \begin{myproof}#1\end{myproof}
}

\newcommand{\qs}[3][]{
  \begin{question}{#2}{#1}#3\end{question}
}

\newcommand{\df}[3][]{
  \begin{definition}{#2}{#1}#3\end{definition}
}

\newcommand{\thm}[3][]{
  \begin{theorem}{#2}{#1}#3\end{theorem}
}

\newcommand{\clm}[3][]{
  \begin{claim}{#2}{#1}#3\end{claim} 
}

\newcommand{\lem}[3][]{
  \begin{lemma}{#2}{#1}#3\end{lemma}
}

\newcommand{\cor}[3][]{
  \begin{corollary}{#2}{#1}#3\end{corollary}
}

\newcommand{\rem}[3][]{
  \begin{remark}{#2}{#1}#3\end{remark}
}

\newcommand{\twoways}[2]{
  \leavevmode\\
  ($\Longrightarrow$): 
  \begin{shift}#1\end{shift}
  ($\Longleftarrow$):
  \begin{shift}#2\end{shift} 
}

\newcommand{\nways}[2]{
  \leavevmode\\
  ($#1$): 
  \begin{shift}#2\end{shift}
}

%%%%%%%%%%%%%%%%%%%%%%%%%%%%%% ENVRN

\newenvironment{myproof}[1][\proofname]{%
	\proof[\bfseries #1: ]
}{\endproof}

\tcbuselibrary{theorems,skins,hooks}
\newtcolorbox{shift}
{%
  before upper={\setlength{\parskip}{5pt}},
  blanker,
	breakable,
	width=0.95\textwidth,
  enlarge left by=0.03\textwidth,
}

\tcbuselibrary{theorems,skins,hooks}
\newtcolorbox{key}
{%
	breakable,
	width=0.95\textwidth,
  enlarge left by=0.03\textwidth,
}

\tcbuselibrary{theorems,skins,hooks}
\newtcolorbox{note}
{%
	enhanced,
	breakable,
	colback = white,
	width=\textwidth,
	frame hidden,
	borderline west = {2pt}{0pt}{black},
	sharp corners,
}

\tcbuselibrary{theorems,skins,hooks}
\newtcbtheorem[]{remark}{Remark}
{%
	enhanced,
	breakable,
	colback = white,
	frame hidden,
	boxrule = 0sp,
	borderline west = {2pt}{0pt}{myrem},
	sharp corners,
	detach title,
  before upper={\setlength{\parskip}{5pt}\tcbtitle\par\smallskip},
	coltitle = myrem,
	fonttitle = \bfseries\sffamily,
	description font = \mdseries,
	separator sign none,
	segmentation style={solid, myrem},
}{rem}

\tcbuselibrary{theorems,skins,hooks}
\newtcbtheorem[number within=section]{lemma}{Lemma}
{%
	enhanced,
	breakable,
	colback = white,
	frame hidden,
	boxrule = 0sp,
	borderline west = {2pt}{0pt}{mylem},
	sharp corners,
	detach title,
  before upper={\setlength{\parskip}{5pt}\tcbtitle\par\smallskip},
	coltitle = mylem,
	fonttitle = \bfseries\sffamily,
	description font = \mdseries,
	separator sign none,
	segmentation style={solid, mylem},
}{lem}

\tcbuselibrary{theorems,skins,hooks}
\newtcbtheorem{claim}{Claim}
{%
  parbox=false,
	enhanced,
	breakable,
	colback = white,
	frame hidden,
	boxrule = 0sp,
	borderline west = {2pt}{0pt}{myclm},
	sharp corners,
	detach title,
  before upper={\setlength{\parskip}{5pt}\tcbtitle\par\smallskip},
	coltitle = myclm,
	fonttitle = \bfseries\sffamily,
	description font = \mdseries,
	separator sign none,
	segmentation style={solid, myclm},
}{clm}

\makeatletter
\newtcbtheorem[number within=section, use counter from=lemma]{theorem}{Theorem}{enhanced,
	breakable,
	colback=white,
	colframe=mythm,
	attach boxed title to top left={yshift*=-\tcboxedtitleheight},
	fonttitle=\bfseries,
	title={#2},
	boxed title size=title,
	boxed title style={%
			sharp corners,
			rounded corners=northwest,
			colback=mythm,
			boxrule=0pt,
		},
	underlay boxed title={%
			\path[fill=mythm] (title.south west)--(title.south east)
			to[out=0, in=180] ([xshift=5mm]title.east)--
			(title.center-|frame.east)
			[rounded corners=\kvtcb@arc] |-
			(frame.north) -| cycle;
		},
	#1
}{thm}
\makeatother

\makeatletter
\newtcbtheorem{question}{Question}{enhanced,
	breakable,
	colback=white,
	colframe=myqst,
	attach boxed title to top left={yshift*=-\tcboxedtitleheight},
	fonttitle=\bfseries,
	title={#2},
	boxed title size=title,
	boxed title style={%
			sharp corners,
			rounded corners=northwest,
			colback=myqst,
			boxrule=0pt,
		},
	underlay boxed title={%
			\path[fill=myqst] (title.south west)--(title.south east)
			to[out=0, in=180] ([xshift=5mm]title.east)--
			(title.center-|frame.east)
			[rounded corners=\kvtcb@arc] |-
			(frame.north) -| cycle;
		},
	#1
}{qs}
\makeatother

\makeatletter
\newtcbtheorem[number within=section]{definition}{Definition}{enhanced,
	breakable,
	colback=white,
	colframe=mydef,
	attach boxed title to top left={yshift*=-\tcboxedtitleheight},
	fonttitle=\bfseries,
	title={#2},
	boxed title size=title,
	boxed title style={%
			sharp corners,
			rounded corners=northwest,
			colback=mydef,
			boxrule=0pt,
		},
	underlay boxed title={%
			\path[fill=mydef] (title.south west)--(title.south east)
			to[out=0, in=180] ([xshift=5mm]title.east)--
			(title.center-|frame.east)
			[rounded corners=\kvtcb@arc] |-
			(frame.north) -| cycle;
		},
	#1
}{def}
\makeatother

\makeatletter
\newtcbtheorem[number within=section, use counter from=lemma]{corollary}{Corollary}{enhanced,
	breakable,
	colback=white,
	colframe=mycor,
	attach boxed title to top left={yshift*=-\tcboxedtitleheight},
	fonttitle=\bfseries,
	title={#2},
	boxed title size=title,
	boxed title style={%
			sharp corners,
			rounded corners=northwest,
			colback=mycor,
			boxrule=0pt,
		},
	underlay boxed title={%
			\path[fill=mycor] (title.south west)--(title.south east)
			to[out=0, in=180] ([xshift=5mm]title.east)--
			(title.center-|frame.east)
			[rounded corners=\kvtcb@arc] |-
			(frame.north) -| cycle;
		},
	#1
}{cor}
\makeatother

% Basic
  \DeclareMathOperator{\lcm}{lcm}
  \newcommand{\Real}{\mathbb{R}}
  \newcommand{\Comp}{\mathbb{C}}
  \newcommand{\Nat}{\mathbb{N}}
  \newcommand{\Rat}{\mathbb{Q}}
  \newcommand{\Int}{\mathbb{Z}}
  \newcommand{\set}[1]{\left\{\, #1 \,\right\}}
  \newcommand{\paren}[1]{\left( \; #1 \; \right)}
  \newcommand{\abs}[1]{\left\lvert #1 \right\rvert}
  \newcommand{\ang}[1]{\left\langle #1 \right\rangle}
  \renewcommand{\to}[1][]{\xrightarrow{\text{#1}}}
  \newcommand{\tol}[1][]{\to{$#1$}}
  \newcommand{\curle}{\preccurlyeq}
  \newcommand{\curge}{\succcurlyeq}
  \newcommand{\mapsfrom}{\leftarrow\!\shortmid}

  \newcommand{\mat}[1]{\begin{bmatrix} #1 \end{bmatrix}}
  \newcommand{\pmat}[1]{\begin{pmatrix} #1 \end{pmatrix}}
  \newcommand{\eqs}[1]{\begin{align*} #1 \end{align*}}
  \newcommand{\case}[1]{\begin{cases} #1 \end{cases}}
  

  % Algebra
  \newcommand{\normSg}[0]{\vartriangleleft}
  \newcommand{\ZMod}[1][n]{\mathbb{Z}/#1\mathbb{Z}}
  \newcommand{\isom}{\simeq}
  \newcommand{\mapHom}{\xrightarrow{\text{hom}}}
  \DeclareMathOperator{\Inn}{Inn}
  \DeclareMathOperator{\Aut}{Aut}
  \DeclareMathOperator{\im}{im}
  \DeclareMathOperator{\ord}{ord}
  \DeclareMathOperator{\Gal}{Gal}
  \DeclareMathOperator{\chr}{char}
  \newcommand{\surjto}{\twoheadrightarrow}
  \newcommand{\injto}{\hookrightarrow}

  % Analysis 
  \newcommand{\limty}[1][k]{\lim_{#1\to\infty}}
  \newcommand{\norm}[1]{\left\lVert#1\right\rVert}
  \newcommand{\darrow}{\rightrightarrows}


\fancyhead[L]{Short Exact Sequence}
\fancyhead[R]{note by \textbf{Touch Sungkawichai}}

\begin{document}

\section{Short Exact Seqeunce}
A short exact sequence is literally a short sequence, a sequence of 3 groups with certain property. The property of the sequence is that 
\[1 \to L \tol[f] G \tol[g] R \to 1 \]
such that all arrows correspond to a group homomorphism, moreover, $\ker \psi = \im \phi$ at all $A \tol[\phi] B \tol[\psi] C$ 
in the sequence.

From the observation that
\begin{itemize}
  \item $1 \to L$ is the trivial group homomorphism that sends $1 \mapsto 1$ with the image being the set of identity in $L$. 
  \item $R \to 1$ is the trivial group homomorphism that sends $r \mapsto 1$ for all $r \in R$, so the kernel is $R$ itself.
\end{itemize}

It follows that $\ker f = 1_L$, thus $f$ is injective, and $\im g = R$, so $g$ is surjective. 

\df{splitting short sequence}{A short sequence is called split if there is a homomorphism $h: R \to G$ such that $h \circ g = id$}

The intuition behind a split short exact sequence is that normally, a seqeunce \[ 1 \to L \to G \tol[g] R \to 1 \] induces an isomorphism 
\[ \frac{G}{\ker g} \isom R \]
This can be viewed as breaking the group $G$ up into partitions according $\ker g$ yields the group structure of $R$. 
When the sequence split, then there is a homomorphism $h$ from $R$ back to $G$ such that $h$ splits $R$ into elements. 
Each elements in $R$ should then be splitted into each of the partition described.

\subsection{Example}
Firstly, the group $D_6 = S_3$ can be viewed as the extension from the sequence 
\[0 \tol[f] \ZMod[3] \to D_6 \tol[g] \ZMod[2] \to 0\] is a short exact sequence with the following homormosphisms
\eqs{
  \ZMod[3] \to D_6 & & \text{given by} & & 0 \mapsto 1   \\ 
                   & &                 & & 1 \mapsto r   \\ 
                   & &                 & & 2 \mapsto r^2 \\ 
  D_6 \to \ZMod[2] & & \text{given by} & & 1, r, r^2 \mapsto 0   \\ 
                   & &                 & & f, fr, fr^2 \mapsto 1 
}
Moreover, the short sequence split by the homomorphism $\ZMod[2] \to D_6$ given by $h$ such that
\[ 0 \mapsto 1 \text{ and } 1 \mapsto gf \]
as it is easy to verify that $g \circ h(0) = g(h(0)) = g(1) = 0$ and $g \circ h (0) = g(h(1)) = g(f) = 1$, which means that
$g \circ h$ is the identity on $R$. 

It can then be seen that the element in $\ZMod[2]$, which are $0$ and $1$ splits into different partition of $D_6$

Another example is the group $\ZMod[6]$, which relates to the following sequence 
\[ 0 \to \ZMod[3] \tol[f] \ZMod[6] \tol[g] \ZMod[2] \to 0 \] with the following homomorphisms
\eqs{
  \ZMod[3] \to \ZMod[6] & & \text{given by} & & 0 \mapsto 0   \\ 
                        & &                 & & 1 \mapsto 2   \\ 
                        & &                 & & 2 \mapsto 4 \\ 
  \ZMod[6] \to \ZMod[2] & & \text{given by} & & 0, 2, 4 \mapsto 0   \\ 
                        & &                 & & 1, 3, 5 \mapsto 1 
}
And the sequence is also split by the homomorphism $h$ given by 
\[ h: 0 \mapsto 0 \text{ and }  1 \mapsto 1 \]

\section{Embedding of Short Exact Seqeunce}

Take the first example given above, \[ 0 \to \ZMod[3] \to D_6 \to \ZMod[2] \to 0 \] for example, 
as the sequence splits, it can be written in another form, embedding all the groups in the sequence as a subgroup of the middle group.

This is \[ 1 \to \set{1, r, r^2} \to \set{1, r, r^2, f, fr, fr^2} \to \set{1, f} \to 1 \]
It can be seen that each pair of corresponding groups are isomorphic.

Generally, every sequence can be embbed in this way.

With this embbeding, $L < G$ and $R < G$ as subgroup. Furthermore, as $L = \im f = \ker g$ and $\ker g \normSg G$, then $L \normSg G$.

\section{Semi Direct Product}

Every short exact sequence that splits induce an extension of two groups. The seqeunce 
\[ 1 \to L \to G \to R \to 1 \] shows an extension of $L$ and $R$ to a bigger group $G$. The group $G$, 
if finite, will be of size $\abs{L}\abs{R}$. It is identified as the semi direct product $G = L \rtimes R$

\df{Semi Direct Product}{
  If there is a map $\phi: R \to \Aut(L)$, then 
  the semi direct product $L \rtimes R$ is the group \[ \set{(l, r) \mid l \in L, r \in R} \] with the operation 
  \[ (l_1, r_1) \cdot (l_2, r_2) = (l_1\phi_{r_1}(l_2), r_1r_2) \] where $\phi_k = \phi(k)$ is an automorphism on $L$.
}

To see that $G$ is the semi direct product, consider the seqeunce 
\[1 \to L \tol[f] G \tol[g] R \to 1 \] with $h: R \to G$ that splits the sequence.

As $g \circ h = id$, then embedding $L$ and $R$ to $G$ gives that any element $r \in R$ is an element of $G$, and every 
element $l \in L$ is also an element of $G$. Hence, $\phi_r(l) = rlr^{-1}$ defines an automorphism of $L$ because $L \normSg G$. 

Therefore, there is $\phi: r \mapsto \phi_r$ that sends $R \to \Aut(L)$, completing the requirement for a semi-direct product.

Thus, every time there is a short exact sequence that splits, the middle group is the, or isomorphic to the, semi direct product of the 
two groups.

\subsection{Direct Product}
A direct product is a special case of semi direct product when $\phi: r \mapsto id$. With this map, the definition of semi direct product
yields $L \rtimes R = \set{ (l, r) \mid l \in L, r \in R }$ with the operation 
\[ (l_1, r_1) \cdot (l_2, r_2) = (l_1l_2, r_1r_2) \] which aligns with the definition of direct product.

\end{document}
